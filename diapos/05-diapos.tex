\documentclass[12pt]{beamer}
\usepackage[spanish]{babel}
\usepackage[utf8]{inputenc}
\usepackage{xcolor}
\usepackage{listings}
\usepackage{textcomp}
\usepackage{mathpazo}
\usepackage{courier}
\usepackage{fancyvrb}
\usepackage{amsmath}
\usepackage{url}
\usepackage{hyperref}

\usepackage{tikz}
\usetikzlibrary{shapes,calc,arrows}

\definecolor{lightbox}{HTML}{FFEEC1}
\definecolor{tablerule}{HTML}{333333}
\usepackage{array}
\usepackage{booktabs}
\usepackage{colortbl}
\newcolumntype{x}{>{\ttfamily}r<{}}
\newcolumntype{z}{!{\color{tablerule}\vline}}
%\newcolumntype{x}{r}

\newcommand{\onelinerule}{\rule[2.3ex]{0pt}{0pt}}
\newcommand{\twolinerule}{\rule[6.2ex]{0pt}{0pt}}
\newcommand{\respuesta}{\framebox[\textwidth]{\twolinerule}}
\newcommand{\nombre}{%
  \begin{tikzpicture}[xscale=.4,yscale=.7]
    \draw (0, 0) rectangle (22, 1);
  \end{tikzpicture}%
}
%\newcommand{\rol}   {\framebox[0.3\textwidth]{\onelinerule}}
\newcommand{\rol}{%
  \begin{tikzpicture}[xscale=.4,yscale=.7]
    \draw[gray!40] ( 0, 0) grid      ( 9, 1);
    \draw          ( 0, 0) rectangle ( 9, 1);
    \draw          (10, 0) rectangle (11, 1);
    \draw (9 + .2, .5) -- (10 - .2, .5);
  \end{tikzpicture}%
}
\newcommand{\li}{\lstinline}
\providecommand{\pond}[1]{[{\small\textbf{#1\%}}]}

\lstdefinelanguage{py}{%
  classoffset=0,%
    morekeywords={%
      False,class,finally,is,return,None,continue,for,lambda,try,%
      True,def,from,nonlocal,while,and,del,global,not,with,print,%
      as,elif,if,or,yield,assert,else,import,pass,break,except,in,raise},%
    keywordstyle=\color{black!80}\bfseries,%
  classoffset=1,
    morekeywords={int,float,str,abs,len,raw_input,exit,range,min,max,%
      set,dict,tuple,list,bool,complex,round,sum,all,any,zip,map,filter,%
      sorted,reversed,dir,file,frozenset,open,%
      array,zeros,ones,arange,linspace,eye,diag,dot},
    keywordstyle=\color{black!50}\bfseries,%
  classoffset=0,%
  sensitive=true,%
  morecomment=[l]\#,%
  morestring=[b]',%
  morestring=[b]",%
  stringstyle=\em,%
}

\lstdefinelanguage{testcase}{%
  moredelim=[is][\bfseries]{`}{`},%
  backgroundcolor=\color{gray!20},%
}

\lstdefinelanguage{file}{%
  frame=single,%
}

\lstset{language=py}
\lstset{basicstyle=\ttfamily}
\lstset{columns=fixed}
\lstset{upquote=true}
\lstset{showstringspaces=false}
\lstset{rangeprefix=\#\ }
\lstset{includerangemarker=false}

\newlist{certamen}{enumerate}{1}
\setlist[certamen]{%
  label=\arabic*.,
  font=\LARGE\bfseries,%
  labelindent=-.5in,%
  leftmargin=0pt,%
  labelsep=1em%
}


\tikzstyle{decision} = [
  diamond,
  very thick,
  draw=red!50!black!50,
  %fill=red!20, 
  aspect=2,
  %text badly centered,
  top color=white,
  bottom color=red!50!black!20,
]
\tikzstyle{stmt} = [
  rectangle,
  very thick,
  draw=blue!50!black!50,
  %fill=blue!20, 
  text centered,
  minimum height=5ex,
  minimum width=5em,
  top color=white,
  bottom color=blue!50!black!20,
]
\tikzstyle{node} = [
  circle,
  very thick,
  draw=orange!50!black!50,
  fill=orange!20,
  minimum size=6ex,
]
\tikzstyle{io} = [
  very thick,
  draw=green!50!black!50,
  trapezium,
  trapezium left angle=80,
  trapezium right angle=-80,
  %fill=green!20!black!10,
  %rounded corners,
  %minimum height=5ex,
  top color=white,
  bottom color=green!50!black!20,
  text centered,
  minimum height=5ex,
  minimum width=5em,
]
\tikzstyle{conn} = [very thick, draw=black!50, -latex']



\usecolortheme{crane}
\usefonttheme{serif}

\title{Estructuras de repetición}
\author{Programación \\ \url{http://progra.usm.cl}}
\date{21 y 22 de marzo de 2011}

\begin{document}
  \begin{frame}
    \maketitle
  \end{frame}

  \begin{frame}
    \frametitle{Problema 1: promedio de 3 números}

    Escriba un programa que muestre el promedio de 3 números reales
    ingresados por el usuario.
    
    \lstinputlisting[language=testcase]{programas/caso-promedio-sin-ciclo.txt}

    \vfill
    \textbf{Solución:}
    \lstinputlisting{programas/promedio-sin-ciclo.py}
  \end{frame}
  

  \begin{frame}
    \frametitle{Problema 2: promedio de \(n\) números}

    Escriba un programa que pregunte al usuario
    cuántos números ingresará,
    le pida que ingrese los números,
    y entregue el promedio como salida.

    \lstinputlisting[language=testcase]{programas/caso-promedio-n-numeros.txt}
  \end{frame}
  
  
  \begin{frame}
    \frametitle{Solución usando ciclo \li!for!}

    \lstinputlisting{programas/promedio-n-numeros.py}
  \end{frame}

  \begin{frame}
    \frametitle{Ruteo}
    \footnotesize
    
    \begin{columns}[t]
      \column{0.40\textwidth}
        \begin{tabular}{xzxzxzxzx}\toprule%
          n &   suma & i &     x & promedio \\ \midrule
          4 &        &   &       &          \\ 
            &   0.0  &   &       &          \\
            &        & 0 &       &          \\
            &        &   &  43.1 &          \\
            &  43.1  &   &       &          \\
            &        & 1 &       &          \\
            &        &   & 12.87 &          \\
            & 55.97  &   &       &          \\
            &        & 2 &       &          \\
            &        &   &   5.0 &          \\
            & 60.97  &   &       &          \\
            &        & 3 &       &          \\
            &        &   & 200.8 &          \\
            & 261.77 &   &       &          \\
            &        &   &       & 65.4425  \\
          \bottomrule
        \end{tabular}
        
      \column{0.50\textwidth}
        \lstinputlisting%
          [basicstyle=\tiny\ttfamily,backgroundcolor=\color{lightbox}]%
          {programas/promedio-n-numeros.py}
    \end{columns}
  \end{frame}

  \begin{frame}
    \frametitle{Ejemplos de rangos}
    \lstinputlisting{programas/ejemplos-rangos.py}
  \end{frame}

  \begin{frame}
    \frametitle{Problema 3: promedio de números hasta cierta suma}

    Escriba un programa que pida al usuario que ingrese varios números.
    Cuando la suma sea mayor que 10,
    el programa debe terminar y entregar el promedio.

    \lstinputlisting[language=testcase]{programas/caso-1-promedio-hasta-suma.txt}

    \lstinputlisting[language=testcase]{programas/caso-2-promedio-hasta-suma.txt}
  \end{frame}
  
  \begin{frame}
    \frametitle{Solución usando ciclo \li!while!}
    \lstinputlisting{programas/promedio-hasta-suma.py}
  \end{frame}

  \begin{frame}
    \frametitle{Ruteo}
    \footnotesize
    
    \begin{columns}
      \column{0.40\textwidth}
        \begin{tabular}{xzxzxzx}\toprule%
           n &  suma &      x & promedio \\ \midrule
           0 &       &        &          \\
             &  0.0  &        &          \\
             &       &    5.0 &          \\
           1 &       &        &          \\
             &  5.0  &        &          \\
             &       &    1.2 &          \\
           2 &       &        &          \\
             &  6.2  &        &          \\
             &       &    2.4 &          \\
           3 &       &        &          \\
             &  8.6  &        &          \\
             &       &    3.3 &          \\
           4 &       &        &          \\
             & 11.9  &        &          \\
             &       &        & 2.975    \\
          \bottomrule
        \end{tabular}

      \column{0.50\textwidth}
        \lstinputlisting%
          [basicstyle=\tiny\ttfamily,backgroundcolor=\color{lightbox}]%
          {programas/promedio-hasta-suma.py}
    \end{columns}
  \end{frame}

  \begin{frame}
    \begin{columns}
      \column{0.5\textwidth}
        \includegraphics[scale=.8]{../diagramas/while-promedio.pdf}
      \column{0.5\textwidth}
        \lstinputlisting[basicstyle=\footnotesize\ttfamily,linerange=3-8]%
          {programas/promedio-hasta-suma.py}
        
    \end{columns}
    
  \end{frame}
\end{document}

