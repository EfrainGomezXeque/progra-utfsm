\documentclass[10pt]{article}
\usepackage{beamerarticle}
\usepackage[spanish]{babel}
\usepackage[utf8]{inputenc}
\usepackage{fullpage}
\usepackage{xcolor}
\usepackage{listings}
\usepackage{textcomp}
\usepackage{mathpazo}
\usepackage{courier}
\usepackage{fancyvrb}
\usepackage{amsmath}
\usepackage{url}
\usepackage{hyperref}
\usepackage{pgfpages}
\usepackage{wrapfig}
\usepackage{enumitem}

\setjobnamebeamerversion{02-interprete-expresiones-tipos-diapos}

\newcommand{\onelinerule}{\rule[2.3ex]{0pt}{0pt}}
\newcommand{\twolinerule}{\rule[6.2ex]{0pt}{0pt}}
\newcommand{\respuesta}{\framebox[\textwidth]{\twolinerule}}
\newcommand{\nombre}{%
  \begin{tikzpicture}[xscale=.4,yscale=.7]
    \draw (0, 0) rectangle (22, 1);
  \end{tikzpicture}%
}
%\newcommand{\rol}   {\framebox[0.3\textwidth]{\onelinerule}}
\newcommand{\rol}{%
  \begin{tikzpicture}[xscale=.4,yscale=.7]
    \draw[gray!40] ( 0, 0) grid      ( 9, 1);
    \draw          ( 0, 0) rectangle ( 9, 1);
    \draw          (10, 0) rectangle (11, 1);
    \draw (9 + .2, .5) -- (10 - .2, .5);
  \end{tikzpicture}%
}
\newcommand{\li}{\lstinline}
\providecommand{\pond}[1]{[{\small\textbf{#1\%}}]}

\lstdefinelanguage{py}{%
  classoffset=0,%
    morekeywords={%
      False,class,finally,is,return,None,continue,for,lambda,try,%
      True,def,from,nonlocal,while,and,del,global,not,with,print,%
      as,elif,if,or,yield,assert,else,import,pass,break,except,in,raise},%
    keywordstyle=\color{black!80}\bfseries,%
  classoffset=1,
    morekeywords={int,float,str,abs,len,raw_input,exit,range,min,max,%
      set,dict,tuple,list,bool,complex,round,sum,all,any,zip,map,filter,%
      sorted,reversed,dir,file,frozenset,open,%
      array,zeros,ones,arange,linspace,eye,diag,dot},
    keywordstyle=\color{black!50}\bfseries,%
  classoffset=0,%
  sensitive=true,%
  morecomment=[l]\#,%
  morestring=[b]',%
  morestring=[b]",%
  stringstyle=\em,%
}

\lstdefinelanguage{testcase}{%
  moredelim=[is][\bfseries]{`}{`},%
  backgroundcolor=\color{gray!20},%
}

\lstdefinelanguage{file}{%
  frame=single,%
}

\lstset{language=py}
\lstset{basicstyle=\ttfamily}
\lstset{columns=fixed}
\lstset{upquote=true}
\lstset{showstringspaces=false}
\lstset{rangeprefix=\#\ }
\lstset{includerangemarker=false}

\newlist{certamen}{enumerate}{1}
\setlist[certamen]{%
  label=\arabic*.,
  font=\LARGE\bfseries,%
  labelindent=-.5in,%
  leftmargin=0pt,%
  labelsep=1em%
}


\newcommand{\diapo}[1]{
  \begin{wrapfigure}{r}{.50\textwidth}
    \hfill\fbox{\includeslide[width=.45\textwidth]{#1}}
  \end{wrapfigure}
}


\title{Uso de intérprete, expresiones y tipos}
\author{Programación \\ \url{http://progra.usm.cl}}
\date{}

\begin{document}
  \maketitle

  \section*{Objetivos de la clase}
  \begin{itemize}
    \item Mostrar cómo ejecutar un programa simple.
    \item Mostrar cómo usar la consola interactiva.
    \item Presentar los conceptos de expresión, asignación y tipos.
    \item Presentar los operadores y sus propiedades.
    \item Presentar las sentencias de entrada y salida.
  \end{itemize}

  \section*{Diapositivas}

  \diapo{prog-ejemplo}

  Esta diapositiva es importante
  porque introduce algunas convenciones
  que serán seguidas durante todo el semestre
  en las diapositivas, los apuntes y las evaluaciones:
  \begin{itemize}
    \item los enunciados explican el problema en prosa;
    \item el caso de prueba muestra cómo se ve la ejecución de un programa,
      donde la entrada ingresada por el usuario aparece en negrita;
    \item el código es el programa que puede ser copiado y pegado
      en un editor de texto para ser guardado y ejecutado.
  \end{itemize}
  Presente estas convenciones en detalle a los estudiantes.

  Antes de mostrar el programa en ejecución,
  explicar también los siguientes aspectos del ejemplo,
  para que los estudiantes comiencen a familiarizarse con el código:
  \begin{itemize}
    \item la primera línea del programa es la entrada,
      la segunda línea es el proceso
      y la tercera línea es la salida;
    \item la función \li!raw_input! recibe como parámetro
      el texto que será mostrado en la pantalla,
      y entrega com resultado el texto ingresado por el usuario;
    \item la función \li!float! convierte el texto a un número real,
      lo que es necesario para poder hacer operaciones aritméticas
      con el valor ingresado;
    \item la expresión \((f - 32)\cdot 5/9\)
      es evaluada con el valor que tiene \(f\) en ese momento,
      y al resultado se le asocia el nombre \li!c!;
    \item la sentencia \li!print!
      imprime el texto entre comillas tal cual aparece,
      y a continuación el valor que tiene \li!c! en ese momento.
  \end{itemize}

  Muestre en vivo cómo ingresar el programa en el editor de texto y ejecutarlo.
  Hágalo de varias maneras:
  \begin{itemize}
    \item primero usando Bloc de Notas + doble clic en el ícono del programa,
    \item luego, ingresando el código en IDLE y ejecutándolo desde ahí
      (presionando F5, o desde el menú),
    \item finalmente,
      ingresando el programa línea por línea en la consola de Python.
  \end{itemize}

  \diapo{expresion}
  
  Explique a grandes rasgos qué hace cada expresión.

  Explique la asignación \li!x = 3.5!.

  En Python, una asignación es un nombre que se le asocia a un valor.
  Mencione que el nombre es llamado \emph{variable},
  y que su nombre puede tener letras, números y underscores.

  A diferencia de C y Java,
  la variable no tiene asociados ni un tipo ni una ubicación en la memoria.
  Cada asignación crea o religa un nombre.

  Las asignaciones siguen la sintaxis \li!nombre = valor!.
  Algunas asignaciones inválidas
  que pueden ser mostradas en la consola son:
  \begin{itemize}
    \item \li!x + 2 = 3!,
    \item \li!7 = a!.
  \end{itemize}
  Estas sentencias arrojan un \emph{error de sintaxis},
  que significa que el código no satisface las reglas del lenguaje.

  Una vez asignada una variable,
  es posible usarla en cualquier expresión.
  Al evaluar la expresión,
  tomará el valor que tiene la variable en ese momento.

  \diapo{operadores-aritmeticos}
  
  Estos operadores son los mismos de matemáticas,
  (salvo que la multiplicación es \li!*!
  y la exponenciación es \li!**!)
  así que lo único que es importante explicar cómo funciona
  es el operador de módulo \li!%!.

  Los usos más comunes del módulo son
  comprobar divisibilidad y extraer dígitos de un entero.
  Muestre algunos ejemplos de esto en la consola.

  Respecto a los tipos, la regla es:
  si al menos uno de los operandos es real,
  entonces el resultado es real.
  El caso más capcioso es la división.
  Muestre en la consola que \li!15/4! es \li!3! y no \li!3.75!.
  La división entera siempre trunca los decimales.

  Muestre también en la consola
  algunos ejemplos de precedencia de operadores.
  Por ejemplo,
  preguntar cuál es el resultado de \li!2 + 3 * 4!
  (la multiplicación se hace primero)
  y de \li!2 ** 3 ** 4!
  (la exponenciación se asocia de derecha a izquierda),
  y mostrar cómo cambiar la precedencia usando paréntesis.
  
  \diapo{operadores-logicos}

  Los operadores lógicos son los mismos de matemáticas.

  Si bien no es buena idea mencionarlo todavía,
  es conveniente saber que estos operadores pueden ser usados
  con operandos de otros tipos.
  Los valores \li!0!, \li!''! y \li!None!, entre otros,
  son considerados falsos,
  y el resto son considerados verdaderos.

  \diapo{funciones}

  El uso de la función para obtener un valor
  se llama \emph{llamar a la función}.

  El valor pasado a la función
  se llama \emph{argumento} o \emph{parámetro}
  (la distinción entre ambos se verá en la unidad de funciones).

  Las funciones matemáticas
  deben ser importadas desde el módulo \li!math!.

  Las funciones \li!min! y \li!max! aceptan
  una cantidad arbitraria de parámetros.

  \diapo{operadores-texto}

  La operación representada por el \li!+! aplicado a strings
  se llama \emph{concatenación}.
  No puede ser una suma, pues no es una operación conmutativa.

  Los caracteres se cuentan desde cero,
  y por eso \li!"amarillo"[4]! es \li!"i"! y no \li!"r"!.
  Los índices de un string \li!s! van desde \li!0!
  hasta \li!len(s) - 1!.

  \diapo{tipos}

  El tipo de un valor
  determina cuál es su dominio,
  qué operaciones se le puede aplicar
  y cómo es representado internamente por el computador.

  El literal \li!6.02e23! está en notación científica.
  Su valor es \(6.02\cdot 10^{23}\).

  Mencione que los valores de tipo texto se llaman \emph{strings}
  y los valores lógicos se llaman \emph{booleanos}.
  Los nombres no son muy descriptivos,
  pero son los que se usan en programación.

  No hay diferencia entre los strings
  representados con comillas simples y dobles:
  \li!"hola" == 'hola'!.

  Explique que las comillas no son parte del string.

  Dos tipos elementales que no aparecen en la diapositiva
  son el valor nulo \li!None! (que tiene su propio tipo).
  y los números complejos (\li!complex!).
  Si bien aparecen mencionados en el apunte oficial,
  no se les dará mucha importancia en la asignatura.

  La parte imaginaria de los números complejos
  se denota con una jota: \li!2+3j!.
  El constructor de \li!complex!
  recibe como parámetros las partes real e imaginaria:
  \li!z = complex(2, 3)!.

  Puede mostrar en la consola algunos ejemplos
  de operaciones con tipos incorrectos.
  Por ejemplo:
  \begin{itemize}
    \item \li!'perro' * 'gato'!
    \item \li!len(2)!
    \item \li!2[0]!
    \item \li!'hola'(2, 3)!
    \item \li!(2+3j) < (1-2j)!
  \end{itemize}
  Estas operaciones arrojan una excepción
  llamada \emph{error de tipo}.
  Enseñe a los estudiantes a leer los mensajes de error.

  \diapo{ejemplos-entrada}

  La función \li!raw_input! siempre retorna un string.
  Además, su parámetro es el mensaje que se mostrará al usuario.

  Las funciones \li!int! y \li!float!
  convierten el string a un valor entero y real,
  respectivamente.

  \diapo{ejemplos-salida}

  La sentencia \li!print! muestra por pantalla
  la representación en texto
  de todos los valores que le son pasados.
  El espacio que separa los valores
  y el salto de línea al final
  son introducidos automáticamente.

  Todo lo que va en el mismo \li!print!
  es impreso en la misma línea en la pantalla.

  Si se pone una coma solitaria al final del \li!print!,
  el salto de línea no es impreso.

  En \url{http://progra.usm.cl/apunte/ejercicios/1/index.html#programas-simples}
  puede encontrar ejercicios para resolver en la segunda mitad de la clase.
  
\end{document}


