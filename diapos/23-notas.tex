\documentclass[10pt]{article}
\usepackage{beamerarticle}
\usepackage[spanish]{babel}
\usepackage[utf8]{inputenc}
\usepackage{fullpage}
\usepackage{xcolor}
\usepackage{listings}
\usepackage{textcomp}
\usepackage{mathpazo}
\usepackage{courier}
\usepackage{fancyvrb}
\usepackage{amsmath}
\usepackage{url}
\usepackage{hyperref}
\usepackage{pgfpages}
\usepackage{wrapfig}
\hyphenation{dia-po-si-tiva ope-ra-ciones}

\setjobnamebeamerversion{23-diapos}

\newcommand{\onelinerule}{\rule[2.3ex]{0pt}{0pt}}
\newcommand{\twolinerule}{\rule[6.2ex]{0pt}{0pt}}
\newcommand{\respuesta}{\framebox[\textwidth]{\twolinerule}}
\newcommand{\nombre}{%
  \begin{tikzpicture}[xscale=.4,yscale=.7]
    \draw (0, 0) rectangle (22, 1);
  \end{tikzpicture}%
}
%\newcommand{\rol}   {\framebox[0.3\textwidth]{\onelinerule}}
\newcommand{\rol}{%
  \begin{tikzpicture}[xscale=.4,yscale=.7]
    \draw[gray!40] ( 0, 0) grid      ( 9, 1);
    \draw          ( 0, 0) rectangle ( 9, 1);
    \draw          (10, 0) rectangle (11, 1);
    \draw (9 + .2, .5) -- (10 - .2, .5);
  \end{tikzpicture}%
}
\newcommand{\li}{\lstinline}
\providecommand{\pond}[1]{[{\small\textbf{#1\%}}]}

\lstdefinelanguage{py}{%
  classoffset=0,%
    morekeywords={%
      False,class,finally,is,return,None,continue,for,lambda,try,%
      True,def,from,nonlocal,while,and,del,global,not,with,print,%
      as,elif,if,or,yield,assert,else,import,pass,break,except,in,raise},%
    keywordstyle=\color{black!80}\bfseries,%
  classoffset=1,
    morekeywords={int,float,str,abs,len,raw_input,exit,range,min,max,%
      set,dict,tuple,list,bool,complex,round,sum,all,any,zip,map,filter,%
      sorted,reversed,dir,file,frozenset,open,%
      array,zeros,ones,arange,linspace,eye,diag,dot},
    keywordstyle=\color{black!50}\bfseries,%
  classoffset=0,%
  sensitive=true,%
  morecomment=[l]\#,%
  morestring=[b]',%
  morestring=[b]",%
  stringstyle=\em,%
}

\lstdefinelanguage{testcase}{%
  moredelim=[is][\bfseries]{`}{`},%
  backgroundcolor=\color{gray!20},%
}

\lstdefinelanguage{file}{%
  frame=single,%
}

\lstset{language=py}
\lstset{basicstyle=\ttfamily}
\lstset{columns=fixed}
\lstset{upquote=true}
\lstset{showstringspaces=false}
\lstset{rangeprefix=\#\ }
\lstset{includerangemarker=false}

\newlist{certamen}{enumerate}{1}
\setlist[certamen]{%
  label=\arabic*.,
  font=\LARGE\bfseries,%
  labelindent=-.5in,%
  leftmargin=0pt,%
  labelsep=1em%
}


\newcommand{\diapo}[1]{
  \begin{wrapfigure}{r}{.50\textwidth}
    \hfill\fbox{\includeslide[width=.45\textwidth]{#1}}
  \end{wrapfigure}
}


\title{Arreglos numéricos}
\author{Programación \\ \url{http://progra.usm.cl}}
\date{}

\begin{document}
  \maketitle

  \section*{Objetivos de la clase}
  \begin{itemize}
    \item Introducir el tipo \li!array!
      del módulo \li!numpy!
      para representar arreglos numéricos.
    \item Explicar la motivación
      para usar arreglos numéricos
      en vez de listas.
  \end{itemize}

  \section*{Diapositivas}

  \diapo{intro-arreglos}

  En esta clase
  comenzaremos a pasar arreglos numéricos.
  Para esto,
  necesitamos instalar la biblioteca NumPy
  que provee los tipos de datos y funciones
  necesarias para operar con arreglos.
  El link presentado en la primera diapositiva
  lleva a la página desde donde se puede descargar
  el instalador para Windows.

  Un \emph{arreglo}
  es una secuencia ordenada de valores,
  de manera muy similar a una lista.
  Sin embargo,
  hay algunas diferencias fundamentales
  entre ambos tipos de datos,
  de las cuales las más relevantes
  para los alumnos son:
  \begin{itemize}
    \item los arreglos en general tienen tamaño fijo,
      mientras que las listas suelen ir creciendo
      o decreciendo durante la ejecución del programa;
    \item todos los elementos de un arreglo
      tienen el mismo tipo, mientras que en las listas
      esto no necesariamente es así.
  \end{itemize}

  Los arreglos de NumPy son, pues,
  muy parecidos a los arreglos que conocíamos
  de C y Pascal.
  Por otra parte,
  proveen operaciones mucho más convenientes
  para trabajar con arreglos completos,
  por lo que no suele ser necesario
  operar elemento a elemento.
  En este sentido,
  son más parecidos a los arreglos de Fortran.

  A diferencia de los arreglos en lenguajes compilados,
  los arreglos de NumPy no son declarados,
  sino que son creados dinámicamente.

  Las ventajas que ofrecen los arreglos son:
  \begin{itemize}
    \item eficiencia (los elementos se guardan contiguos
      en la memoria, a diferencia de las listas);
    \item proveen operaciones que operan
      sobre todos los datos a la vez,
      y no uno por uno;
    \item proveen funciones convenientes
      para crearlos y operar sobre ellos.
  \end{itemize}
  Por estos motivos,
  los arreglos numéricos son las estructuras de datos
  más usadas en aplicaciones de ingeniería.

  En la clase de hoy
  cubriremos sólo arreglos de una dimensión.
  En la próxima clase veremos arreglos
  de varias dimensiones.

  \diapo{crear-arreglo-valores}

  Para crear un arreglo
  que tenga los valores que uno desea,
  la manera más simple es
  pasar la lista de valores
  a la función \li!array!,
  que es el constructor del tipo de datos
  del mismo nombre.

  El arreglo tiene un tipo asociado,
  que es el más general de los tipos
  de sus elementos.
  Por ejemplo,
  si hay enteros y reales,
  el arreglo será de reales.
  Si apareciera un número complejo,
  entonces el arreglo completo será de complejos.

  Una manera de convertir un arreglo a otro tipo
  es usar el método \li!astype!,
  que recibe el nuevo tipo como parámetro.

  \diapo{crear-arreglo-funciones}

  Muchas veces no suele ser conveniente
  crear los arreglos indicando sus elementos
  uno por uno.
  Esto es especialmente cierto
  cuando los valores tienen
  algún tipo de estructura.

  La función \li!zeros(n)!
  crea un arreglo de tamaño \li!n!
  inicializado con ceros.

  La función \li!ones(n)!
  crea un arreglo de tamaño \li!n!
  inicializado con unos.

  La función \li!arange(a, b, c)!
  funciona igual que la función \li!range!,
  con la diferencia que sus parámetros
  pueden ser de tipo \li!float!
  y el valor retornado siempre es un arreglo.
  Los parámetros \li!b! y \li!c!
  son opcionales.

  La función \li!linspace(a, b, n)!
  crea un arreglo de \li!n! valores equiespaciados
  que van desde \li!a! hasta \li!b!.

  Explique cada función
  hasta que quede claro qué es lo retorna cada una.

  Estas funciones se pueden combinar
  con operaciones sobre arreglos
  para crear otros arreglos
  con estructuras similares,
  lo que está ejemplificado en los ejercicios
  de esta clase.

  En esta diapositiva,
  se usa la forma \li!from modulo import *!
  para importar todas las funciones y clases
  que están definidas en un módulo.
  En este caso, se hace por brevedad,
  ya que el espacio de la diapositiva es reducido.
  También lo puede hacer así en clases.
  Lo que hay que tener claro
  es que en general esto no es una buena práctica,
  principalmente porque los nombres importados
  pueden ocultar objetos que existían desde antes
  y que están asignados con el mismo nombre.
  Un ejemplo concreto son las funciones
  \li!sum!, \li!max! y \li!min!
  que están definidas por NumPy;
  al importar todos los nombres,
  las funciones homónimas de Python
  ya no están disponibles directamente.

  En las diapositivas siguientes,
  se supondrá que los nombres a utilizar
  ya han sido importados.

  \diapo{operaciones-arreglos}

  Los arreglos soportan
  todas las operaciones aritméticas elementales.
  La principal característica de los arreglos
  es que todas las operaciones se aplican
  elemento por elemento,
  independientemente del tamaño del arreglo.

  Cuando los operandos son
  un arreglo y un escalar
  (es decir, un único valor),
  se usa el escalar como operando
  para cada elemento del arreglo.

  Cuando los dos operandos son arreglos,
  la operación es aplicada
  elemento a elemento.
  Para esto, los arreglos deben tener
  exactamente el mismo tamaño,
  y el resultado será también de ese tamaño.

  Explique los ejemplos de la diapositiva
  para que esto quede muy claro.

  \diapo{comparacion-arreglos}

  Los operadores relacionales
  también operan elemento a elemento,
  lo que puede resultar un poco menos intuitivo.
  Al comparar dos arreglos
  (por ejemplo, usando \li!<! o \li!==!),
  el resultado no es \li!True! ni \li!False!,
  sino un arreglo de valores booleanos
  indicando si los elementos correspondientes
  son o no iguales.

  Para reducir los valores a un único valor,
  se puede usar las funciones \li!any! y \li!all!,
  dependiendo de lo que se desee saber.
  \li!any! retorna \li!True!
  si por lo menos un elemento del arreglo
  es verdadero.
  \li!all! retorna \li!True!
  si todos los elementos del arreglo
  son verdaderos.

  Por lo tanto,
  para saber si dos arreglos son iguales,
  debe hacerse así: \li!all(a == b)!.

  Las funciones \li!any! y \li!all!
  deben ser importadas desde NumPy.
  Python provee dos funciones con el mismo nombre
  que funcionan con listas,
  pero no están pensadas
  para trabajar con arreglos.

  \diapo{modificar-arreglos}

  Los elementos de los arreglos
  pueden ser obtenidos
  a través de su índice.
  Los elementos se pueden editar
  al igual que en las listas.

  Además, se puede obtener y modificar
  rebanadas de los arreglos
  usando la sintaxis \li!arreglo[a:b:c]!,
  que significia
  «los elementos desde el \li!a!-ésimo hasta el \li!b!-ésimo,
  tomando de \li!c! en \li!c!» (sin incluir el \li!b!-ésimo).

  En los ejemplos,
  se modifica secciones enteras del arreglo
  asignándoles un valor escalar.
  También podría asignarse
  un arreglo más pequeño
  a una sección de un arreglo más grande.

  \diapo{arreglos-aleatorios}

  El módulo \li!numpy.random!
  provee muchas funciones
  para generar arreglos de números aleatorios.
  Aquí sólo presentaremos la función \li!random!,
  que crea arreglos aleatorios
  con valores reales entre 0 y 1.

  Explique qué es un valor aleatorio,
  por qué al llamar la misma función varias veces
  se obtienen resultados distintos,
  y haga notar en los ejemplos
  que los valores siempre están entre 0 y 1.

  \diapo{ejercicio-crear-arreglos}

  Ejercicio práctico
  para crear arreglos con ciertas estructuras.
  El objetivo es resolverlos usando
  sólo operaciones sobre arreglos,
  evitando utilizar ciclos.

  Guíe a los alumnos
  para que a ellos mismos se les ocurra las soluciones.
  Pruebe en la consola
  las ideas que ellos propongan
  para ponerlas a prueba.

  Las soluciones están en los archivos anexos
  a la clase.

  \diapo{ejercicio-diferencias}

  Ejercicio sencillo:
  obtener las diferencias finitas de un arreglo
  (las diferencias entre elementos consecutivos).
  Nuevamente, la idea es evitar usar ciclos.

  Es importante notar
  que el resultado no tiene el mismo tamaño
  que el argumento,
  sino uno menos.
  Procure que los alumnos se den cuenta
  de este detalle mientras intentan resolver
  el ejercicio.

  La solución es obtener dos rebanadas,
  cada una omitiendo uno de los extremos,
  y luego restarlas.

  En la solución incluida en los archivos anexos,
  se usa el atributo \li!a.size!
  para obtener el tamaño del arreglo.
  La función \li!len! funciona correctamente
  con arreglos de una dimensión,
  por lo que podría ser utilizada.
  Sin embargo, no entrega el tamaño correcto
  con los arreglos que veremos en la próxima clase.

  \diapo{vectorizacion}

  NumPy provee también
  funciones matemáticas
  que, al igual que las operaciones básicas,
  también son aplicadas elemento a elemento.

  En este ejemplo,
  se crea un arreglo de 5 elementos
  que van desde 0 hasta \(\pi/2\),
  y se le aplica la función seno.
  Como se puede ver en el resultado,
  se obtienen valores que van
  creciendo desde 0 a 1,
  tal como lo hace la función seno
  en el intervalo \([0, \pi/2]\).

  Puede mostrar en la consola
  cuál es el resultado de la llamada
  a \li!linspace!,
  y hacer ver cómo cada elemento de \li!a!
  corresponde a uno del arreglo \li!sin(a)!.

  \label{ejercicio-integral}

  Ejercicio para calcular
  la suma del área de rectángulos.
  Todos los rectángulos tienen la misma base,
  y las alturas están dadas por una función.

  La solución es obtener un arreglo
  con las coordenadas \(x\)
  del lado izquierdo de cada rectángulo,
  y luego aplicarle la función \li!exp!
  (que debe ser importada desde NumPy).
  Con esto,
  se obtiene el arreglo
  de las alturas de los rectángulos.
  Luego, basta con multiplicar
  todo el arreglo
  por el largo de la base
  de los rectángulos
  y sumarlos.

  Para sumar los elementos de un arreglo,
  puede usar la función \li!sum!
  provista por NumPy,
  que está diseñada para trabajar sobre arreglos.

  Nuevamente,
  intente que a los alumnos se les ocurra la solución.
  También discuta cómo obtener el largo de la base
  de manera simple.

  La solución está incluida
  en los archivos anexos.
  Una vez discutido el problema con los alumnos,
  explique la solución en detalle.

\end{document}

