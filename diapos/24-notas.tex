\documentclass[10pt]{article}
\usepackage{beamerarticle}
\usepackage[spanish]{babel}
\usepackage[utf8]{inputenc}
\usepackage{fullpage}
\usepackage{xcolor}
\usepackage{listings}
\usepackage{textcomp}
\usepackage{mathpazo}
\usepackage{courier}
\usepackage{fancyvrb}
\usepackage{amsmath}
\usepackage{url}
\usepackage{hyperref}
\usepackage{pgfpages}
\usepackage{wrapfig}
\hyphenation{}

\setjobnamebeamerversion{24-diapos}

\newcommand{\onelinerule}{\rule[2.3ex]{0pt}{0pt}}
\newcommand{\twolinerule}{\rule[6.2ex]{0pt}{0pt}}
\newcommand{\respuesta}{\framebox[\textwidth]{\twolinerule}}
\newcommand{\nombre}{%
  \begin{tikzpicture}[xscale=.4,yscale=.7]
    \draw (0, 0) rectangle (22, 1);
  \end{tikzpicture}%
}
%\newcommand{\rol}   {\framebox[0.3\textwidth]{\onelinerule}}
\newcommand{\rol}{%
  \begin{tikzpicture}[xscale=.4,yscale=.7]
    \draw[gray!40] ( 0, 0) grid      ( 9, 1);
    \draw          ( 0, 0) rectangle ( 9, 1);
    \draw          (10, 0) rectangle (11, 1);
    \draw (9 + .2, .5) -- (10 - .2, .5);
  \end{tikzpicture}%
}
\newcommand{\li}{\lstinline}
\providecommand{\pond}[1]{[{\small\textbf{#1\%}}]}

\lstdefinelanguage{py}{%
  classoffset=0,%
    morekeywords={%
      False,class,finally,is,return,None,continue,for,lambda,try,%
      True,def,from,nonlocal,while,and,del,global,not,with,print,%
      as,elif,if,or,yield,assert,else,import,pass,break,except,in,raise},%
    keywordstyle=\color{black!80}\bfseries,%
  classoffset=1,
    morekeywords={int,float,str,abs,len,raw_input,exit,range,min,max,%
      set,dict,tuple,list,bool,complex,round,sum,all,any,zip,map,filter,%
      sorted,reversed,dir,file,frozenset,open,%
      array,zeros,ones,arange,linspace,eye,diag,dot},
    keywordstyle=\color{black!50}\bfseries,%
  classoffset=0,%
  sensitive=true,%
  morecomment=[l]\#,%
  morestring=[b]',%
  morestring=[b]",%
  stringstyle=\em,%
}

\lstdefinelanguage{testcase}{%
  moredelim=[is][\bfseries]{`}{`},%
  backgroundcolor=\color{gray!20},%
}

\lstdefinelanguage{file}{%
  frame=single,%
}

\lstset{language=py}
\lstset{basicstyle=\ttfamily}
\lstset{columns=fixed}
\lstset{upquote=true}
\lstset{showstringspaces=false}
\lstset{rangeprefix=\#\ }
\lstset{includerangemarker=false}

\newlist{certamen}{enumerate}{1}
\setlist[certamen]{%
  label=\arabic*.,
  font=\LARGE\bfseries,%
  labelindent=-.5in,%
  leftmargin=0pt,%
  labelsep=1em%
}


\newcommand{\diapo}[1]{
  \begin{wrapfigure}{r}{.50\textwidth}
    \hfill\fbox{\includeslide[width=.45\textwidth]{#1}}
  \end{wrapfigure}
}


\title{Arreglos bidimensionales}
\author{Programación \\ \url{http://progra.usm.cl}}
\date{Clase 24}

\begin{document}
  \maketitle

  \section*{Objetivos de la clase}
  \begin{itemize}
    \item Presentar el concepto
      de arreglo bidimensional.
    \item Enseñar algunas operaciones básicas
      de los arreglos bidimensionales de NumPy.
  \end{itemize}

  \section*{Diapositivas}

  Los arreglos que vimos en la clase anterior
  son secuencias lineales de valores.

  Los arreglos bidimensionales, por otra parte,
  son tablas de valores.
  Cada elemento de un arreglo bidimensional
  está simultáneamente en una fila y en una columna.

  Posiblemente,
  muchos alumnos no hayan visto aún en algún curso de matemáticas
  el concepto de matriz.
  Mencione que «matriz» es simplemente otro nombre
  para referirse a los arreglos bidimensionales.
  Al menos en programación,
  ambos términos pueden ser considerados sinónimos.

  La mayoría de los conceptos enseñados en la clase anterior
  son aplicables también a los arreglos bidimensionales.
  Por ejemplo, las operaciones se hacen término a término,
  se puede modificar elementos o regiones del arreglo, y
  todos los elementos son del mismo tipo.

  \diapo{crear-arreglo-bidmensional}

  La manera de crear un arreglo bidimensional
  indicando sus valores explícitamente
  es pasar una lista de listas
  a la función \li!array!.
  Cada una de las listas interiores
  representa una fila de la matriz.
  Las listas deben ser todas del mismo largo,
  o si no ocurrirá un error de valor.

  Todos los arreglos tienen un atributo \li!shape!
  que indica su forma.
  Es una tupla con los tamaños a lo largo de cada dimensión.
  En el caso de los arreglos bidimensionales,
  los valores de la tupla son los números de filas y de columnas.
  Típicamente,
  un algoritmo que debe recorrer los elementos de la matriz
  comienza con \li!m, n = a.shape!.

  El valor del atributo \li!size! es la cantidad de elementos
  que tiene el arreglo, independiente de su forma.

  Ojo con la función \li!len!, que no entrega el tamaño del arreglo,
  sino el número de filas.

  \diapo{crear-arreglo-bidmensional-funciones}

  Las funciones \li!zeros! y \li!ones!
  también permiten crear arreglos de más de una dimensión.
  Para esto, debe pasarse la tupla con la forma del arreglo
  como argumento.

  Esta regla se aplica también a la mayoría de las funciones
  que reciben dimensiones de arreglos como parámetros.
  Por ejemplo,
  la función \li!numpy.random.random! también acepta
  una tupla como parámetro.

  \diapo{cambiar-forma}

  El método \li!reshape! de los arreglos
  retorna un arreglo que tiene los mismos elementos
  pero con otra forma.
  El reordenamiento se hace siempre a lo largo de las filas.

  El parámetro de \li!reshape! es una tupla
  indicando la nueva forma del arreglo.
  El número de elementos en la nueva forma
  (es decir, el producto de los elementos de la tupla)
  debe coincidir con el tamaño del arreglo.

  Note que el arreglo original no es modificado,
  sino que un arreglo nuevo es retornado.
  Sin embargo, ambos arreglos comparten el buffer de datos,
  por lo que al modificar uno de los arreglos,
  los cambios se ven reflejados también en el otro.
  En general esto es así en las operaciones de NumPy:
  las operaciones de arreglos entregan vistas
  a arreglos existentes, no copias de arreglos.

  \diapo{indices}

  Los arreglos se indexan usando tuplas \li!i, j!
  (al igual que Pascal,
  y a diferencia de C, donde se ocupan dos pares de corchetes).
  \li!a[i, j]! es el elemento en la fila \li!i! y la columna \li!j!.

  En el primer ejemplo,
  se obtiene el elemento en la posición \((2, 3)\).

  En el segundo ejemplo,
  se obtiene la fila 2 completa del arreglo.
  Esto se hace indicando que el índice de las columnas
  toma todos los valores posibles.
  La sintaxis \li!:! significa «desde el principio hasta el final».
  El ejemplo también podría haber sido escrito como \li!a[2, 0:4]!.

  El tercer ejemplo entrega una región más complicada del arreglo.

  \diapo{ejercicio-crear-arreglo}

  Resuelva estos ejercicios con calma y detalladamente,
  hasta que quede claro cómo ocupar las operaciones enseñadas anteriormente.

  Las soluciones a los problemas
  están en los archivos anexos al material de clases.

  \diapo{transponer}

  La transposición consiste en cambiar filas por columnas y viceversa.
  El efecto de la operación es comparable a usar la diagonal como un espejo.

  La transposición se obtiene usando el método \li!transpose!,
  y entrega una vista al arreglo existente.

  Por conveniencia,
  NumPy permite obtener la transpuesta mediante la sintaxis \li!a.T!,
  pero es preferible no enseñar esto
  para mantener la consistencia en las operaciones,
  y no obligar a los alumnos a estar pendientes
  de cuáles operaciones son métodos y cuáles no.

  \diapo{diagonales}

  La función \li!diag! permite obtener la diagonal de un arreglo bidimensional.
  Además de la diagonal principal,
  es posible obtener otras diagonales
  pasando un segundo parámetro a la función.

  No se muestra en el ejemplo,
  pero la misma función \li!diag!
  permite hacer la operación inversa:
  al recibir un arreglo de una dimensión,
  entrega un arreglo bidimensional
  con sus elementos a lo largo de la diagonal.

  \diapo{cuadrado-magico}

  Lea el enunciado del ejercicio en voz alta.
  En el cuadrado mágico del ejemplo,
  todas las filas, columnas y diagonales suman 34.
  Para ilustrar la propiedad de los cuadrados mágicos,
  haga algunas de las sumas para mostrar que son iguales.

  Discuta con los estudiantes
  las estrategias para resolver el problema,
  y procure que ellos vayan proponiendo
  las operaciones concretas a realizar.

  La parte comparativamente más difícil
  es obtener los elementos de la antidiagonal,
  ya que no hay ninguna función para obtenerlos directamente.
  Una posible idea (errónea) es obtener la diagonal
  de la matriz transpuesta, pero esto no sirve,
  ya que la transposición no altera la diagonal de la matriz.

  El problema resuelto de varias maneras distintas
  está en los archivos anexos.

\end{document}

