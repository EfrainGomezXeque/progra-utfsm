\documentclass[10pt]{article}
\usepackage{beamerarticle}
\usepackage[spanish]{babel}
\usepackage[utf8]{inputenc}
\usepackage{fullpage}
\usepackage{xcolor}
\usepackage{listings}
\usepackage{textcomp}
\usepackage{mathpazo}
\usepackage{courier}
\usepackage{fancyvrb}
\usepackage{amsmath}
\usepackage{url}
\usepackage{hyperref}
\usepackage{pgfpages}
\usepackage{wrapfig}
\hyphenation{va-ria-bles}

\setjobnamebeamerversion{19-diapos}

\newcommand{\onelinerule}{\rule[2.3ex]{0pt}{0pt}}
\newcommand{\twolinerule}{\rule[6.2ex]{0pt}{0pt}}
\newcommand{\respuesta}{\framebox[\textwidth]{\twolinerule}}
\newcommand{\nombre}{%
  \begin{tikzpicture}[xscale=.4,yscale=.7]
    \draw (0, 0) rectangle (22, 1);
  \end{tikzpicture}%
}
%\newcommand{\rol}   {\framebox[0.3\textwidth]{\onelinerule}}
\newcommand{\rol}{%
  \begin{tikzpicture}[xscale=.4,yscale=.7]
    \draw[gray!40] ( 0, 0) grid      ( 9, 1);
    \draw          ( 0, 0) rectangle ( 9, 1);
    \draw          (10, 0) rectangle (11, 1);
    \draw (9 + .2, .5) -- (10 - .2, .5);
  \end{tikzpicture}%
}
\newcommand{\li}{\lstinline}
\providecommand{\pond}[1]{[{\small\textbf{#1\%}}]}

\lstdefinelanguage{py}{%
  classoffset=0,%
    morekeywords={%
      False,class,finally,is,return,None,continue,for,lambda,try,%
      True,def,from,nonlocal,while,and,del,global,not,with,print,%
      as,elif,if,or,yield,assert,else,import,pass,break,except,in,raise},%
    keywordstyle=\color{black!80}\bfseries,%
  classoffset=1,
    morekeywords={int,float,str,abs,len,raw_input,exit,range,min,max,%
      set,dict,tuple,list,bool,complex,round,sum,all,any,zip,map,filter,%
      sorted,reversed,dir,file,frozenset,open,%
      array,zeros,ones,arange,linspace,eye,diag,dot},
    keywordstyle=\color{black!50}\bfseries,%
  classoffset=0,%
  sensitive=true,%
  morecomment=[l]\#,%
  morestring=[b]',%
  morestring=[b]",%
  stringstyle=\em,%
}

\lstdefinelanguage{testcase}{%
  moredelim=[is][\bfseries]{`}{`},%
  backgroundcolor=\color{gray!20},%
}

\lstdefinelanguage{file}{%
  frame=single,%
}

\lstset{language=py}
\lstset{basicstyle=\ttfamily}
\lstset{columns=fixed}
\lstset{upquote=true}
\lstset{showstringspaces=false}
\lstset{rangeprefix=\#\ }
\lstset{includerangemarker=false}

\newlist{certamen}{enumerate}{1}
\setlist[certamen]{%
  label=\arabic*.,
  font=\LARGE\bfseries,%
  labelindent=-.5in,%
  leftmargin=0pt,%
  labelsep=1em%
}


\newcommand{\diapo}[1]{
  \begin{wrapfigure}{r}{.50\textwidth}
    \hfill\fbox{\includeslide[width=.45\textwidth]{#1}}
  \end{wrapfigure}
}


\title{Archivos de texto}
\author{Programación \\ \url{http://progra.usm.cl}}
\date{Clase 19}

\begin{document}
  \maketitle

  \section*{Objetivos de la clase}
  \begin{itemize}
    \item Explicar las diferencias fundamentales
      entre almacenamientos volátil (RAM) y persistente (disco duro).
    \item Enseñar a crear y leer archivos de texto en Python.
    \item Introducir los archivos de texto con separadores (CSV)
      para almacenar datos tabulares.
  \end{itemize}

  \section*{Diapositivas}

  Comience la clase explicando la diferencia entre el almacenamiento volátil
  (memoria RAM) y el almacenamiento persistente (disco duro, entre otros).

  Simplísticamente, la memoria RAM es donde están los valores de todas las variables
  que están siendo usadas por el programa.
  Todos estos datos se pierden cuando el programa se termina,
  o cuando el computador se apaga.

  La motivación para usar almacenamiento persistente
  es poder recuperar datos provistos por un programa
  después de que éste ha terminado.
  Los datos se almacenan en archivos.

  En este ramo,
  sólo pasaremos los archivos de texto,
  que son muy fáciles de usar en Python.
  Los archivos binarios deben ser leídos con ayuda del módulo \li!struct!,
  pero esto no lo enseñaremos como parte de la materia.

  \diapo{archivo-texto}

  Esta diapositiva muestra un archivo de texto de ejemplo.
  Explique que este archivo puede ser creado
  usando un editor de texto como el Bloc de Notas
  o el mismo IDLE.

  La convención usual es que los archivos de texto tienen extensión \li!.txt!,
  pero esto no es mandatorio.
  De todos modos, conviene que en los programas que hagamos
  siempre sigamos esta convención
  para evitar confundir a los alumnos.

  Si bien solemos visualizar el archivo como varias líneas,
  en verdad un archivo es una única secuencia de caracteres,
  entre los que puede haber saltos de línea (\li!'\n'!).

  Debe quedar claro que al final de cada línea del archivo del ejemplo
  hay un salto de línea, aunque no se vea al abrirlo en un editor.

  \diapo{leer-archivo-texto}

  Para leer datos de un archivo desde un programa,
  primero hay que abrirlo usando la función \li!open!,
  que recibe como parámetro el nombre del archivo.

  Si el archivo está en el mismo directorio que el programa,
  basta con poner el nombre del archivo como un string.
  Si estuviera en otro lado, hay que poner la ruta completa
  (puede ser absoluta o relativa).
  Para simplificar, pongamos siempre los archivos en la misma carpeta.

  Siempre que se abre un archivo,
  hay que cerrarlo una vez que se ha terminado de usarlo.
  Para esto, se usa el método \li!close!.

  Hay varias maneras de leer datos de un archivo ya abierto.
  La más simple es hacerlo línea por línea.
  En Python, los archivos son objetos iterables,
  por lo que pueden ser usados en un ciclo \li!for!.
  En cada iteración,
  se obtiene una nueva línea del archivo.
  La línea leída siempre contiene el caracter de salto de línea al final,
  que puede ser eliminado usando el método \li!strip!.

  El programa del ejemplo
  lee línea por línea el contenido del archivo,
  la convierte a mayúsculas y luego la imprime.
  La salida del programa es la que se muestra en la diapositiva.

  Si no se hubiera usado \li!strip!,
  el salto de línea hubiera permanecido,
  y habría sido impreso además del salto de línea que agrega siempre
  la sentencia \li!print!.
  En este caso, en la salida habrían aparecido las líneas del archivo
  separadas por líneas en blanco.

  Es importante explicar que los archivos se leen secuencialmente,
  y que una vez que un dato ha sido leído,
  ya no puede volver a serlo.
  (Técnicamente sí se puede, rebobinando el archivo con el método \li!seek!,
  pero omitiremos este detalle por ahora).
  Un error típico puede ser algo como lo siguiente:
\begin{lstlisting}
for linea in archivo:
    print linea
for linea in archivo:
    print linea
\end{lstlisting}
   En este caso, el segundo \li!for! hará cero iteraciones,
   ya que el iterable \li!archivo! no generará más datos.

   Para leer una única línea desde el archivo,
   se puede usar el método \li!readline!.
   El método \li!read! sin parámetro lee todo el contenido del archivo
   desde la posición actual.
   También puede recibir un parámetro entero,
   que indica cuántos bytes hay que leer.
   Estas formas de lectura no las enseñaremos en la asignatura.

  \diapo{crear-archivo-texto}

  Para crear un archivo de texto y abrirlo para escritura,
  hay que pasar como segundo parámetro a la función \li!open!
  el string \li!'w'!.

  Explique el efecto usual de abrir un archivo para escritura:
  \begin{itemize}
    \item si el archivo no existe, es creado;
    \item si el archivo ya existe, es sobreescrito.
  \end{itemize}

  Para escribir texto en el archivo,
  se usa el método \li!write!.
  Los saltos de línea deben ser escritos explícitamente
  (a diferencia de la sentencia \li!print!).

  El método \li!write! sólo acepta strings como parámetros.
  La llamada \li!archivo.write(10)! arrojará un error de tipo.

  Al terminar de escribir,
  hay que cerrar el archivo.

  \diapo{agregar-a-archivo-texto}

  Además del modo \li!'w'!, también existe el modo \li!'a'! (\emph{append})
  que permite abrir un archivo existente
  para escribir datos al final de él.

  El ejemplo de la diapositiva abre el archivo creado en la diapositiva anterior,
  y agrega un par de líneas adicionales al final de él.

  \diapo{ejercicio-archivos-texto}

  Lea el enunciado en voz alta y explíquelo.

  Enfatice los pasos típicos de los ejercicios con archivos:
  se abre el archivo, se lee línea por línea usando un \li!for!
  y al final se cierra.

  Para resolver este problema,
  hay que usar tres contadores.
  Para evitar contar los espacios,
  puede eliminarlos usando el método \li!replace!.

  \diapo{archivo-csv}

  Los archivos con separadores (o archivos CSV)
  son archivos de texto comúnes y corrientes
  en los que se guardan datos tabulares.
  Cada fila de la tabla es una línea del archivo,
  y los valores de cada fila se ponen separados por algún caracter a elección.

  En el ejemplo de la diapositiva,
  el archivo tiene los datos de una tabla de alumnos
  en los que cada fila tiene nombre, apellido y tres notas.

  \diapo{leer-archivo-csv}

  La lectura de archivos CSV se hace igual que para los archivos de texto ordinarios,
  sólo que ahora interesa separar los datos usando el método \li!split!.

  Al ocupar \li!split!, se obtiene una lista de valores.
  En general, suele ser útil asignar estos valores a variables individuales
  para poder convertirlas al tipo apropiado por separado.

  Cuando hay muchos campos,
  el operador de rebanado de las listas (\li!valores[inicio:final]!)
  y la función \li!map!
  pueden ayudar a hacer la asignación más brevemente.

  Las precauciones usuales siguen vigentes:
  hay que eliminar el salto de línea al leer cada línea,
  y cerrar el archivo al final.

  \diapo{crear-archivo-csv}

  Para crear un archivo CSV,
  se hace igual que para los archivos de texto ordinarios.
  La parte que es propia de los archivos CSV
  es crear el string con la línea a escribir.
  Para esto, puede ser útil el método \li!join! de los strings,
  pero en general no es necesario,
  ya que se puede construir la línea manualmente,
  o incluso escribir los valores y los separadores
  directamente en el archivo.

  \diapo{ejercicio-archivos-csv}

  Lea el enunciado en voz alta y explique el ejercicio.

  En la solución,
  hay que tener tres archivos abiertos (uno para lectura y dos para escritura).

\end{document}

