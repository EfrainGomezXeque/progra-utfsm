\documentclass[12pt]{beamer}
\usepackage[spanish]{babel}
\usepackage[utf8]{inputenc}
\usepackage{xcolor}
\usepackage{listings}
\usepackage{textcomp}
\usepackage{mathpazo}
\usepackage{courier}
\usepackage{fancyvrb}
\usepackage{amsmath}
\usepackage{url}
\usepackage{hyperref}
\usepackage{enumitem}

\newcommand{\onelinerule}{\rule[2.3ex]{0pt}{0pt}}
\newcommand{\twolinerule}{\rule[6.2ex]{0pt}{0pt}}
\newcommand{\respuesta}{\framebox[\textwidth]{\twolinerule}}
\newcommand{\nombre}{%
  \begin{tikzpicture}[xscale=.4,yscale=.7]
    \draw (0, 0) rectangle (22, 1);
  \end{tikzpicture}%
}
%\newcommand{\rol}   {\framebox[0.3\textwidth]{\onelinerule}}
\newcommand{\rol}{%
  \begin{tikzpicture}[xscale=.4,yscale=.7]
    \draw[gray!40] ( 0, 0) grid      ( 9, 1);
    \draw          ( 0, 0) rectangle ( 9, 1);
    \draw          (10, 0) rectangle (11, 1);
    \draw (9 + .2, .5) -- (10 - .2, .5);
  \end{tikzpicture}%
}
\newcommand{\li}{\lstinline}
\providecommand{\pond}[1]{[{\small\textbf{#1\%}}]}

\lstdefinelanguage{py}{%
  classoffset=0,%
    morekeywords={%
      False,class,finally,is,return,None,continue,for,lambda,try,%
      True,def,from,nonlocal,while,and,del,global,not,with,print,%
      as,elif,if,or,yield,assert,else,import,pass,break,except,in,raise},%
    keywordstyle=\color{black!80}\bfseries,%
  classoffset=1,
    morekeywords={int,float,str,abs,len,raw_input,exit,range,min,max,%
      set,dict,tuple,list,bool,complex,round,sum,all,any,zip,map,filter,%
      sorted,reversed,dir,file,frozenset,open,%
      array,zeros,ones,arange,linspace,eye,diag,dot},
    keywordstyle=\color{black!50}\bfseries,%
  classoffset=0,%
  sensitive=true,%
  morecomment=[l]\#,%
  morestring=[b]',%
  morestring=[b]",%
  stringstyle=\em,%
}

\lstdefinelanguage{testcase}{%
  moredelim=[is][\bfseries]{`}{`},%
  backgroundcolor=\color{gray!20},%
}

\lstdefinelanguage{file}{%
  frame=single,%
}

\lstset{language=py}
\lstset{basicstyle=\ttfamily}
\lstset{columns=fixed}
\lstset{upquote=true}
\lstset{showstringspaces=false}
\lstset{rangeprefix=\#\ }
\lstset{includerangemarker=false}

\newlist{certamen}{enumerate}{1}
\setlist[certamen]{%
  label=\arabic*.,
  font=\LARGE\bfseries,%
  labelindent=-.5in,%
  leftmargin=0pt,%
  labelsep=1em%
}



\usecolortheme{seahorse}
\usefonttheme{serif}

\title{Módulos, funciones avanzadas}
\author{
  Programación \\ \url{http://progra.usm.cl}
}
\date{}

\begin{document}
  \begin{frame}
    \maketitle
  \end{frame}

  \begin{frame}
    \frametitle{Uso de módulos}
    \label{uso-modulos}
    Dos maneras de importar objetos:
    \vfill
    \footnotesize
    \lstinputlisting[frame=single]{programas/modulo-from-import.py}
    \lstinputlisting[frame=single]{programas/modulo-import.py}
  \end{frame}

  \begin{frame}
    \frametitle{Creación de módulos}
    \label{creacion-modulos}
    programa.py
    {\footnotesize
    \lstinputlisting[frame=single]{programas/programa-circulos.py}}
    \vfill
    circulos.py
    {\footnotesize
    \lstinputlisting[frame=single]{programas/circulos.py}}
  \end{frame}

  \begin{frame}
    \frametitle{Programas como módulos}
    \label{programa-modulo}
    comb.py
    \footnotesize
    \lstinputlisting[frame=single]{programas/comb.py}
  \end{frame}

  \begin{frame}
    \frametitle{Funciones con múltiples valores de retorno}
    \label{mult-retornos}
    \lstinputlisting[frame=single]{programas/conversion-segundos.py}
    \lstinputlisting{programas/caso-conversion-segundos.py}
  \end{frame}

  \begin{frame}
    \frametitle{Funciones sin valor de retorno}
    \label{sin-retorno}
    \footnotesize
    \lstinputlisting[frame=single]{programas/funcion-imprimir.py}
  \end{frame}

  \begin{frame}
    \frametitle{Parámetros con valores por omisión}
    \label{parametros-por-omision}
    \lstinputlisting[frame=single]{programas/parametros-omision.py}
    \lstinputlisting{programas/caso-parametros-omision.py}
  \end{frame}

  \begin{frame}
    \frametitle{Funciones como parámetros}
    \label{func-parametros}
    \footnotesize
    \lstinputlisting[frame=single]{programas/func-parametros.py}
  \end{frame}

  \begin{frame}
    \frametitle{Funciones recursivas}
    \label{func-recursivas}
    \begin{align*}
        n! &= 1\cdot 2\cdot 3\cdot\;\cdots\;\cdot (n - 2)\cdot (n - 1)\cdot n \\
           &=  (n - 1)! \cdot n
    \end{align*}
    \vfill
    \lstinputlisting[frame=single]{programas/factorial-recursivo.py}
  \end{frame}

\end{document}

