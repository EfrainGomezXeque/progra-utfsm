\documentclass[12pt]{beamer}
\usepackage[spanish]{babel}
\usepackage[utf8]{inputenc}
\usepackage{xcolor}
\usepackage{listings}
\usepackage{textcomp}
\usepackage{mathpazo}
\usepackage{courier}
\usepackage{fancyvrb}
\usepackage{amsmath}
\usepackage{url}
\usepackage{hyperref}

\newcommand{\onelinerule}{\rule[2.3ex]{0pt}{0pt}}
\newcommand{\twolinerule}{\rule[6.2ex]{0pt}{0pt}}
\newcommand{\respuesta}{\framebox[\textwidth]{\twolinerule}}
\newcommand{\nombre}{%
  \begin{tikzpicture}[xscale=.4,yscale=.7]
    \draw (0, 0) rectangle (22, 1);
  \end{tikzpicture}%
}
%\newcommand{\rol}   {\framebox[0.3\textwidth]{\onelinerule}}
\newcommand{\rol}{%
  \begin{tikzpicture}[xscale=.4,yscale=.7]
    \draw[gray!40] ( 0, 0) grid      ( 9, 1);
    \draw          ( 0, 0) rectangle ( 9, 1);
    \draw          (10, 0) rectangle (11, 1);
    \draw (9 + .2, .5) -- (10 - .2, .5);
  \end{tikzpicture}%
}
\newcommand{\li}{\lstinline}
\providecommand{\pond}[1]{[{\small\textbf{#1\%}}]}

\lstdefinelanguage{py}{%
  classoffset=0,%
    morekeywords={%
      False,class,finally,is,return,None,continue,for,lambda,try,%
      True,def,from,nonlocal,while,and,del,global,not,with,print,%
      as,elif,if,or,yield,assert,else,import,pass,break,except,in,raise},%
    keywordstyle=\color{black!80}\bfseries,%
  classoffset=1,
    morekeywords={int,float,str,abs,len,raw_input,exit,range,min,max,%
      set,dict,tuple,list,bool,complex,round,sum,all,any,zip,map,filter,%
      sorted,reversed,dir,file,frozenset,open,%
      array,zeros,ones,arange,linspace,eye,diag,dot},
    keywordstyle=\color{black!50}\bfseries,%
  classoffset=0,%
  sensitive=true,%
  morecomment=[l]\#,%
  morestring=[b]',%
  morestring=[b]",%
  stringstyle=\em,%
}

\lstdefinelanguage{testcase}{%
  moredelim=[is][\bfseries]{`}{`},%
  backgroundcolor=\color{gray!20},%
}

\lstdefinelanguage{file}{%
  frame=single,%
}

\lstset{language=py}
\lstset{basicstyle=\ttfamily}
\lstset{columns=fixed}
\lstset{upquote=true}
\lstset{showstringspaces=false}
\lstset{rangeprefix=\#\ }
\lstset{includerangemarker=false}

\newlist{certamen}{enumerate}{1}
\setlist[certamen]{%
  label=\arabic*.,
  font=\LARGE\bfseries,%
  labelindent=-.5in,%
  leftmargin=0pt,%
  labelsep=1em%
}



\usecolortheme{crane}
\usefonttheme{serif}

\title{Patrones comunes}
\author{
  Programación \\ \url{http://progra.usm.cl}
}
\date{23 y 24 de marzo de 2011}

\begin{document}
  \begin{frame}
    \maketitle
  \end{frame}

  \begin{frame}
    \frametitle{Sumar cosas}
    Escriba un programa que entregue la suma
    de los cuadrados de los números naturales
    menores que mil:
    \[
      1^2 + 2^2 + 3^2 + \cdots + 999^2 + 1000^2.
    \]
  \end{frame}

  \begin{frame}
    \frametitle{Solución:}
    \lstinputlisting{programas/sumar-cuadrados.py}
  \end{frame}

  \begin{frame}
    \frametitle{Sumar cosas}
    Escriba un programa que entregue la suma
    de los cuadrados de los números ingresados por el usuario.
    El programa debe terminar
    cuando se ingrese un cero:
    \lstinputlisting[language=testcase]{programas/caso-sumar-cuadrados-ingresados.txt}
  \end{frame}

  \begin{frame}
    \frametitle{Solución:}
    \lstinputlisting{programas/sumar-cuadrados-ingresados.py}
  \end{frame}

  \begin{frame}
    \frametitle{Patrón \emph{sumar cosas}:}
    \lstinputlisting[
      classoffset=0,morekeywords={ciclo},
      classoffset=1,morekeywords={calcular},
      basicstyle=\ttfamily\LARGE,
    ]{programas/patron-sumar.py}
  \end{frame}

  \begin{frame}
    \frametitle{Patrón \emph{multiplicar cosas}:}
    \lstinputlisting[
      classoffset=0,morekeywords={ciclo},
      classoffset=1,morekeywords={calcular},
      basicstyle=\ttfamily\LARGE,
    ]{programas/patron-multiplicar.py}
  \end{frame}

  \begin{frame}
    \frametitle{Patrón \emph{contar cosas}:}
    \lstinputlisting[
      classoffset=0,morekeywords={ciclo},
      classoffset=1,morekeywords={calcular},
      basicstyle=\ttfamily\LARGE,
    ]{programas/patron-contar.py}
  \end{frame}

  \begin{frame}
    \frametitle{Encontrar el máximo de enteros positivos}
    Escriba un programa que pida al usuario
    ingresar diez enteros positivos,
    e indique cuál es el mayor de ellos:
    \lstinputlisting[language=testcase]{programas/caso-mayor.txt}
  \end{frame}

  \begin{frame}
    \frametitle{Solución 1:}
    \lstinputlisting{programas/mayor-1.py}
  \end{frame}

  \begin{frame}
    \frametitle{Solución 2:}
    \lstinputlisting{programas/mayor-2.py}
  \end{frame}

  \begin{frame}
    \frametitle{Encontrar el máximo de números reales}
    Escriba un programa que pida al usuario
    ingresar varios números reales (positivos o negativos)
    hasta que ingrese un cero,
    y que indique cuál es el mayor de ellos:
    \lstinputlisting[language=testcase]{programas/caso-mayor-reales.txt}
  \end{frame}

  \begin{frame}
    \frametitle{Solución:}
    \lstinputlisting{programas/mayor-reales.py}
  \end{frame}

  \begin{frame}
    \frametitle{Patrón \emph{encontrar el mayor}:}
    \lstinputlisting[
      classoffset=0,morekeywords={ciclo},
      classoffset=1,morekeywords={calcular},
      basicstyle=\ttfamily\LARGE,
    ]{programas/patron-mayor.py}
  \end{frame}

  \begin{frame}
    \frametitle{Patrón \emph{encontrar el menor}:}
    \lstinputlisting[
      classoffset=0,morekeywords={ciclo},
      classoffset=1,morekeywords={calcular},
      basicstyle=\ttfamily\LARGE,
    ]{programas/patron-menor.py}
  \end{frame}

  \begin{frame}
    \frametitle{Combinaciones de dados}
    Escriba un programa que muestre
    todas las combinaciones de dos dados
    que entreguen un puntaje mayor que siete.
  \end{frame}

  \begin{frame}
    \frametitle{Solución:}
    \lstinputlisting{programas/dados-7.py}
  \end{frame}

  \begin{frame}
    \frametitle{Patrón \emph{pares de cosas}:}
    \lstinputlisting[
      classoffset=0,morekeywords={ciclo},
      classoffset=1,morekeywords={calcular},
      basicstyle=\ttfamily\LARGE,
    ]{programas/patron-pares.py}
  \end{frame}

\end{document}

