\documentclass[12pt]{beamer}
\usepackage[spanish]{babel}
\usepackage[utf8]{inputenc}
\usepackage{xcolor}
\usepackage{listings}
\usepackage{textcomp}
\usepackage{mathpazo}
\usepackage{courier}
\usepackage{fancyvrb}
\usepackage{amsmath}
\usepackage{url}
\usepackage{hyperref}

\usepackage{tikz}
\usetikzlibrary{arrows,calc,shapes,chains}
\definecolor{lightbox}{HTML}{FFEEC1}
\definecolor{questionmark}{HTML}{FCBC05}

\newcommand{\onelinerule}{\rule[2.3ex]{0pt}{0pt}}
\newcommand{\twolinerule}{\rule[6.2ex]{0pt}{0pt}}
\newcommand{\respuesta}{\framebox[\textwidth]{\twolinerule}}
\newcommand{\nombre}{%
  \begin{tikzpicture}[xscale=.4,yscale=.7]
    \draw (0, 0) rectangle (22, 1);
  \end{tikzpicture}%
}
%\newcommand{\rol}   {\framebox[0.3\textwidth]{\onelinerule}}
\newcommand{\rol}{%
  \begin{tikzpicture}[xscale=.4,yscale=.7]
    \draw[gray!40] ( 0, 0) grid      ( 9, 1);
    \draw          ( 0, 0) rectangle ( 9, 1);
    \draw          (10, 0) rectangle (11, 1);
    \draw (9 + .2, .5) -- (10 - .2, .5);
  \end{tikzpicture}%
}
\newcommand{\li}{\lstinline}
\providecommand{\pond}[1]{[{\small\textbf{#1\%}}]}

\lstdefinelanguage{py}{%
  classoffset=0,%
    morekeywords={%
      False,class,finally,is,return,None,continue,for,lambda,try,%
      True,def,from,nonlocal,while,and,del,global,not,with,print,%
      as,elif,if,or,yield,assert,else,import,pass,break,except,in,raise},%
    keywordstyle=\color{black!80}\bfseries,%
  classoffset=1,
    morekeywords={int,float,str,abs,len,raw_input,exit,range,min,max,%
      set,dict,tuple,list,bool,complex,round,sum,all,any,zip,map,filter,%
      sorted,reversed,dir,file,frozenset,open,%
      array,zeros,ones,arange,linspace,eye,diag,dot},
    keywordstyle=\color{black!50}\bfseries,%
  classoffset=0,%
  sensitive=true,%
  morecomment=[l]\#,%
  morestring=[b]',%
  morestring=[b]",%
  stringstyle=\em,%
}

\lstdefinelanguage{testcase}{%
  moredelim=[is][\bfseries]{`}{`},%
  backgroundcolor=\color{gray!20},%
}

\lstdefinelanguage{file}{%
  frame=single,%
}

\lstset{language=py}
\lstset{basicstyle=\ttfamily}
\lstset{columns=fixed}
\lstset{upquote=true}
\lstset{showstringspaces=false}
\lstset{rangeprefix=\#\ }
\lstset{includerangemarker=false}

\newlist{certamen}{enumerate}{1}
\setlist[certamen]{%
  label=\arabic*.,
  font=\LARGE\bfseries,%
  labelindent=-.5in,%
  leftmargin=0pt,%
  labelsep=1em%
}


\tikzstyle{decision} = [
  diamond,
  very thick,
  draw=red!50!black!50,
  %fill=red!20, 
  aspect=2,
  %text badly centered,
  top color=white,
  bottom color=red!50!black!20,
]
\tikzstyle{stmt} = [
  rectangle,
  very thick,
  draw=blue!50!black!50,
  %fill=blue!20, 
  text centered,
  minimum height=5ex,
  minimum width=5em,
  top color=white,
  bottom color=blue!50!black!20,
]
\tikzstyle{node} = [
  circle,
  very thick,
  draw=orange!50!black!50,
  fill=orange!20,
  minimum size=6ex,
]
\tikzstyle{io} = [
  very thick,
  draw=green!50!black!50,
  trapezium,
  trapezium left angle=80,
  trapezium right angle=-80,
  %fill=green!20!black!10,
  %rounded corners,
  %minimum height=5ex,
  top color=white,
  bottom color=green!50!black!20,
  text centered,
  minimum height=5ex,
  minimum width=5em,
]
\tikzstyle{conn} = [very thick, draw=black!50, -latex']



\usecolortheme{crane}
\usefonttheme{serif}

\title{Introducción a la programación}
\author{Programación \\ \url{http://progra.usm.cl}}
\date{}

\begin{document}
  \begin{frame}
    \maketitle
  \end{frame}

  \begin{frame}
    \frametitle{Evaluaciones}
    \label{evaluaciones}
    \begin{itemize}
      \item Tres \textbf{certamenes} (20\%, 25\%, 25\%)
      \item \textbf{Certamen recuperativo}
        (reemplaza al peor certamen)
      \vfill
      \item \textbf{Controles presenciales} (15\%)
        \begin{itemize}
          \item en la sala de clases,
          \item duran 30 minutos.
        \end{itemize}
      \vfill
      \item \textbf{Controles en línea} (15\%)
        \begin{itemize}
          \item consiste en hacer programas reales,
          \item se puede hacer desde la casa,
          \item plazo de 1 hora 20 minutos para resolverlo.
        \end{itemize}
    \end{itemize}
  \end{frame}

  \begin{frame}
    \frametitle{Página web del ramo}
    \label{web}
    \begin{itemize}
      \item \url{http://progra.usm.cl}
        \begin{itemize}
          \item información del ramo,
          \item materia,
          \item ejercicios,
          \item controles en línea.
        \end{itemize}
      \vfill
      \item \url{http://twitter.com/progra_usm}
      \item \url{http://facebook.com/}\(\;\rightarrow\;\) Programación USM
        \begin{itemize}
          \item noticias,
          \item anuncios.
        \end{itemize}
    \end{itemize}
  \end{frame}

  \tikzstyle{i} = [fill=lightbox, text width=10em]
  \tikzstyle{b} = [draw, circle, right of=entrada, fill=questionmark]
  \tikzstyle{o} = [fill=lightbox, text width=10em, right of=caja]
  \tikzstyle{a} = [>=stealth, thick, ->]
  \tikzstyle{p} = [node distance=8em]

  \newcommand{\problema}[2]{
    \begin{tikzpicture}[p]
      \node[i] (entrada) {#1};
      \node[b] (caja)    {¿?};
      \node[o] (salida)  {#2};
      \draw[a] (entrada) -- (caja);
      \draw[a] (caja)    -- (salida);
    \end{tikzpicture}
  }

  \begin{frame}
  \end{frame}

  \begin{frame}
    \frametitle{Programación}
    \label{conceptos}
    \begin{block}{Problema}
      Entrada \(\longrightarrow\) Salida
    \end{block}

    \begin{block}{Algoritmo}
      Secuencia de pasos para resolver un problema
    \end{block}

    \begin{block}{Programa}
      Secuencia de instrucciones
      descritas en un lenguaje
      que puede ser entendido por el computador
    \end{block}
  \end{frame}

  \begin{frame}
    \frametitle{Ejemplos de problemas}
    \label{problema-ceros-recta}
    \problema{Una función lineal \(y = ax + b\)}%
             {Los ceros \\ de la función}
  \end{frame}

  \begin{frame}
    \frametitle{Ejemplos de problemas}
    \label{problema-ceros-funcion}
    \problema{Una función real cualquiera \(f(x)\)}%
             {Los ceros \\ de la función}
  \end{frame}

  \begin{frame}
    \frametitle{Ejemplos de problemas}
    \label{problema-ordenamiento}
    \problema{Un conjunto de números}%
             {Los números \\ ordenados \\ de menor a mayor}
  \end{frame}

  \begin{frame}
    \frametitle{Ejemplos de problemas}
    \label{problema-ciudades}
    \problema{Un conjunto \\ de ciudades}%
             {El camino más \\ corto que recorre \\ las ciudades}
  \end{frame}

  \begin{frame}
    \frametitle{Ejemplos de problemas}
    \label{problema-spam}
    \problema{Un mensaje de email}%
             {La probabilidad de que sea spam}
  \end{frame}

  \begin{frame}
    \frametitle{Ejemplos de problemas}
    \label{problema-tsunami}
    \problema{Mediciones de sismógrafos}%
             {``Sí habra tsunami'' \\ o ``no habrá tsunami''}
  \end{frame}

  \begin{frame}
    \frametitle{Ejercicio}
    \label{ejercicio}
    \textbf{Ejercicio}: diseñe un algoritmo para determinar si un número natural \(n\)
    es primo o compuesto.
    \vfill
    \problema{Un número natural \\ \(n\)}%
             {``\(n\) es primo'' o \\ ``\(n\) es compuesto''}
  \end{frame}

  \begin{frame}
    \frametitle{Solución en lenguaje natural}
    \label{sol-natural}
    Buscar algún valor \(d\)
    que esté entre 2 y \(n - 1\)
    que sea divisor de \(n\).

    Si existe por lo menos uno de estos valores,
    entonces \(n\) es compuesto;
    o si no, es primo.
  \end{frame}

  \begin{frame}
    \frametitle{Solución en diagrama de flujo}
    \label{sol-diagrama}
    \includegraphics[scale=.7]{../diagramas/primos.pdf}
  \end{frame}

  \begin{frame}
    \frametitle{Solución en pseudocódigo}
    \label{sol-pseudocodigo}
    \begin{itemize}
      \item \textbf{leer} \(n\)
      \item es\_primo = verdadero
      \item \textbf{para} \(d\) \textbf{de} \(2\) \textbf{a} \(n - 1\):
      \begin{itemize}
        \item \textbf{si} \(n\) es divisible por \(d\):
        \begin{itemize}
          \item es\_primo = falso
        \end{itemize}
      \end{itemize}
      \item \textbf{si} es\_primo es verdadero:
      \begin{itemize}
        \item \textbf{escribir} ``n es primo''
      \end{itemize}
      \item \textbf{o si no}:
      \begin{itemize}
        \item \textbf{escribir} ``n es compuesto''
      \end{itemize}
    \end{itemize}
  \end{frame}

  \begin{frame}
    \frametitle{Solución en Python}
    \label{sol-python}
    \lstinputlisting[language=py]{../_static/programas/primo.py}
  \end{frame}

  \begin{frame}
  \end{frame}
\end{document}
