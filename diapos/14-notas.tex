\documentclass[10pt]{article}
\usepackage{beamerarticle}
\usepackage[spanish]{babel}
\usepackage[utf8]{inputenc}
\usepackage{fullpage}
\usepackage{xcolor}
\usepackage{listings}
\usepackage{textcomp}
\usepackage{mathpazo}
\usepackage{courier}
\usepackage{fancyvrb}
\usepackage{amsmath}
\usepackage{url}
\usepackage{hyperref}
\usepackage{pgfpages}
\usepackage{wrapfig}

\setjobnamebeamerversion{14-diapos}

\newcommand{\onelinerule}{\rule[2.3ex]{0pt}{0pt}}
\newcommand{\twolinerule}{\rule[6.2ex]{0pt}{0pt}}
\newcommand{\respuesta}{\framebox[\textwidth]{\twolinerule}}
\newcommand{\nombre}{%
  \begin{tikzpicture}[xscale=.4,yscale=.7]
    \draw (0, 0) rectangle (22, 1);
  \end{tikzpicture}%
}
%\newcommand{\rol}   {\framebox[0.3\textwidth]{\onelinerule}}
\newcommand{\rol}{%
  \begin{tikzpicture}[xscale=.4,yscale=.7]
    \draw[gray!40] ( 0, 0) grid      ( 9, 1);
    \draw          ( 0, 0) rectangle ( 9, 1);
    \draw          (10, 0) rectangle (11, 1);
    \draw (9 + .2, .5) -- (10 - .2, .5);
  \end{tikzpicture}%
}
\newcommand{\li}{\lstinline}
\providecommand{\pond}[1]{[{\small\textbf{#1\%}}]}

\lstdefinelanguage{py}{%
  classoffset=0,%
    morekeywords={%
      False,class,finally,is,return,None,continue,for,lambda,try,%
      True,def,from,nonlocal,while,and,del,global,not,with,print,%
      as,elif,if,or,yield,assert,else,import,pass,break,except,in,raise},%
    keywordstyle=\color{black!80}\bfseries,%
  classoffset=1,
    morekeywords={int,float,str,abs,len,raw_input,exit,range,min,max,%
      set,dict,tuple,list,bool,complex,round,sum,all,any,zip,map,filter,%
      sorted,reversed,dir,file,frozenset,open,%
      array,zeros,ones,arange,linspace,eye,diag,dot},
    keywordstyle=\color{black!50}\bfseries,%
  classoffset=0,%
  sensitive=true,%
  morecomment=[l]\#,%
  morestring=[b]',%
  morestring=[b]",%
  stringstyle=\em,%
}

\lstdefinelanguage{testcase}{%
  moredelim=[is][\bfseries]{`}{`},%
  backgroundcolor=\color{gray!20},%
}

\lstdefinelanguage{file}{%
  frame=single,%
}

\lstset{language=py}
\lstset{basicstyle=\ttfamily}
\lstset{columns=fixed}
\lstset{upquote=true}
\lstset{showstringspaces=false}
\lstset{rangeprefix=\#\ }
\lstset{includerangemarker=false}

\newlist{certamen}{enumerate}{1}
\setlist[certamen]{%
  label=\arabic*.,
  font=\LARGE\bfseries,%
  labelindent=-.5in,%
  leftmargin=0pt,%
  labelsep=1em%
}


\newcommand{\diapo}[1]{
  \begin{wrapfigure}{r}{.50\textwidth}
    \hfill\fbox{\includeslide[width=.45\textwidth]{#1}}
  \end{wrapfigure}
}


\title{Estructuras de datos anidadas}
\author{Programación \\ \url{http://progra.usm.cl}}
\date{}

\begin{document}
  \maketitle

  \section*{Objetivos de la clase}
  \begin{itemize}
    \item Enseñar el uso de estructuras de datos anidadas mediante ejemplos.
    \item Mostrar cómo el uso de estructuras de datos anidadas
      permite representar relaciones relativamente complejas en problemas reales.
  \end{itemize}

  \section*{Diapositivas}

  Esta clase no tiene materia nueva.
  Los conceptos aplicados son los vistos en las clases de las últimas semanas.

  La clase está preparada en torno a tres problemas con varios ejercicios
  que deben ser propuestos a los estudiantes para que al menos intenten resolverlos.
  Lo más probable es que no alcance a hacer todos los ejercicios durante la clase,
  por lo que puede alterar el orden si le parece más razonable
  darle prioridad a alguno de los problemas.

  Las soluciones a los problemas no están en las diapositivas,
  sino en archivos aparte para ser descargados.
  De este modo, es más fácil probar las soluciones en el intérprete.

  \diapo{repaso}

  Comience repasando las estructuras de datos aprendidas en la semana anterior.
  Indique las propiedades de cada una (aparecen en la diapositiva),
  repase cómo crear una instancia de ellas,
  y recuerde algunas de las operaciones y métodos más importantes.
  Idealmente, ayúdese de la consola para hacer algunos ejemplos muy sencillos.

  \diapo{problema-alumnos-1}
  \diapo{problema-alumnos-2}

  En estas diapositivas se presenta las estructuras de datos
  que contienen la información para el problema de las asignaturas.
  Describa detalladamente la estructura de ambas,
  haciendo énfasis en los tipos de datos.

  \diapo{problema-alumnos-3}

  Lea el enunciado en voz alta.
  Pregunte a los alumnos cuál debe ser el nombre de la función,
  cuál debe ser el parámetro y su tipo,
  y cuál es el tipo del valor de retorno.
  Aunque esto aparezca casi explícitamente en el enunciado,
  son muchos los alumnos que tienen problemas identificando esta información.

  Discuta con los alumnos la estrategia para resolver el problema.
  Lo que hay que hacer es recorrer el diccionario \li!inscritos!
  buscando al alumno en cuestión,
  y agregar el ramo a una lista cada vez que sea encontrado
  entre los inscritos a un ramo.
  La solución no es complicada,
  y seguramente algunos alumnos podrán escribir código bien encaminado.

  \diapo{problema-alumnos-4}

  Lea el enunciado en voz alta,
  y nuevamente discuta parámetros y valor de retorno.
  Note que la función no retorna la tupla del alumno,
  sino su nombre completo.

  Como esta función no tiene parámetro,
  entonces uno podría preguntarse de dónde obtiene la información.
  La respuesta es que los datos están en variables globales,
  por lo que cualquier función puede acceder a ellos
  sin necesidad de pasárselos explícitamente.

  El problema de entontrar al más joven
  es equivalente a buscar la mayor fecha de nacimiento
  (es decir, la más reciente).
  La solución debe ser una variante del patrón «buscar mayor».
  Una opción es inicializar el cumpleaños del más joven en una fecha muy antigua
  (en el programa adjunto, se usa la fecha \li!(0, 0, 0)!).
  Otra alternativa es meter todas las fechas de nacimiento a una lista
  y ocupar la función \li!max!.

  \diapo{problema-futbol-1}

  En esta diapositiva se presenta la información
  para el problema del campeonato de fútbol
  en un único diccionario.
  Nuevamente describa en detalle la estructura,
  con énfasis en los tipos de datos contenidos.

  \diapo{problema-futbol-2}

  Lea el enunciado en voz alta.
  Si bien en el diccionario de resultados
  aparece toda la información sobre el campeonato,
  la lista de equipos no aparece explícitamente.

  Para obtener la lista,
  hay que recorrer todos los partidos
  para ver qué equipos fueron los que jugaron.
  Como cada equipo participa en varios partidos,
  habrá equipos repetidos,
  por lo que es conveniente usar un conjunto para irlos guardando.

  Note que ésta es una decisión del programador que desarrolla la función
  que debe ser transparente para quien la usa.
  La función no puede retornar el conjunto,
  ya que el enunciado y el caso de prueba
  indican explícitamente que se debe retornar una lista.
  Enfatice esto, pues en los controles en línea un error muy común entre los alumnos
  fue usar operaciones de strings en algoritmos numéricos
  (por ejemplo, para extraer dígitos),
  pero sin convertir el resultado final de vuelta al tipo \li!int!.
  Es importante que al probar la función en la consola,
  la sesión se vea exactamente igual que el caso de prueba del enunciado.

  \diapo{problema-futbol-3}
  \diapo{problema-futbol-4}

  Los dos ejercicios siguientes son parecidos.
  En ambos hay que recorrer el diccionario,
  y ver si cada partido involucra al equipo en cuestión,
  y si es así, si se trata del primer o del segundo equipo de la tupla.

  Nuevamente,
  sea claro en relación a los tipos de parámetros y retorno.

  \diapo{problema-asistencia-1}

  Lea el enunciado del problema en voz alta,
  explique cómo debe interpretarse la tabla,
  y cómo se traduce en una lista de listas
  (el equivalente a un arreglo bidimensional de Pascal o C).

  A diferencia de los diccionarios,
  en los que la asociación llave-valor es explícita,
  acá hay una asociación implícita dada por los índices del alumno y de la clase.
  Por ejemplo,
  la asistencia de Fulanita es la tercera lista de la tabla \li!asistencia!,
  pues Fulanita es el tercer elemento de la lista de alumnos.

  \diapo{problema-asistencia-2}

  Lea el enunciado en voz alta y discuta los tipos de parámetros y retorno.
  En este caso,
  la función recibe la estructura de datos como parámetro,
  y no accede a ella a través de la variable global.

  Es un buen momento para repasar la diferencia entre parámetro
  (el nombre con el que uno se refiere al dato dentro de la función)
  y argumento
  (el valor que es pasado a la función).
  En el enunciado, el parámetro se llama \li!tabla! dentro de la función,
  pero al momento de la llamada se le pasa como argumento la variable \li!asistencia!,
  que es como se llama la estructura afuera de la función.

  Al recorrer la tabla de asistencia,
  se van obteniendo cada una de las listas que la componen.
  Lo que hay que hacer es ir contando cuántas veces aparece \li!True!
  en cada una de ellas, y agregar en orden los totales a la lista que será retornada.
  En el programa adjunto,
  se muestran varias maneras de hacerlo,
  en orden creciente de sofisticación.

  Las versiones 3 y 4 usan la función \li!sum! para contar el total de una lista.
  Esto es posible ya que los valores booleanos \li!True! y \li!False!
  valen 0 y 1 respectivamente en un contexto numérico.
  Más aún, el tipo \li!bool! es una subclase de \li!int!,
  por lo que los valores booleanos son enteros en toda su regla:
  \begin{lstlisting}
>>> isinstance(True, int)
True
  \end{lstlisting}

  \diapo{problema-asistencia-3}

  Lea el enunciado en voz alta.
  Aunque este ejercicio parece similar al anterior,
  es más difícil, pues para cada clase la información está distribuida
  entre todas las listas de la tabla.
  Por lo tanto, hay que ir recorriendo las filas en paralelo.

  La solución adjunta opera \emph{à la Pascal}:
  inicializa una lista de ceros, y recorre filas y columnas
  usando dos \li!for! anidados, y actualizando en cada iteración
  el contador asociado a la clase.

  Una solución más sofisticada (y que es mejor no mencionar)
  es usar la función \li!zip! para trasponer la tabla:
  \begin{lstlisting}
def total_por_clase(tabla):
    return map(sum, zip(*tabla))
  \end{lstlisting}
  El asterisco significa «cada uno de los argumentos de la función
  será tomado de la lista \li!tabla!».

  \diapo{problema-asistencia-4}

  Lea el enunciado en voz alta, y discuta los tipos de datos.

  Esta función es sencilla de resolver usando la función \li!total_por_alumno!
  desarrollada anteriormente, junto con operaciones ya provistas por Python.

  Al discutir la estrategia para resolver este problema,
  hay que dejar claro cuáles son cada uno de los pasos:
  \begin{itemize}
    \item calcular los totales por alumnos,
    \item obtener el máximo de ellos, y
    \item ver cuál alumno es el que tiene ese máximo de asistencias.
  \end{itemize}
  Si bien el ejercicio podría resolverse con un solo ciclo que vaya sumando
  totales y llevando la cuenta de cuál es el alumno que lleva más,
  es una buena práctica descomponer el problema en subproblemas más pequeños.
  En muchos casos, uno se encuentra que estos subproblemas ya están resueltos,
  como ocurre en este caso, en que se usa la función \li!total_por_alumno!,
  la función \li!max! y el método \li!index! de las listas.

\end{document}

