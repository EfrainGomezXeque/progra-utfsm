\documentclass[10pt]{article}
\usepackage{beamerarticle}
\usepackage[spanish]{babel}
\usepackage[utf8]{inputenc}
\usepackage{fullpage}
\usepackage{xcolor}
\usepackage{listings}
\usepackage{textcomp}
\usepackage{mathpazo}
\usepackage{courier}
\usepackage{fancyvrb}
\usepackage{amsmath}
\usepackage{url}
\usepackage{hyperref}
\usepackage{pgfpages}
\usepackage{wrapfig}
\hyphenation{}

\setjobnamebeamerversion{30-diapos}

\newcommand{\onelinerule}{\rule[2.3ex]{0pt}{0pt}}
\newcommand{\twolinerule}{\rule[6.2ex]{0pt}{0pt}}
\newcommand{\respuesta}{\framebox[\textwidth]{\twolinerule}}
\newcommand{\nombre}{%
  \begin{tikzpicture}[xscale=.4,yscale=.7]
    \draw (0, 0) rectangle (22, 1);
  \end{tikzpicture}%
}
%\newcommand{\rol}   {\framebox[0.3\textwidth]{\onelinerule}}
\newcommand{\rol}{%
  \begin{tikzpicture}[xscale=.4,yscale=.7]
    \draw[gray!40] ( 0, 0) grid      ( 9, 1);
    \draw          ( 0, 0) rectangle ( 9, 1);
    \draw          (10, 0) rectangle (11, 1);
    \draw (9 + .2, .5) -- (10 - .2, .5);
  \end{tikzpicture}%
}
\newcommand{\li}{\lstinline}
\providecommand{\pond}[1]{[{\small\textbf{#1\%}}]}

\lstdefinelanguage{py}{%
  classoffset=0,%
    morekeywords={%
      False,class,finally,is,return,None,continue,for,lambda,try,%
      True,def,from,nonlocal,while,and,del,global,not,with,print,%
      as,elif,if,or,yield,assert,else,import,pass,break,except,in,raise},%
    keywordstyle=\color{black!80}\bfseries,%
  classoffset=1,
    morekeywords={int,float,str,abs,len,raw_input,exit,range,min,max,%
      set,dict,tuple,list,bool,complex,round,sum,all,any,zip,map,filter,%
      sorted,reversed,dir,file,frozenset,open,%
      array,zeros,ones,arange,linspace,eye,diag,dot},
    keywordstyle=\color{black!50}\bfseries,%
  classoffset=0,%
  sensitive=true,%
  morecomment=[l]\#,%
  morestring=[b]',%
  morestring=[b]",%
  stringstyle=\em,%
}

\lstdefinelanguage{testcase}{%
  moredelim=[is][\bfseries]{`}{`},%
  backgroundcolor=\color{gray!20},%
}

\lstdefinelanguage{file}{%
  frame=single,%
}

\lstset{language=py}
\lstset{basicstyle=\ttfamily}
\lstset{columns=fixed}
\lstset{upquote=true}
\lstset{showstringspaces=false}
\lstset{rangeprefix=\#\ }
\lstset{includerangemarker=false}

\newlist{certamen}{enumerate}{1}
\setlist[certamen]{%
  label=\arabic*.,
  font=\LARGE\bfseries,%
  labelindent=-.5in,%
  leftmargin=0pt,%
  labelsep=1em%
}


\newcommand{\diapo}[1]{
  \begin{wrapfigure}{r}{.50\textwidth}
    \hfill\fbox{\includeslide[width=.45\textwidth]{#1}}
  \end{wrapfigure}
}


\title{Más sobre interfaces gráficas}
\author{Programación \\ \url{http://progra.usm.cl}}
\date{}

\begin{document}
  \maketitle

  \section*{Objetivos de la clase}
  \begin{itemize}
    \item Mostrar algunas configuraciones de estilo
      que es posible aplicar a los widgets.
    \item Enseñar dos maneras nuevas de posicionar widgets:
      empaquetando hacia el lado (\li!pack(side='left')!)
      y usando una grilla (\li!grid()!).
    \item Introducir el uso de marcos (\emph{frames})
      para organizar los widgets de una ventana.
  \end{itemize}

  \section*{Diapositivas}

  En la clase de hoy presentaremos dos conceptos importantes nuevos:
  las distintas maneras de agregar los widgets a la interfaz
  (dependiendo de la geometría deseada)
  y cómo agrupar widgets usando marcos.
  Además,
  veremos cómo modificar algunos atributos visuales de los widgets.

  Para crear una interfaz completamente funcional,
  es necesario usar modelos y controladores.
  Sin embargo,
  en la clase de hoy nos enfocaremos sólo
  en aspectos de las vistas.
  Los ejemplos y ejercicios que veremos hoy
  son la cáscara de un programa.
  Terminar estos programas introduciendo los modelos y controladores apropiados
  puede ser un ejercicio futuro, o para el final de la clase si queda tiempo.

  \diapo{repaso-estructura}

  Primero que todo,
  haga un resumen de los conceptos introducidos en la clase anterior.
  Comience mostrando la estructura básica de un programa
  que tiene una interfaz gráfica.

  Los elementos que hay que destacar del programa mostrado son:
  \begin{itemize}
    \item la importación del módulo,
    \item la creación de la ventana principal
      usando el constructor \li!Tk!, y
    \item el inicio del ciclo de eventos
      mediante la llamada al método \li!mainloop!.
  \end{itemize}
  Todos los programas que usan Tkinter
  siguen esta estructura,
  y el resto de la interfaz debe ser construida
  en la parte indicada con puntos suspensivos.
  El ciclo de eventos
  hace que el programa se quede esperando
  acciones por parte del usuario,
  que gatillan llamadas a funciones
  que están asociadas a los widgets.

  \diapo{repaso-widgets}

  Repase el concepto de widget
  (cada uno de los componentes de la interfaz),
  y recuerde los tres widgets básicos
  presentados en la clase anterior:
  la etiqueta (\li!Label!),
  el botón (\li!Button!)
  y el campo de entrada (\li!Entry!).

  Recuerde también cuál es el uso de cada uno.
  La etiqueta es para mostrar datos
  (análogo al \li!print!),
  el campo de entrada es para que el usuario introduzca datos
  (análogo al \li!raw_input!)
  y el botón es un «llamador de funciones».

  \diapo{repaso-crear-widgets}

  Para agregar un widget a una interfaz,
  hay que hacer las siguientes dos acciones:
  \begin{itemize}
    \item instanciar el widget,
      pasando como primer parámetro la ventanta que lo contendrá, y
    \item llamar al método \li!pack!
      para que sea agregado a la ventana.
  \end{itemize}
  Al crear los widgets de esta manera,
  serán apilados verticalmente en la ventana,
  de arriba hacia abajo.

  \diapo{estilo}
  \diapo{configuraciones-estilo}

  Al momento de instanciar un widget,
  es posible especificar sus características
  usando parámetros con nombre.
  Aquí mostraremos algunos de los posibles parámetros
  que sirven para modificar la apariencia de un widget.

  El objetivo aquí es simplemente mostrar a los estudiantes
  que es posible modificar la apariencia de la interfaz
  para que lo tengan presente.
  La lista de atributos presentados aquí
  no es de ninguna manera completa.

  Explique el programa de ejemplo,
  y asocie a cada línea de código
  su widget correspondiente en la interfaz mostrada.

  \li!width! y \li!height!
  especifican el ancho y la altura del widget
  en unidades de texto.

  \li!fg! y \li!bg!
  especifican, respectivamente,
  el color de las letras y del fondo.
  El valor debe ser un string
  con el nombre del color en inglés
  o con el color en la notación de HTML
  (por ejemplo, \li!'#5A27F0'!).

  \li!font! especifica el tipo y el tamaño de letra.
  El valor debe ser una tupla de dos valores:
  el primero es el nombre de la familia
  (por ejemplo \li!'Arial'!, \li!'Times'! o \li!'Courier'!),
  y el segundo es un entero indicando el tamaño.
  También se puede pasar una tupla de tres valores,
  de los que el tercero es \li!'bold'! o \li!'italic'!
  para indicar negrita o cursiva, respectivamente.

  \li!borderwidth! especifica el ancho del borde del widget en pixeles.

  Puede mostrar el programa adjunto
  y hacer modificaciones en vivo
  para dejar más claro cómo se usan los atributos,
  y cuáles son sus valores posibles.

  \diapo{pack-left}

  En las interfaces que presentamos durante la clase pasada,
  todos los widgets eran apilados verticalmente.
  Ahora enseñaremos cómo hacerlo horizontalmente.
  Basta con pasar el parámetro \li!side='left'!
  en la llamada a \li!pack!.

  El parámetro \li!side! especifica cuál es el lado del widget
  que irá pegado al borde de la ventana,
  o a los widgets puestos anteriormente.
  Sus valores posibles son:
  \li!'top'! (arriba),
  \li!'bottom'! (abajo),
  \li!'left'! (izquierda) y
  \li!'right'! (derecha).
  Si el parámetro no es especificado,
  el valor por omisión es \li!'top'!.
  En general, para interfaces sencillas
  debe bastar con conocer el \li!'left'! y el \li!'top'!.

  Explique el código, y deje bien clara cuál es la diferencia
  entre este programa y los que hicimos antes.
  El código está incluido en los programas adjuntos.
  Puede mostrar el programa en ejecución,
  y modificar la llamada a \li!pack!
  para ilustrar en vivo cuál es la diferencia.

  \diapo{grillas}

  Además de \li!pack!,
  que va pegando los widgets uno después del otro,
  existe otro mecanismo para ubicar los componentes
  dentro de la interfaz,
  que es usando una grilla.

  Para crear una grilla de widgets,
  hay que agregarlos a la ventana usando el método \li!grid!,
  e indicando en qué fila y columna (\li!row! y \li!column!)
  de la grilla irá cada uno.

  \li!pack! y \li!grid! son dos tipos diferentes de \emph{geometry managers},
  y son excluyentes: todos los widgets que van en un mismo contenedor
  deben ser agregados usando el mismo geometry manager.
  Bajo ninguna circunstancia deben mezclarse ambos
  dentro de un mismo contenedor.

  Explique el código,
  y muestre la correspondencia de cada línea
  con la interfaz mostrada en la figura.
  Enfatice que aquí no se está usando \li!pack!.
  Puede mostrar el programa en ejecución,
  y modificar las llamadas
  para ilustrar en vivo cómo son agregados los widgets.

  Aquí,
  haga un resumen de los tres mecanismos posibles
  para agregar un widget a la ventana:
  \begin{itemize}
    \item \li!widget.pack()! (apilar verticalmente),
    \item \li!widget.pack(side='left')! (apilar horizontalmente), y
    \item \li!widget.grid(row=i, column=j)! (grilla).
  \end{itemize}

  \diapo{ejemplo-marcos}

  En las interfaces que hemos creado hasta ahora,
  los widgets han estado siempre contenidos directamente
  dentro de la ventana principal.

  Muestre la interfaz de ejemplo,
  y discuta sobre cuál es el geometry manager indicado para crearla.
  La respuesta es que es una mezcla de ambos:
  los campos del formulario están puestos en una grilla,
  y los botones están apilados horizontalmente
  sin estar alineados con la grilla.
  Ya dijimos que no era posible mezclar \li!pack! y \li!grid!,
  por lo que esta solución no es aceptable.

  Aquí debe introducir el concepto de marco (\li!Frame!).
  Un marco es un contenedor de widgets
  que a su vez va contenido dentro de la ventana principal.
  Los marcos sirven para agrupar widgets
  e irlos agregando agrupadamente a la interfaz.

  Muestre cómo la interfaz de la figura
  puede separarse en dos partes:
  la grilla superior
  y la tira de botones inferior.
  Ambas de estas partes, a su vez,
  están apiladas verticalmente dentro de la ventana principal.

  \diapo{marcos-0}

  Muestre primero cómo crear la ventana principal
  junto con los dos marcos que la componen.
  Haga notar que los \li!Frame!s reciben como primer parámetro
  la ventana principal, pues es en ella que estarán contenidos.
  Haga corresponder a cada una de las variables creadas
  su parte de la interfaz asociada en la figura.

  \diapo{marcos-1}

  A continuación,
  muestre cómo agregar los campos del formulario
  al primero de los marcos.
  Haga notar que el primer parámetro de los widgets
  debe ser el marco correspondiente,
  y no la ventana principal.
  Por brevedad,
  no aparecen las llamadas a \li!grid! en la diapositiva,
  pero es importante señalar que ellas deben ir en el programa,
  o de otro modo los widgets no serían agregados a la interfaz.

  \diapo{marcos-2}

  A continuación,
  muestre como agregar los botones
  al segundo de los marcos.
  Nuevamente haga notar que el primer parámetro de los widgets
  debe ser el marco correspondiente,
  y no la ventana principal o el marco anterior.

  Dentro de este marco,
  los botones son agregados usando \li!pack!,
  lo que no entra en conflicto con la grilla del marco anterior
  pues se trata de contenedores distintos.

  \diapo{programa-marcos}

  Finalmente muestre el programa terminado.
  El código de la diapositiva está separado
  en nueve «párrafos», agrupados por propósito.
  Vaya mostrando uno por uno de los párrafos,
  y explique qué hace cada uno.
  En orden de arriba hacia abajo:
  \begin{itemize}
    \item se importa el módulo,
    \item se crea la ventana principal,
    \item se crea los dos marcos,
    \item se crea los campos y etiquetas del formulario,
    \item se crea los botones,
    \item se agrega los campos y etiquetas a su marco usando una grilla,
    \item se agrega los botones a su marco apilándolos horizontalmente,
    \item se agrega los dos marcos a la ventana apilándolos verticalmente,
    \item se inicia el ciclo de eventos.
  \end{itemize}

  Es importante hacer notar nuevamente
  que al instanciar cada widget
  hay que indicar cuál es su contenedor,
  que puede ser un marco y la ventana principal.
  Puede mostrar el programa en ejecución,
  y hacer modificaciones para mostrar qué ocurre.

  \diapo{ejercicio-edad}
  \diapo{ejercicio-tabla}
  \diapo{ejercicio-cachipun}

  En estos tres ejercicios,
  el objetivo es escribir el código
  para crear una interfaz como la de cada figura.
  Inicialmente,
  no se preocupe de los controladores y los modelos,
  sólo de la vista.

  Primero que todo,
  hay que determinar si es necesario usar marcos
  para agrupar los widgets.
  Luego, hay que determinar cuáles son los widgets que van en cada marcos,
  y con qué geometry manager deben ser agregados.
  Finalmente,
  hay que decidir cómo empaquetar los marcos.

  Las soluciones están incluidas
  en los programas adjuntos al material de clase.
  Si le queda tiempo,
  puede completar alguna de las interfaces
  con los modelos y controladores apropiados
  para que quede completamente funcional.

\end{document}

