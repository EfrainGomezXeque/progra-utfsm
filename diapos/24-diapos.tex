\documentclass[12pt]{beamer}
\usepackage[spanish]{babel}
\usepackage[utf8]{inputenc}
\usepackage{xcolor}
\usepackage{listings}
\usepackage{textcomp}
\usepackage{mathpazo}
\usepackage{courier}
\usepackage{fancyvrb}
\usepackage{amsmath}
\usepackage{url}
\usepackage{hyperref}

\newcommand{\onelinerule}{\rule[2.3ex]{0pt}{0pt}}
\newcommand{\twolinerule}{\rule[6.2ex]{0pt}{0pt}}
\newcommand{\respuesta}{\framebox[\textwidth]{\twolinerule}}
\newcommand{\nombre}{%
  \begin{tikzpicture}[xscale=.4,yscale=.7]
    \draw (0, 0) rectangle (22, 1);
  \end{tikzpicture}%
}
%\newcommand{\rol}   {\framebox[0.3\textwidth]{\onelinerule}}
\newcommand{\rol}{%
  \begin{tikzpicture}[xscale=.4,yscale=.7]
    \draw[gray!40] ( 0, 0) grid      ( 9, 1);
    \draw          ( 0, 0) rectangle ( 9, 1);
    \draw          (10, 0) rectangle (11, 1);
    \draw (9 + .2, .5) -- (10 - .2, .5);
  \end{tikzpicture}%
}
\newcommand{\li}{\lstinline}
\providecommand{\pond}[1]{[{\small\textbf{#1\%}}]}

\lstdefinelanguage{py}{%
  classoffset=0,%
    morekeywords={%
      False,class,finally,is,return,None,continue,for,lambda,try,%
      True,def,from,nonlocal,while,and,del,global,not,with,print,%
      as,elif,if,or,yield,assert,else,import,pass,break,except,in,raise},%
    keywordstyle=\color{black!80}\bfseries,%
  classoffset=1,
    morekeywords={int,float,str,abs,len,raw_input,exit,range,min,max,%
      set,dict,tuple,list,bool,complex,round,sum,all,any,zip,map,filter,%
      sorted,reversed,dir,file,frozenset,open,%
      array,zeros,ones,arange,linspace,eye,diag,dot},
    keywordstyle=\color{black!50}\bfseries,%
  classoffset=0,%
  sensitive=true,%
  morecomment=[l]\#,%
  morestring=[b]',%
  morestring=[b]",%
  stringstyle=\em,%
}

\lstdefinelanguage{testcase}{%
  moredelim=[is][\bfseries]{`}{`},%
  backgroundcolor=\color{gray!20},%
}

\lstdefinelanguage{file}{%
  frame=single,%
}

\lstset{language=py}
\lstset{basicstyle=\ttfamily}
\lstset{columns=fixed}
\lstset{upquote=true}
\lstset{showstringspaces=false}
\lstset{rangeprefix=\#\ }
\lstset{includerangemarker=false}

\newlist{certamen}{enumerate}{1}
\setlist[certamen]{%
  label=\arabic*.,
  font=\LARGE\bfseries,%
  labelindent=-.5in,%
  leftmargin=0pt,%
  labelsep=1em%
}



\usecolortheme{crane}
\usefonttheme{serif}

\title{Arreglos bidimensionales}
\author{
  Programación \\ \url{http://progra.usm.cl}
}
\date{18 y 19 de mayo de 2011}

\begin{document}
  \begin{frame}
    \maketitle
  \end{frame}

  \begin{frame}
    \label{crear-arreglo-bidmensional}
    \frametitle{Arreglos bidimensionales}
    \lstinputlisting{programas/arreglo-bidimensional-crear-valores.py}
  \end{frame}

  \begin{frame}
    \label{crear-arreglo-bidmensional-funciones}
    \frametitle{Otras maneras de crear arreglos 2D}
    \lstinputlisting{programas/arreglo-bidimensional-crear-funciones.py}
  \end{frame}

  \begin{frame}
    \label{cambiar-forma}
    \frametitle{Cambiar la forma del arreglo}
    \lstinputlisting{programas/arreglo-bidimensional-cambiar-forma.py}
  \end{frame}

  \begin{frame}
    \label{indices}
    \frametitle{Elementos y secciones del arreglo}
    \lstinputlisting{programas/arreglo-bidimensional-indices.py}
  \end{frame}

  \begin{frame}
    \label{ejercicio-crear-arreglo}
    \frametitle{Ejercicio: crear arreglos}
    Crear los siguientes arreglos:
    \[
      \begin{bmatrix}
        9 & 9 & 9 & 9 & 9 \\
        9 & 9 & 9 & 9 & 9 \\
        9 & 9 & 0 & 9 & 9 \\
        9 & 9 & 9 & 9 & 9 \\
        9 & 9 & 9 & 9 & 9 \\
      \end{bmatrix}
      \begin{bmatrix}
        8 & 8 & 8 & 8 & 8 \\
        8 & 8 & 8 & 8 & 8 \\
        8 & 8 & 8 & 8 & 8 \\
        1 & 1 & 1 & 1 & 1 \\
        8 & 8 & 8 & 8 & 8 \\
      \end{bmatrix}
      \begin{bmatrix}
        8 & 8 & 8 & 8 & 8 \\
        8 & 8 & 8 & 8 & 8 \\
        8 & 8 & 8 & 8 & 8 \\
        0 & 1 & 2 & 3 & 4 \\
        8 & 8 & 8 & 8 & 8 \\
      \end{bmatrix}
    \]
    \[
      \begin{bmatrix}
        5 & 5 & 5 & 5 & 5 \\
        5 & 0 & 0 & 0 & 5 \\
        5 & 0 & 0 & 0 & 5 \\
        5 & 0 & 0 & 0 & 5 \\
        5 & 5 & 5 & 5 & 5 \\
      \end{bmatrix}
      \begin{bmatrix}
        0 & 1 & 2 & 3 & 4 \\
        4 & 3 & 2 & 1 & 0 \\
        0 & 1 & 2 & 3 & 4 \\
        4 & 3 & 2 & 1 & 0 \\
        0 & 1 & 2 & 3 & 4 \\
      \end{bmatrix}
      \begin{bmatrix}
        1 & 2 & 3 & 1 & 2 \\
        3 & 1 & 2 & 3 & 1 \\
        2 & 3 & 1 & 2 & 3 \\
        1 & 2 & 3 & 1 & 2 \\
        3 & 1 & 2 & 3 & 1 \\
      \end{bmatrix}
    \]
  \end{frame}

  \begin{frame}
    \label{transponer}
    \frametitle{Transponer arreglo}
    \lstinputlisting{programas/arreglo-bidimensional-transponer.py}
  \end{frame}

  \begin{frame}
    \label{diagonales}
    \frametitle{Diagonales}
    \lstinputlisting{programas/arreglo-bidimensional-diagonales.py}
  \end{frame}

  %\begin{frame}
  %  \frametitle{Suma de elementos}
  %  \lstinputlisting{programas/arreglo-bidimensional-sumar.py}
  %\end{frame}

  \begin{frame}
    \label{cuadrado-magico}
    \frametitle{Ejercicio: cuadrado mágico}
    Una matriz es un \emph{cuadrado mágico}
    si las sumas de cada fila, de cada columna
    y de las dos diagonales son iguales.

    Por ejemplo:
    \[
      m =
      \begin{bmatrix}
         4 & 15 & 14 &  1 \\
         9 &  6 &  7 & 12 \\
         5 & 10 & 11 &  8 \\
        16 &  3 &  2 & 13 \\
      \end{bmatrix}
    \]
    Escriba una función que indique \\
    si una matriz es o no un cuadrado mágico:
    \lstinputlisting{programas/caso-cuadrado-magico.py}
  \end{frame}

\end{document}

