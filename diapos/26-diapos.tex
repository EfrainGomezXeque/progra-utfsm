\documentclass[12pt]{beamer}
\usepackage[spanish]{babel}
\usepackage[utf8]{inputenc}
\usepackage{xcolor}
\usepackage{listings}
\usepackage{textcomp}
\usepackage{mathpazo}
\usepackage{courier}
\usepackage{fancyvrb}
\usepackage{amsmath}
\usepackage{url}
\usepackage{hyperref}

\newcommand{\onelinerule}{\rule[2.3ex]{0pt}{0pt}}
\newcommand{\twolinerule}{\rule[6.2ex]{0pt}{0pt}}
\newcommand{\respuesta}{\framebox[\textwidth]{\twolinerule}}
\newcommand{\nombre}{%
  \begin{tikzpicture}[xscale=.4,yscale=.7]
    \draw (0, 0) rectangle (22, 1);
  \end{tikzpicture}%
}
%\newcommand{\rol}   {\framebox[0.3\textwidth]{\onelinerule}}
\newcommand{\rol}{%
  \begin{tikzpicture}[xscale=.4,yscale=.7]
    \draw[gray!40] ( 0, 0) grid      ( 9, 1);
    \draw          ( 0, 0) rectangle ( 9, 1);
    \draw          (10, 0) rectangle (11, 1);
    \draw (9 + .2, .5) -- (10 - .2, .5);
  \end{tikzpicture}%
}
\newcommand{\li}{\lstinline}
\providecommand{\pond}[1]{[{\small\textbf{#1\%}}]}

\lstdefinelanguage{py}{%
  classoffset=0,%
    morekeywords={%
      False,class,finally,is,return,None,continue,for,lambda,try,%
      True,def,from,nonlocal,while,and,del,global,not,with,print,%
      as,elif,if,or,yield,assert,else,import,pass,break,except,in,raise},%
    keywordstyle=\color{black!80}\bfseries,%
  classoffset=1,
    morekeywords={int,float,str,abs,len,raw_input,exit,range,min,max,%
      set,dict,tuple,list,bool,complex,round,sum,all,any,zip,map,filter,%
      sorted,reversed,dir,file,frozenset,open,%
      array,zeros,ones,arange,linspace,eye,diag,dot},
    keywordstyle=\color{black!50}\bfseries,%
  classoffset=0,%
  sensitive=true,%
  morecomment=[l]\#,%
  morestring=[b]',%
  morestring=[b]",%
  stringstyle=\em,%
}

\lstdefinelanguage{testcase}{%
  moredelim=[is][\bfseries]{`}{`},%
  backgroundcolor=\color{gray!20},%
}

\lstdefinelanguage{file}{%
  frame=single,%
}

\lstset{language=py}
\lstset{basicstyle=\ttfamily}
\lstset{columns=fixed}
\lstset{upquote=true}
\lstset{showstringspaces=false}
\lstset{rangeprefix=\#\ }
\lstset{includerangemarker=false}

\newlist{certamen}{enumerate}{1}
\setlist[certamen]{%
  label=\arabic*.,
  font=\LARGE\bfseries,%
  labelindent=-.5in,%
  leftmargin=0pt,%
  labelsep=1em%
}



\usecolortheme{crane}
\usefonttheme{serif}

\title{Resolución de sistemas lineales}
\author{
  Programación \\ \url{http://progra.usm.cl}
}
\date{}

\begin{document}
  \begin{frame}
    \maketitle
  \end{frame}

  \begin{frame}
    \label{producto-matriz-vector}
    \frametitle{Repaso: producto matriz-vector}
    \includegraphics[width=\textwidth]{../diagramas/matriz-vector.pdf}
  \end{frame}

  \begin{frame}
    \label{producto-matriz-vector-numpy}
    \frametitle{Repaso: producto matriz-vector en NumPy}
    \lstinputlisting[linerange=MATRIZ\ VECTOR-FIN\ MATRIZ\ VECTOR]
        {programas/producto-interno.py}
  \end{frame}

  \begin{frame}
    \label{ejercicio-nutrientes}
    \frametitle{Variación del ejercicio de la clase anterior}
    La siguiente tabla muestra cuántos gramos de nutrientes
    son aportados por cada 100 gramos de ciertos alimentos:
    \vspace{2ex}

    {\footnotesize
    \begin{tabular}{lp{6em}p{6em}p{4.5em}}
      & \raggedright Leche descremada & \raggedright Harina de soya & {\raggedright Suero de leche} \\\hline
      Proteínas     &  36 &   51 &   13  \\\hline
      Carbohidratos &  52 &   34 &   74  \\\hline
      Grasa         &   0 &    7 &  1.1  \\\hline
    \end{tabular}}

    \begin{enumerate}
      \vspace{2ex}
      \item
        Obtener el \textbf{aporte en nutrientes} al consumir \\
        10 gramos de leche descremada,
        50 gramos de harina de soya y
        30 gramos de suero de leche.
      \vspace{2ex}
      \item
        Obtener las \textbf{cantidades de alimentos} para consumir
        33 gramos de proteínas,
        45 gramos de carbohidratos y
        3 gramos de grasa.
    \end{enumerate}
  \end{frame}

  \begin{frame}
    \label{diagramas-dieta}
    \begin{columns}
      \column{0.45\textwidth}
        \begin{block}{Problema 1}
          \vspace{2ex}
          \includegraphics[width=.9\textwidth]{../diagramas/dieta-1.pdf}
        \end{block}
      \column{0.45\textwidth}
        \begin{block}{Problema 2}
          \vspace{2ex}
          \includegraphics[width=.9\textwidth]{../diagramas/dieta-2.pdf}
        \end{block}
    \end{columns}
  \end{frame}

  \begin{frame}
    \label{solucion-nutrientes}
    \frametitle{Solución problema 1}
    \lstinputlisting[linerange=CALCULAR\ NUTRIENTES-FIN\ CALCULAR\ NUTRIENTES]
        {programas/ecuaciones-matriciales.py}
  \end{frame}

  \begin{frame}
    \label{solucion-alimentos}
    \frametitle{Solución problema 2}
    \lstinputlisting[linerange=CALCULAR\ ALIMENTOS-FIN\ CALCULAR\ ALIMENTOS]
        {programas/ecuaciones-matriciales.py}
  \end{frame}

  \begin{frame}
    \label{enunciado-feria}
    \frametitle{Ejercicio: feria de entretenciones}
    La entrada a una feria de entretenciones \\
    vale \$500 para niños y \$1\,800 para adultos.
    \vspace{2ex}

    En cierto día,
    2\,200 personas fueron a la feria, \\
    y se recaudó \$2\,010\,000 en total.
    \vspace{2ex}

    ¿Cuántos niños y cuántos adultos asistieron?
    \pause
    \[
      \begin{array}{cc}
        &
        \begin{bmatrix} n \\ a \\ \end{bmatrix} \vspace{3ex} \\
        \begin{bmatrix} 1 & 1 \\ 500 & 1800 \\ \end{bmatrix} &
        \begin{bmatrix} 2200  \\ 2010000    \\ \end{bmatrix}
      \end{array}
    \]
  \end{frame}

  \begin{frame}
    \label{ejercicio-parabola}
    \frametitle{Ejercicio: parábola}
    Una parábola \(f(x) = ax^2 + bx + c\)
    pasa por los puntos
    \((3, 9)\), \((4, 5)\) y \((5, 12)\).
    \vspace{2ex}

    ¿Cuál es esa parábola? Es decir,
    ¿cuánto valen \(a\), \(b\) y \(c\)?

    \pause
    \[
      \begin{array}{cc}
        &
        \begin{bmatrix} a \\ b \\ c \\ \end{bmatrix} \vspace{3ex} \\
        \begin{bmatrix}
          3^2 & 3 & 1 \\
          4^2 & 4 & 1 \\
          5^2 & 5 & 1 \\
        \end{bmatrix} &
        \begin{bmatrix} 9 \\ 5 \\ 12  \\ \end{bmatrix}
      \end{array}
    \]

  \end{frame}

  \begin{frame}
    \label{problema-parabola}
    Escriba un programa que pida al usuario \\
    ingresar las coordenadas de tres puntos, \\
    y muestre cuál es la parábola que pasa por esos puntos:
    \lstinputlisting[language=testcase]{programas/caso-parabola.txt}
  \end{frame}

  \begin{frame}
    \label{ejercicio-migracion}
    \frametitle{Migración urbana}
    Estudios demográficos muestran que anualmente
    el 5\% de la población de una ciudad
    se muda a los suburbios (y el 95\% se queda en la ciudad),
    mientras que el 3\% de la población de los suburbios
    se muda a la ciudad (y el 97\% se queda).
    \vspace{2ex}

    En el año 2011,
    600 mil personas viven en la ciudad,
    y 400 mil viven en los suburbios.
    ¿Cuáles serán las poblaciones en el año 2012?
    \pause
    \[
      \begin{array}{cc}
        &
        \begin{bmatrix} 600 \\ 400 \\ \end{bmatrix} \vspace{3ex} \\
        \begin{bmatrix}
          0.95 & 0.03 \\
          0.05 & 0.97 \\
        \end{bmatrix} &
        \begin{bmatrix} \;c\; \\ \;s\; \\ \end{bmatrix}
      \end{array}
    \]
  \end{frame}

  \begin{frame}
    \label{problema-migracion}
    Escriba un programa que pida al usuario \\
    las poblaciones iniciales de la ciudad y los suburbios, \\
    y muestre una tabla de las poblaciones \\
    para los próximos 10 años.
    \footnotesize
    \lstinputlisting[language=testcase]{programas/caso-migracion.txt}
  \end{frame}

\end{document}

