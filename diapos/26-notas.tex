\documentclass[10pt]{article}
\usepackage{beamerarticle}
\usepackage[spanish]{babel}
\usepackage[utf8]{inputenc}
\usepackage{fullpage}
\usepackage{xcolor}
\usepackage{listings}
\usepackage{textcomp}
\usepackage{mathpazo}
\usepackage{courier}
\usepackage{fancyvrb}
\usepackage{amsmath}
\usepackage{url}
\usepackage{hyperref}
\usepackage{pgfpages}
\usepackage{wrapfig}
\hyphenation{}

\setjobnamebeamerversion{26-diapos}

\newcommand{\onelinerule}{\rule[2.3ex]{0pt}{0pt}}
\newcommand{\twolinerule}{\rule[6.2ex]{0pt}{0pt}}
\newcommand{\respuesta}{\framebox[\textwidth]{\twolinerule}}
\newcommand{\nombre}{%
  \begin{tikzpicture}[xscale=.4,yscale=.7]
    \draw (0, 0) rectangle (22, 1);
  \end{tikzpicture}%
}
%\newcommand{\rol}   {\framebox[0.3\textwidth]{\onelinerule}}
\newcommand{\rol}{%
  \begin{tikzpicture}[xscale=.4,yscale=.7]
    \draw[gray!40] ( 0, 0) grid      ( 9, 1);
    \draw          ( 0, 0) rectangle ( 9, 1);
    \draw          (10, 0) rectangle (11, 1);
    \draw (9 + .2, .5) -- (10 - .2, .5);
  \end{tikzpicture}%
}
\newcommand{\li}{\lstinline}
\providecommand{\pond}[1]{[{\small\textbf{#1\%}}]}

\lstdefinelanguage{py}{%
  classoffset=0,%
    morekeywords={%
      False,class,finally,is,return,None,continue,for,lambda,try,%
      True,def,from,nonlocal,while,and,del,global,not,with,print,%
      as,elif,if,or,yield,assert,else,import,pass,break,except,in,raise},%
    keywordstyle=\color{black!80}\bfseries,%
  classoffset=1,
    morekeywords={int,float,str,abs,len,raw_input,exit,range,min,max,%
      set,dict,tuple,list,bool,complex,round,sum,all,any,zip,map,filter,%
      sorted,reversed,dir,file,frozenset,open,%
      array,zeros,ones,arange,linspace,eye,diag,dot},
    keywordstyle=\color{black!50}\bfseries,%
  classoffset=0,%
  sensitive=true,%
  morecomment=[l]\#,%
  morestring=[b]',%
  morestring=[b]",%
  stringstyle=\em,%
}

\lstdefinelanguage{testcase}{%
  moredelim=[is][\bfseries]{`}{`},%
  backgroundcolor=\color{gray!20},%
}

\lstdefinelanguage{file}{%
  frame=single,%
}

\lstset{language=py}
\lstset{basicstyle=\ttfamily}
\lstset{columns=fixed}
\lstset{upquote=true}
\lstset{showstringspaces=false}
\lstset{rangeprefix=\#\ }
\lstset{includerangemarker=false}

\newlist{certamen}{enumerate}{1}
\setlist[certamen]{%
  label=\arabic*.,
  font=\LARGE\bfseries,%
  labelindent=-.5in,%
  leftmargin=0pt,%
  labelsep=1em%
}


\newcommand{\diapo}[1]{
  \begin{wrapfigure}{r}{.50\textwidth}
    \hfill\fbox{\includeslide[width=.45\textwidth]{#1}}
  \end{wrapfigure}
}


\title{Resolución de sistemas lineales}
\author{Programación \\ \url{http://progra.usm.cl}}
\date{}

\begin{document}
  \maketitle

  \section*{Objetivos de la clase}
  \begin{itemize}
    \item Enseñar cómo resolver sistemas lineales usando SciPy.
    \item Plantear sistemas lineales sencillos
      en forma matricial, para ser resueltos en el computador.
  \end{itemize}

  \section*{Diapositivas}

  En esta clase,
  veremos cómo resolver sistemas lineales
  de la forma \(Ax = b\),
  donde \(A\) es una matriz dada,
  \(b\) es un vector dado
  y \(x\) es el vector cuyo valor se desea resolver.
  Para esto, usaremos la función \li!solve!
  que está provista por el módulo \li!numpy.linalg!.

  El objetivo no es enseñar la teoría detras de los sistemas de ecuaciones,
  sino mostrarlos como una manera de plantear un problema
  de modo de resolverlo usando una técnica ya implementada.

  Si en la clase pasada no alcanzó a cubrir todas las diapositivas,
  retome esa materia desde donde quedó,
  y luego continúe con la materia de esta clase.

  \diapo{producto-matriz-vector}
  \diapo{producto-matriz-vector-numpy}

  Antes de comenzar, haga un repaso de la operación
  de producto entre una matriz y un vector.
  Enfatice que todos los elementos del resultado
  son simplemente productos internos entre las filas de la matriz
  y el vector.

  El resultado de esta operación
  es obtenido en NumPy usando la función \li!dot!,
  pasándole como argumentos la matriz \(A\) y el vector \(x\).
  Si la matriz \(A\) es de \(m\times n\),
  entonces el vector \(x\) debe tener \(n\) elementos,
  y el resultado \(b = Ax\) es un vector de \(m\) elementos.

  \diapo{ejercicio-nutrientes}

  Este ejercicio es una variación de uno de los problemas vistos
  la clase pasada. La tabla asocia a cada alimento su aporte
  en cada uno de los nutrientes.
  Lea el enunciado en voz alta,
  y discuta qué es lo que se pide obtener en ambas preguntas.

  En la primera pregunta, la cantidad de alimentos está dada,
  y se pide obtener el aporte por cada nutriente.
  En la segunda es al revés: el aporte está dado,
  y se pide obtener cuánto hay que consumir de cada alimento.
  (Esta es la motivación original para el problema,
  ya que las cantidades de comida de una dieta
  están propuestas en función del aporte que se espera obtener).

  Proponga a los estudiantes que formulen ambas preguntas
  usando matrices y vectores.
  Idealmente,
  deberían poder plantear los problemas
  de manera parecida a lo que se muestra en la próxima diapositiva.

  \diapo{diagramas-dieta}

  El diagrama de esta diapostiva
  ilustra claramente cuáles son los datos que están dados,
  y cuáles son las incógnitas en las dos preguntas planteadas.
  En ambos casos,
  cada elemento del vector \li!x! está asociado a un alimento
  (es decir, a una columna de \li!a!)
  y cada elemento del vector \li!b! está asociado a un nutriente
  (es decir, a una fila de \li!a!).

  Tómese su tiempo para explicar el diagrama,
  qué significa cada valor
  y a qué están asociadas las filas y las columnas.

  Pregunte a los estudiantes
  cómo resolverían cada uno de los problemas mediante un programa.
  En el primer caso,
  la solución es simplemente calcular el producto usando la función \li!dot!.
  Para el segundo caso se necesita la nueva función que enseñaremos,
  pero en principio se puede resolver usando métodos que algunos alumnos
  ya podrían conocer, como aplicar operaciones elementales entre las filas
  para despejar las incógnitas.

  \diapo{solucion-nutrientes}

  La solución al primer problema
  consiste simplemente en obtener el producto entre la matriz y el vector
  usando la función \li!dot!.

  \diapo{solucion-alimentos}

  La solución al segundo problema
  es obtenida usando la función \li!solve!
  que hay que importar desde el módulo \li!numpy.linalg!.

  Enfatice cuál es el tipo de problemas que resuelve la función \li!solve!:
  aquéllos que tienen la forma del segundo de los diagramas presentados anteriormente,
  donde la incógnita es el vector que se está multiplicando por la matriz.

  \diapo{enunciado-feria}

  Ejercicio simple de planteo.
  La idea es formular la solución en función de matrices y vectores
  de una manera que pueda ser resuelta usando la función \li!solve!.
  Proponga a los estudiantes que lo planteen.

  Primero hay que identificar las dos incógnitas.
  Para poder usar \li!solve!, ambas incógnitas deben aparecer
  como elementos del vector \li!b!.

  La estrategia para plantear el sistema
  es identificar cuáles son los productos internos involucrados.
  En este caso, hay uno para el total de personas
  (en que las ponderaciones son 1 y 1)
  y otro para el total recaudado
  (donde las ponderaciones son los precios).

  La respuesta al ejercicio de planteo está incluida en la diapositiva.

  \diapo{ejercicio-parabola}

  Otro ejercicio de planteo.
  Nuevamente, hay que formular el problema
  en función de matrices y vectores.
  Otra vez, dé unos minutos a los estudiantes para que lo descubran por sí mismos.

  La clave sigue siendo identificar
  cuáles son los productos internos involucrados.
  La evaluación del polinomio cuadrático \(f\)
  es en efecto un producto interno
  entre los coeficientes del polinomio
  y las potencias de la variable \(x\).

  Una pregunta que cabe discutir es:
  ¿cómo se decide si asignar los coeficientes a la matriz \(A\)
  y las potencias al vector \(b\), o viceversa?
  Una regla simple para notarlo es
  analizar cuáles son los datos que cambian en cada evaluación.
  En este caso, son las potencias las que cambian (ya que \(x\) cambia),
  y por lo tanto son las que deben ir en cada fila de la matriz.
  Los coeficientes del polinomio son siempre los mismos,
  y por lo tanto van en el vector \(b\).

  La respuesta al ejercicio de planteo está incluida en la diapositiva.

  \diapo{problema-parabola}

  Ejercicio práctico para implementar una solución
  para el problema planteado en el ejercicio anterior.

  Lea el enunciado, explique el caso de prueba
  y discuta las estrategias para resolverlo.

  Al final,
  las partes más engorrosas del programa son la entrada y la salida.
  La parte del proceso es muy simple
  ya que es sólo una llamada a la función \li!solve!.
  La tarea del programador es crear las matrices con los elementos indicados
  en el orden apropiado para poder usar el resolvedor ya disponible.

  En los programas adjuntos,
  se ilustra tres maneras distintas de llenar la matriz.

  \diapo{ejercicio-migracion}

  Ejercicio simple de planteo.
  Si le queda tiempo en la clase,
  pida nuevamente a los estudiantes que lo formulen
  en términos de matrices y vectores.

  En este problema,
  la incógnita no es el vector \(x\),
  por lo que no hay que usar la función \li!solve!,
  sino la función \li!dot!.
  Procure que los estudiantes lo descubran por sí mismos.
  Nuevamente, la clave es identificar cuáles son los productos internos involucrados.

  La respuesta al ejercicio de planteo está incluida en la diapositiva.

  \diapo{problema-migracion}

  Si le queda tiempo en la clase,
  resuelva este problema en el computador.
  Se trata simplemente de ir premultiplicando el vector de poblaciones
  por la matriz de migración,
  de esta manera: \li!p = dot(m, p)!.

  El programa está incluido en los archivos anexos.





\end{document}

