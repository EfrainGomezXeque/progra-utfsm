\documentclass[10pt]{article}
\usepackage{beamerarticle}
\usepackage[spanish]{babel}
\usepackage[utf8]{inputenc}
\usepackage{fullpage}
\usepackage{xcolor}
\usepackage{listings}
\usepackage{textcomp}
\usepackage{mathpazo}
\usepackage{courier}
\usepackage{fancyvrb}
\usepackage{amsmath}
\usepackage{url}
\usepackage{hyperref}
\usepackage{pgfpages}
\usepackage{wrapfig}

\setjobnamebeamerversion{04-diapos}

\newcommand{\onelinerule}{\rule[2.3ex]{0pt}{0pt}}
\newcommand{\twolinerule}{\rule[6.2ex]{0pt}{0pt}}
\newcommand{\respuesta}{\framebox[\textwidth]{\twolinerule}}
\newcommand{\nombre}{%
  \begin{tikzpicture}[xscale=.4,yscale=.7]
    \draw (0, 0) rectangle (22, 1);
  \end{tikzpicture}%
}
%\newcommand{\rol}   {\framebox[0.3\textwidth]{\onelinerule}}
\newcommand{\rol}{%
  \begin{tikzpicture}[xscale=.4,yscale=.7]
    \draw[gray!40] ( 0, 0) grid      ( 9, 1);
    \draw          ( 0, 0) rectangle ( 9, 1);
    \draw          (10, 0) rectangle (11, 1);
    \draw (9 + .2, .5) -- (10 - .2, .5);
  \end{tikzpicture}%
}
\newcommand{\li}{\lstinline}
\providecommand{\pond}[1]{[{\small\textbf{#1\%}}]}

\lstdefinelanguage{py}{%
  classoffset=0,%
    morekeywords={%
      False,class,finally,is,return,None,continue,for,lambda,try,%
      True,def,from,nonlocal,while,and,del,global,not,with,print,%
      as,elif,if,or,yield,assert,else,import,pass,break,except,in,raise},%
    keywordstyle=\color{black!80}\bfseries,%
  classoffset=1,
    morekeywords={int,float,str,abs,len,raw_input,exit,range,min,max,%
      set,dict,tuple,list,bool,complex,round,sum,all,any,zip,map,filter,%
      sorted,reversed,dir,file,frozenset,open,%
      array,zeros,ones,arange,linspace,eye,diag,dot},
    keywordstyle=\color{black!50}\bfseries,%
  classoffset=0,%
  sensitive=true,%
  morecomment=[l]\#,%
  morestring=[b]',%
  morestring=[b]",%
  stringstyle=\em,%
}

\lstdefinelanguage{testcase}{%
  moredelim=[is][\bfseries]{`}{`},%
  backgroundcolor=\color{gray!20},%
}

\lstdefinelanguage{file}{%
  frame=single,%
}

\lstset{language=py}
\lstset{basicstyle=\ttfamily}
\lstset{columns=fixed}
\lstset{upquote=true}
\lstset{showstringspaces=false}
\lstset{rangeprefix=\#\ }
\lstset{includerangemarker=false}

\newlist{certamen}{enumerate}{1}
\setlist[certamen]{%
  label=\arabic*.,
  font=\LARGE\bfseries,%
  labelindent=-.5in,%
  leftmargin=0pt,%
  labelsep=1em%
}


\newcommand{\diapo}[1]{
  \begin{wrapfigure}{r}{.50\textwidth}
    \hfill\fbox{\includeslide[width=.45\textwidth]{#1}}
  \end{wrapfigure}
}


\title{Estructuras condicionales}
\author{Programación \\ \url{http://progra.usm.cl}}
\date{}

\begin{document}
  \maketitle

  \section*{Objetivos de la clase}
  \begin{itemize}
    \item Presentar las sentencias \li!if!, \li!if-elif! e \li!if-elif-else!.
    \item Enseñar el uso de estructuras condicionales.
    \item Resolver ejercicios con estructuras condicionales.
  \end{itemize}

  \section*{Diapositivas}

  Los tipos de sentencias condicionales
  son presentados usando problemas sencillos
  relacionados con el cálculo del promedio de notas de un alumno.
  Al mostrar cada programa,
  explique con detención cómo es usada la sentencia condicional respectiva.

  \diapo{problema-if}

  A esta altura, los estudiantes deberían ser capaces
  de escribir por su cuenta
  la parte del programa que lee la entrada y que calcula el promedio.
  Es una buena oportunidad para ponerlos a prueba.

  Hay que tener cuidado en cómo calcular el promeedio
  para evitar errores al dividir y al redondear.
  La solución en la diapo siguiente
  aproxima el promedio al entero más cercano.

  \diapo{solucion-if}

  Aquí se introduce la sentencia \li!if!.

  Mencionar que \emph{if} en inglés significa «si».

  El \li!if! tiene una condición,
  que es una expresión lógica.
  Al llegar al \li!if!, la expresión es evaluada,
  y su resultado decide si las sentencias son ejecutadas o no.

  Éste es el primer ejemplo
  en que es necesario indentar.
  Explique que el primer \li!print! sí está dentro del \li!if!
  mientras que el segundo no lo está.

  Un error común es olvidar poner los dos puntos después de la condición.

  No es legal que la línea que sigue al \li!if! no esté indentada.

  El estándar es usar cuatro espacios para indentar.

  \diapo{problema-if-else}

  Nuevamente los estudiantes deberían ser capaces de desarrollar la primera parte por sí solos.

  También podría ocurrírseles resolver el problema usando dos \li!if!s,
  con condiciones \li!promedio < 55! y \li!promedio >= 55!.

  \diapo{solucion-if-else}

  Aquí se introduce la sentencia \li!if-else!.
  
  Mencionar que \emph{else} en inglés significa «o si no».

  Todo \li!else! siempre está asociado a un \li!if!.
  Ambos deben partir siempre en la misma columna.

  \diapo{problema-if-elif}

  Este problema también puede ser resuelto
  usando varios \li!if!s.

  \diapo{solucion-if-elif}

  Aquí se introduce la sentencia \li!if-elif-else!.

  No es legal en Python usar \li!else if ...:!,
  y por eso la sentencia \li!elif! es necesaria.
  Como es una contracción de \li!else if!,
  en español vendría a ser algo como «o si no, si».

  El uso de varios \li!elif!s encadenados
  hace que el programa sea mucho más claro.
  Si se usara varios \li!if!s desacoplados,
  no sería evidente a primera vista
  que el programa seguirá sólo una de las alternativas.

  \diapo{ejercicio-bisiesto}

  Este ejercicio está pensado para que lo resuelvan los estudiantes.

  Tras leer el enunciado,
  analizar cuál es la condición de la regla del año bisiesto
  que hace que cada caso de prueba entregue el resultado mostrado.
  
  \diapo{bisiesto-1}

  Primera manera de resolver el problema.
  Se cubre el caso más específico primero,
  y luego se va refinando hacia el caso más general.

  Explicar cuál es el flujo de ejecución
  para los casos de prueba presentados.

  \diapo{bisiesto-2}

  Segunda manera de resolver el problema.
  Sirve para introducir sentencias de control anidadas.

  En este caso, se cubre el caso más general primero,
  y luego es refinado con otro \li!if!.

  Nuevamente, explicar cuál es el flujo de ejecución
  para los casos de prueba presentados.

  \diapo{bisiesto-3}

  Tercera manera de resolver el problema.
  En este caso, una sola expresión lógica indica el camino a seguir.
  
  El año es bisiesto si es divisible por 400, o si es divisible por 4 y no por 100.

  La expresión puede abarcar dos líneas de código
  ya que hay un paréntesis abierto sin cerrar.
  Ésta es la manera de continuar líneas en Python.

  Además, se usa paréntesis para forzar la precedencia entre los operadores \li!and! y \li!or!.

  Para el resto de la clase,
  puede encontrar ejercicios para resolver en
  \url{http://progra.usm.cl/apunte/ejercicios/1/index.html#estructuras-condicionales}.
\end{document}


