\documentclass[10pt]{article}
\usepackage{beamerarticle}
\usepackage[spanish]{babel}
\usepackage[utf8]{inputenc}
\usepackage{fullpage}
\usepackage{xcolor}
\usepackage{listings}
\usepackage{textcomp}
\usepackage{mathpazo}
\usepackage{courier}
\usepackage{fancyvrb}
\usepackage{amsmath}
\usepackage{url}
\usepackage{hyperref}
\usepackage{pgfpages}
\usepackage{wrapfig}

\setjobnamebeamerversion{10-diccionarios-conjuntos-diapos}

\newcommand{\onelinerule}{\rule[2.3ex]{0pt}{0pt}}
\newcommand{\twolinerule}{\rule[6.2ex]{0pt}{0pt}}
\newcommand{\respuesta}{\framebox[\textwidth]{\twolinerule}}
\newcommand{\nombre}{%
  \begin{tikzpicture}[xscale=.4,yscale=.7]
    \draw (0, 0) rectangle (22, 1);
  \end{tikzpicture}%
}
%\newcommand{\rol}   {\framebox[0.3\textwidth]{\onelinerule}}
\newcommand{\rol}{%
  \begin{tikzpicture}[xscale=.4,yscale=.7]
    \draw[gray!40] ( 0, 0) grid      ( 9, 1);
    \draw          ( 0, 0) rectangle ( 9, 1);
    \draw          (10, 0) rectangle (11, 1);
    \draw (9 + .2, .5) -- (10 - .2, .5);
  \end{tikzpicture}%
}
\newcommand{\li}{\lstinline}
\providecommand{\pond}[1]{[{\small\textbf{#1\%}}]}

\lstdefinelanguage{py}{%
  classoffset=0,%
    morekeywords={%
      False,class,finally,is,return,None,continue,for,lambda,try,%
      True,def,from,nonlocal,while,and,del,global,not,with,print,%
      as,elif,if,or,yield,assert,else,import,pass,break,except,in,raise},%
    keywordstyle=\color{black!80}\bfseries,%
  classoffset=1,
    morekeywords={int,float,str,abs,len,raw_input,exit,range,min,max,%
      set,dict,tuple,list,bool,complex,round,sum,all,any,zip,map,filter,%
      sorted,reversed,dir,file,frozenset,open,%
      array,zeros,ones,arange,linspace,eye,diag,dot},
    keywordstyle=\color{black!50}\bfseries,%
  classoffset=0,%
  sensitive=true,%
  morecomment=[l]\#,%
  morestring=[b]',%
  morestring=[b]",%
  stringstyle=\em,%
}

\lstdefinelanguage{testcase}{%
  moredelim=[is][\bfseries]{`}{`},%
  backgroundcolor=\color{gray!20},%
}

\lstdefinelanguage{file}{%
  frame=single,%
}

\lstset{language=py}
\lstset{basicstyle=\ttfamily}
\lstset{columns=fixed}
\lstset{upquote=true}
\lstset{showstringspaces=false}
\lstset{rangeprefix=\#\ }
\lstset{includerangemarker=false}

\newlist{certamen}{enumerate}{1}
\setlist[certamen]{%
  label=\arabic*.,
  font=\LARGE\bfseries,%
  labelindent=-.5in,%
  leftmargin=0pt,%
  labelsep=1em%
}


\newcommand{\diapo}[1]{
  \begin{wrapfigure}{r}{.50\textwidth}
    \hfill\fbox{\includeslide[width=.45\textwidth]{#1}}
  \end{wrapfigure}
}


\title{Diccionarios y conjuntos}
\author{Programación \\ \url{http://progra.usm.cl}}
\date{}

\begin{document}
  \maketitle

  \section*{Objetivos de la clase}
  \begin{itemize}
    \item Enseñar el uso de diccionarios y conjuntos en Python.
  \end{itemize}

  \section*{Diapositivas}

  \diapo{dicc-telefonos}

  Un \emph{diccionario} es una estructura de datos
  que sirve para asociar pares llave-valor.
  Las llaves son usadas como índices
  para acceder a los valores.

  En el ejemplo,
  se ha usado un diccionario
  para asociar a cada persona su número de teléfono.
  Para obtener el teléfono de alguien,
  hay que usar el operador de índice \li![]!
  aplicado sobre su nombre.
  En este caso,
  los nombres son las llaves,
  y los teléfonos los valores.

  En Python, los diccionarios se crean usando paréntesis de llave (\li!{}!),
  y representando cada par llave-valor unido por dos puntos (\li!:!).

  Los diccionarios no tienen equivalente en C y Pascal,
  pero sí en otros lenguajes:
  los hashes de Perl,
  los arreglos asociativos de PHP,
  los maps de C++
  y los Hashtables de Java.
  Internamente,
  los diccionarios de Python están implementados como tablas hash,
  al igual que en Java y a diferencia de C++, donde son árboles binarios.

  La ventaja de las tablas hash
  es que el tiempo de acceso para obtener un valor es \(O(1)\) amortizado
  (es decir, prácticamente constante),
  independientemente del tamaño del diccionario.
  Esto ocurre ya que el mismo valor de la llave
  indica dónde está el valor asociado en la memoria,
  por lo que no es necesario recorrer una estructura de datos
  para encontrarlo.

  \diapo{dicc-agregar}

  Para crear una nueva asociación en el diccionario,
  basta con asignar el valor al diccionario indexado por la llave.
  En las tres primeras asignaciones del ejemplo,
  se ha asignado un número entero
  a cada una de las letras \li!'a'!, \li!'x'! y \li!'m'!.

  Es importante indicar que el diccionario no guarda los pares
  en ningún orden específico.
  A pesar que asignamos el valor asociado a \li!'x'! antes que el asociado a \li!'m'!,
  al mostrar el diccionario aparece la \li!'m'! antes.
  Uno jamás debe esperar que los elementos de un diccionario
  estén en algún orden determinístico.

  En las siguientes dos asignaciones,
  se ha asociado a \li!'f'! un valor que ya estaba en el diccionario (5),
  y se ha cambiado el valor asociado a la llave \li!'x'!.
  Al mostrar el nuevo estado del diccionario,
  queda claro que éste no puede contener llaves repetidas,
  pero valores repetidos sí.

  En resumen,
  la asignación de un valor en un diccionario
  puede cumplir uno de dos roles:
  crear una asociación nueva (cuando la llave no existía),
  o modificar una existente.

  \diapo{dicc-operaciones}

  El diccionario de este ejemplo
  asocia a cada animal su cantidad de patas.

  Un diccionario es un iterable que,
  al ser recorrido, entrega sus llaves.
  Una de las consecuencias de esto
  es que, al convertir el diccionario en una lista,
  se obtiene la lista de las llaves.

  Para obtener la lista de los valores,
  hay que usar el método \li!d.values()!.

  El operador \li!in! sirve para buscar una llave en el diccionario,
  no un valor.
  Explique esto con detención usando el ejemplo como referencia:
  el número 8 está en el diccionario (asociado al pulpo),
  pero no es una llave, sino un valor,
  por lo que \li!8 in patas! retorna falso.

  La función \li!len! entrega cuántos pares llave-valor tiene el diccionario.
  No cuenta las llaves y los valores por separado.

  \diapo{ejercicio-contar-letras}
  \diapo{solucion-contar-letras}

  Un uso típico de los diccionarios
  es crear una colección de contadores.
  En este ejercicio,
  no hay que contar una de las letras en la oración,
  sino todas ellas.
  Explique los ejemplos para que quede claro qué es lo que se pide.

  La solución está basada en el típico patrón de contar cosas.
  La diferencia es que hay un contador por cada letra.
  Para cada letra, hay que revisar si ya ha sido agregada previamente al diccionario;
  si no es así, hay que hacerlo asociándola al valor cero.
  Si no se hiciera esto,
  ocurriría un \emph{error de llave} (\li!KeyError!)
  al intentar incrementar el valor asociado a una llave inexistente.

  \diapo{recorrer-diccionarios}

  Al igual que todas las estructuras de datos,
  los diccionarios pueden ser recorridos usando el ciclo \li!for!.
  Las maneras de hacerlo son tres:
  \begin{enumerate}
    \item recorrer las llaves (usando el diccionario directamente),
    \item recorrer los valores (usando el método \li!d.values()!), y
    \item recorrer los pares llave-valor (usando el método \li!d.items()!).
  \end{enumerate}

  Haga notar que, al recorrer las llaves,
  de todas maneras podemos obtener el valor usando el operador de indexación
  como en el primer ciclo,
  pero al recorrer los valores no es posible obtener directamente la llave asociada.
  Por eso, de cada capital sólo podemos decir que es linda, pero no a qué país pertenece.

  \diapo{conjuntos-crear}

  Los \emph{conjuntos} son colecciones desordenadas de elementos no repetidos.
  Son muy similares a los conjuntos de matemáticas
  que ya son familiares para los estudiantes.

  Con la sintaxis de los conjuntos hay que tener cuidado.
  El uso de llaves (\li!{! y \li!}!) para crear conjuntos
  fue introducido recién en Python 2.7.
  En versiones anteriores, esta manera no es válida.
  Ojo que hay estudiantes que tienen instalada la versión 2.6 del intérprete,
  especialmente los que ocupan Mac o Linux.

  Por la misma razón,
  al mostrar un conjunto en la consola,
  aparece como una lista pasada a la función \li!set!.

  En las versiones 3.x de Python,
  este problema está solucionado,
  y las llaves aparecen tanto al crear los conjuntos
  como en su representación en la consola.
  Por ahora, tenemos que lidiar con esta pequeña inconsistencia.

  En los ejemplos,
  muestre que en ningún caso aparecen los elementos repetidos en el conjunto,
  y que el orden de los elementos no es relevante de ninguna manera.

  \diapo{conjuntos-modificar}

  Pregunta capciosa para los estudiantes:
  ¿por qué los conjuntos vacíos se crean con \li!set()! y no con \li!{}!?
  La respuesta es que \li!{}! es el diccionario vacío.

  Modificar conjuntos es fácil:
  se agregan elementos con el método \li!add!
  y se eliminan con el método \li!remove!.
  Además,
  hay un método \li!discard! que también elimina un elemento,
  pero no lanza un error si el elemento no estaba en el conjunto.

  Al igual que en las demás estructuras de datos,
  el operador \li!in! (junto con su hermano \li!not in!)
  permite saber si un elemento está (o no) en el conjunto.

  \diapo{conjuntos-operaciones}

  Las operaciones son las mismas de teoría de conjuntos,
  pero representadas por caracteres ASCII:
  \begin{itemize}
    \item \li!&! es la intersección,
    \item \li!|! es la unión,
    \item \li!-! es la diferencia,
    \item \li!^! es la diferencia simétrica,
    \item \li!<! significa «es subconjunto propio de».
  \end{itemize}
  Los demás operadores relacionales \li!>!, \li!<=! y \li!>=!
  tienen también los significados obvios.

  \diapo{ejercicio-letras-comun}
  \diapo{solucion-letras-comun}

  Un ejercicio muy sencillo: contar las letras en común que tienen dos palabras.
  La solución consiste en meter las letras de las palabras a sendos conjuntos, y obtener su intersección.
\end{document}

