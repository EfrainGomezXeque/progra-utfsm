\documentclass[10pt]{article}
\usepackage{beamerarticle}
\usepackage[spanish]{babel}
\usepackage[utf8]{inputenc}
\usepackage{fullpage}
\usepackage{xcolor}
\usepackage{listings}
\usepackage{textcomp}
\usepackage{mathpazo}
\usepackage{courier}
\usepackage{fancyvrb}
\usepackage{amsmath}
\usepackage{url}
\usepackage{hyperref}
\usepackage{pgfpages}
\usepackage{wrapfig}
\usepackage{enumitem}

\setjobnamebeamerversion{07-funciones-diapos}

\newcommand{\onelinerule}{\rule[2.3ex]{0pt}{0pt}}
\newcommand{\twolinerule}{\rule[6.2ex]{0pt}{0pt}}
\newcommand{\respuesta}{\framebox[\textwidth]{\twolinerule}}
\newcommand{\nombre}{%
  \begin{tikzpicture}[xscale=.4,yscale=.7]
    \draw (0, 0) rectangle (22, 1);
  \end{tikzpicture}%
}
%\newcommand{\rol}   {\framebox[0.3\textwidth]{\onelinerule}}
\newcommand{\rol}{%
  \begin{tikzpicture}[xscale=.4,yscale=.7]
    \draw[gray!40] ( 0, 0) grid      ( 9, 1);
    \draw          ( 0, 0) rectangle ( 9, 1);
    \draw          (10, 0) rectangle (11, 1);
    \draw (9 + .2, .5) -- (10 - .2, .5);
  \end{tikzpicture}%
}
\newcommand{\li}{\lstinline}
\providecommand{\pond}[1]{[{\small\textbf{#1\%}}]}

\lstdefinelanguage{py}{%
  classoffset=0,%
    morekeywords={%
      False,class,finally,is,return,None,continue,for,lambda,try,%
      True,def,from,nonlocal,while,and,del,global,not,with,print,%
      as,elif,if,or,yield,assert,else,import,pass,break,except,in,raise},%
    keywordstyle=\color{black!80}\bfseries,%
  classoffset=1,
    morekeywords={int,float,str,abs,len,raw_input,exit,range,min,max,%
      set,dict,tuple,list,bool,complex,round,sum,all,any,zip,map,filter,%
      sorted,reversed,dir,file,frozenset,open,%
      array,zeros,ones,arange,linspace,eye,diag,dot},
    keywordstyle=\color{black!50}\bfseries,%
  classoffset=0,%
  sensitive=true,%
  morecomment=[l]\#,%
  morestring=[b]',%
  morestring=[b]",%
  stringstyle=\em,%
}

\lstdefinelanguage{testcase}{%
  moredelim=[is][\bfseries]{`}{`},%
  backgroundcolor=\color{gray!20},%
}

\lstdefinelanguage{file}{%
  frame=single,%
}

\lstset{language=py}
\lstset{basicstyle=\ttfamily}
\lstset{columns=fixed}
\lstset{upquote=true}
\lstset{showstringspaces=false}
\lstset{rangeprefix=\#\ }
\lstset{includerangemarker=false}

\newlist{certamen}{enumerate}{1}
\setlist[certamen]{%
  label=\arabic*.,
  font=\LARGE\bfseries,%
  labelindent=-.5in,%
  leftmargin=0pt,%
  labelsep=1em%
}


\newcommand{\diapo}[1]{
  \begin{wrapfigure}{r}{.50\textwidth}
    \hfill\fbox{\includeslide[width=.45\textwidth]{#1}}
  \end{wrapfigure}
}


\title{Funciones}
\author{Programación \\ \url{http://progra.usm.cl}}
\date{}

\begin{document}
  \maketitle

  \section*{Objetivos de la clase}
  \begin{itemize}
    \item Explicar el concepto de función.
    \item Enseñar las ventajas de usar funciones.
    \item Introducir los conceptos de parámetro, variables locales y globales,
      y valor de retorno.
  \end{itemize}

  \section*{Diapositivas}

  \diapo{uso-funcion-abs}

  Esta diapositiva presenta dos maneras de calcular
  el valor absoluto de un número real \(x\):
  implementando la definición matemática de \(\lvert x\rvert\),
  y usando la función \li!abs! provista por Python.

  Indique que el hecho de que la función ya exista
  permite ahorrar trabajo, aunque se trate de una operación sencilla.
  Lo mismo es aplicable a otras operaciones más difíciles de implementar,
  como la raíz cuadrada.

  \diapo{uso-funcion-factorial}

  

  (Las últimas versiones de Python incluyen una función \li!factorial!
  en el módulo \li!math!,
  pero para efectos pedagógicos
  omitiremos esta información).

  \diapo{def-fn-factorial}

  Éste es el programa completo
  que calcula el factorial
  usando una función.
  Las últimas tres líneas son las mismas
  en la parte derecha de la diapositiva anterior.

  Una función es una sección de código que hace algo
  de manera independiente al resto del programa.
  En esencia,
  una función es como un mini programa,
  que tiene una entrada (los parámetros)
  y una salida (el valor de retorno).

  Las sentencias nuevas que son introducidas en esta diapositiva
  son \li!def! y \li!return!.
  La sentencia \li!return! indica cuál es el resultado de la función.
  Cuando la ejecución alcanza la sentencia \li!return!,
  la función termina inmediatamente.

  \diapo{conceptos-funciones}

  Introduzca los conceptos de
  parámetro, variables locales y globales y valor de retorno
  usando el ejemplo.

  Los \emph{parámetros} son los nombres que son asociados
  a los valores que son pasados a la función en cada llamada.
  En el ejemplo,
  la función \li!factorial! tiene un solo parámetro llamado \li!n!.
  Una función puede tener más de un parámetro,
  o no tener ninguno.

  Al valor que es ligado al parámetro en una llamada se le llama \emph{argumento}.
  Se suele usar los terminos «parámetro» y «argumento» indistintamente,
  por lo que la diferencia no es tan relevante en la práctica.

  Las \emph{variables locales}
  son las variables que son definidas
  dentro de la función.
  En el ejemplo,
  las variables locales son \li!n!, \li!prod! e \li!i!.
  Los parámetros siempre están incluidos entre las variables locales.

  Las variables locales dejan de existir cuando la llamada termina.
  Sin embargo, no ocurre de la misma manera que en lenguajes como C o Pascal.
  En Python, las variables son nombres (o referencias), no valores.
  Después de la llamada, el nombre deja de existir,
  pero el valor que estaba ligado a ella puede permanecer en la memoria.
  En la implementación estándar de Python,
  un objeto es eliminado de la memoria
  sólo cuando no queda ningún nombre ligado a él
  (el nombre técnico de esto es «recolección de basura por conteo de referencias»).

  Las \emph{variables globales}
  son las variables que están definidas fuera de una función.
  En el ejemplo,
  \li!n! y \li!f! son variables globales.

  Los valores de las variables globales pueden ser usados
  desde cualquier parte del programa, incluso desde dentro de una función.
  Una diferencia importante con otros lenguajes es la siguiente:
  al intentar asignar un valor a una variable global
  desde dentro de una función, lo que realmente se está haciendo
  es crear una variable local con el mismo nombre.
  Para poder modificar una variable global en una función,
  hay que declarar la variable usando la sentencia \li!global!,
  pero esto rara vez es una buena idea en Python,
  ya que existen técnicas más convenientes
  para las circunstancias en las que en otros lenguajes
  se usan variables globales.
  Por lo mismo, la sentencia \li!global! no es parte de la materia.

  Respecto al mecanismo de paso de parámetros:
  en Python es mejor olvidarse de los conceptos de paso por valor y por referencia.
  Los parámetros son en efecto referencias,
  por lo que los argumentos son compartidos entre el programa y la función.
  Sin embargo, al asignarle un valor al parámetro,
  no se está modificando el objeto, sino que se está religando el nombre.
  La única manera de modificar un objeto (en Python, todos los valores son objetos)
  es mediante llamadas a métodos.
  Por lo tanto, el paso de referencias no funciona de la misma manera que en Pascal.

  El mecanismo de paso de parámetros usado por Python
  se llama \href{http://effbot.org/zone/call-by-object.htm}{paso por objeto},
  pero rara vez es necesario tener esto en consideración.
  La regla que realmente hay que recordar es:
  la asignación modifica nombres, no valores.

  Cuando aparece una variable en una expresión dentro de una función,
  primero el intérprete busca si se trata de una variable local.
  Si no hay una variable local con ese nombre,
  busca si hay una global.
  Por esto,
  si hay una variable local con el mismo nombre de una global,
  la variable global no es accesible desde dentro de la función.

  El \emph{valor de retorno} es el resultado de la función,
  y es indicado con la sentencia \li!return!.
  En este ejemplo se está retornando el valor de una variable,
  pero en general se puede poner cualquier expresión
  en el \li!return!.

  \diapo{ruteo-fn-factorial}

  Ruteo del programa que calcula el factorial.
  Para denotar que las variables locales existen
  sólo durante la llamada,
  se las ha pintado en gris cuando la función todavía no es ejecutada.

  Técnicamente, las variables locales no son creadas todas
  al momento de la llamada (como en C o Pascal),
  sino en la primera asignación.
  Pero por claridad,
  en el ruteo aparecen creadas todas al mismo tiempo.

  Hacer notar que hay dos variables llamadas \li!n!, una local y una global.

  \diapo{funciones-consola}

  Éste es un consejo práctico para probar las funciones que uno desarrolla.
  Uno puede copiar el código de la función desde el editor
  y pegarlo en la consola.
  A continuación,
  uno puede llamar la función todas las veces que quiera,
  y ver los resultados en vivo y en directo,
  sin la necesidad de escribir un programa para eso.
  Así es como se espera que los alumnos
  prueben sus funciones para los controles en línea.

  Todos los ejercicios con funciones de aquí en adelante
  tendrán los casos de prueba presentados
  como una sesión de consola.

  \diapo{ejercicio-binomial}
  \diapo{solucion-binomial}

  Este ejercicio es una extensión del anterior,
  ya que también utiliza la función \li!factorial!.
  Sirve para mostrar que es posible llamar una función desde otra,
  y que es posible reutilizar la misma función del ejemplo anterior.
  (Lo ideal sería importar la función desde el programa anterior
  para no tener que reescribirla.
  En la próxima clase veremos cómo hacer eso).

  En la solución,
  hacer notar que el cuerpo del programa ocupa sólo tres líneas
  gracias a las funciones que definimos.
  Si quisiéramos escribir este programa sin usar funciones,
  sería mucho más largo.
  Además,
  el programa es mucho más fácil de entender,
  ya que el nombre de cada función deja claro cuál es el resultado que entrega.

  \diapo{ejercicio-contar-letras}
  \diapo{solucion-contar-letras}

  Ejemplo de una función con dos parámetros.
  Servirá además para repasar operaciones con strings.

  La manera en que se recorre el string con las vocales:
  \begin{lstlisting}
vocales  = 'aeiou'
for i in range(5):
    v = vocales[i]
    c = contar(v, oracion)
  \end{lstlisting}
  no es la manera típica de hacerlo en Python.
  La forma idiomática es ésta:
  \begin{lstlisting}
for v in 'aeiou':
    c = contar(v, oracion)
  \end{lstlisting}

  En rigor, la función es innecesaria,
  ya que los strings tienen un método \li!count!:
  \begin{lstlisting}
>>> 'viva la programacion'.count('a')
4
  \end{lstlisting}



  Todo esto se verá más adelante en la unidad sobre procesamiento de texto.

\end{document}

