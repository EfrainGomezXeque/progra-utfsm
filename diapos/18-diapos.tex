\documentclass[12pt]{beamer}
\usepackage[spanish]{babel}
\usepackage[utf8]{inputenc}
\usepackage{xcolor}
\usepackage{listings}
\usepackage{textcomp}
\usepackage{mathpazo}
\usepackage{courier}
\usepackage{fancyvrb}
\usepackage{amsmath}
\usepackage{url}
\usepackage{hyperref}

\newcommand{\onelinerule}{\rule[2.3ex]{0pt}{0pt}}
\newcommand{\twolinerule}{\rule[6.2ex]{0pt}{0pt}}
\newcommand{\respuesta}{\framebox[\textwidth]{\twolinerule}}
\newcommand{\nombre}{%
  \begin{tikzpicture}[xscale=.4,yscale=.7]
    \draw (0, 0) rectangle (22, 1);
  \end{tikzpicture}%
}
%\newcommand{\rol}   {\framebox[0.3\textwidth]{\onelinerule}}
\newcommand{\rol}{%
  \begin{tikzpicture}[xscale=.4,yscale=.7]
    \draw[gray!40] ( 0, 0) grid      ( 9, 1);
    \draw          ( 0, 0) rectangle ( 9, 1);
    \draw          (10, 0) rectangle (11, 1);
    \draw (9 + .2, .5) -- (10 - .2, .5);
  \end{tikzpicture}%
}
\newcommand{\li}{\lstinline}
\providecommand{\pond}[1]{[{\small\textbf{#1\%}}]}

\lstdefinelanguage{py}{%
  classoffset=0,%
    morekeywords={%
      False,class,finally,is,return,None,continue,for,lambda,try,%
      True,def,from,nonlocal,while,and,del,global,not,with,print,%
      as,elif,if,or,yield,assert,else,import,pass,break,except,in,raise},%
    keywordstyle=\color{black!80}\bfseries,%
  classoffset=1,
    morekeywords={int,float,str,abs,len,raw_input,exit,range,min,max,%
      set,dict,tuple,list,bool,complex,round,sum,all,any,zip,map,filter,%
      sorted,reversed,dir,file,frozenset,open,%
      array,zeros,ones,arange,linspace,eye,diag,dot},
    keywordstyle=\color{black!50}\bfseries,%
  classoffset=0,%
  sensitive=true,%
  morecomment=[l]\#,%
  morestring=[b]',%
  morestring=[b]",%
  stringstyle=\em,%
}

\lstdefinelanguage{testcase}{%
  moredelim=[is][\bfseries]{`}{`},%
  backgroundcolor=\color{gray!20},%
}

\lstdefinelanguage{file}{%
  frame=single,%
}

\lstset{language=py}
\lstset{basicstyle=\ttfamily}
\lstset{columns=fixed}
\lstset{upquote=true}
\lstset{showstringspaces=false}
\lstset{rangeprefix=\#\ }
\lstset{includerangemarker=false}

\newlist{certamen}{enumerate}{1}
\setlist[certamen]{%
  label=\arabic*.,
  font=\LARGE\bfseries,%
  labelindent=-.5in,%
  leftmargin=0pt,%
  labelsep=1em%
}



\usecolortheme{crane}
\usefonttheme{serif}

\title{Procesamiento de texto}
\author{
  Programación \\ \url{http://progra.usm.cl}
}
\date{Clase 18}

\begin{document}
  \begin{frame}
    \maketitle
  \end{frame}

  \begin{frame}
    \label{saltos-linea}
    \frametitle{Saltos de línea}
    \lstinputlisting{programas/saltos-de-linea.py}
  \end{frame}

  \begin{frame}
    \label{reemplazar}
    \frametitle{Reemplazar secciones de un string}
    \lstinputlisting{programas/replace.py}
  \end{frame}

  \begin{frame}
    \label{partir-string-palabras}
    \frametitle{Partir un string por palabras}
    \lstinputlisting{programas/split-1.py}
  \end{frame}

  \begin{frame}
    \label{partir-string-separador}
    \frametitle{Partir un string con separador}
    \lstinputlisting{programas/split-2.py}
  \end{frame}

  \begin{frame}
    \label{unir-strings}
    \frametitle{Unir una secuencia de strings}
    \lstinputlisting{programas/join-1.py}
  \end{frame}

  \begin{frame}
    \label{unir-valores}
    \frametitle{Unir una secuencia de valores (no strings)}
    \lstinputlisting{programas/join-2.py}
  \end{frame}

  \begin{frame}
    \label{interpolacion-posicion}
    \frametitle{Interpolación de valores por posición}
    \lstinputlisting{programas/format-1.py}
  \end{frame}

  \begin{frame}
    \label{interpolacion-nombre}
    \frametitle{Interpolación de valores por nombre}
    \lstinputlisting{programas/format-2.py}
  \end{frame}

  \begin{frame}
    \label{mayusculas-minusculas}
    \frametitle{Mayúsculas y minúsculas}
    \lstinputlisting{programas/mayusculas-minusculas.py}
  \end{frame}

  \begin{frame}
    \label{problema-adn-enunciado}
    \frametitle{Problema: ADN}
    Una cadena de ADN es una secuencia
    de bases nitrogenadas llamadas
    adenina, citosina, timina y guanina. \\[2ex]

    En un programa,
    una cadena se representa como un string
    de caracteres \li!'a'!, \li!'c'!, \li!'t'! y \li!'g'!. \\[2ex]

    A cada cadena,
    le corresponde una cadena complementaria,
    que se obtiene intercambiando las adeninas con las timinas,
    y las citosinas con las guaninas:

    \lstinputlisting[linerange=1-2]{programas/cadena-adn.py}
  \end{frame}

  \begin{frame}
    \label{problema-adn-aleatorio}
    \frametitle{Problema: ADN---Ejercicio 1}
    Escriba la función \li!cadena_al_azar(n)!
    que retorne una cadena aleatoria de ADN
    de largo \li!n!:
    \lstinputlisting[linerange=1-4]{programas/caso-adn-cadena-al-azar.py}

    Pista:
    \lstinputlisting[linerange=6-10]{programas/caso-adn-cadena-al-azar.py}
  \end{frame}

  \begin{frame}
    \label{problema-adn-complementaria}
    \frametitle{Problema: ADN---Ejercicio 2}
    Escriba la función \li!complementaria(c)!
    que entregue la cadena complementaria de \li!c!:
    \lstinputlisting{programas/caso-adn-cadena-complementaria.py}
  \end{frame}

  \begin{frame}
    \label{problema-reporte-promedios}
    \frametitle{Problema: reporte de promedios}
    Escriba un programa que haga lo siguiente:
    \lstinputlisting[language=testcase]{programas/caso-reporte-promedios.txt}
  \end{frame}

  \begin{frame}
    \label{problema-cuenta-palabras}
    \frametitle{Problema: cuenta de palabras}
    Escriba la función \li!contar_palabras(oracion)!,
    que retorne un diccionario indicando cuántas veces
    aparece cada palabra en la oración:
    \footnotesize
    \lstinputlisting{programas/caso-cuenta-oracion.py}
  \end{frame}

\end{document}

