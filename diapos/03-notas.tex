\documentclass[10pt]{article}
\usepackage{beamerarticle}
\usepackage[spanish]{babel}
\usepackage[utf8]{inputenc}
\usepackage{fullpage}
\usepackage{xcolor}
\usepackage{listings}
\usepackage{textcomp}
\usepackage{mathpazo}
\usepackage{courier}
\usepackage{fancyvrb}
\usepackage{amsmath}
\usepackage{url}
\usepackage{hyperref}
\usepackage{pgfpages}
\usepackage{wrapfig}

\setjobnamebeamerversion{03-diapos}

\newcommand{\onelinerule}{\rule[2.3ex]{0pt}{0pt}}
\newcommand{\twolinerule}{\rule[6.2ex]{0pt}{0pt}}
\newcommand{\respuesta}{\framebox[\textwidth]{\twolinerule}}
\newcommand{\nombre}{%
  \begin{tikzpicture}[xscale=.4,yscale=.7]
    \draw (0, 0) rectangle (22, 1);
  \end{tikzpicture}%
}
%\newcommand{\rol}   {\framebox[0.3\textwidth]{\onelinerule}}
\newcommand{\rol}{%
  \begin{tikzpicture}[xscale=.4,yscale=.7]
    \draw[gray!40] ( 0, 0) grid      ( 9, 1);
    \draw          ( 0, 0) rectangle ( 9, 1);
    \draw          (10, 0) rectangle (11, 1);
    \draw (9 + .2, .5) -- (10 - .2, .5);
  \end{tikzpicture}%
}
\newcommand{\li}{\lstinline}
\providecommand{\pond}[1]{[{\small\textbf{#1\%}}]}

\lstdefinelanguage{py}{%
  classoffset=0,%
    morekeywords={%
      False,class,finally,is,return,None,continue,for,lambda,try,%
      True,def,from,nonlocal,while,and,del,global,not,with,print,%
      as,elif,if,or,yield,assert,else,import,pass,break,except,in,raise},%
    keywordstyle=\color{black!80}\bfseries,%
  classoffset=1,
    morekeywords={int,float,str,abs,len,raw_input,exit,range,min,max,%
      set,dict,tuple,list,bool,complex,round,sum,all,any,zip,map,filter,%
      sorted,reversed,dir,file,frozenset,open,%
      array,zeros,ones,arange,linspace,eye,diag,dot},
    keywordstyle=\color{black!50}\bfseries,%
  classoffset=0,%
  sensitive=true,%
  morecomment=[l]\#,%
  morestring=[b]',%
  morestring=[b]",%
  stringstyle=\em,%
}

\lstdefinelanguage{testcase}{%
  moredelim=[is][\bfseries]{`}{`},%
  backgroundcolor=\color{gray!20},%
}

\lstdefinelanguage{file}{%
  frame=single,%
}

\lstset{language=py}
\lstset{basicstyle=\ttfamily}
\lstset{columns=fixed}
\lstset{upquote=true}
\lstset{showstringspaces=false}
\lstset{rangeprefix=\#\ }
\lstset{includerangemarker=false}

\newlist{certamen}{enumerate}{1}
\setlist[certamen]{%
  label=\arabic*.,
  font=\LARGE\bfseries,%
  labelindent=-.5in,%
  leftmargin=0pt,%
  labelsep=1em%
}


\newcommand{\diapo}[1]{
  \begin{wrapfigure}{r}{.50\textwidth}
    \hfill\fbox{\includeslide[width=.45\textwidth]{#1}}
  \end{wrapfigure}
}


\title{Programas simples}
\author{Programación \\ \url{http://progra.usm.cl}}
\date{Clase 3}

\begin{document}
  \maketitle

  \section*{Objetivos de la clase}
  \begin{itemize}
    \item Desarrollar programas lineales sencillos.
    \item Reforzar los conceptos de
      entrada, salida, expresiones, asignaciones, tipos y errores.
    \item Enseñar algunas técnicas recurrentes de programación.
  \end{itemize}

  \section*{Diapositivas}

  \diapo{problema-gravitacion}

  Éste es un ejercicio típico de evaluar una fórmula.
  Con lo visto en la clase anterior,
  todos deberían poder resolverlo.
  Discutir cuál es la entrada, cuál es la salida,
  y cuáles son los tipos de los valores
  (en este caso son todos reales).

  \diapo{gravitacion}

  Las dos prácticas que son introducidas en este programa son
  los usos de constantes y comentarios.

  Python no tiene constantes como característica del lenguaje,
  por lo que nada impide que el valor de \li!G! sea modificado.
  De todos modos,
  siempre es más claro asignar las constantes al principio del programa
  y darles un nombre representativo.

  En Python,
  los comentarios comienzan con \li!#! y terminan al final de la línea.
  Los comentarios son ignorados por el intérprete
  y sirven como notas para los programadores que lean el programa.

  El uso de paréntesis en el denominador de la expresión
  no es necesario, ya que el operador \li!**! tiene mayor precedencia
  que \li!*! y \li!/!. De todos modos, es más claro para el lector
  usar los paréntesis.

  \diapo{problema-serrucho}

  La función continúa de la misma forma a lo largo de todo el eje real.
  Los círculos rellenos indican el valor de la función en los extremos de los tramos.

  Aquí el desafío es encontrar una expresión que describa la función,
  ya que aún no hemos pasado el \li!if!.

  El truco aquí es notar que, en todos los tramos,
  la función es un pedazo de la recta de pendiente \(3/2\)
  que pasa por el origen, desplazado verticalmente.

  Más precisamente,
  el desplazamiento vertical siempre es un múltiplo de 3:
  el primer diente está desplazado 0 unidades,
  el segundo 3 hacia abajo, el tercero 6, etcétera.

  El número de orden del diente al que pertenece \(x\) es \(\lfloor x/2\rfloor\).

  \diapo{serrucho}
  
  Las dos soluciones presentadas son equivalentes.
  La primera calcula el resultado en una única expresión,
  por lo que no es muy clara en una lectura inicial.

  En la segunda solución,
  la última asignación expresa casi literalmente
  cuál es la función:
  es una recta de cierta pendiente,
  desplazada en tres veces el orden de cada diente.

  Discutir si el programa entrega el resultado correcto
  en las fronteras de cada tramo.

  Técnicamente,
  la solución no describe correctamente la función serrucho,
  ya que el número de diente debe ser \(\lfloor x/2\rfloor\)
  (redondeo hacia abajo),
  mientras que la función \li!int! trunca los decimales
  (redondeo hacia cero).
  Por lo tanto,
  el programa entrega la solución incorrecta
  para valores negativos de \(x\).
  Para corregir esto,
  hay que usar la función \li!floor! que debe ser importada
  del módulo \li!math!.

  \diapo{problema-rimas}

  Este ejercicio es útil para repasar las operaciones sobre strings.

  \diapo{rima1}

  Como \li!raw_input! retorna un string,
  no es necesario ``castear'' el resultado.
  De todos modos, no es incorrecto poner \li!str(raw_input(...))!.

  La expresión lógica en el \li!print!
  puede continuar en la línea siguiente
  ya que quedó un paréntesis sin cerrar en la sentencia.
  Ésta es la manera de separar una sentencia
  en varias líneas en Python.
  La segunda línea puede ser indentada libremente.

  Es una buena oportunidad para repasar la diferencia
  entre \li!=! (sentencia de asignación)
  y \li!==! (operador de igualdad).

  Esta solución no es la manera idomática
  de resolver el problema en Python,
  pero es parecida a cómo se haría en C,
  por lo que no es mala idea presentarla.

  El programa arroja un \emph{error de índice}
  (lanzado por el operador \li![]!)
  si alguna de las palabras tiene menos de cuatro letras.

  \diapo{rima2}

  Esta solución utiliza la propiedad del operador \li![]!
  de que los índices negativos sirven para contar
  desde el final del string hacia atrás.
  \li!s[-1]! es siempre el último caracter del string.

  En este programa no es necesario calcular el largo del string.

  \diapo{rima3}

  Aquí se usa el operador de rebanado (\emph{slice})
  de los strings (ver \url{http://docs.python.org/library/stdtypes.html}, sección \emph{Sequence Types}).

  \li!s[a:b]! significa ``desde el \li!a!-ésimo caracter hasta el \li!b!-ésimo caracter
  del string \li!s!''.
  Si se omite \li!a!, es desde el principio. Si se omite \li!b!, es hasta el final.
  
  Este operador será pasado en la materia del segundo certamen,
  por lo que no es necesario que los estudiantes lo aprendan ahora.

  Este programa funciona incluso si las palabras tienen menos de cuatro letras.

  \diapo{problema-tazas}

  Éste es un ejercicio de aritmética modular.
  El desafío es resolverlo sin usar \li!for! ni \li!while!.

  \diapo{diagrama-tazas}

  Aquí conviene explicar de nuevo el enunciado usando el diagrama,
  para que quede completamente claro.
  Considerar los casos en que los índices se pasan de largo
  del principio o del final del círculo.
  No basta con restar \li!i! y sumar \li!j!.

  \diapo{tazas}

  Mencione que los ejercicios para recorrer elementos circularmente
  suelen resolverse usando el operador de módulo.
  Es una buena oportunidad para repasar este operador.

  Más ejercicios para hacer en clases
  pueden ser sacados de \url{http://progra.usm.cl/apunte/ejercicios/1/index.html#programas-simples}.

\end{document}


