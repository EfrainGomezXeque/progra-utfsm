\documentclass[12pt]{beamer}
\usepackage[spanish]{babel}
\usepackage[utf8]{inputenc}
\usepackage{xcolor}
\usepackage{listings}
\usepackage{textcomp}
\usepackage{mathpazo}
\usepackage{courier}
\usepackage{fancyvrb}
\usepackage{amsmath}
\usepackage{url}
\usepackage{hyperref}

\newcommand{\onelinerule}{\rule[2.3ex]{0pt}{0pt}}
\newcommand{\twolinerule}{\rule[6.2ex]{0pt}{0pt}}
\newcommand{\respuesta}{\framebox[\textwidth]{\twolinerule}}
\newcommand{\nombre}{%
  \begin{tikzpicture}[xscale=.4,yscale=.7]
    \draw (0, 0) rectangle (22, 1);
  \end{tikzpicture}%
}
%\newcommand{\rol}   {\framebox[0.3\textwidth]{\onelinerule}}
\newcommand{\rol}{%
  \begin{tikzpicture}[xscale=.4,yscale=.7]
    \draw[gray!40] ( 0, 0) grid      ( 9, 1);
    \draw          ( 0, 0) rectangle ( 9, 1);
    \draw          (10, 0) rectangle (11, 1);
    \draw (9 + .2, .5) -- (10 - .2, .5);
  \end{tikzpicture}%
}
\newcommand{\li}{\lstinline}
\providecommand{\pond}[1]{[{\small\textbf{#1\%}}]}

\lstdefinelanguage{py}{%
  classoffset=0,%
    morekeywords={%
      False,class,finally,is,return,None,continue,for,lambda,try,%
      True,def,from,nonlocal,while,and,del,global,not,with,print,%
      as,elif,if,or,yield,assert,else,import,pass,break,except,in,raise},%
    keywordstyle=\color{black!80}\bfseries,%
  classoffset=1,
    morekeywords={int,float,str,abs,len,raw_input,exit,range,min,max,%
      set,dict,tuple,list,bool,complex,round,sum,all,any,zip,map,filter,%
      sorted,reversed,dir,file,frozenset,open,%
      array,zeros,ones,arange,linspace,eye,diag,dot},
    keywordstyle=\color{black!50}\bfseries,%
  classoffset=0,%
  sensitive=true,%
  morecomment=[l]\#,%
  morestring=[b]',%
  morestring=[b]",%
  stringstyle=\em,%
}

\lstdefinelanguage{testcase}{%
  moredelim=[is][\bfseries]{`}{`},%
  backgroundcolor=\color{gray!20},%
}

\lstdefinelanguage{file}{%
  frame=single,%
}

\lstset{language=py}
\lstset{basicstyle=\ttfamily}
\lstset{columns=fixed}
\lstset{upquote=true}
\lstset{showstringspaces=false}
\lstset{rangeprefix=\#\ }
\lstset{includerangemarker=false}

\newlist{certamen}{enumerate}{1}
\setlist[certamen]{%
  label=\arabic*.,
  font=\LARGE\bfseries,%
  labelindent=-.5in,%
  leftmargin=0pt,%
  labelsep=1em%
}



\usecolortheme{seahorse}
\usefonttheme{serif}

\title{Listas y tuplas}
\author{
  Programación \\ \url{http://progra.usm.cl}
}
\date{}

\begin{document}
  \begin{frame}
    \maketitle
  \end{frame}

  \begin{frame}
    \frametitle{Desviación estándar}
    \label{def-desviacion}
    \begin{itemize}
      \item Datos: \([5.8\quad 3.1\quad 2.5]\)
      \item Promedio:
        \[
          \bar{x} =
          \frac{1}{3}
          (5.8 + 3.1 + 2.5) = 3.8
        \]
      \item Desviación estándar:
        \begin{align*}
          \sigma_x &=
          \sqrt{
            \frac{1}{2}
            \Bigl(
              (5.8 - 3.8)^2 +
              (3.1 - 3.8)^2 +
              (2.5 - 3.8)^2
            \Bigr)
          } \\[2ex]
          &\approx 1.76
        \end{align*}
    \end{itemize}

  \end{frame}

  \begin{frame}
    \label{ejercicio-desviacion}
    \frametitle{Ejercicio}
    Escriba un programa que pregunte \\
    cuántos datos se ingresará, pida los datos, \\
    y entregue la desviación estándar:
    \lstinputlisting[language=testcase]{programas/caso-desviacion.txt}
  \end{frame}

  \begin{frame}
    \label{programa-desviacion}
    \frametitle{Solución}
    \footnotesize
    \lstinputlisting{programas/desviacion-1.py}
  \end{frame}

  \begin{frame}
    \label{programa-desviacion-funciones}
    \frametitle{Solución con funciones}
    \tiny
    \lstinputlisting{programas/desviacion-2.py}
  \end{frame}

  \begin{frame}
    \label{listas-agregar}
    \frametitle{Agregar cosas a una lista}
    \lstinputlisting{programas/listas-append.py}
  \end{frame}

  \begin{frame}
    \label{listas-indices}
    \frametitle{Índices}
    \lstinputlisting{programas/listas-indices.py}
  \end{frame}

  \begin{frame}
    \label{listas-modificar}
    \frametitle{Modificar elementos de una lista}
    \lstinputlisting{programas/listas-modificar.py}
  \end{frame}

  \begin{frame}
    \label{listas-operaciones}
    \frametitle{Operaciones sobre listas}
    \lstinputlisting{programas/listas-operaciones.py}
  \end{frame}

  \begin{frame}
    \label{listas-iterar}
    \frametitle{Iterar sobre listas}
    \lstinputlisting{programas/listas-iterar.py}
  \end{frame}

  \begin{frame}
    \label{listas-copiar}
    \frametitle{Copiado de listas (¡cuidado!)}
    \lstinputlisting{programas/listas-copiar.py}
  \end{frame}

  \begin{frame}
    \label{tuplas}
    \frametitle{Tuplas}
    \lstinputlisting{programas/tuplas-ejemplos.py}
  \end{frame}

  \begin{frame}
    \label{tuplas-desempaquetar}
    \frametitle{Desempaquetado de tuplas}
    \lstinputlisting{programas/tuplas-desempaquetar.py}
  \end{frame}

  \begin{frame}
    \label{tuplas-comparar}
    \frametitle{Comparación de tuplas}
    \lstinputlisting{programas/tuplas-comparar.py}
  \end{frame}

  \begin{frame}
    \label{listas-de-tuplas}
    \frametitle{Listas de tuplas}
    \lstinputlisting{programas/lista-tuplas.py}
  \end{frame}

  \begin{frame}
    \label{ejercicio-distancia}
    \frametitle{Ejercicio: distancias}
    Escriba una función \li!distancia(p1, p2)!
    que retorne la distancia entre los puntos \li!p1! y \li!p2!:
    \lstinputlisting{programas/caso-distancia.py}
  \end{frame}

  \begin{frame}
    \label{solucion-ejercicio-distancia}
    \frametitle{Solución}
    \lstinputlisting{programas/distancia.py}
  \end{frame}

  \begin{frame}
    \label{ejercicio-perimetro}
    \frametitle{Ejercicio}
    Un polígono está determinado por la lista de sus vértices.

    Escriba una función \li!perimetro(vertices)!
    que entregue el perímetro del polígono
    definido por la lista \li!vertices!:
    \lstinputlisting{programas/caso-perimetro.py}
  \end{frame}

  \begin{frame}
    \label{solucion-ejercicio-perimetro}
    \frametitle{Solución}
    \lstinputlisting{programas/perimetro.py}
  \end{frame}

  \begin{frame}
    \label{solucion-ejercicio-perimetro-2}
    \frametitle{Solución 2}
    \lstinputlisting{programas/perimetro-2.py}
  \end{frame}

\end{document}

