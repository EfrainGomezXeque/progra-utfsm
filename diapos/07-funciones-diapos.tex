\documentclass[12pt]{beamer}
\usepackage[spanish]{babel}
\usepackage[utf8]{inputenc}
\usepackage{xcolor}
\usepackage{listings}
\usepackage{textcomp}
\usepackage{mathpazo}
\usepackage{courier}
\usepackage{fancyvrb}
\usepackage{amsmath}
\usepackage{url}
\usepackage{hyperref}

\definecolor{lightbox}{HTML}{FFEEC1}
\definecolor{tablerule}{HTML}{333333}
\definecolor{novar}{HTML}{CCCCCC}
\usepackage{array}
\usepackage{booktabs}
\usepackage{colortbl}
\newcolumntype{x}{>{\ttfamily}r<{}}
\newcolumntype{z}{!{\color{tablerule}\vline}}
\newcommand{\onelinerule}{\rule[2.3ex]{0pt}{0pt}}
\newcommand{\twolinerule}{\rule[6.2ex]{0pt}{0pt}}
\newcommand{\respuesta}{\framebox[\textwidth]{\twolinerule}}
\newcommand{\nombre}{%
  \begin{tikzpicture}[xscale=.4,yscale=.7]
    \draw (0, 0) rectangle (22, 1);
  \end{tikzpicture}%
}
%\newcommand{\rol}   {\framebox[0.3\textwidth]{\onelinerule}}
\newcommand{\rol}{%
  \begin{tikzpicture}[xscale=.4,yscale=.7]
    \draw[gray!40] ( 0, 0) grid      ( 9, 1);
    \draw          ( 0, 0) rectangle ( 9, 1);
    \draw          (10, 0) rectangle (11, 1);
    \draw (9 + .2, .5) -- (10 - .2, .5);
  \end{tikzpicture}%
}
\newcommand{\li}{\lstinline}
\providecommand{\pond}[1]{[{\small\textbf{#1\%}}]}

\lstdefinelanguage{py}{%
  classoffset=0,%
    morekeywords={%
      False,class,finally,is,return,None,continue,for,lambda,try,%
      True,def,from,nonlocal,while,and,del,global,not,with,print,%
      as,elif,if,or,yield,assert,else,import,pass,break,except,in,raise},%
    keywordstyle=\color{black!80}\bfseries,%
  classoffset=1,
    morekeywords={int,float,str,abs,len,raw_input,exit,range,min,max,%
      set,dict,tuple,list,bool,complex,round,sum,all,any,zip,map,filter,%
      sorted,reversed,dir,file,frozenset,open,%
      array,zeros,ones,arange,linspace,eye,diag,dot},
    keywordstyle=\color{black!50}\bfseries,%
  classoffset=0,%
  sensitive=true,%
  morecomment=[l]\#,%
  morestring=[b]',%
  morestring=[b]",%
  stringstyle=\em,%
}

\lstdefinelanguage{testcase}{%
  moredelim=[is][\bfseries]{`}{`},%
  backgroundcolor=\color{gray!20},%
}

\lstdefinelanguage{file}{%
  frame=single,%
}

\lstset{language=py}
\lstset{basicstyle=\ttfamily}
\lstset{columns=fixed}
\lstset{upquote=true}
\lstset{showstringspaces=false}
\lstset{rangeprefix=\#\ }
\lstset{includerangemarker=false}

\newlist{certamen}{enumerate}{1}
\setlist[certamen]{%
  label=\arabic*.,
  font=\LARGE\bfseries,%
  labelindent=-.5in,%
  leftmargin=0pt,%
  labelsep=1em%
}



\usecolortheme{seahorse}
\usefonttheme{serif}

\title{Funciones}
\author{
  Programación \\ \url{http://progra.usm.cl}
}
\date{}

\begin{document}
  \begin{frame}
    \maketitle
  \end{frame}

  \begin{frame}
    \frametitle{Uso de funciones}
    \label{uso-funcion-abs}
    \begin{columns}[t]
      \column{0.5\textwidth}
        Cálculo del valor absoluto de \(x\)
        usando la definción:
      \column{0.5\textwidth}
        Cálculo del valor absoluto de \(x\)
        usando la función \li!abs!:
    \end{columns}
    \vspace{2ex}
    \begin{columns}[t]
      \column{0.5\textwidth}
        \footnotesize
        \lstinputlisting{programas/valor-absoluto-if.py}
      \column{0.5\textwidth}
        \footnotesize
        \lstinputlisting{programas/valor-absoluto-fn.py}
    \end{columns}
  \end{frame}

  \begin{frame}
    \frametitle{Uso de funciones}
    \label{uso-funcion-factorial}
    \begin{columns}[t]
      \column{0.5\textwidth}
        Cálculo del factorial \(n! = 1\cdot 2\cdot\;\cdots\;\cdot n\):
      \column{0.5\textwidth}
        Nos gustaría
        poder hacerlo así:
    \end{columns}
    \vspace{2ex}
    \begin{columns}[t]
      \column{0.5\textwidth}
        \footnotesize
        \lstinputlisting{programas/factorial-for.py}
      \column{0.5\textwidth}
        \footnotesize
        \lstinputlisting[linerange=7-9]{programas/factorial-fn.py}
    \end{columns}
  \end{frame}

  \begin{frame}
    \frametitle{Definición de funciones}
    \label{def-fn-factorial}
    \lstinputlisting{programas/factorial-fn.py}
  \end{frame}

  \begin{frame}
    \label{conceptos-funciones}
    \begin{block}{Conceptos}
      \begin{itemize}
        \item Parámetros
        \item Variables \textbf{locales} de \li!factorial!
        \item Variables \textbf{globales}
        \item Valor de retorno
      \end{itemize}
    \end{block}
    \footnotesize
    \lstinputlisting{programas/factorial-fn.py}
  \end{frame}

  \begin{frame}
    \frametitle{Ruteo}
    \label{ruteo-fn-factorial}
    \footnotesize

    \begin{columns}[t]
      \column{0.40\textwidth}
        \begin{tabular}{xzx||xzxzx}\toprule%
          \multicolumn{2}{c||}{Globales} &
          \multicolumn{3}{c}{Locales} \\
          n &  f & n & i & prod  \\ \midrule
          4 &    &  \multicolumn{3}{c}{\cellcolor{novar}} \\
            &    & 4 &   &       \\
            &    &   &   & 1     \\
            &    &   & 1 &       \\
            &    &   &   & 1     \\
            &    &   & 2 &       \\
            &    &   &   & 2     \\
            &    &   & 3 &       \\
            &    &   &   & 6     \\
            &    &   & 4 &       \\
            &    &   &   & 24    \\
            &    & \multicolumn{3}{c}{\cellcolor{novar}} \\
            & 24 & \multicolumn{3}{c}{\cellcolor{novar}} \\
          \bottomrule
        \end{tabular}

      \column{0.35\textwidth}
        \tiny
        \lstinputlisting[backgroundcolor=\color{lightbox}]%
          {programas/factorial-fn.py}
    \end{columns}
  \end{frame}

  \begin{frame}
    \label{funciones-consola}
    \frametitle{Probar funciones en la consola}
    \lstinputlisting{programas/fn-factorial-consola.py}
  \end{frame}

  \begin{frame}
    \label{ejercicio-binomial}
    \frametitle{Ejercicio: coeficiente binomial}
    Escriba un programa que calcule
    el \emph{coeficiente binomial}:
    \[
      \binom{n}{k} = \frac{n!}{(n - k)!\;k!}
    \]
    \lstinputlisting[language=testcase]{programas/caso-binomial.txt}
  \end{frame}

  \begin{frame}
    \label{solucion-binomial}
    \frametitle{Solución}
    \lstinputlisting{programas/binomial.py}
  \end{frame}

  \begin{frame}
    \label{ejercicio-contar-letras}
    \frametitle{Ejercicio: contar letras}
    Escriba la función \li!contar! para completar este programa:
    \footnotesize
    \lstinputlisting[linerange=9-14]{programas/contar-letras.py}
    \lstinputlisting[language=testcase]{programas/caso-contar-letras.txt}
  \end{frame}

  \begin{frame}
    \label{solucion-contar-letras}
    \frametitle{Solución}
    \lstinputlisting{programas/contar-letras.py}
  \end{frame}

\end{document}

