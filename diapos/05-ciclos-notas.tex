\documentclass[10pt]{article}
\usepackage{beamerarticle}
\usepackage[spanish]{babel}
\usepackage[utf8]{inputenc}
\usepackage{fullpage}
\usepackage{xcolor}
\usepackage{listings}
\usepackage{textcomp}
\usepackage{mathpazo}
\usepackage{courier}
\usepackage{fancyvrb}
\usepackage{amsmath}
\usepackage{url}
\usepackage{hyperref}
\usepackage{pgfpages}
\usepackage{wrapfig}
\usepackage{enumitem}

\setjobnamebeamerversion{05-ciclos-diapos}

\newcommand{\onelinerule}{\rule[2.3ex]{0pt}{0pt}}
\newcommand{\twolinerule}{\rule[6.2ex]{0pt}{0pt}}
\newcommand{\respuesta}{\framebox[\textwidth]{\twolinerule}}
\newcommand{\nombre}{%
  \begin{tikzpicture}[xscale=.4,yscale=.7]
    \draw (0, 0) rectangle (22, 1);
  \end{tikzpicture}%
}
%\newcommand{\rol}   {\framebox[0.3\textwidth]{\onelinerule}}
\newcommand{\rol}{%
  \begin{tikzpicture}[xscale=.4,yscale=.7]
    \draw[gray!40] ( 0, 0) grid      ( 9, 1);
    \draw          ( 0, 0) rectangle ( 9, 1);
    \draw          (10, 0) rectangle (11, 1);
    \draw (9 + .2, .5) -- (10 - .2, .5);
  \end{tikzpicture}%
}
\newcommand{\li}{\lstinline}
\providecommand{\pond}[1]{[{\small\textbf{#1\%}}]}

\lstdefinelanguage{py}{%
  classoffset=0,%
    morekeywords={%
      False,class,finally,is,return,None,continue,for,lambda,try,%
      True,def,from,nonlocal,while,and,del,global,not,with,print,%
      as,elif,if,or,yield,assert,else,import,pass,break,except,in,raise},%
    keywordstyle=\color{black!80}\bfseries,%
  classoffset=1,
    morekeywords={int,float,str,abs,len,raw_input,exit,range,min,max,%
      set,dict,tuple,list,bool,complex,round,sum,all,any,zip,map,filter,%
      sorted,reversed,dir,file,frozenset,open,%
      array,zeros,ones,arange,linspace,eye,diag,dot},
    keywordstyle=\color{black!50}\bfseries,%
  classoffset=0,%
  sensitive=true,%
  morecomment=[l]\#,%
  morestring=[b]',%
  morestring=[b]",%
  stringstyle=\em,%
}

\lstdefinelanguage{testcase}{%
  moredelim=[is][\bfseries]{`}{`},%
  backgroundcolor=\color{gray!20},%
}

\lstdefinelanguage{file}{%
  frame=single,%
}

\lstset{language=py}
\lstset{basicstyle=\ttfamily}
\lstset{columns=fixed}
\lstset{upquote=true}
\lstset{showstringspaces=false}
\lstset{rangeprefix=\#\ }
\lstset{includerangemarker=false}

\newlist{certamen}{enumerate}{1}
\setlist[certamen]{%
  label=\arabic*.,
  font=\LARGE\bfseries,%
  labelindent=-.5in,%
  leftmargin=0pt,%
  labelsep=1em%
}


\newcommand{\diapo}[1]{
  \begin{wrapfigure}{r}{.50\textwidth}
    \hfill\fbox{\includeslide[width=.45\textwidth]{#1}}
  \end{wrapfigure}
}


\title{Estructuras de repetición}
\author{Programación \\ \url{http://progra.usm.cl}}
\date{}

\begin{document}
  \maketitle

  \section*{Objetivos de la clase}
  \begin{itemize}
    \item Presentar las sentencias \li!for! y \li!while!.
    \item Enseñar el uso de ciclos y el criterio para elegir cuál usar.
    \item Introducir el uso de ruteos.
    \item Resolver ejercicios con ciclos.
  \end{itemize}

  \section*{Diapositivas}

  \diapo{problema-promedio-3-numeros}

  Este es un problema sencillo que todos los estudiantes
  deberían ser capaces de resolver hasta ahora.
  Como la cantidad de valores a leer es fija,
  es posible usar una variable para cada uno de ellos.
  En este caso, son tres.
  La solución está en la misma diapositiva.

  \diapo{problema-promedio-n-numeros}

  Lea el enunciado del problema y explique el caso de prueba.

  A diferencia del problema anterior,
  aquí no es posible tener una cantidad fija de variables,
  ni una cantidad fija de sentencias de entrada.
  Es necesario usar un ciclo
  para leer la entrada
  e ir acumulando la suma en cada iteración.

  \diapo{solucion-promedio-n-numeros}

  Aquí es introducido el \li!for! con rango.

  La variable \li!suma! acumula la suma de los valores ingresados.
  Cada valor leído es siempre asignado a una única variable \li!x!.
  En cada iteración, el valor anterior de \li!x! se pierde.

  Es indispensable inicializar la variable \li!suma! antes del ciclo.
  Si no se hace, ocurrirá un \li!NameError!
  al intentar evaluar la expresión \li!suma + x!.

  Introduzca el concepto de \emph{iteración}:
  es cada una de las veces que se ejecuta el ciclo.
  En el ejemplo del caso de prueba,
  el ciclo hace cuatro iteraciones.

  El cuerpo del ciclo se escribe indentado
  respecto a la línea del \li!for!.
  Las sentencias que están indentadas a la misma altura del \li!for!
  ya no pertenecen a él.

  La variable \li!i! es la \emph{variable de control}
  del ciclo \li!for!. En este ejemplo,
  su valor no es usado dentro del ciclo,
  por lo que su importancia no es evidente.
  Explique que, de todos modos,
  la variable \li!i! va tomando valores desde 0 hasta \(n - 1\).

  Ejecute el programa en el computador,
  para mostrar que en efecto resuelve el problema.

  El \li!for! con rango es similar al \li!for! de Pascal,
  pero no es la única manera de usarlo.
  En lugar del \li!range!, es posible poner cualquier objeto iterable.
  Por ejemplo: un string, una lista, un iterador y casi cualquier estructura de datos.
  Estos usos los veremos en la materia del segundo certamen.

  \diapo{ruteo-promedio-n-numeros}

  Éste es el primer ruteo mostrado en las diapositivas del ramo.
  El ruteo es una representación paso a paso
  del estado del programa después de cada sentencia.

  La convención para los ruteos es:
  cada columna tiene los estados de una variable,
  y cada fila del ruteo está asociada a una línea del programa.
  Si dos valores son modificados en líneas diferentes del programa,
  entonces van en filas distintas del ruteo.

  Este ruteo usa la entrada del caso de prueba de la diapositiva con el enunciado.

  A diferencia de los programas en C o Java,
  aquí la variable de control no se ``pasa de largo''.
  Si el \li!for! va de 0 a 3, entonces la variable nunca toma el valor 4.

  \diapo{ejemplos-rangos}

  La función \li!range! entrega la lista de valores incluidos en el rango
  indicado por sus parámetros.
  A pesar de que no hemos enseñando listas aún,
  los ejemplos deberían ser bastante claros.

  La regla que siempre se cumple para los rangos es:
  el valor inicial siempre es incluido,
  y el final nunca lo es.

  (La razón para preferir estos rangos semiabiertos está justificada en el
  \href{http://www.cs.utexas.edu/users/EWD/transcriptions/EWD08xx/EWD831.html}{manuscrito EWD831}
  de Dijkstra).

  La forma más usual de usar \li!range! es con un solo argumento.
  \li!range(n)! va desde 0 hasta \li!n - 1!, de uno en uno.

  Cuando tiene dos argumentos, se toman como los valores inicial y final.

  Cuando tiene tres argumentos, el tercero de ellos indica
  cada cuántos valores serán considerados.
  Este argumento se llama el \emph{paso} del rango.

  El caso en que el valor final es menor que el inicial
  no es un error, sino un rango vacío.
  Si se pone en un ciclo \li!for!,
  entonces ninguna iteración será ejecutada.
  Esto es útil en algunos algoritmos.

  \diapo{uso-variable-control}

  La variable de control toma cada uno de los valores del rango
  a medida que las iteraciones del ciclo van siendo ejecutadas.
  Estos valores pueden ser usados,
  como los de cualquier otra variable del programa.
  En este ejemplo,
  se muestra una tabla de los cuadrados
  de los números del 2 al 6.

  \diapo{problema-promedio-hasta-cierta-suma}

  Lea el enunciado del problema y explique el caso de prueba.

  En este problema,
  no sólo no es posible tener una cantidad fija de variables,
  sino que tampoco se sabe antes de comenzar el ciclo
  cuántas iteraciones serán ejecutadas.

  La lógica es la misma que en el problema anterior:
  hay que ir leyendo los valores y acumulando la suma
  dentro del ciclo.
  Lo que cambia es el control del término del ciclo:
  en este caso, hay que usar un ciclo \li!while!
  con la condición de término indicada en el enunciado.

  \diapo{solucion-promedio-hasta-cierta-suma}

  Aquí es introducido el ciclo \li!while!.

  Muchas de las observaciones sobre el ciclo \li!for!
  también se aplican para el \li!while!:
  la variable \li!suma! acumula el total,
  debe ser inicializada en cero,
  cada pasada por el ciclo se llama iteración y
  el cuerpo del ciclo se escribe indentado

  Una diferencia es que ya no existe una variable de control.
  El ciclo es ejecutado mientras la condición es verdadera.

  Como la cantidad de valores no es conocida,
  es necesario además llevar la cuenta a medida que van siendo ingresados,
  para poder dividir por el total al calcular el promedio.
  La variable \li!n! debe ser inicializada en cero,
  e incrementada en uno en cada iteración.

  La condición del \li!while! es evaluada al principio de cada iteración.
  El ciclo podría ejecutarse cero veces,
  lo que ocurre cuando la condición es falsa desde el principio.

  Ejecute el programa en el computador,
  para mostrar que en efecto resuelve el problema.

  \diapo{ruteo-promedio-hasta-cierta-suma}

  Este ruteo usa la entrada del caso de prueba de la diapositiva con el enunciado.

  \diapo{diagrama-flujo-while}

  Para mayor claridad en caso de dudas,
  el diagrama de flujo representa cómo es ejecutada
  la parte principal del programa,
  que está presentada en la mitad derecha.

  El diagrama de flujo para el \li!for! no fue incluido
  porque es más engorroso.

  Para el resto de la clase,
  puede encontrar ejercicios para resolver en
  \url{http://progra.usm.cl/apunte/ejercicios/1/index.html#ciclos}.
  En estos ejercicios, puede introducir el uso de \li!break! y \li!continue!.
  Intente hacer por lo menos un ejercicio con \li!if! dentro de un ciclo,
  y otro con ciclos anidados.

  Al hacer el resumen de la clase,
  explique el criterio para elegir \li!for! o \li!while!.
  El \li!for! se usa cuando se sabe justo antes del ciclo
  cuántas iteraciones serán ejecutadas.
  El \li!while! se usa cuando se sabe qué es lo que debe ocurrir
  para detener las iteraciones.

\end{document}

