\documentclass[12pt]{beamer}
\usepackage[spanish]{babel}
\usepackage[utf8]{inputenc}
\usepackage{xcolor}
\usepackage{listings}
\usepackage{textcomp}
\usepackage{mathpazo}
\usepackage{courier}
\usepackage{fancyvrb}
\usepackage{amsmath}
\usepackage{url}
\usepackage{hyperref}

\usepackage{tikz}
\usetikzlibrary{arrows}

\newcommand{\onelinerule}{\rule[2.3ex]{0pt}{0pt}}
\newcommand{\twolinerule}{\rule[6.2ex]{0pt}{0pt}}
\newcommand{\respuesta}{\framebox[\textwidth]{\twolinerule}}
\newcommand{\nombre}{%
  \begin{tikzpicture}[xscale=.4,yscale=.7]
    \draw (0, 0) rectangle (22, 1);
  \end{tikzpicture}%
}
%\newcommand{\rol}   {\framebox[0.3\textwidth]{\onelinerule}}
\newcommand{\rol}{%
  \begin{tikzpicture}[xscale=.4,yscale=.7]
    \draw[gray!40] ( 0, 0) grid      ( 9, 1);
    \draw          ( 0, 0) rectangle ( 9, 1);
    \draw          (10, 0) rectangle (11, 1);
    \draw (9 + .2, .5) -- (10 - .2, .5);
  \end{tikzpicture}%
}
\newcommand{\li}{\lstinline}
\providecommand{\pond}[1]{[{\small\textbf{#1\%}}]}

\lstdefinelanguage{py}{%
  classoffset=0,%
    morekeywords={%
      False,class,finally,is,return,None,continue,for,lambda,try,%
      True,def,from,nonlocal,while,and,del,global,not,with,print,%
      as,elif,if,or,yield,assert,else,import,pass,break,except,in,raise},%
    keywordstyle=\color{black!80}\bfseries,%
  classoffset=1,
    morekeywords={int,float,str,abs,len,raw_input,exit,range,min,max,%
      set,dict,tuple,list,bool,complex,round,sum,all,any,zip,map,filter,%
      sorted,reversed,dir,file,frozenset,open,%
      array,zeros,ones,arange,linspace,eye,diag,dot},
    keywordstyle=\color{black!50}\bfseries,%
  classoffset=0,%
  sensitive=true,%
  morecomment=[l]\#,%
  morestring=[b]',%
  morestring=[b]",%
  stringstyle=\em,%
}

\lstdefinelanguage{testcase}{%
  moredelim=[is][\bfseries]{`}{`},%
  backgroundcolor=\color{gray!20},%
}

\lstdefinelanguage{file}{%
  frame=single,%
}

\lstset{language=py}
\lstset{basicstyle=\ttfamily}
\lstset{columns=fixed}
\lstset{upquote=true}
\lstset{showstringspaces=false}
\lstset{rangeprefix=\#\ }
\lstset{includerangemarker=false}

\newlist{certamen}{enumerate}{1}
\setlist[certamen]{%
  label=\arabic*.,
  font=\LARGE\bfseries,%
  labelindent=-.5in,%
  leftmargin=0pt,%
  labelsep=1em%
}



\usecolortheme{crane}
\usefonttheme{serif}

\title{Programas simples}
\author{Programación \\ \url{http://progra.usm.cl}}
\date{14 y 15 de marzo de 2011}

\begin{document}
  \begin{frame}
    \maketitle
  \end{frame}

  \begin{frame}
    \label{problema-gravitacion}
    \frametitle{Problema}
    La fuerza de atracción gravitacional
    entre dos planetas de masas \(m_1\) y \(m_2\)
    separados por una distancia de \(r\) kilómetros
    está dada por la fórmula:
    \[ F = G\frac{m_1 m_2}{r^2}, \]
    donde
    \(G = 6.67428\cdot 10^{-11} [\text{m}^3\,\text{kg}^{-1}\,\text{s}^{-2}]\)
    es la constante de gravitación universal.

    Escriba un programa que pregunte las masas de los planetas y su distancia,
    y entregue la fuerza de atracción entre ellos.
  \end{frame}

  \begin{frame}
    \label{gravitacion}
    \frametitle{Solución}
    \lstinputlisting{programas/gravitacion.py}
  \end{frame}

  \begin{frame}
    \label{problema-serrucho}
    \frametitle{Problema}
    Escriba un programa que pida al usuario un valor real \(x\),
    y entregue \(y = s(x)\),
    donde \(s(x)\) es la \emph{función serrucho}:

    \begin{center}
      \begin{tikzpicture}[scale=.8]
        \draw[gray] (0, 0) grid (8.5, 3.5);
        \draw[->, thick] (0, 0) -- (8.5, 0);
        \draw[->, thick] (0, 0) -- (0, 3.5);
        \foreach\x in {0,2,...,8} \draw[*-o, red, thick] (\x, 0) -- (\x + 2, 3);
        \foreach\x in {0,...,8} \node at (\x, -.3) {\footnotesize \x};
        \foreach\y in {1,...,3} \node at (-.3, \y) {\footnotesize \y};
      \end{tikzpicture}
    \end{center}

    \lstinputlisting[language=testcase]{programas/caso-serrucho.txt}
  \end{frame}

  \begin{frame}
    \label{serrucho}
    \frametitle{Dos soluciones}
    \lstinputlisting{programas/serrucho.py}
  \end{frame}

  \begin{frame}
    \label{problema-rimas}
    \frametitle{Problema}
    Dos palabras riman si sus cuatro últimas letras son iguales.

    Escriba un programa que pregunte dos palabras,
    y muestre \li!True! o \li!False! dependiendo si riman o no.

    \lstinputlisting[language=testcase]{programas/caso-1-rima.txt}
    \lstinputlisting[language=testcase]{programas/caso-2-rima.txt}
  \end{frame}

  \begin{frame}
    \label{rima1}
    \frametitle{Primera solución}
    \lstinputlisting{programas/rima1.py}
  \end{frame}

  \begin{frame}
    \label{rima2}
    \frametitle{Solución con truco}
    \lstinputlisting{programas/rima2.py}
  \end{frame}

  \begin{frame}
    \label{rima3}
    \frametitle{Solución con más truco}
    \lstinputlisting{programas/rima3.py}
  \end{frame}

  \begin{frame}
    \label{problema-tazas}
    \frametitle{Problema}
    En una mesa hay 23 tazas formando un círculo.
    Las tazas 4 y 18 tienen una moneda en su interior.

    La moneda de la taza 4 es movida
    \(i\) posiciones en el sentido de las agujas del reloj,
    y la de la taza 18, \(j\) posiciones en el sentido contrario.

    Escriba un programa que pida al usuario ingresar \(i\) y \(j\),
    e imprima \li!True! o \li!False! dependiendo de si las monedas
    quedaron o no en la misma taza.

    \lstinputlisting[language=testcase]{programas/caso-tazas.txt}
  \end{frame}

  \begin{frame}
    \label{diagrama-tazas}
    \begin{center}
      \begin{tikzpicture}[rotate=90]
        \def\A{{-360/23}}
        \foreach\i in {0,...,22}
          \node[draw, fill=blue!20, circle, minimum size=2em] at (\i * \A:4cm) {\i};
        \foreach\i in {4, 18}
          \node[draw, fill=yellow, circle, minimum size=1em] at (\i * \A:4cm) {};
        \draw[thick, ->] ( 4 * \A:3cm) arc ( 4 * \A: 7.5 * \A:3cm);
        \draw[thick, ->] (18 * \A:3cm) arc (18 * \A: 12.5 * \A:3cm);
        \node at ( 7 * \A : 2.6cm) {\(i\)};
        \node at (12 * \A : 2.6cm) {\(j\)};
      \end{tikzpicture}
    \end{center}
  \end{frame}

  \begin{frame}
    \label{tazas}
    \frametitle{Solución}
    \lstinputlisting{programas/tazas.py}
  \end{frame}

\end{document}

