\documentclass[10pt]{article}
\usepackage{beamerarticle}
\usepackage[spanish]{babel}
\usepackage[utf8]{inputenc}
\usepackage{fullpage}
\usepackage{xcolor}
\usepackage{listings}
\usepackage{textcomp}
\usepackage{mathpazo}
\usepackage{courier}
\usepackage{fancyvrb}
\usepackage{amsmath}
\usepackage{url}
\usepackage{hyperref}
\usepackage{pgfpages}
\usepackage{wrapfig}
\usepackage{enumitem}

\setjobnamebeamerversion{06-patrones-comunes-diapos}

\newcommand{\onelinerule}{\rule[2.3ex]{0pt}{0pt}}
\newcommand{\twolinerule}{\rule[6.2ex]{0pt}{0pt}}
\newcommand{\respuesta}{\framebox[\textwidth]{\twolinerule}}
\newcommand{\nombre}{%
  \begin{tikzpicture}[xscale=.4,yscale=.7]
    \draw (0, 0) rectangle (22, 1);
  \end{tikzpicture}%
}
%\newcommand{\rol}   {\framebox[0.3\textwidth]{\onelinerule}}
\newcommand{\rol}{%
  \begin{tikzpicture}[xscale=.4,yscale=.7]
    \draw[gray!40] ( 0, 0) grid      ( 9, 1);
    \draw          ( 0, 0) rectangle ( 9, 1);
    \draw          (10, 0) rectangle (11, 1);
    \draw (9 + .2, .5) -- (10 - .2, .5);
  \end{tikzpicture}%
}
\newcommand{\li}{\lstinline}
\providecommand{\pond}[1]{[{\small\textbf{#1\%}}]}

\lstdefinelanguage{py}{%
  classoffset=0,%
    morekeywords={%
      False,class,finally,is,return,None,continue,for,lambda,try,%
      True,def,from,nonlocal,while,and,del,global,not,with,print,%
      as,elif,if,or,yield,assert,else,import,pass,break,except,in,raise},%
    keywordstyle=\color{black!80}\bfseries,%
  classoffset=1,
    morekeywords={int,float,str,abs,len,raw_input,exit,range,min,max,%
      set,dict,tuple,list,bool,complex,round,sum,all,any,zip,map,filter,%
      sorted,reversed,dir,file,frozenset,open,%
      array,zeros,ones,arange,linspace,eye,diag,dot},
    keywordstyle=\color{black!50}\bfseries,%
  classoffset=0,%
  sensitive=true,%
  morecomment=[l]\#,%
  morestring=[b]',%
  morestring=[b]",%
  stringstyle=\em,%
}

\lstdefinelanguage{testcase}{%
  moredelim=[is][\bfseries]{`}{`},%
  backgroundcolor=\color{gray!20},%
}

\lstdefinelanguage{file}{%
  frame=single,%
}

\lstset{language=py}
\lstset{basicstyle=\ttfamily}
\lstset{columns=fixed}
\lstset{upquote=true}
\lstset{showstringspaces=false}
\lstset{rangeprefix=\#\ }
\lstset{includerangemarker=false}

\newlist{certamen}{enumerate}{1}
\setlist[certamen]{%
  label=\arabic*.,
  font=\LARGE\bfseries,%
  labelindent=-.5in,%
  leftmargin=0pt,%
  labelsep=1em%
}


\newcommand{\diapo}[1]{
  \begin{wrapfigure}{r}{.50\textwidth}
    \hfill\fbox{\includeslide[width=.45\textwidth]{#1}}
  \end{wrapfigure}
}


\title{Patrones comunes}
\author{Programación \\ \url{http://progra.usm.cl}}
\date{}

\begin{document}
  \maketitle

  \section*{Objetivos de la clase}
  Enseñar los siguientes patrones típicos de programación:
  \begin{itemize}
    \item sumar cosas,
    \item multiplicar cosas,
    \item contar cosas,
    \item encontrar el máximo y el mínimo,
    \item generar pares de elementos.
  \end{itemize}

  \section*{Diapositivas}

  \diapo{problema-suma-cuadrados-menores-que-mil}
  \diapo{solucion-suma-cuadrados-menores-que-mil}

  Éste es el primero de dos ejercicios
  que consisten en sumar cuadrados de números.

  Como todos los programas en que hay que sumar elementos en un ciclo,
  éste usa una variable inicializada en cero para ir acumulando el resultado.

  Rutee unas pocas iteraciones para demostrar que en efecto
  la variable va acumulando el resultado deseado.

  En las diapositivas se utiliza la notación tradicional
  para aumentar el valor de una variable:
  \li!suma = suma + n!.
  En Python también se puede usar la \emph{asignación aumentada}:
  \li!suma += n!.
  Puede introducir esta notación al resolver problemas en el computador durante la clase.

  Olvidar inicializar la variable acumuladora es un error común.
  Enfatice que hacerlo es imprescindible.
  Si no es inicializada, el programa arrojará un error de valor en la primera iteración.

  \diapo{problema-suma-cuadrados-hasta-cero}
  \diapo{solucion-suma-cuadrados-hasta-cero}

  Éste es el segundo ejercicio de sumar cuadrados de números.
  Tiene dos diferencias importantes con el anterior:
  \begin{itemize}
    \item el número de iteraciones no es conocido de antemano,
      por lo que hay que usar un ciclo \li!while!, y
    \item el valor de \li!n! es leído desde la entrada estándar
      dentro del ciclo.
  \end{itemize}
  
  El uso de \li!while! infinito con \li!break! es necesario
  ya que la variable \li!n! no ha sido creada
  cuando la condición del \li!while! es evaluada por primera vez.
  Una alternativa sería inicializar \li!n! con cualquier valor distinto de cero
  antes del ciclo, y poner la condición \li+n != 0+ en el \li!while!.

  Discuta con los estudiantes qué es lo que tienen en común ambos programas,
  que es lo que se presentará a continuación como el patrón \emph{sumar cosas}.

  \diapo{patron-sumar}

  Explicación en pseudocódigo del patrón \emph{sumar cosas}.
  Aclare que se trata de pseudocódigo,
  por lo que \li!ciclo! y \li!calcular!  no deben entenderse literalmente,
  sino que deben ser adaptados dependiendo del problema.

  \diapo{patron-multiplicar}
  \diapo{patron-contar}

  Dos patrones relacionados son multiplicar y contar.
  Puede preguntar a los estudiantes si se les ocurre como hacerlos
  antes de mostrar las diapositivas.
  En ambos casos, lo que cambia es el valor inicial del acumulador
  y la actualización de él que es realizada al final del ciclo.

  Como comentario al margen,
  estos tres patrones son casos especiales
  de las \href{http://en.wikipedia.org/wiki/Reduce_(higher-order_function)}{reducciones} de programación funcional,
  que también pueden ser aplicadas en Python:
  \begin{lstlisting}
>>> from operator import add, mul
>>> reduce(add, [3, 4, 5], 0)
12
>>> reduce(mul, [3, 4, 5], 1)
60
  \end{lstlisting}

  \diapo{problema-maximo-positivos}
  \diapo{solucion-1-maximo-positivos}
  \diapo{solucion-2-maximo-positivos}

  Éste es el primero de dos ejercicios
  acerca de encontrar el máximo de una secuencia de valores.

  La estrategia para encontrar el máximo es:
  en cada iteración, recordar cuál es el máximo hasta el momento,
  y actualizarlo si es que se encuentra un valor que sea mayor a él.

  Algunos alumnos pueden sugerir usar la función \li!max!.
  Esto no es conveniente, ya que requiere recibir todos los valores como argumentos.
  Además, no es posible usarla si la cantidad de valores no es sabida de antemano,
  a menos que se utilice listas, que aún no han sido enseñadas.

  Para encontrar el mayor,
  el acumulador debe ser inicializado en un valor que sea menor
  a todos los posibles valores que serán examinados.
  El valor inicial que se elija
  de todos modos se perderá después de la primera iteración.
  Como en este caso los números son positivos,
  se ha elegido el valor \(-1\).

  Otra estrategia posible de inicialización
  es usar el primero de los valores por examinar.
  En las diapositivas no se la utiliza,
  pero sí está explicada en el apunte.

  La diferencia entre las dos soluciones mostradas
  es la actualización del valor mayor:
  en la primera, se usa el \li!if! para decidir si actualizarlo o no,
  mientras que la segunda tiene una expresión más breve.
  La primera es la forma «à la C», pues este lenguaje no tiene una función \li!max!.

  \diapo{problema-maximo-reales}
  \diapo{solucion-maximo-reales}

  Segundo ejercicio de encontrar el máximo.
  Tiene dos diferencias importantes con el anterior:
  \begin{itemize}
    \item la cantidad de valores no es conocida,
      por lo que hay que usar un ciclo \li!while!, y
    \item los valores pueden ser negativos,
      por lo que ya no sirve inicializar el acumulador en \(-1\).
  \end{itemize}

  La mala práctica suele ser inicializar el acumulador
  en un valor «muy negativo», como \(-999999\).
  Esta idea es poco robusta, pues no cubre todos los casos.
  
  El tipo \li!float! permite representar el valor infinito
  (no sólo en Python, sino en cualquier lenguaje que soporte el tipo \li!float! estándar).
  Por lo tanto,
  la mejor estrategia es usar \(-\infty\),
  representado como \li!-float('inf')! en Python,
  para inicializar el acumulador.
  
  \diapo{patron-mayor}
  \diapo{patron-menor}

  Explicación en pseudocódigo de los patrones \emph{encontrar mayor} y \emph{encontrar menor}.
  Después de mostrar el primero,
  puede preguntar a los estudiantes si se les ocurre qué hay que modificar
  para obtener el segundo.
  
  Un ejercicio para ver si entendieron puede ser:
  ¿cómo inicializar las variables
  si se desea encontrar el mínimo y el máximo
  de notas de certámenes de la UTFSM?
  (Respuesta: \(101\) y \(-1\)).

  \diapo{problema-combinaciones-dados}
  \diapo{solucion-combinaciones-dados}

  Ejercicio sobre generación de pares de valores.
  La técnica es usar dos ciclos anidados
  que vayan generando los pares.
  Dentro del ciclo interior,
  ambas variables están disponibles para ser usadas.

  Un error típico es equivocadamente usar la misma variable de control en ambos ciclos \li!for!.
  Esto es más común en C o Java,
  donde usualmente se debe repetir tres veces el nombre de la variable.

  \diapo{patron-pares}

  Explicación en pseudocódigo.
  Este esquema debe ser modificado
  para problemas en que los pares son desordenados (por ejemplo, fichas de dominó),
  o cuando se debe excluir pares de valores repetidos (por ejemplo, parejas de personas).
  Estas situaciones están explicadas en el apunte.

\end{document}

