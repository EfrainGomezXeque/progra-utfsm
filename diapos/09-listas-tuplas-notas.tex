\documentclass[10pt]{article}
\usepackage{beamerarticle}
\usepackage[spanish]{babel}
\usepackage[utf8]{inputenc}
\usepackage{fullpage}
\usepackage{xcolor}
\usepackage{listings}
\usepackage{textcomp}
\usepackage{mathpazo}
\usepackage{courier}
\usepackage{fancyvrb}
\usepackage{amsmath}
\usepackage{url}
\usepackage{hyperref}
\usepackage{pgfpages}
\usepackage{wrapfig}

\setjobnamebeamerversion{12-diapos}

\newcommand{\onelinerule}{\rule[2.3ex]{0pt}{0pt}}
\newcommand{\twolinerule}{\rule[6.2ex]{0pt}{0pt}}
\newcommand{\respuesta}{\framebox[\textwidth]{\twolinerule}}
\newcommand{\nombre}{%
  \begin{tikzpicture}[xscale=.4,yscale=.7]
    \draw (0, 0) rectangle (22, 1);
  \end{tikzpicture}%
}
%\newcommand{\rol}   {\framebox[0.3\textwidth]{\onelinerule}}
\newcommand{\rol}{%
  \begin{tikzpicture}[xscale=.4,yscale=.7]
    \draw[gray!40] ( 0, 0) grid      ( 9, 1);
    \draw          ( 0, 0) rectangle ( 9, 1);
    \draw          (10, 0) rectangle (11, 1);
    \draw (9 + .2, .5) -- (10 - .2, .5);
  \end{tikzpicture}%
}
\newcommand{\li}{\lstinline}
\providecommand{\pond}[1]{[{\small\textbf{#1\%}}]}

\lstdefinelanguage{py}{%
  classoffset=0,%
    morekeywords={%
      False,class,finally,is,return,None,continue,for,lambda,try,%
      True,def,from,nonlocal,while,and,del,global,not,with,print,%
      as,elif,if,or,yield,assert,else,import,pass,break,except,in,raise},%
    keywordstyle=\color{black!80}\bfseries,%
  classoffset=1,
    morekeywords={int,float,str,abs,len,raw_input,exit,range,min,max,%
      set,dict,tuple,list,bool,complex,round,sum,all,any,zip,map,filter,%
      sorted,reversed,dir,file,frozenset,open,%
      array,zeros,ones,arange,linspace,eye,diag,dot},
    keywordstyle=\color{black!50}\bfseries,%
  classoffset=0,%
  sensitive=true,%
  morecomment=[l]\#,%
  morestring=[b]',%
  morestring=[b]",%
  stringstyle=\em,%
}

\lstdefinelanguage{testcase}{%
  moredelim=[is][\bfseries]{`}{`},%
  backgroundcolor=\color{gray!20},%
}

\lstdefinelanguage{file}{%
  frame=single,%
}

\lstset{language=py}
\lstset{basicstyle=\ttfamily}
\lstset{columns=fixed}
\lstset{upquote=true}
\lstset{showstringspaces=false}
\lstset{rangeprefix=\#\ }
\lstset{includerangemarker=false}

\newlist{certamen}{enumerate}{1}
\setlist[certamen]{%
  label=\arabic*.,
  font=\LARGE\bfseries,%
  labelindent=-.5in,%
  leftmargin=0pt,%
  labelsep=1em%
}


\newcommand{\diapo}[1]{
  \begin{wrapfigure}{r}{.50\textwidth}
    \hfill\fbox{\includeslide[width=.45\textwidth]{#1}}
  \end{wrapfigure}
}


\title{Listas y tuplas}
\author{Programación \\ \url{http://progra.usm.cl}}
\date{}

\begin{document}
  \maketitle

  \section*{Objetivos de la clase}
  \begin{itemize}
    \item Enseñar el uso de listas y tuplas en Python.
    \item Justificar por qué es necesario
      usar listas y tuplas.
  \end{itemize}

  \section*{Diapositivas}

  \diapo{def-desviacion}

  Para introducir el uso de listas,
  usaremos como ejemplo el cálculo de la desviación estándar.
  Es un concepto que los estudiantes de primer año
  ya han visto en el curso de física.

  Puede explicar el concepto de desviación estándar,
  para que no parezca que se trata sólo de evaluar una fórmula.
  La desviación estándar indica cuán dispersos están los datos.
  Por ejemplo,
  un alumno que aprueba con promedio 55 puede lograrlo
  sacándose puros 55 (en este caso, la desviación es cero),
  o sacándose un 10, un 55 y un 100 (en este caso, la desviación es 45).
  El promedio no dice cuánto variaron las notas,
  y para eso se usa la desviación estándar.

  En la primera diapositiva,
  hay un ejemplo de cálculo de la desviación
  con tres datos.
  Explique paso a paso cómo se calcula,
  para que los estudiantes se familiaricen con el procedimiento.
  En palabras,
  lo que hay que hacer es lo siguiente:
  calcular el promedio,
  restar el promedio a cada uno de los datos,
  sumar los cuadrados de la diferencia,
  dividir por \(n - 1\) y sacar la raíz cuadrada.

  Quizás algunos alumnos hayan visto antes
  la fórmula de la desviación dividiendo por \(n\) en lugar de \(n - 1\),
  y lo mencionen en clases.
  Ambas fórmulas pueden ser usadas,
  pero hay razones estadísticas para preferir la fórmula con \(n - 1\).

  \diapo{ejercicio-desviacion}

  El ejercicio es un programa que recibe una cantidad variable de datos,
  y entrega la desviación estándar.
  Discuta con los alumnos cómo hacerlo.

  La dificultad radica en que hay que hacer dos recorridos por los datos:
  uno para calcular el promedio, y otro para sumar las diferencias al cuadrado.
  No se pueden fusionar en un único ciclo,
  ya que el segundo necesita el resultado del primero.

  El promedio se puede calcular a medida que se van leyendo los datos,
  pero no se puede hacer lo mismo con la desviación,
  ya que habría que volver a preguntar los datos.
  La alternativa es guardar los datos en variables,
  pero no sabemos de antemano cuántos datos son,
  por lo que no es posible hacerlo.

  Ésta es la motivación para introducir las listas.
  Si se tienen todos los datos en una lista,
  se puede hacer con ellos cualquier cosa
  todas las veces que se quiera,
  y refiriéndose a ellos mediante una única variable.

  \diapo{programa-desviacion}

  Programa para calcular la desviación estándar.
  Explique los pasos del programa uno por uno.
  Está organizado en cinco «párrafos»:
  \begin{itemize}
    \item Primero, se lee la cantidad de datos \li!n!.
    \item Se piden los \li!n! datos y se van metiendo a la lista \li!datos!.
      Las operaciones nuevas aquí son la creación de la lista vacía (\li![]!)
      y el método para meter el dato a la lista (\li!append!).
      Por ahora, mencione qué significa cada línea.
      Después explicaremos qué es un método.
    \item Se suman los datos, recorriendo la lista.
      Lo nuevo acá es usar el \li!for! sin \li!range!,
      para ir tomando los elementos uno por uno.
    \item Se recorre nuevamente la lista,
      ahora para ir sumando los cuadrados de las diferencias.
    \item Finalmente se muestra el resultado.
  \end{itemize}

  \diapo{programa-desviacion-funciones}

  Otra manera de resolver el mismo problema usando funciones.
  La función \li!leer_datos(n)! pide \li!n! datos al usuario y retorna la lista de ellos.
  Las funciones \li!promedio! y \li!desviacion! reciben una lista
  y entregan respectivamente el promedio y la desviación estándar.

  Como siempre que se usa funciones,
  haga alarde de lo breve y fácil de entender (excepto por el tamaño de la letra)
  que queda la parte principal del programa.

  \diapo{listas-agregar}

  Éste es el patrón típico para agregar valores dinámicamente a una lista:
  se crea la lista vacía y luego se meten los valores uno por uno
  usando el método \li!append!.

  Ésta es la primera vez que los alumnos ven métodos en el ramo.
  Una manera sencilla de explicar lo que es un método es la siguiente:
  un método es una función que pertenece a la lista.
  Cada lista tiene su propio método \li!append!,
  que afecta a esa lista en particular.
  Esta explicación no es del todo correcta,
  pero para los propósitos del ramo es suficientemente precisa.

  Los métodos, al igual que las funciones,
  pueden retornar un valor o no hacerlo.
  Esto último es más común que en las funciones,
  ya que los métodos pueden tener como única misión
  modificar el estado interno del objeto.
  El método \li!append! es un ejemplo de esto.

  En C y Pascal,
  la manera de almacenar \(n\) valores es declarando un arreglo de tamaño \(n\) (o más)
  y luego ir reemplazando los valores uno por uno.
  En Python,
  hacerlo así no es una buena práctica;
  si se necesita una lista de 10 valores,
  no hay que partir llenándola con diez ceros y luego reemplazarlos.

  Internamente,
  las listas de Python son arreglos de C.
  La memoria no es agregada después de cada \li!append!,
  sino que siempre hay memoria reservada de sobra.
  Cuando la capacidad es excedida,
  dinámicamente se reserva otro arreglo más grande y se copian los datos.
  El algoritmo está bien optimizado,
  por lo que para ejemplos razonablemente simples
  no hay degradación notoria de rendimiento.

  La lista vacía también puede ser creada
  llamando a la función \li!list()! sin ningún argumento.

  Las listas son el primer tipo de datos \emph{mutable} que vemos.
  Los tipos que hemos enseñado hasta ahora (strings, números, booleanos)
  son inmutables: sus valores nunca cambian (el número 8 no puede transformarse en el número 9).
  Este nuevo comportamiento puede causar confusión entre los estudiantes.

  \diapo{listas-indices}

  Otra manera de crear una lista
  es poniendo todos los valores iniciales entre corchetes
  y separados por comas.
  Esto es ideal para cuando se conocen de antemano los valores.

  Al igual que los strings y que los arreglos de otros lenguajes,
  los elementos de una lista se pueden obtener individualmente
  a través de su índice.
  Los índices parten desde cero.

  Si en vez de índice se pone un rango (como \li!1:3! en el ejemplo),
  no se obtiene un elemento de la lista, sino una 
  Esta operación se llama \emph{rebanado} (en inglés: \emph{slice}).
  Los índices del rebanado no se refieren a los elementos,
  sino a los espacios entre elementos:
  \begin{verbatim}
    +------------+------------+------------+------------+
    | 'azul'     | 'rojo'     | 'verde'    | 'amarillo' |
    +------------+------------+------------+------------+
    0            1            2            3            4\end{verbatim}
  Por esto, \li!colores[1:3]! no incluye el elemento de índice \li!3!.

  Los índices negativos sirven para referirse a los elementos
  por su posición desde el final hacia atrás, partiendo de 1.

  El método \li!index! opera al revés que el operador \li![]!:
  a partir del valor entrega el índice.
  Éste es un ejemplo de método que retorna algo, a diferencia del \li!append!.

  La función \li!len! entrega el largo de la lista,
  al igual que con los strings.

  \diapo{listas-modificar}

  Las listas pueden modificarse elemento por elemento.
  Para modificar un elemento refiriéndose a él por su índice,
  basta con asignarle un nuevo valor.
  Para eliminar un elemento refiriéndose a él por su índice,
  se usa la sentencia \li!del!.

  El método \li!insert! permite insertar un elemento
  en cualquier posición,
  en contraste con \li!append! que siempre agrega al final.

  La asignación \li!colores[0] = 'fucsia'!
  técnicamente no es una asignación
  (las asignaciones en Python no pueden modificar un objeto),
  sino que es una conveniencia sintáctica
  que se traduce por debajo a una llamada a un método.
  Por supuesto, esto no hay que mencionarlo en clases,
  pero es bueno saberlo para entender algunos casos rebuscados que puedan surgir
  como consecuencia de cómo funcionan las asignaciones en Python
  (que es diferente a C y Pascal).

  \diapo{listas-operaciones}

  Operaciones muy convenientes sobre listas:
  \li!len! entrega el número de elementos,
  \li!sum! suma los elementos,
  el método \li!sort! ordena la lista,
  el método \li!reverse! invierte el orden de los elementos,
  y el operador \li!in! permite saber si un elemento está o no en la lista.

  También hay un operador \li!not in! que entrega el resultado de \li!in! negado.

  Los métodos \li!sort! y \li!reverse! no retornan nada,
  sino que modifican la lista en su lugar.
  Un error común es hacer algo como \li!ordenada = desordenada.sort()!.

  \diapo{listas-iterar}

  El tipo \li!list! es \emph{iterable}.
  En la práctica, esto significa que se puede poner una lista en un ciclo \li!for!
  para obtener sus valores uno por uno. De hecho, la función \li!range! que usamos hasta ahora
  retorna una lista.

  En el primer ejemplo,
  para cada iteración del ciclo la variable de control toma como valor
  uno de los elementos de la lista.

  El segundo ejemplo muestra cómo recorrer listas \emph{à la Pascal},
  refiriéndose a cada elemento por su índice.
  En Python hay que evitar hacerlo de este modo siempre que sea posible.
  En este ejemplo, se hace para poder recorrer las dos listas en paralelo.

  Otra manera de recorrer listas en paralelo
  es usando la función \li!zip!:
  \begin{lstlisting}
    for ramo, hora in zip(ramos, horas):
        print 'Tengo', ramo, 'a las', hora
  \end{lstlisting}
  No es necesario que mencione esto.

  \diapo{listas-copiar}

  Aquí hay un ejemplo en el que la semántica de las asignaciones de Python
  nos juega una mala pasada cuando pensamos como en C o Pascal.
  En el primer ejemplo,
  se crea una lista \li!a!, y se la asigna al nombre \li!b!.
  La asignación no crea una copia de la lista,
  sino que crea un nombre que se refiere a la lista original.
  Por lo tanto, al modificar \li!b! también se modifica \li!a!.

  Una manera de copiar los elementos a una nueva lista
  es llamar a la función \li!list!
  pasando la lista original como argumento.

  Otra manera de obtener una copia de la lista
  es hacer un rebanado con todos los elementos: \li!b = a[:]!.
  Esta forma es un poco más críptica,
  pero aparece muy frecuentemente en libros y documentación en la red.
  Puede omitir mencionarla en clases.

  \diapo{tuplas}

  La segunda materia de la clase es el uso de \emph{tuplas}.
  Las tuplas sirven para empaquetar valores
  que por su naturaleza deben ir juntos.
  Mencione esta definición a los alumnos.

  Las tuplas sirven para el mismo propósito
  que los registros de Pascal o las estructuras de C,
  pero la manera de usarlas es diferente.
  Los elementos de las tuplas se acceden por posición
  y no por nombre.

  Las tuplas son inmutables:
  una vez creada, una tupla no se puede modificar.

  Los ejemplos de la diapositiva son los siguientes.
  \begin{itemize}
    \item La posición de un alfil en un tablero de ajedrez
      puede representarse empaquetando su fila y su columna.
    \item Un alumno puede representarse
      empaquetando su nombre, su apellido y su rol.
    \item En un juego de naipes,
      una carta está representada por su valor y su palo.
    \item Una fecha se representa empaquetando
      el año, el mes y el día.
    \item Una tupla puede estar compuesta de otras tuplas.
      Por ejemplo, un triángulo está compuesto de tres puntos,
      que están compuestos de dos coordenadas.
    \item Otro ejemplo de tuplas anidadas:
      un personaje histórico tiene fechas de nacimiento y defunción.
  \end{itemize}

  Los paréntesis no son necesarios para crear una tupla,
  basta con las comas.

  La principal motivación para usar tuplas
  es poder comunicar varios valores empaquetados entre funciones.
  Esto evita tener que manejar todos los valores por separado.

  \diapo{tuplas-desempaquetar}

  Para desempaquetar los valores de una tupla,
  se puede usar asignaciones con varios nombres.
  En el segundo ejemplo,
  se desempaqueta primero el personaje y luego las fechas.

  También es posible obtener los valores a través de su índice.
  Por ejemplo, el año de nacimiento es \li!personaje[1][0]!.

  \diapo{tuplas-comparar}

  Las tuplas se comparan usando el algoritmo de orden lexicográfico.
  El orden de las tuplas está determinado por el primer elemento
  que es diferente entre ambas tuplas.

  En el primer ejemplo,
  el orden está determinado por el primer elemento (\li!4 < 7!),
  y en el segundo ejemplo,
  por el segundo elemento (\li!5 > 2!).
  Cuando los primeros elementos son todos iguales,
  la tupla más corta es la menor.

  Esto es muy conveniente para comparar fechas.
  Si en la tupla se ponen los campos en el orden año-mes-día,
  entonces se puede saber qué fecha ocurrió antes o después
  simplemente usando los operadores relacionales.

  \diapo{listas-de-tuplas}

  Las listas de tuplas son el equivalente a los arreglos de registros.
  En el ejemplo de la diapositiva,
  se tiene una lista de ramos representados por su nombre, su sigla y su número de créditos.

  Al iterar sobre un contenedor de tuplas,
  se puede desempaquetar los valores directamente en el ciclo \li!for!,
  como se muestra en el ejemplo.

  \diapo{ejercicio-distancia}
  \diapo{solucion-ejercicio-distancia}

  Ejercicio con tuplas. La función debe recibir como parámetros
  dos tuplas de tres elementos.

  Repase la materia de funciones mientras explica el ejercicio.
  Enfatice que la función debe retornar el resultado, y no imprimirlo.
  Esta es una confusión recurrente entre los estudiantes.
  La función tampoco debe solicitar la entrada del programa,
  sino recibir todos sus datos como parámetros.

  La solución no tiene ciencia:
  sólo hay que desempaquetar los valores con cuidado.

  \diapo{ejercicio-perimetro}
  \diapo{solucion-ejercicio-perimetro}
  \diapo{solucion-ejercicio-perimetro-2}

  Ejercicio con lista de tuplas.
  Si le queda tiempo para resolver este problema,
  explique el enunciado en detalle.

  Para resolverlo de manera conveniente,
  hay que crear una función \li!distancia!
  como la del problema anterior.
  En este problema, los puntos tienen dos dimensiones en vez de tres,
  pero el código es prácticamente el mismo.
  En la solución,
  se ha usado la sintaxis del desempaquetado para asignar
  varios valores a la vez, por lo que el código tiene menos líneas.

  Aquí se presenta dos maneras de escribir la función \li!perimetro!.
  En la primera, se recorre la lista usando el índice,
  y en cada iteración se calcula la distancia con el vértice siguiente.
  Explique cuál es la razón para usar el operador módulo.

  En la segunda solución,
  se ha creado tres listas dentro de la función:
  \begin{itemize}
    \item una que tiene todos los vértices,
    \item otra que tiene los mismos vértices, pero moviendo el primero de ellos al final, y
    \item otra con las distancias entre los elementos correspondientes de las dos anteriores.
  \end{itemize}
  Aquí aparece el operador \li!+! usado para concatenar dos listas.
  Note que para usar este operador se requiere que ambos operandos sean listas,
  por lo que es necesario poner los corchetes alrededor de \li!vertices[0]!
  para convertirlo en una lista de un elemento.
  Con el rebanado no es necesario, ya que esta operación sí retorna una lista.

\end{document}

