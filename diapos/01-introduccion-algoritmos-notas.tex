\documentclass[10pt]{article}
\usepackage{beamerarticle}
\usepackage[spanish]{babel}
\usepackage[utf8]{inputenc}
\usepackage{fullpage}
\usepackage{xcolor}
\usepackage{listings}
\usepackage{textcomp}
\usepackage{mathpazo}
\usepackage{courier}
\usepackage{fancyvrb}
\usepackage{amsmath}
\usepackage{url}
\usepackage{hyperref}
\usepackage{pgfpages}
\usepackage{wrapfig}
\usepackage{enumitem}

\hyphenation{re-fe-ren-cia}

\setjobnamebeamerversion{01-introduccion-algoritmos-diapos}


\newcommand{\onelinerule}{\rule[2.3ex]{0pt}{0pt}}
\newcommand{\twolinerule}{\rule[6.2ex]{0pt}{0pt}}
\newcommand{\respuesta}{\framebox[\textwidth]{\twolinerule}}
\newcommand{\nombre}{%
  \begin{tikzpicture}[xscale=.4,yscale=.7]
    \draw (0, 0) rectangle (22, 1);
  \end{tikzpicture}%
}
%\newcommand{\rol}   {\framebox[0.3\textwidth]{\onelinerule}}
\newcommand{\rol}{%
  \begin{tikzpicture}[xscale=.4,yscale=.7]
    \draw[gray!40] ( 0, 0) grid      ( 9, 1);
    \draw          ( 0, 0) rectangle ( 9, 1);
    \draw          (10, 0) rectangle (11, 1);
    \draw (9 + .2, .5) -- (10 - .2, .5);
  \end{tikzpicture}%
}
\newcommand{\li}{\lstinline}
\providecommand{\pond}[1]{[{\small\textbf{#1\%}}]}

\lstdefinelanguage{py}{%
  classoffset=0,%
    morekeywords={%
      False,class,finally,is,return,None,continue,for,lambda,try,%
      True,def,from,nonlocal,while,and,del,global,not,with,print,%
      as,elif,if,or,yield,assert,else,import,pass,break,except,in,raise},%
    keywordstyle=\color{black!80}\bfseries,%
  classoffset=1,
    morekeywords={int,float,str,abs,len,raw_input,exit,range,min,max,%
      set,dict,tuple,list,bool,complex,round,sum,all,any,zip,map,filter,%
      sorted,reversed,dir,file,frozenset,open,%
      array,zeros,ones,arange,linspace,eye,diag,dot},
    keywordstyle=\color{black!50}\bfseries,%
  classoffset=0,%
  sensitive=true,%
  morecomment=[l]\#,%
  morestring=[b]',%
  morestring=[b]",%
  stringstyle=\em,%
}

\lstdefinelanguage{testcase}{%
  moredelim=[is][\bfseries]{`}{`},%
  backgroundcolor=\color{gray!20},%
}

\lstdefinelanguage{file}{%
  frame=single,%
}

\lstset{language=py}
\lstset{basicstyle=\ttfamily}
\lstset{columns=fixed}
\lstset{upquote=true}
\lstset{showstringspaces=false}
\lstset{rangeprefix=\#\ }
\lstset{includerangemarker=false}

\newlist{certamen}{enumerate}{1}
\setlist[certamen]{%
  label=\arabic*.,
  font=\LARGE\bfseries,%
  labelindent=-.5in,%
  leftmargin=0pt,%
  labelsep=1em%
}


\newcommand{\diapo}[1]{
  \begin{wrapfigure}{r}{.50\textwidth}
    \hfill\fbox{\includeslide[width=.45\textwidth]{#1}}
  \end{wrapfigure}
}



\title{Introducción a la programación, algoritmos}
\author{Programación \\ \url{http://progra.usm.cl}}
\date{}

\begin{document}
  \maketitle

  \section*{Objetivos de la clase}
  \begin{itemize}
    \item Entegar información administrativa de la asignatura.
    \item Presentar los conceptos de programa y algoritmo.
    \item Presentar algunos problemas interesantes
      que pueden ser resueltos mediante programación.
    \item Guiar a los alumnos para que diseñen un algoritmo simple.
    \item Presentar un algoritmo simple en lenguaje natural,
      diagrama de flujo, pseudocódigo y programa en Python.
    \item Mostrar la ejecución del programa en Python.
  \end{itemize}

  \section*{Diapositivas}

  \diapo{evaluaciones}

  La información oficial siempre es la que aparece en la página.
  Haga hincapié en esto.

  \diapo{web}

  Recordar a los estudiantes
  visitar la página lo antes posible
  para leer la información administrativa
  y para comenzar a leer la materia.
  
  También sugerirles suscribirse al Twitter y al Facebook
  si es que ocupan esas aplicaciones.
  Deben poner «Programación USM» en el buscador de Facebook
  para encontrar la página.

  \diapo{conceptos}

  Antes de presentar los conceptos hay una diapo en blanco
  para que cada profesor haga el preludio del ramo a su manera.

  Los tres conceptos importantes (problema, algoritmo y programa)
  son presentados en las diapositivas con definiciones sencillas,
  pensando en que los estudiantes las retengan más fácilmente.
  Una vez explicados los conceptos de esta manera,
  se puede profundizar en las características importantes de los algoritmos:
  \begin{itemize}
    \item deben tener inicio y final,
    \item deben funcionar para todas las entradas posibles,
    \item los pasos deben ser precisos y bien definidos,
    \item están compuestos de entrada, proceso y salida.
  \end{itemize}

  La relación entre problema y algoritmo es:
  un problema especifica para una entrada cuál debe ser la salida,
  y el algoritmo indica cómo obtener esa salida.

  Los ejemplos típicos de algoritmos son:
  receta de cocina,
  buscar una palabra en el diccionario (búsqueda binaria),
  resolver ecuación cuadrática,
  multiplicar números a mano.
  \newpage

  A continuación,
  se presentará algunos problemas computacionales típicos
  para ser analizados junto con los estudiantes.
  Discutir acerca de si es posible crear algoritmos,
  y si son sencillos de describir o no.

  (La idea de partir presentando problemas interesantes
  está sacada del libro \emph{Introduction to Algorithms},
  de Cormen et~al).
 
  \diapo{problema-ceros-recta}

  Explicar antes qué es un cero de la función
  (es el valor de \(x\) para el que \(y = 0\)).
  En este caso, el algoritmo es muy sencillo:
  sólo hay que calcular \(x_0 = -b/a\).
  
  \diapo{problema-ceros-funcion}

  Para este ejemplo,
  enfatizar que el algoritmo debe funcionar para cualquier función.
  Incentivar a los estudiantes para que se les ocurra
  una manera sistemática de hacerlo.
  
  Para este problema,
  los algoritmos más comunes son los iterativos
  que se van acercando cada vez más a los ceros exactos,
  pero sin llegar a ellos.
  Esto se puede explicar gráficamente en la pizarra.
  Como referencia, puede tomarse estas animaciones:
  \url{http://tinyurl.com/6cfc5w2} y
  \url{http://tinyurl.com/24fzfkp}.

  \diapo{problema-ordenamiento}

  Este problema se explica por sí solo.
  Nuevamente enfatizar que debe funcionar para todas las entradas posibles.

  \diapo{problema-ciudades}

  Típico ejemplo de problema de optimización.
  Contar que, si bien los algoritmos pueden ser fáciles de describir,
  son muy costosos de ejecutar,
  ya que hay que probar todas las combinaciones posibles
  para encontrar la mejor solución,
  y que hay algoritmos más rápidos,
  pero no encuentran necesariamente la mejor.

  \diapo{problema-spam}

  Éste es un buen ejemplo de un problema real
  con el que quizás los estudiantes estén familiarizados.
  Mencionar que se trata de algoritmos de aprendizaje,
  que deben ser entrenados para entregar sus resultados.
  
  A grandes rasgos,
  los típicos filtros de spam
  funcionan correlacionando las frecuencias de las palabras
  en correos deseados y no deseados,
  y luego usando fórmulas estadísticas
  para estimar la probabilidad de que un nuevo correo sea o no spam.

  \diapo{problema-tsunami}

  Este es un ejemplo de un problema de simulación.
  Hay que tener un modelo de la tierra.
  y usar las leyes de la física
  para simular el comportamiento de todo el océano.

  Es posible hacerlo (de hecho se hace)
  pero es extremadamente costoso.

  \diapo{ejercicio}

  Proponga el ejercicio de diseñar un algoritmo
  para determinar si un número entero es primo o compuesto.

  Repase la definición de número primo:
  es el que es divisible sólo por 1 y por sí mismo.
  Esto debería dar pistas sobre el algoritmo obvio:
  buscar un divisor que esté entre 2 y \(n - 1\).

  \diapo{sol-natural}
  \diapo{sol-diagrama}
  \diapo{sol-pseudocodigo}

  Presente las notaciones de lenguaje natural,
  diagrama de flujo y pseudocódigo.
  En cada caso, identifique claramente:
  entrada y salida del algoritmo,
  orden en que se hacen los pasos,
  variables y tipos.

  La lógica del algoritmo es:
  partir suponiendo que el número es primo;
  si encontramos algún divisor,
  modificamos el supuesto.
  Al final, verificamos el estado del supuesto
  para entegar la respuesta.

  Un buen ejemplo para ilustrarlo
  es preguntar a los estudiantes que respondan rápidamente si 91 es primo.
  Generalmente dicen que sí porque no se les ocurre ningún divisor,
  pero \(91 = 7\times 13\).

  \diapo{sol-python}

  Explique a grandes rasgos la solución en Python
  (sin entrar en tanto detalle).
  Muestre en vivo
  cómo introducir el programa en el computador
  y cómo ejecutarlo.

\end{document}


