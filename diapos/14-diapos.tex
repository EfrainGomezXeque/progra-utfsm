\documentclass[12pt]{beamer}
\usepackage[spanish]{babel}
\usepackage[utf8]{inputenc}
\usepackage{xcolor}
\usepackage{listings}
\usepackage{textcomp}
\usepackage{mathpazo}
\usepackage{courier}
\usepackage{fancyvrb}
\usepackage{amsmath}
\usepackage{url}
\usepackage{hyperref}

\newcommand{\onelinerule}{\rule[2.3ex]{0pt}{0pt}}
\newcommand{\twolinerule}{\rule[6.2ex]{0pt}{0pt}}
\newcommand{\respuesta}{\framebox[\textwidth]{\twolinerule}}
\newcommand{\nombre}{%
  \begin{tikzpicture}[xscale=.4,yscale=.7]
    \draw (0, 0) rectangle (22, 1);
  \end{tikzpicture}%
}
%\newcommand{\rol}   {\framebox[0.3\textwidth]{\onelinerule}}
\newcommand{\rol}{%
  \begin{tikzpicture}[xscale=.4,yscale=.7]
    \draw[gray!40] ( 0, 0) grid      ( 9, 1);
    \draw          ( 0, 0) rectangle ( 9, 1);
    \draw          (10, 0) rectangle (11, 1);
    \draw (9 + .2, .5) -- (10 - .2, .5);
  \end{tikzpicture}%
}
\newcommand{\li}{\lstinline}
\providecommand{\pond}[1]{[{\small\textbf{#1\%}}]}

\lstdefinelanguage{py}{%
  classoffset=0,%
    morekeywords={%
      False,class,finally,is,return,None,continue,for,lambda,try,%
      True,def,from,nonlocal,while,and,del,global,not,with,print,%
      as,elif,if,or,yield,assert,else,import,pass,break,except,in,raise},%
    keywordstyle=\color{black!80}\bfseries,%
  classoffset=1,
    morekeywords={int,float,str,abs,len,raw_input,exit,range,min,max,%
      set,dict,tuple,list,bool,complex,round,sum,all,any,zip,map,filter,%
      sorted,reversed,dir,file,frozenset,open,%
      array,zeros,ones,arange,linspace,eye,diag,dot},
    keywordstyle=\color{black!50}\bfseries,%
  classoffset=0,%
  sensitive=true,%
  morecomment=[l]\#,%
  morestring=[b]',%
  morestring=[b]",%
  stringstyle=\em,%
}

\lstdefinelanguage{testcase}{%
  moredelim=[is][\bfseries]{`}{`},%
  backgroundcolor=\color{gray!20},%
}

\lstdefinelanguage{file}{%
  frame=single,%
}

\lstset{language=py}
\lstset{basicstyle=\ttfamily}
\lstset{columns=fixed}
\lstset{upquote=true}
\lstset{showstringspaces=false}
\lstset{rangeprefix=\#\ }
\lstset{includerangemarker=false}

\newlist{certamen}{enumerate}{1}
\setlist[certamen]{%
  label=\arabic*.,
  font=\LARGE\bfseries,%
  labelindent=-.5in,%
  leftmargin=0pt,%
  labelsep=1em%
}



\usecolortheme{crane}
\usefonttheme{serif}

\title{Estructuras de datos anidadas}
\author{
  Programación \\ \url{http://progra.usm.cl}
}
\date{18 y 19 de abril de 2011}

\begin{document}
  \begin{frame}
    \maketitle
  \end{frame}

  \begin{frame}
    \label{repaso}
    \frametitle{Repaso}
    \small
    \def\sep{-1ex}
    \begin{columns}[T]
      \column{0.5\textwidth}
        Listas:
        \begin{itemize}
          \setlength{\itemsep}{\sep}
          \item tienen orden
          \item mutables
          \item \li!list()!
          \item \li![1, 2, 3]!
        \end{itemize}
      \column{0.5\textwidth}
        Tuplas:
        \begin{itemize}
          \setlength{\itemsep}{\sep}
          \item tienen orden
          \item inmutables
          \item \li!tuple()!
          \item \li!(1, 2, 3)!
        \end{itemize}
    \end{columns}
    \vfill
    \begin{columns}[T]
      \column{0.5\textwidth}
        Diccionarios:
        \begin{itemize}
          \setlength{\itemsep}{\sep}
          \item no tienen orden
          \item pares llave-valor
          \item llaves no repetidas
          \item mutables
          \item \li!dict()!
          \item \li!\{1: 2, 3: 4\}!
        \end{itemize}
      \column{0.5\textwidth}
        Conjuntos:
        \begin{itemize}
          \setlength{\itemsep}{\sep}
          \item no tienen orden
          \item elementos no repetidos
          \item mutables
          \item \li!set()!
          \item \li!\{1, 2, 3\}!
        \end{itemize}
    \end{columns}
  \end{frame}

  \begin{frame}
    \frametitle{Problema 1}
    \label{problema-alumnos-1}
    Los datos de los alumnos de una universidad \\
    están almacenados en un diccionario. \\
    Las llaves son los roles,
    y los valores son tuplas \\
    con el nombre, el apellido
    y la fecha de nacimiento: \\
    \tiny
    \lstinputlisting[linerange=1-10]{programas/alumnos.py}
  \end{frame}

  \begin{frame}
    \frametitle{Problema 1 (continuación)}
    \label{problema-alumnos-2}
    Los inscritos de cada ramo están registrados \\
    en otro diccionario. Las llaves son los nombres de los ramos,
    y cada valor es el conjunto de los roles \\
    de los alumnos que están tomando el ramo:
    \tiny
    \lstinputlisting[linerange=12-17]{programas/alumnos.py}
  \end{frame}

  \begin{frame}
    \frametitle{Problema 1 (continuación)}
    \label{problema-alumnos-3}
    \textbf{Ejercicio 1} \\
    Escriba la función \li!ramos_alumno(rol)! \\
    que retorne la lista de los ramos que está tomando \\
    el alumno con el rol entregado como parámetro:
    \lstinputlisting{programas/caso-ramos-alumno.py}
  \end{frame}

  \begin{frame}
    \frametitle{Problema 1 (continuación)}
    \label{problema-alumnos-4}
    \textbf{Ejercicio 2} \\
    Escriba la función \li!alumno_mas_joven()! \\
    que retorne el nombre completo del alumno más joven:
    \lstinputlisting{programas/caso-alumno-mas-joven.py}
  \end{frame}

\end{document}

