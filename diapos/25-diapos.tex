\documentclass[12pt]{beamer}
\usepackage[spanish]{babel}
\usepackage[utf8]{inputenc}
\usepackage{xcolor}
\usepackage{listings}
\usepackage{textcomp}
\usepackage{mathpazo}
\usepackage{courier}
\usepackage{fancyvrb}
\usepackage{amsmath}
\usepackage{url}
\usepackage{hyperref}

\newcommand{\onelinerule}{\rule[2.3ex]{0pt}{0pt}}
\newcommand{\twolinerule}{\rule[6.2ex]{0pt}{0pt}}
\newcommand{\respuesta}{\framebox[\textwidth]{\twolinerule}}
\newcommand{\nombre}{%
  \begin{tikzpicture}[xscale=.4,yscale=.7]
    \draw (0, 0) rectangle (22, 1);
  \end{tikzpicture}%
}
%\newcommand{\rol}   {\framebox[0.3\textwidth]{\onelinerule}}
\newcommand{\rol}{%
  \begin{tikzpicture}[xscale=.4,yscale=.7]
    \draw[gray!40] ( 0, 0) grid      ( 9, 1);
    \draw          ( 0, 0) rectangle ( 9, 1);
    \draw          (10, 0) rectangle (11, 1);
    \draw (9 + .2, .5) -- (10 - .2, .5);
  \end{tikzpicture}%
}
\newcommand{\li}{\lstinline}
\providecommand{\pond}[1]{[{\small\textbf{#1\%}}]}

\lstdefinelanguage{py}{%
  classoffset=0,%
    morekeywords={%
      False,class,finally,is,return,None,continue,for,lambda,try,%
      True,def,from,nonlocal,while,and,del,global,not,with,print,%
      as,elif,if,or,yield,assert,else,import,pass,break,except,in,raise},%
    keywordstyle=\color{black!80}\bfseries,%
  classoffset=1,
    morekeywords={int,float,str,abs,len,raw_input,exit,range,min,max,%
      set,dict,tuple,list,bool,complex,round,sum,all,any,zip,map,filter,%
      sorted,reversed,dir,file,frozenset,open,%
      array,zeros,ones,arange,linspace,eye,diag,dot},
    keywordstyle=\color{black!50}\bfseries,%
  classoffset=0,%
  sensitive=true,%
  morecomment=[l]\#,%
  morestring=[b]',%
  morestring=[b]",%
  stringstyle=\em,%
}

\lstdefinelanguage{testcase}{%
  moredelim=[is][\bfseries]{`}{`},%
  backgroundcolor=\color{gray!20},%
}

\lstdefinelanguage{file}{%
  frame=single,%
}

\lstset{language=py}
\lstset{basicstyle=\ttfamily}
\lstset{columns=fixed}
\lstset{upquote=true}
\lstset{showstringspaces=false}
\lstset{rangeprefix=\#\ }
\lstset{includerangemarker=false}

\newlist{certamen}{enumerate}{1}
\setlist[certamen]{%
  label=\arabic*.,
  font=\LARGE\bfseries,%
  labelindent=-.5in,%
  leftmargin=0pt,%
  labelsep=1em%
}



\usecolortheme{crane}
\usefonttheme{serif}

\title{Productos matriciales}
\author{
  Programación \\ \url{http://progra.usm.cl}
}
\date{}

\begin{document}
  \begin{frame}
    \maketitle
  \end{frame}

  \begin{frame}
    \label{repaso-arreglos}
    \frametitle{Repaso de arreglos}
    \lstinputlisting%[linerange=]
        {programas/repaso-arreglos.py}
  \end{frame}

  \begin{frame}
    \label{producto-interno}
    \frametitle{Producto interno}
    \includegraphics[width=\textwidth]{../diagramas/producto-interno.pdf}
  \end{frame}

  \begin{frame}
    \label{producto-interno-numpy}
    \frametitle{Producto interno en NumPy}
    \lstinputlisting[linerange=DOT-FIN\ DOT]
        {programas/producto-interno.py}
  \end{frame}

  \begin{frame}
    \label{ejercicios-producto-interno}
    \frametitle{Ejercicio 1}
    ¿Cuáles son los resultados
    de las siguientes operaciones?
    \lstinputlisting[linerange=EJERCICIO\ 1-FIN\ EJERCICIO\ 1]%
        {programas/producto-interno.py}
  \end{frame}

  \begin{frame}
    \label{ejemplos-producto-interno}
    \frametitle{Ejemplos de uso del producto interno}
    \lstinputlisting[linerange=EJEMPLOS\ PRODUCTO\ INTERNO-FIN\ EJEMPLOS\ PRODUCTO\ INTERNO]%
        {programas/producto-interno.py}
  \end{frame}

  \begin{frame}
    \label{producto-matriz-vector}
    \frametitle{Producto matriz-vector}
    \includegraphics[width=\textwidth]{../diagramas/matriz-vector.pdf}
  \end{frame}

  \begin{frame}
    \label{producto-matriz-vector-numpy}
    \frametitle{Producto matriz-vector en NumPy}
    \lstinputlisting[linerange=MATRIZ\ VECTOR-FIN\ MATRIZ\ VECTOR]
        {programas/producto-interno.py}
  \end{frame}

  \begin{frame}
    \label{ejercicios-producto-matriz-vector}
    \frametitle{Ejercicio 2}
    ¿Cuáles son los resultados
    de las siguientes operaciones?
    \lstinputlisting[linerange=EJERCICIO\ 2-FIN\ EJERCICIO\ 2]%
        {programas/producto-interno.py}
  \end{frame}

  \begin{frame}
    \label{ejercicio-nutrientes}
    \frametitle{Ejercicio 3}
    La siguiente tabla muestra cuántos gramos de nutrientes
    son aportados por cada 100 gramos de ciertos alimentos:
    \vspace{2ex}

    {\footnotesize
    \begin{tabular}{lp{6em}p{6em}p{4.5em}}
      & \raggedright Leche descremada & \raggedright Harina de soya & {\raggedright Suero de leche} \\\hline
      Proteínas     &  36 &   51 &   13  \\\hline
      Carbohidratos &  52 &   34 &   74  \\\hline
      Grasa         &   0 &    7 &  1.1  \\\hline
    \end{tabular}}

    \vspace{2ex}
    El doctor Alan Howard está diseñando una nueva dieta,
    y desea calcular su aporte nutritivo
    en función de las cantidades recomendadas de alimentos.

    \vspace{2ex}
    Escriba un programa que pregunte la cantidades de leche, harina y suero
    recomendadas por la dieta, y que muestre como salida
    los aportes en cada uno de los nutrientes.
  \end{frame}

  \begin{frame}
    \label{caso-nutrientes}
    \frametitle{Ejercicio 3---caso de prueba}
    \lstinputlisting[language=testcase]{programas/caso-dieta.txt}
  \end{frame}

  \begin{frame}
    \label{producto-matriz-matriz}
    \frametitle{Producto matriz-matriz}
    \includegraphics[width=\textwidth]{../diagramas/matriz-matriz.pdf}
  \end{frame}

  \begin{frame}
    \label{producto-matriz-matriz-numpy}
    \frametitle{Producto matriz-matriz en NumPy}
    \lstinputlisting[linerange=MATRIZ\ MATRIZ-FIN\ MATRIZ\ MATRIZ]
        {programas/producto-interno.py}
  \end{frame}

  \begin{frame}
    \label{ejercicio-acero}
    \frametitle{Ejercicio 4}
    Para producir acero,
    la empresa VÖST Alpine necesita
    hierro y carbón. Las demandas en toneladas por semana son:

    \begin{center}
      {\footnotesize
      \begin{tabular}{lrr}
        & Hierro & Carbón \\ \hline
        Semana 1 &  9 &   8 \\\hline
        Semana 2 &  5 &   7 \\\hline
        Semana 3 &  6 &   4 \\\hline
      \end{tabular}}
      \vfill
    \end{center}

    Tres proveedores ofrecen estas materias primas. \\
    ¿Cuál ofrece mayor beneficio?
    \vfill

    \begin{center}
      {\footnotesize
      \begin{tabular}{lrrr}
        & Ruhr AG & Iron One & Hard Coal \\\hline
        Hierro & \$540 & \$630 & \$530 \\\hline
        Carbón & \$420 & \$410 & \$440 \\\hline
      \end{tabular}}
      \vspace{2ex}
    \end{center}
  \end{frame}
\end{document}

