\documentclass[10pt]{article}
\usepackage{beamerarticle}
\usepackage[spanish]{babel}
\usepackage[utf8]{inputenc}
\usepackage{fullpage}
\usepackage{xcolor}
\usepackage{listings}
\usepackage{textcomp}
\usepackage{mathpazo}
\usepackage{courier}
\usepackage{fancyvrb}
\usepackage{amsmath}
\usepackage{url}
\usepackage{hyperref}
\usepackage{pgfpages}
\usepackage{wrapfig}
\hyphenation{}

\setjobnamebeamerversion{29-diapos}

\newcommand{\onelinerule}{\rule[2.3ex]{0pt}{0pt}}
\newcommand{\twolinerule}{\rule[6.2ex]{0pt}{0pt}}
\newcommand{\respuesta}{\framebox[\textwidth]{\twolinerule}}
\newcommand{\nombre}{%
  \begin{tikzpicture}[xscale=.4,yscale=.7]
    \draw (0, 0) rectangle (22, 1);
  \end{tikzpicture}%
}
%\newcommand{\rol}   {\framebox[0.3\textwidth]{\onelinerule}}
\newcommand{\rol}{%
  \begin{tikzpicture}[xscale=.4,yscale=.7]
    \draw[gray!40] ( 0, 0) grid      ( 9, 1);
    \draw          ( 0, 0) rectangle ( 9, 1);
    \draw          (10, 0) rectangle (11, 1);
    \draw (9 + .2, .5) -- (10 - .2, .5);
  \end{tikzpicture}%
}
\newcommand{\li}{\lstinline}
\providecommand{\pond}[1]{[{\small\textbf{#1\%}}]}

\lstdefinelanguage{py}{%
  classoffset=0,%
    morekeywords={%
      False,class,finally,is,return,None,continue,for,lambda,try,%
      True,def,from,nonlocal,while,and,del,global,not,with,print,%
      as,elif,if,or,yield,assert,else,import,pass,break,except,in,raise},%
    keywordstyle=\color{black!80}\bfseries,%
  classoffset=1,
    morekeywords={int,float,str,abs,len,raw_input,exit,range,min,max,%
      set,dict,tuple,list,bool,complex,round,sum,all,any,zip,map,filter,%
      sorted,reversed,dir,file,frozenset,open,%
      array,zeros,ones,arange,linspace,eye,diag,dot},
    keywordstyle=\color{black!50}\bfseries,%
  classoffset=0,%
  sensitive=true,%
  morecomment=[l]\#,%
  morestring=[b]',%
  morestring=[b]",%
  stringstyle=\em,%
}

\lstdefinelanguage{testcase}{%
  moredelim=[is][\bfseries]{`}{`},%
  backgroundcolor=\color{gray!20},%
}

\lstdefinelanguage{file}{%
  frame=single,%
}

\lstset{language=py}
\lstset{basicstyle=\ttfamily}
\lstset{columns=fixed}
\lstset{upquote=true}
\lstset{showstringspaces=false}
\lstset{rangeprefix=\#\ }
\lstset{includerangemarker=false}

\newlist{certamen}{enumerate}{1}
\setlist[certamen]{%
  label=\arabic*.,
  font=\LARGE\bfseries,%
  labelindent=-.5in,%
  leftmargin=0pt,%
  labelsep=1em%
}


\newcommand{\diapo}[1]{
  \begin{wrapfigure}{r}{.50\textwidth}
    \hfill\fbox{\includeslide[width=.45\textwidth]{#1}}
  \end{wrapfigure}
}


\title{Interfaces gráficas}
\author{Programación \\ \url{http://progra.usm.cl}}
\date{}

\begin{document}
  \maketitle

  \section*{Objetivos de la clase}
  \begin{itemize}
    \item Presentar la biblioteca Tkinter
      para la creación de interfaces gráficas en Python.
    \item Presentar los widgets básicos
      para hacer entrada y salida:
      \li!Entry!, \li!Label! y \li!Button!.
    \item Explicar el patrón modelo-vista-controlador
      que subyace a las interfaces gráficas de Tkinter.
  \end{itemize}

  \section*{Diapositivas}

  Existen muchas bibliotecas para crear interfaces gráficas en Python.
  Tkinter es la «oficial»,
  pues viene incorporada en la biblioteca estándar de Python.
  La ventaja es que no suele ser necesario instalar nada para usarla,
  excepto en algunas distribuciones de Linux, en las que debe ser instalada por separado.

  En esta clase,
  enseñaremos lo mínimo que hay que saber para crear una interfaz funcional simple.
  Omitiremos muchos detalles y variantes,
  para evitar introducir confusiones,
  y nos enfocaremos en los conceptos importantes.

  Antes de comenzar la materia,
  explique a los estudiantes qué es una interfaz gráfica.
  Simplemente es otra manera de hacer entrada y salida en un programa.
  Recuerde a los estudiantes qué son la entrada y la salida del programa,
  y mencione los mecanismos de entrada que hemos visto hasta ahora
  (por teclado y por archivo)
  así como los mecanismos de salida
  (por pantalla y por archivo).

  Durante la clase,
  use las diapositivas para explicar los conceptos,
  y muestre en vivo cómo se crean los programas,
  usando la consola o los programas adjuntos.

  \diapo{crear-ventana}

  Esta diapositiva muestra lo mínimo que debe hacerse
  para crear una interfaz gráfica:
  crear la ventana principal
  e iniciar el ciclo de eventos.

  Primero hay que importar todos los objetos del módulo Tkinter.
  El módulo está diseñado para ser usado de este modo.

  A continuación,
  hay que crear la ventana principal
  llamando a su constructor llamado \li!Tk! (ojo con la mayúscula).
  Por convención,
  siempre llamaremos \li!w! a la variable
  que asociaremos a la ventana recién creada
  (se suele llamarla \li!window!, pero por brevedad mejor dejar sólo la \li!w!).

  Al final del programa,
  hay que iniciar el ciclo de eventos de la ventana
  invocando su método \li!mainloop!.
  Aquí es importante notar la diferencia fundamental
  con respecto a las interfaces de consola.
  Las interfaces gráficas están permanentemente en un ciclo infinito,
  escuchando si ocurre algún evento (como el clic a un botón).
  El instante en que se ejecuta el código no es sabido de antemano.
  Esto requiere estructurar los programas de otra manera,
  que es la que iremos explicando durante esta clase.

  En general,
  toda la parte del programa en que se crea la interfaz
  va entre la creación de la ventana
  y la llamada a \li!mainloop!.
  Esto se ha indicado  con los puntos suspensivos en la diapositiva.
  Mencione que esta es la plantilla en la que nos basaremos
  para crear todos los programas que hagamos en adelante.

  Muestre en la consola el programa sencillo que crea la ventana vacía,
  y que el programa se queda «pegado» al ejecutar el \li!mainloop!,
  hasta que uno cierra la ventana.

  \diapo{widgets}

  Los componentes de una interfaz gráfica
  se llaman \emph{widgets}.
  Un widget es simplemente cualquier cosa que podemos poner en una ventana.
  En la clase de hoy,
  aprenderemos a crear sólo tres widgets,
  que son suficientes para crear programas simples:
  \begin{itemize}
    \item las \textbf{etiquetas} (\li!Label!)
      sirven para mostrar datos en la interfaz,
    \item los \textbf{botones} (\li!Button!)
      sirven para hacer que algo ocurra en el programa, y
    \item los \textbf{campos de entrada} (\li!Entry!)
      sirven para ingresar datos al programa.
  \end{itemize}

  Para hacer una correspondencia con los programas con interfaz de consola,
  el \li!Entry! es análogo al \li!raw_input!
  y el \li!Label! al \li!print!.
  Los botones no tienen un análogo,
  pues en un programa con interfaz de consola
  las funciones no son llamadas ante eventos del usuario.
  El rol de los botones puede ser explicado
  como el de un «llamador de funciones»:
  cada vez que uno presione un botón,
  se ejecutará la función que tiene asociada.

  La diapositiva muestra
  cómo se ve un programa que tiene un widget de cada tipo.
  Un programa puede tener tantos widgets de cualquier tipo
  como uno le indique.

  \diapo{crear-widgets}

  Cada uno de los tipos de widgets tiene un constructor,
  que se llaman respectivamente \li!Label!, \li!Button! y \li!Entry!.
  Al llamar al constructor,
  el primer parámetro \emph{siempre} debe ser
  la ventana en la que estará contenido el widget.
  Éste es el único parámetro obligatorio.
  Los demás parámetros son opcionales
  y se pasan con la sintaxis de asignación de parámetros por nombre.
  En el ejemplo,
  el único parámetro opcional que es usado es \li!text!,
  que sirve para indicar el texto que será mostrado en el widget.

  Al crear el widget mediante su constructor,
  \emph{no} es agregado a la ventana inmediatamente.
  Debe ser agregado explícitamente
  llamando a su método \li!pack!
  (``empaquetar'' en inglés).
  La llamada puede ser vista como una orden: «agrégate a la ventana».

  Los widgets van apareciendo en el orden en que son empaquetados,
  no en el que los constructores son creados.
  Las llamadas a \li!pack! sin parámetros
  siempre apila los widgets verticalmente,
  de arriba hacia abajo.
  En la próxima clase veremos cómo hacerlo de otras maneras.
  Por ahora, sólo usaremos el comportamiento por omisión.

  Explique con detención los conceptos importantes:
  \begin{itemize}
    \item primer parámetro debe ser la ventana padre,
    \item hay que asignar el widget a una variable,
    \item agregar el widget a la ventana
      llamando al método \li!pack!.
  \end{itemize}

  El programa de la diapositiva
  está incluído en los archivos adjuntos.
  Muéstrelo en ejecución.
  Demuestre que al hacer clic en los botones no ocurre absolutamente nada,
  y anuncie que lo que haremos a continuación
  es hacer que los botones hagan algo.

  \diapo{controladores}

  La interfaz que está creada en este programa
  es exactamente la misma del ejemplo anterior,
  sólo que se asoció acciones a los botones.
  Las acciones deben ser encapsuladas en funciones
  que reciben el nombre de \emph{controladores}.
  Cada vez que se haga clic al botón,
  la función será ejecutada.

  Muestre las dos diferencias entre este programa y el anterior:
  \begin{itemize}
    \item se creó una función llamada \li!saludar!,
      que imprime el mensaje \li!Hola! en la consola, y
    \item se asoció comandos a los botones
      usando el parámetro \li!command!.
  \end{itemize}

  El parámetro \li!command! permite especificar el controlador asociado al botón.
  El controlador debe ser una función que no recibe parámetros.
  Hay que tener cuidado de no llamar a la función al momento de crear el botón:
  \begin{lstlisting}
b1 = Button(w, text='Saludar', command=saludar())  # INCORRECTO
b1 = Button(w, text='Saludar', command=saludar)    # CORRECTO
  \end{lstlisting}

  El segundo botón está asociado a la función \li!exit!,
  que ya existe en Python y lo que hace es terminar el programa.

  Muestre el código del programa y ejecútelo.
  Muestre en vivo lo que ocurre al hacer clic en los botones.

  \diapo{modelos}

  Hasta aquí hemos visto como crear una interfaz estática.
  Una interfaz también puede tener un estado
  que va mutando a medida que ocurren eventos.

  Un \emph{modelo} es un dato almacenado
  que está asociado a la interfaz.
  En cualquier momento,
  se puede modificar el dato contenido en el modelo,
  y también se puede obtener cuál es su valor actual.

  En Tkinter, los modelos se llaman \emph{variables}
  (que no son lo mismo que las variables de Python,
  lo que puede llevar a confusión).
  En nuestros programas,
  usaremos sólo modelos de tipo strings,
  que son creados usando el constructor \li!StringVar!.
  Hay modelos de otros tipos,
  pero las conversiones de datos entre strings y otros tipos
  son tan fáciles que no vale la pena tener que memorizar
  todos los tipos de modelos que hay (y que además tienen nombres confusos,
  como \li!DoubleVar! para datos de tipo \li!float!).

  Todos los datos de la interfaz
  que cambian durante el programa
  deben ser asociados a un modelo.
  En el ejemplo de la diapositiva,
  se muestra una interfaz que tiene dos \li!Label!s.
  El primero tiene una leyenda estática,
  que permanecerá igual durante todo el programa,
  y que es indicada usando el parámetro \li!text!.
  El segundo está asociado a un modelo creado anteriormente,
  que debe ser indicado usando el parámetro \li!textvariable!.
  En todo momento,
  la leyenda mostrada en el \li!Label!
  será el dato almacenado en el modelo \li!v!.
  Al modificar el valor del modelo,
  también cambiará la leyenda.

  Para modificar el valor del modelo,
  debe usarse su método \li!set!,
  pasando como argumento el nuevo valor.
  Para obtener su valor actual,
  debe usarse el método \li!get! sin parámetros.

  Los dos \li!Label!s se ven exactamente iguales en la interfaz,
  pero difieren en que la leyenda del segundo puede ser modificada
  durante el programa.

  \diapo{ejercicio-mas-uno}

  Lea y explique el enunciado de este ejercicio a los alumnos.
  Puede partir ejecutando el programa (adjunto)
  sin mostrar el código,
  para dejar claro qué es lo que tiene que hacer.

  Primero que todo,
  haga que identifiquen:
  \begin{itemize}
    \item qué widgets hay que crear
      (tres: un \li!Label! y dos \li!Button!s),
    \item qué modelos hay que crear
      (uno, para guardar el valor que se muestra en el \li!Label!),
    \item qué controladores hay que crear
      (dos: uno para cada botón)
  \end{itemize}

  Escriba el código en varias iteraciones
  junto con los alumnos.
  Primero,
  haga la estructura básica de la interfaz:
  crear la ventana, crear los widgets e iniciar el \li!mainloop!;
  pruebe el programa para ver que se ve como en el ejemplo.
  A continuación,
  cree el modelo
  y asócielo al \li!Label!.
  Finalmente,
  cree los controladores
  y asócielos a los botones.
  Es más claro escribir el código de esta manera
  que hacerlo inmediatamente de principio a fin.

  Ojo con el orden en que se definen las cosas en el código:
  el controlador debe ser definido después que el modelo,
  pues se necesita una referencia a éste para modificar su valor
  desde dentro de la función.

  \diapo{entrada}

  Para ingresar datos al programa,
  hay que usar un widget \li!Entry! asociado a un modelo.
  En cualquier momento que se desee obtener el valor
  que está ingresado en el \li!Entry!,
  basta con llamar al método \li!get! de su modelo asociado.
  Esto debe hacerse dentro de un controlador para que tenga sentido.

  En el ejemplo,
  aparecen dos controladores:
  uno que muestra por consola el valor contenido en el campo de entrada,
  y otro que cambia su valor a ``\li!Hola!''.
  Estos controladores podrían ser asociados a botones,
  que han sido omitidos en el ejemplo por simplicidad.

  \diapo{resumen-conceptos}

  Repaso de los conceptos importantes
  que hay que tener en cuenta para crear una interfaz funcional en Tkinter:
  los modelos son donde se guardan los datos asociados a la interfaz,
  y los controladores son las acciones que son gatilladas por los eventos
  capturados por la interfaz.

  Esta terminología es la usada en el libro guía
  \emph{Practical Programming: An introduction to Computer Science using Python},
  y se refiere al conocido patrón \href{http://es.wikipedia.org/wiki/Modelo_Vista_Controlador}{modelo-vista-controlador}
  de arquitectura de software.
  En todas las interfaces que diseñemos de ahora en adelante,
  tendremos que comenzar considerando cómo intervienen cada uno de estos componentes.

  \diapo{ejercicio-transformar-texto}
  \diapo{ejercicio-temperatura}
  \diapo{ejercicio-estadisticos}

  Para resolver los ejercicios,
  comience identificando cuáles son la vista (los widgets),
  los modelos y los controladores.
  Luego implemente la vista (el ``esqueleto'' de la interfaz),
  siga creando los modelos
  y finalmente cree los controladores.

  Avance al mismo ritmo que los estudiantes.
  Dé prioridad a que los conceptos queden muy claros,
  repasando los ejemplos anteriores si es necesario.
  No es necesario que alcance a hacer todos los ejercicios.
  Puede continuar en la próxima clase si es necesario.

  Las soluciones a los problemas están incluídas
  en los programas adjuntos como referencia.
  Puede mostrarlos en ejecución
  para que queden claros los requerimientos.
  No muestre el código de las respuestas de inmediato;
  desarrolle los programas progresivamente.
  Deje los detalles para el final
  (por ejemplo, el formato de salida del conversor de temperaturas,
  o el título de la ventana).

\end{document}

