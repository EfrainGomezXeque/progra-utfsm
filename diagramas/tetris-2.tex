\documentclass{article}
\usepackage[pdftex,active,tightpage]{preview}
\usepackage[utf8]{inputenc}
\usepackage[spanish]{babel}
\usepackage{mathpazo}
\usepackage{courier}
\usepackage{tikz}
\usetikzlibrary{calc,arrows,decorations,shapes,patterns}

\begin{document}
\begin{preview}
\begin{tikzpicture}[yscale=-1, scale=.6]

  \def\juego{
    \draw[gray] (0, 0) grid (7, 10);

    \fill[blue!80!black, draw=black] (0, 10)
      -- ++( 1, 0) -- ++( 0,-2) -- ++( 1, 0) -- ++( 0,-1) -- ++(-2, 0) -- cycle;
    \fill[orange!90!black, draw=black] (3, 10)
      -- ++( 1, 0) -- ++( 0,-1) -- ++( 1, 0) -- ++( 0,-1)
      -- ++(-3, 0) -- ++( 0, 1) -- ++( 1, 0) -- cycle;
    \fill[yellow!90!black, draw=black] (6, 10)
      -- ++( 1, 0) -- ++( 0,-2) -- ++(-1, 0) -- ++( 0,-1)
      -- ++(-1, 0) -- ++( 0, 2) -- ++( 1, 0) -- cycle;
    \fill[yellow!90!black, draw=black] (1, 7)
      -- ++( 1, 0) -- ++( 0,-2) -- ++(-1, 0) -- ++( 0,-1)
      -- ++(-1, 0) -- ++( 0, 2) -- ++( 1, 0) -- cycle;
    \fill[red!80!black, draw=black] (2, 7) rectangle ++( 2,-2);
    \fill[green!80!black, draw=black] (2, 8)
      -- ++( 3, 0) -- ++( 0,-2) -- ++(-1, 0) -- ++( 0, 1) -- ++(-2, 0) -- cycle;
    \fill[orange!90!black, draw=black] (4, 6)
      -- ++( 1, 0) -- ++( 0,-3) -- ++(-1, 0) -- ++( 0, 1)
      -- ++(-1, 0) -- ++( 0, 1) -- ++( 1, 0) -- cycle;
    \fill[red!40!blue!80!black, draw=black] (1, 5)
      -- ++( 2, 0) -- ++( 0,-1) -- ++(-1, 0) -- ++( 0,-1)
      -- ++(-2, 0) -- ++( 0, 1) -- ++( 1, 0) --cycle;
  }

  \def\pieza{
    \fill[green!80!black, draw=black] (6, 8)
      -- ++( 1, 0) -- ++( 0,-3) -- ++(-2, 0) -- ++( 0, 1) -- ++( 1, 0) -- cycle;
  }

  \begin{scope}
    \juego
    \draw[thick] (0, 0) rectangle (7, 10);
    \node[anchor=south] at (3.5, 0) {\texttt{juego}};
    \node (c) at (5.5, 10) {};
    \node[anchor=north] (l) at (5.5, 11) {\texttt{columna}};
    \draw[-latex', thick] (l) -- (c);
    \node (from1) at (7, 5) {};
  \end{scope}

  \begin{scope}[yshift=-10cm, xshift=-2.5cm]
    \draw[gray] (5, 5) grid ++(2, 3);
    \pieza
    \draw[thick] (5, 5) rectangle ++(2, 3);
    \node[anchor=south] at (6, 5) {\texttt{pieza}};
  \end{scope}

  \begin{scope}[xshift=12cm]
    \juego
    \fill[green!80!black, draw=black] (6, 8)
      -- ++( 1, 0) -- ++( 0,-3) -- ++(-2, 0) -- ++( 0, 1) -- ++( 1, 0) -- cycle;
    \draw[thick] (0, 0) rectangle (7, 10);
    \fill[pattern color=gray, pattern=north west lines] (0, 6) rectangle ++(7, -1);
    \fill[pattern color=gray, pattern=north west lines] (0, 8) rectangle ++(7, -1);
    \node (to1) at (0, 5) {};
    \node (from2) at (7, 5) {};
  \end{scope}

  \begin{scope}[xshift=24cm]
    \draw[gray] (0, 0) grid (7, 10);

    \fill[blue!80!black, draw=black] (0, 10) rectangle ++( 1,-2);
      -- ++( 1, 0) -- ++( 0,-2) -- ++( 1, 0) -- ++( 0,-1) -- ++(-2, 0) -- cycle;
    \fill[orange!90!black, draw=black] (3, 10)
      -- ++( 1, 0) -- ++( 0,-1) -- ++( 1, 0) -- ++( 0,-1)
      -- ++(-3, 0) -- ++( 0, 1) -- ++( 1, 0) -- cycle;
    \fill[yellow!90!black, draw=black] (6, 10)
      -- ++( 1, 0) -- ++( 0,-2) -- ++(-2, 0) -- ++( 0, 1) -- ++( 1, 0) -- cycle;
    \fill[yellow!90!black, draw=black] (1, 8) rectangle ++(1,-1);
    \fill[yellow!90!black, draw=black] (0, 7) rectangle ++(1,-1);
    \fill[red!80!black, draw=black] (2, 8) rectangle ++( 2,-1);
    \fill[green!80!black, draw=black] (4, 8) rectangle ++( 1,-1);
    \fill[orange!90!black, draw=black] (3, 7)
      -- ++( 2, 0) -- ++( 0,-2) -- ++(-1, 0) -- ++( 0, 1)
      -- ++(-1, 0) -- cycle;
    \fill[red!40!blue!80!black, draw=black] (1, 7)
      -- ++( 2, 0) -- ++( 0,-1) -- ++(-1, 0) -- ++( 0,-1)
      -- ++(-2, 0) -- ++( 0, 1) -- ++( 1, 0) --cycle;
    \fill[green!80!black, draw=black] (6, 8) rectangle ++( 1,-1);
    \node (to2) at (0, 5) {};
    \draw[thick] (0, 0) rectangle (7, 10);
  \end{scope}

  \draw[very thick, -latex'] (from1) -- (to1);
  \node[font=\footnotesize,yshift=-.5cm] at ($ (from1)!.5!(to1) $) {\verb+poner_pieza+};

  \draw[very thick, -latex'] (from2) -- (to2);
  \node[font=\footnotesize,yshift=-.5cm] at ($ (from2)!.5!(to2) $) {\verb+limpiar_filas+};

\end{tikzpicture}
\end{preview}
\end{document}
