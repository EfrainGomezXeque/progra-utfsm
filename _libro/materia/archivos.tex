\section{Archivos}

Todos los datos que un programa utiliza durante su ejecución se
encuentran en sus variables, que están almacenadas en la
\href{http://es.wikipedia.org/wiki/Memoria\_RAM}{memoria RAM} del
computador.

La memoria RAM es un medio de almacenamiento \textbf{volátil}: cuando el
programa termina, o cuando el computador se apaga, todos los datos se
pierden para siempre.

Para que un programa pueda guardar datos de manera permanente, es
necesario utilizar un medio de almacenamiento \textbf{persistente}, de
los cuales el más importante es el
\href{http://es.wikipedia.org/wiki/Disco\_duro}{disco duro}.

Los datos en el disco duro están organizados en
\href{http://es.wikipedia.org/wiki/Archivo\_(informática)}{archivos}. Un
\textbf{archivo} es una secuencia de datos almacenados en un medio
persistente que están disponibles para ser utilizados por un programa.
Todos los archivos tienen un nombre y una ubicación dentro del sistema
de archivos del sistema operativo.

Los datos en un archivo siguen estando presentes después de que termina
el programa que lo ha creado. Un programa puede guardar sus datos en
archivos para usarlos en una ejecución futura, e incluso puede leer
datos desde archivos creados por otros programas.

Un programa no puede manipular los datos de un archivo directamente.
Para usar un archivo, un programa siempre abrir el archivo y asignarlo a
una variable, que llamaremos el \textbf{archivo lógico}. Todas las
operaciones sobre un archivo se realizan a través del archivo lógico.

Dependiendo del contenido, hay muchos tipos de archivos. Nosotros nos
preocuparemos sólo de los \textbf{archivos de texto}, que son los que
contienen texto, y pueden ser abiertos y modificados usando un editor de
texto como el Bloc de Notas. Los archivos de texto generalmente tienen
un nombre terminado en \lstinline!.txt!.

\subsection{Lectura de archivos}

Para leer datos de un archivo, hay que abrirlo de la siguiente manera:

\begin{lstlisting}
archivo = open(nombre)
\end{lstlisting}

\lstinline!nombre! es un string que tiene el nombre del archivo.
\lstinline!archivo! es el archivo lógico a través del que se manipulará
el archivo.

Si el archivo no existe, ocurrirá un \textbf{error de entrada y salida}
(\lstinline!IOError!).

Es importante recordar que la variable \lstinline!archivo! es una
representación abstracta del archivo, y no los contenidos del mismo.

La manera más simple de leer el contenido es hacerlo línea por línea.
Para esto, basta con poner el archivo lógico en un ciclo for:

\begin{lstlisting}
for linea in archivo:
    # hacer algo
\end{lstlisting}

Una vez que los datos han sido leídos del archivo, hay que cerrarlo:

\begin{lstlisting}
archivo.close()
\end{lstlisting}

Por ejemplo, supongamos que tenemos el archivo \lstinline!himno.txt! que
tiene el siguiente contenido:

\begin{lstlisting}
Puro Chile
es tu cielo azulado
puras brisas
te cruzan también.
\end{lstlisting}

El archivo tiene cuatro líneas. Cada línea termina con un salto de línea
(\lstinline!\n!), que indica que a continuación comienza una línea
nueva.

El siguiente programa imprime la primera letra de cada línea del himno:

\begin{lstlisting}
archivo = open('himno.txt')
for linea in archivo:
    print linea[0]
archivo.close()
\end{lstlisting}

El ciclo for es ejecutado cuatro veces, una por cada línea del archivo.
La salida del programa es:

Otro ejemplo: el siguiente programa imprime cuántos símbolos hay en cada
línea:

\begin{lstlisting}
archivo = open('himno.txt')
for linea in archivo:
    print len(linea)
archivo.close()
\end{lstlisting}

La salida es:

Note que el salto de línea (el ``enter'') es considerado en la cuenta:

\begin{lstlisting}
+---+---+---+---+---+---+---+---+---+---+---+
| P | u | r | o |   | C | h | i | l | e | \n| = 11 símbolos
+---+---+---+---+---+---+---+---+---+---+---+
\end{lstlisting}

Para obtener el string sin el salto de línea, se puede usar el método
\lstinline!strip!, que elimina todos los símbolos de espaciado al
principio y al final del string:

\begin{lstlisting}
>>> s = '   Hola\n'
>>> s.strip()
'Hola'
\end{lstlisting}

Si modificamos el programa para eliminar el salto de línea:

\begin{lstlisting}
archivo = open('himno.txt')
for linea in archivo:
    print len(linea.strip())
archivo.close()
\end{lstlisting}

entonces la salida es:

Lo importante es comprender que los archivos son leídos línea por línea
usando el ciclo \lstinline!for!.

\subsection{Escritura en archivos}

Los ejemplos anteriores suponen que el archivo por leer existe, y está
listo para ser abierto y leído. Ahora veremos cómo crear los archivos y
cómo escribir datos en ellos, para que otro programa después pueda
abrirlos y leerlos.

Uno puede crear un archivo vacío abriéndolo de la siguiente manera:

\begin{lstlisting}
archivo = open(nombre, 'w')
\end{lstlisting}

El segundo parámetro de la función \lstinline!open! indica el uso que se
le dará al archivo. \lstinline!'w'! significa «escribir» (\emph{write}
en inglés).

Si el archivo señalado no existe, entonces será creado. Si ya existe,
entonces será sobreescrito. Hay que tener cuidado entonces, pues esta
operación elimina los datos del archivo que existía previamente.

Una vez abierto el archivo, uno puede escribir datos en él usando el
método \lstinline!write!:

\begin{lstlisting}
a = open('prueba.txt', 'w')
a.write('Hola ')
a.write('mundo.')
a.close()
\end{lstlisting}

Una vez ejecutado este programa, el archivo \lstinline!prueba.txt! será
creado (o sobreescrito, si ya existía). Al abrirlo en el Bloc de Notas,
veremos este contenido:

\begin{lstlisting}
Hola mundo.
\end{lstlisting}

Para escribir varias líneas en el archivo, es necesario agregar
explícitamente los saltos de línea en cada string que sea escrito. Por
ejemplo, para crear el archivo \lstinline!himno.txt! que usamos más
arriba, podemos hacerlo así:

\begin{lstlisting}
a = open('himno.txt', 'w')
a.write('Puro Chile\n')
a.write('es tu cielo azulado\n')
a.write('puras brisas\n')
a.write('te cruzan también.\n')
a.close()
\end{lstlisting}

Además del modo \lstinline!'w'! (\emph{write}), también existe el modo
\lstinline!'a'! (\emph{append}), que permite escribir datos al final de
un archivo existente. Por ejemplo, el siguiente programa abre el archivo
\lstinline!prueba.txt! que creamos más arriba, y agrega más texto al
final de él:

\begin{lstlisting}
a = open('prueba.txt', 'a')
a.write('\n')
a.write('Chao ')
a.write('pescao.')
a.close()
\end{lstlisting}

Si abrimos el archivo \lstinline!prueba.txt! en el Bloc de Notas,
veremos esto:

\begin{lstlisting}
Hola mundo.

Chao pescao.
\end{lstlisting}

De haber abierto el archivo en modo \lstinline!'w'! en vez de
\lstinline!'a'!, el contenido anterior (la frase \lstinline!Hola mundo!)
se habría borrado.

\subsection{Archivos de valores con separadores}

Una manera usual de almacenar datos con estructura de tabla en un
archivo es la siguiente: cada línea del archivo representa una fila de
la tabla, y los datos de una fila se ponen separados por algún símbolo
especial.

Por ejemplo, supongamos que queremos guardar en un archivo los datos de
esta tabla:

\ctable[pos = H, center, botcap]{llllll}
{% notes
}
{% rows
\FL
Nombre & Apellido & Nota 1 & Nota 2 & Nota 3 & Nota 4
\ML
Perico & Los Palotes & 90 & 75 & 38 & 65
\\\noalign{\medskip}
Yayita & Vinagre & 39 & 49 & 58 & 55
\\\noalign{\medskip}
Fulana & De Tal & 96 & 100 & 36 & 71
\LL
}

Si usamos el símbolo \lstinline!:! como separador, el archivo, que
llamaremos \lstinline!alumnos.txt!, debería quedar así:

\begin{lstlisting}
Perico:Los Palotes:90:75:38:65
Yayita:Vinagre:39:49:58:55
Fulanita:De Tal:96:100:36:71
\end{lstlisting}

El formato de estos archivos se suele llamar
\href{http://en.wikipedia.org/wiki/CSV\_(file\_format)}{CSV}, que en
inglés son las siglas de \emph{comma-separated values} (significa
«valores separados por comas», aunque técnicamente el separador puede
ser cualquier símbolo). A pesar del nombre especial que reciben, los
archivos CSV son archivos de texto como cualquier otro, y se pueden
tratar como tales.

Los archivos de valores con separadores son muy fáciles de leer y
escribir, y por esto son muy usados. Como ejemplo práctico, si usted
desea hacer un programa que analice los datos de una hoja de cálculo
Excel, puede guardar el archivo con el formato CSV directamente en el
Excel, y luego abrirlo desde su programa escrito en Python.

Para leer los datos de un archivo de valores con separadores, debe
hacerlo línea por línea, eliminar el salto de línea usando el método
\lstinline!strip! y luego extraer los valores de la línea usando el
método \lstinline!split!. Por ejemplo, al leer la primera línea del
archivo de más arriba obtendremos el siguiente string:

\begin{lstlisting}
'Perico:Los Palotes:90:75:38:65\n'
\end{lstlisting}

Para separar los seis valores, lo podemos hacer así:

\begin{lstlisting}
>>> linea.strip().split(':')
['Perico', 'Los Palotes', '90', '75', '38', '65']
\end{lstlisting}

Como se trata de un archivo de texto, todos los valores son strings. Una
manera de convertir los valores a sus tipos apropiados es hacerlo uno
por uno:

\begin{lstlisting}
valores = linea.strip().split(':')
nombre   = valores[0]
apellido = valores[1]
nota1 = int(valores[2])
nota2 = int(valores[3])
nota3 = int(valores[4])
nota4 = int(valores[5])
\end{lstlisting}

Una manera más breve es usar las rebanadas y la función \lstinline!map!:

\begin{lstlisting}
valores = linea.strip().split(':')
nombre, apellido = valores[0:2]
nota1, nota2, nota3, nota4 = map(int, valores[2:6])
\end{lstlisting}

O podríamos dejar las notas en una lista, en vez de usar cuatro
variables diferentes:

\begin{lstlisting}
notas = map(int, valores[2:6])
\end{lstlisting}

Por ejemplo, un programa para imprimir el promedio de todos los alumnos
se puede escribir así:

\begin{lstlisting}
archivo_alumnos = open('alumnos.txt')
for linea in archivo_alumnos:
    valores = linea.strip().split(':')
    nombre, apellido = valores[0:2]
    notas = map(int, valores[2:6])
    promedio = sum(notas) / 4.0
    print '{0} obtuvo promedio {1}'.format(nombre, promedio)
archivo_alumnos.close()
\end{lstlisting}

Para escribir los datos en un archivo, hay que hacer el proceso inverso:
convertir todos los datos al tipo string, pegarlos en un único string,
agregar el salto de línea al final y escribir la línea en el archivo.

Si los datos de la línea ya están en una lista o una tupla, podemos
convertirlos a string usando la función \lstinline!map! y pegarlos
usando el método \lstinline!join!:

\begin{lstlisting}
alumno = ('Perico', 'Los Palotes', 90, 75, 38, 65)
linea = ':'.join(map(str, alumno)) + '\n'
archivo.write(linea)
\end{lstlisting}

Otra manera es armar el string parte por parte:

\begin{lstlisting}
linea = '{0}:{1}:{2}:{3}:{4}:{5}\n'.format(nombre, apellido,
                                           nota1, nota2, nota3, nota4)
archivo.write(linea)
\end{lstlisting}

Como siempre, usted debe preferir la manera que le parezca más simple de
entender.
