\chapter{Tuplas}

Una \textbf{tupla} es una secuencia de valores agrupados.

Una tupla sirve para agrupar, como si fueran un único valor, varios
valores que, por su naturaleza, deben ir juntos.

El tipo de datos que representa a las tuplas se llama \lstinline!tuple!.
El tipo \lstinline!tuple! es inmutable: una tupla no puede ser
modificada una vez que ha sido creada.

Una tupla puede ser creada poniendo los valores separados por comas y
entre paréntesis. Por ejemplo, podemos crear una tupla que tenga el
nombre y el apellido de una persona:

\begin{lstlisting}
>>> persona = ('Perico', 'Los Palotes')
>>> persona
('Perico', 'Los Palotes')
\end{lstlisting}

\section{Desempaquetado de tuplas}

Los valores individuales de una tupla pueden ser recuperados asignando
la tupla a las variables respectivas. Esto se llama
\textbf{desempaquetar la tupla} (en inglés: \emph{unpack}):

\begin{lstlisting}
>>> nombre, apellido = persona
>>> nombre
'Perico'
\end{lstlisting}

Si se intenta desempaquetar una cantidad incorrecta de valores, ocurre
un error de valor:

\begin{lstlisting}
>>> a, b, c = persona
Traceback (most recent call last):
  File "<stdin>", line 1, in <module>
ValueError: need more than 2 values to unpack
\end{lstlisting}

Además, también es posible extraer los valores usando su índice, al
igual que con las listas:

\begin{lstlisting}
>>> persona[1]
'Los Palotes'
\end{lstlisting}

A diferencia de las listas, los elementos no se pueden modificar:

\begin{lstlisting}
>>> persona[1] = 'Smith'
Traceback (most recent call last):
  File "<console>", line 1, in <module>
TypeError: 'tuple' object does not support item assignment
\end{lstlisting}

\section{Comparación de tuplas}

Dos tuplas son iguales cuando tienen el mismo tamaño y cada uno de sus
elementos correspondientes tienen el mismo valor:

\begin{lstlisting}
>>> (1, 2) == (3 / 2, 1 + 1)
True
>>> (6, 1) == (6, 2)
False
>>> (6, 1) == (6, 1, 0)
False
\end{lstlisting}

Para determinar si una tupla es menor que otra, se utiliza lo que se
denomina \textbf{orden lexicográfico}. Si los elementos en la primera
posición de ambas tuplas son distintos, ellos determinan el ordenamiento
de las tuplas:

\begin{lstlisting}
>>> (1, 4, 7) < (2, 0, 0, 1)
True
>>> (1, 9, 10) < (0, 5)
False
\end{lstlisting}

La primera comparación es \lstinline!True! porque \lstinline!1 < 2!. La
segunda comparación es \lstinline!False! porque \lstinline!1 > 0!. No
importa el valor que tengan los siguientes valores, o si una tupla tiene
más elementos que la otra.

Si los elementos en la primera posición son iguales, entonces se usa el
valor siguiente para hacer la comparación:

\begin{lstlisting}
>>> (6, 1, 8) < (6, 2, 8)
True
>>> (6, 1, 8) < (6, 0)
False
\end{lstlisting}

La primera comparación es \lstinline!True! porque \lstinline!6 == 6! y
\lstinline!1 < 2!. La segunda comparación es \lstinline!False! porque
\lstinline!6 == 6! y \lstinline!1 > 0!.

Si los elementos respectivos siguen siendo iguales, entonces se sigue
probando con los siguientes uno por uno, hasta encontrar dos distintos.
Si a una tupla se le acaban los elementos para comparar antes que a la
otra, entonces es considerada menor que la otra:

\begin{lstlisting}
>>> (1, 2) < (1, 2, 4)
True
>>> (1, 3) < (1, 2, 4)
False
\end{lstlisting}

La primera compación es \lstinline!True! porque \lstinline!1 == 1!,
\lstinline!2 == 2!, y ahí se acaban los elementos de la primera tupla.
La segunda comparación es \lstinline!False! porque \lstinline!1 == 1! y
\lstinline!3 < 2!; en este caso sí se alcanza a determinar el resultado
antes que se acaben los elementos de la primera tupla.

Este método de comparación es el mismo que se utiliza para poner
palabras en orden alfabético (por ejemplo, en guías telefónicas y
diccionarios):

\begin{lstlisting}
>>> 'auto' < 'auxilio'
True
>>> 'auto' < 'autos'
True
>>> 'mes' < 'mesa' < 'mesadas' < 'mesas' < 'meses' < 'mi'
True
\end{lstlisting}

\section{Usos típicos de las tuplas}

Las tuplas se usan siempre que es necesario agrupar valores.
Generalmente, conceptos del mundo real son representados como tuplas que
agrupan información sobre ellos. Por ejemplo, un partido de fútbol se
puede representar como una tupla de los equipos que lo juegan:

\begin{lstlisting}
partido1 = ('Milan', 'Bayern')
\end{lstlisting}

Para representar puntos en el plano, se puede usar tuplas de dos
elementos \lstinline!(x, y)!. Por ejemplo, podemos crear una función
\lstinline!distancia! que recibe dos puntos y entrega la distancia entre
ellos:

\begin{lstlisting}
def distancia(p1, p2):
    x1, y1 = p1
    x2, y2 = p2
    dx = x2 - x1
    dy = y2 - y1
    return (dx ** 2 + dy ** 2) ** 0.5
\end{lstlisting}

Al llamar a la función, se le debe pasar dos tuplas:

\begin{lstlisting}
>>> a = (2, 3)
>>> b = (7, 15)
>>> distancia(a, b)
13.0
\end{lstlisting}

Las fechas generalmente se representan como tuplas agrupando el año, el
mes y el día. La ventaja de hacerlo en este orden (el año primero) es
que las operaciones relacionales permiten saber en qué orden ocurrieron
las fechas:

\begin{lstlisting}
>>> hoy = (2011, 4, 19)
>>> ayer = (2011, 4, 18)
>>> navidad = (2011, 12, 25)
>>> anno_nuevo = (2012, 1, 1)
>>> hoy < ayer
False
>>> hoy < navidad < anno_nuevo
True
\end{lstlisting}

Una tupla puede contener otras tuplas. Por ejemplo, una persona puede
ser descrita por su nombre, su rut y su fecha de nacimiento:

\begin{lstlisting}
persona = ('Perico Los Palotes', '12345678-9', (1980, 5, 14))
\end{lstlisting}

En este caso, los datos se pueden desempaquetar así:

\begin{lstlisting}
>>> nombre, rut, (a, m, d) = persona
>>> m
5
\end{lstlisting}

A veces a uno le interesa sólo uno de los valores de la tupla. Para
evitar crear variables innecesarias, se suele asignar estos valores a la
variable \lstinline!_!. Por ejemplo, si sólo nos interesa el mes en que
nació la persona, podemos obtenerlo así:

\begin{lstlisting}
>>> _, _, (_, mes, _) = persona
>>> mes
5
\end{lstlisting}

Una tabla de datos generalmente se representa como una lista de tuplas.
Por ejemplo, la información de los alumnos que están tomando un ramo
puede ser representada así:

\begin{lstlisting}
alumnos = [
    ('Perico', 'Los Palotes', '201199001-5', 'Civil'),
    ('Fulano', 'De Tal',      '201199002-6', 'Electrica'),
    ('Fulano', 'De Tal',      '201199003-7', 'Mecanica'),
]
\end{lstlisting}

En este caso, se puede desempaquetar los valores automáticamente al
recorrer la lista en un ciclo \lstinline!for!:

\begin{lstlisting}
for nombre, apellido, rol, carrera in alumnos:
    print nombre, 'estudia', carrera
\end{lstlisting}

O, ya que el apellido y el rol no son usados:

\begin{lstlisting}
for nombre, _, _, carrera in alumnos:
    print nombre, 'estudia', carrera
\end{lstlisting}

Es posible crear tuplas de largo uno dejando una coma a continuación del
único valor:

\begin{lstlisting}
>>> t = (12,)
>>> len(t)
1
\end{lstlisting}

En otros lenguajes, las tuplas reciben el nombre de \textbf{registros}.
Este nombre es común, por lo que conviene conocerlo.

\section{Iteración sobre tuplas}

Al igual que las listas, las tuplas son iterables:

\begin{lstlisting}
for valor in (6, 1):
    print valor ** 2
\end{lstlisting}

Además, se puede convertir una tupla en una lista usando la función
\lstinline!list!, y una lista en una tupla usando la función
\lstinline!tuple!:

\begin{lstlisting}
>>> a = (1, 2, 3)
>>> b = [4, 5, 6]
>>> list(a)
[1, 2, 3]
>>> tuple(b)
(4, 5, 6)
\end{lstlisting}

