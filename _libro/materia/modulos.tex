\chapter{Módulos}

Un \textbf{módulo} (o \textbf{biblioteca}) es una colección de
definiciones de variables, funciones y tipos (entre otras cosas) que
pueden ser importadas para ser usadas desde cualquier programa.

Ya hemos visto algunos ejemplos de cómo usar módulos, particularmente el
módulo matemático, del que podemos importar funciones como la
exponencial y el coseno, y las constantes \(\pi\) y \(e\):

\begin{lstlisting}
>>> from math import exp, cos
>>> from math import pi, e
>>> print cos(pi / 3)
0.5
\end{lstlisting}

Las ventajas de usar módulos son:

\begin{itemize}
\item
  las funciones y variables deben ser definidas sólo una vez, y luego
  pueden ser utilizadas en muchos programas sin necesidad de reescribir
  el código;
\item
  permiten que un programa pueda ser organizado en varias secciones
  lógicas, puestas cada una en un archivo separado;
\item
  hacen más fácil compartir componentes con otros programadores.
\end{itemize}

Python viene «de fábrica» con muchos módulos listos para ser usados.
Además, es posible descargar de internet e instalar módulos
prácticamente para hacer cualquier cosa. Aquí aprenderemos a
crear nuestros propios módulos.

\section{Módulos provistos por Python}

Éstos son algunos de los módulos estándares de Python, que pueden ser
usados desde cualquier programa.

El módulo \href{http://docs.python.org/library/math.html}{\lstinline!math!} contiene
funciones y constantes matemáticas:

\begin{lstlisting}
>>> from math import floor, radians
>>> floor(-5.9)
-6.0
>>> radians(180)
3.1415926535897931
\end{lstlisting}

El módulo \href{http://docs.python.org/library/random.html}{\lstinline!random!}
contiene funciones para producir números aleatorios (es decir, al azar):

\begin{lstlisting}
>>> from random import choice, randrange, shuffle
>>> choice(['cara', 'sello'])
'cara'
>>> choice(['cara', 'sello'])
'sello'
>>> choice(['cara', 'sello'])
'sello'
>>> randrange(10)
7
>>> randrange(10)
2
>>> randrange(10)
5
>>> r = range(5)
>>> r
[0, 1, 2, 3, 4]
>>> shuffle(r)
>>> r
[4, 2, 0, 3, 1]
\end{lstlisting}

El módulo \href{http://docs.python.org/library/datetime.html}{\lstinline!datetime!}
provee tipos de datos para manipular fechas y horas:

\begin{lstlisting}
>>> from datetime import date
>>> hoy = date(2011, 5, 31)
>>> fin_del_mundo = date(2012, 12, 21)
>>> (fin_del_mundo - hoy).days
570
\end{lstlisting}

El módulo
\href{http://docs.python.org/library/fractions.html}{\lstinline!fractions!} provee
un tipo de datos para representar números racionales:

\begin{lstlisting}
>>> from fractions import Fraction
>>> a = Fraction(5, 12)
>>> b = Fraction(9, 7)
>>> a + b
Fraction(143, 84)
\end{lstlisting}

El módulo \href{http://docs.python.org/library/turtle.html}{\lstinline!turtle!}
permite manejar una tortuga (¡haga la prueba!):

\begin{lstlisting}
>>> from turtle import Turtle
>>> t = Turtle()
>>> t.forward(10)
>>> t.left(45)
>>> t.forward(20)
>>> t.left(45)
>>> t.forward(30)
>>> for i in range(10):
...   t.right(30)
...   t.forward(10 * i)
...
>>>
\end{lstlisting}

La lista completa de módulos de Python puede ser encontrada en la
\href{http://docs.python.org/library/index.html}{documentación de la
biblioteca estándar}.

\section{Importación de nombres}

La sentencia \lstinline!import! importa objetos desde un módulo para
poder ser usados en el programa actual.

Una manera de usar \lstinline!import! es importar sólo los nombres
específicos que uno desea utilizar en el programa:

\begin{lstlisting}
from math import sin, cos
print sin(10)
print cos(20)
\end{lstlisting}

En este caso, las funciones \lstinline!sin! y \lstinline!cos! no fueron
creadas por nosotros, sino importadas del módulo de matemáticas, donde
están definidas.

La otra manera de usar \lstinline!import! es importando el módulo
completo, y accediendo a sus objetos mediante un punto:

\begin{lstlisting}
import math
print math.sin(10)
print math.cos(10)
\end{lstlisting}

Las dos formas son equivalentes.

\section{Creación de módulos}

Un módulo sencillo es simplemente un archivo con código en Python. El
nombre del archivo indica cuál es el nombre del módulo.

Por ejemplo, podemos crear un archivo llamado \lstinline!pares.py! que
tenga funciones relacionadas con los números pares:

\begin{lstlisting}
def es_par(n):
    return n % 2 == 0

def es_impar(n):
    return not es_par(n)

def pares_hasta(n):
    return range(0, n, 2)
\end{lstlisting}

En este caso, el nombre del módulo es \lstinline!pares!. Para poder usar
estas funciones desde otro programa, el archivo \lstinline!pares.py!
debe estar en la misma carpeta que el programa.

Por ejemplo, un programa \lstinline!mostrar_pares.py! puede ser escrito así:
%
\begin{lstlisting}
from pares import pares_hasta

n = int(raw_input('Ingrese un entero: '))
print 'Los numeros pares hasta', n, 'son:'
for i in pares_hasta(n):
    print i
\end{lstlisting}
%
Y un programa \lstinline!ver_si_es_par.py! puede ser escrito así:

\begin{lstlisting}
import pares

n = int(raw_input('Ingrese un entero: '))
if pares.es_par(n):
    print n, 'es par'
else:
    print n, 'no es par'
\end{lstlisting}

Como se puede ver, ambos programas pueden usar los objetos definidos en
el módulo simplemente importándolos.

\section{Usar módulos como programas}

Un archivo con extensión \lstinline!.py! puede ser un módulo o un
programa. Si es un módulo, contiene definiciones que pueden ser
importadas desde un programa o desde otro módulo. Si es un programa,
contiene código para ser ejecutado.

A veces, un programa también contiene definiciones (por ejemplo,
funciones y variables) que también pueden ser útiles desde otro
programa. Sin embargo, no pueden ser importadas, ya que al usar la
sentencia \lstinline!import! el programa completo sería ejecutado. Lo
que ocurriría en este caso es que, al ejecutar el segundo programa,
también se ejecutaría el primero.

Existe un truco para evitar este problema: siempre que hay código siendo
ejecutado, existe una variable llamada \lstinline!__name__!. Cuando se
trata de un programa, el valor de esta variable es
\lstinline!'__main__'!, mientras que en un módulo, es el nombre del
módulo.

Por lo tanto, se puede usar el valor de esta variable para marcar la
parte del programa que debe ser ejecutada al ejecutar el archivo, pero
no al importarlo.

Por ejemplo, el programa listado a continuación
convierte unidades de medidas de longitud.
Este programa es útil por sí solo, pero además sus cuatro funciones y
las constantes \lstinline!km_por_milla! y \lstinline!cm_por_pulgada!
podrían ser útiles en otro programa:

\begin{lstlisting}
km_por_milla = 1.609344
cm_por_pulgada = 2.54

def millas_a_km(mi):
    return mi * km_por_milla

def km_a_millas(km):
    return km / km_por_milla

def pulgadas_a_cm(p):
    return p * cm_por_pulgada

def cm_a_pulgadas(cm):
    return cm / cm_por_pulgada

if __name__ == '__main__':
    print 'Que conversion desea hacer?'
    print '1) millas a kilometros'
    print '2) kilometros a millas'
    print '3) pulgadas a centimetros'
    print '4) centimetros a pulgadas'
    opcion = int(raw_input('> '))

    if opcion == 1:
        x = float(raw_input('Ingrese millas: '))
        print x, 'millas =', millas_a_km(x), 'km'
    elif opcion == 2:
        x = float(raw_input('Ingrese kilometros: '))
        print x, 'km =', km_a_millas(x), 'millas'
    elif opcion == 3:
        x = float(raw_input('Ingrese pulgadas: '))
        print x, 'in =', pulgadas_a_cm(x), 'cm'
    elif opcion == 4:
        x = float(raw_input('Ingrese centimetros: '))
        print x, 'cm =', cm_a_pulgadas(x), 'in'
    else:
        print 'Opcion incorrecta'
\end{lstlisting}

Al poner el cuerpo del programa dentro del
\lstinline!if __name__ == '__main__'!, el archivo puede ser usado como
un módulo. Si no hiciéramos esto, cada vez que otro programa importe una
función se ejecutaría el programa completo.

Haga la prueba:
ejecute este programa, y luego escriba otro programa que importe alguna
de las funciones. A continuación, haga lo mismo, pero eliminando el
\lstinline!if __name__ == '__main__'!.
