\chapter{Patrones comunes}

Como hemos visto hasta ahora, los programas son una combinación de
asignaciones, condicionales y ciclos, organizados de tal manera que
describan el algoritmo que queremos ejecutar.

Existen algunas tareas muy comunes y que casi siempre se resuelven de la
misma manera. Por lo tanto, es conveniente conocerlas.

En programación, se llama \textbf{patrón} a una solución que es
aplicable a un problema que ocurre a menudo. A continuación veremos
algunos patrones comunes que ocurren en programación.

\section{Sumar y multiplicar cosas}

La suma y la multiplicación son operaciones binarias: operan sobre dos
valores.

Para sumar y multiplicar más valores, generalmente dentro de un ciclo
que los vaya generando, hay que usar una variable para ir guardando el
resultado parcial de la operación. Esta variable se llama
\textbf{acumulador}.

En el caso de la suma, el acumulador debe partir con el valor cero. Para
la multiplicación, con el valor uno. En general, el acumulador debe ser
inicializado con el
\href{http://es.wikipedia.org/wiki/Elemento\_neutro}{elemento neutro} de
la operación que será aplicada.

Por ejemplo, el siguiente programa entrega el producto de los mil
primeros números naturales:

\begin{lstlisting}
producto = 1
for i in range(1, 1001):
    producto = producto * i

print producto
\end{lstlisting}

El siguiente programa entrega la suma de los cubos de los números
naturales cuyo cuadrado es menor que mil:

\begin{lstlisting}
i = 1
suma = 0
while i ** 2 < 1000:
    valor = i ** 3
    i = i + 1
    suma = suma + valor

print suma
\end{lstlisting}

En todos los casos, el patrón a seguir es algo como esto:

\begin{lstlisting}
acumulador = valor_inicial
ciclo:
    valor = ...
    ...
    acumulador = acumulador operacion valor
\end{lstlisting}

El cómo adaptar esta plantilla a cada situación de modo que entregue el
resultado correcto es responsabilidad del programador.

\section{Contar cosas}

Para contar cuántas veces ocurre algo, hay que usar un acumulador, al
que se le suele llamar \textbf{contador}.

Tal como en el caso de la suma, debe ser inicializado en cero, y cada
vez que aparezca lo que queremos contar, hay que incrementarlo en uno.

Por ejemplo, el siguiente programa cuenta cuántos de los números
naturales menores que mil tienen un cubo terminado en siete:

\begin{lstlisting}
c = 0
for i in range(1000):
    ultimo_digito = (i ** 3) % 10
    if ultimo_digito == 7:
        c = c + 1

print c
\end{lstlisting}

\section{Encontrar el mínimo y el máximo}

Para encontrar el máximo de una secuencia de valores, hay que usar un
acumulador para recordar cuál es el mayor valor visto hasta el momento.
En cada iteración, hay que examinar cuál es el valor actual, y si es
mayor que el máximo, actualizar el acumulador.

El acumulador debe ser inicializado con un valor que sea menor a todos
los valores que vayan a ser examinados.

Por ejemplo, el siguiente programa pide al usuario que ingrese diez
números enteros positivos, e indica cuál es el mayor de ellos:

\begin{lstlisting}
print 'Ingrese diez numeros positivos'

mayor = -1
for i in range(10):
    n = int(raw_input())
    if n > mayor:
        mayor = n

print 'El mayor es', mayor
\end{lstlisting}

Otra manera de hacerlo es reemplazando esta parte:
\begin{lstlisting}
if n > mayor:
    mayor = n
\end{lstlisting}
por ésta:
\begin{lstlisting}
mayor = max(mayor, n)
\end{lstlisting}

En este caso, como todos los números ingresados son positivos,
inicializamos el acumulador en \lstinline!-1!, que es menor que todos
los valores posibles, por lo que el que sea el mayor eventualmente lo
reemplazará.

¿Qué hacer cuando no exista un valor inicial que sea menor a todas las
entradas posibles? Una solución tentativa es poner un número «muy negativo», y
rezar para que el usuario no ingrese uno menor que él. Ésta no es la
mejor solución, ya que no cubre todos los casos posibles:

\begin{lstlisting}
mayor = -999999999
for i in range(10):
    n = int(raw_input())
    mayor = max(mayor, n)
\end{lstlisting}

Una opción más robusta es usar el primero de los valores por examinar:

\begin{lstlisting}
mayor = int(raw_input())   # preguntar el primer valor
for i in range(9):         # preguntar los nueve siguientes
    n = int(raw_input())
    mayor = max(mayor, n)
\end{lstlisting}

La otra buena solución es usar explícitamente el valor \(-\infty\), que en
Python puede representarse usando el tipo \lstinline!float! de la
siguiente manera:

\begin{lstlisting}
mayor = -float('inf')     # asi se dice "infinito" en Python
for i in range(10):
    n = int(raw_input())
    mayor = max(mayor, n)
\end{lstlisting}

Si sabemos de antemano que todos los números por revisar son positivos,
podemos simplemente inicializar el acumulador en \(-1\).

Por supuesto, para obtener el menor valor se hace de la misma manera,
pero inicializando el acumulador con un número muy grande, y
actualizándolo al encontrar un valor menor.

\section{Generar pares}

Para generar pares de cosas en un programa, es necesario usar dos ciclos
anidados (es decir, uno dentro del otro).
Ambos ciclos, el exterior y el interior, van asignando valores a sus
variables de control, y ambas son accesibles desde dentro del doble
ciclo.

Por ejemplo, todas las casillas de un tablero de ajedrez pueden ser
identificadas mediante un par \lstinline!(fila, columna)!. Para recorrer
todas las casillas del tablero, se puede hacer de la siguiente manera:

\begin{lstlisting}
for i in range(1, 9):
    for i in range(1, 9):
        print 'Casilla', i, j
\end{lstlisting}

Cuando los pares son desordenados (es decir, el par \((a, b)\) es el mismo
que el par \((b, a)\)), el ciclo interior no debe partir desde cero, sino
desde el valor que tiene la variable de control del ciclo interior.

Por ejemplo, el siguiente programa muestra todas las piezas de un juego
de dominó:

\begin{lstlisting}
for i in range(7):
    for j in range(i, 7):
        print i, j
\end{lstlisting}

Además, otros tipos de restricciones pueden ser necesarias. Por ejemplo,
en un campeonato de fútbol, todos los equipos deben jugar entre ellos
dos veces, una como local y una como visita. Por supuesto, no pueden
jugar consigo mismos, por lo que es necesario excluir los pares
compuestos por dos valores iguales. El siguiente programa muestra todos
los partidos que se deben jugar en un campeonato con 6 equipos,
suponiendo que los equipos están numerados del 0 al 5:

\begin{lstlisting}
for i in range(6):
    for j in range(6):
        if i != j:
            print i, j
\end{lstlisting}

Otra manera de escribir el mismo código es:

\begin{lstlisting}
for i in range(6):
    for j in range(6):
        if i == j:
            continue
        print i, j
\end{lstlisting}

