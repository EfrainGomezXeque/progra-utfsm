\chapter{Arreglos bidimensionales}

Los \textbf{arreglos bidimensionales} son tablas de valores. Cada
elemento de un arreglo bidimensional está simultáneamente en una fila y
en una columna.

En matemáticas, a los arreglos bidimensionales se les llama
\href{http://es.wikipedia.org/wiki/Matriz\_(matem\%C3\%A1tica)}{matrices},
y son muy utilizados en problemas de Ingeniería.

En un arreglo bidimensional, cada elemento tiene una posición que se
identifica mediante dos índices: el de su fila y el de su columna.

\section{Crear arreglos bidimensionales}

Los arreglos bidimensionales también son provistos por NumPy, por lo que
debemos comenzar importando las funciones de este módulo:

\begin{lstlisting}
from numpy import *
\end{lstlisting}

Al igual que los arreglos de una dimensión, los arreglos bidimensionales
también pueden ser creados usando la función \lstinline!array!, pero
pasando como argumentos una lista con las filas de la matriz:

\begin{lstlisting}
a = array([[5.1, 7.4, 3.2, 9.9],
           [1.9, 6.8, 4.1, 2.3],
           [2.9, 6.4, 4.3, 1.4]])
\end{lstlisting}

Todas las filas deben ser del mismo largo, o si no ocurre un error de
valor:

\begin{lstlisting}
>>> array([[1], [2, 3]])
Traceback (most recent call last):
  File "<stdin>", line 1, in <module>
ValueError: setting an array element with a sequence.
\end{lstlisting}

Los arreglos tienen un atributo llamado \lstinline!shape!, que es una
tupla con los tamaños de cada dimensión. En el ejemplo, \lstinline!a! es
un arreglo de dos dimensiones que tiene tres filas y cuatro columnas:

\begin{lstlisting}
>>> a.shape
(3, 4)
\end{lstlisting}

Los arreglos también tienen otro atributo llamado \lstinline!size! que
indica cuántos elementos tiene el arreglo:

\begin{lstlisting}
>>> a.size
12
\end{lstlisting}

Por supuesto, el valor de \lstinline!a.size! siempre es el producto de
los elementos de \lstinline!a.shape!.

Hay que tener cuidado con la función \lstinline!len!, ya que no retorna
el tamaño del arreglo, sino su cantidad de filas:

\begin{lstlisting}
>>> len(a)
3
\end{lstlisting}

Las funciones \lstinline!zeros! y \lstinline!ones! también sirven para
crear arreglos bidimensionales. En vez de pasarles como argumento un
entero, hay que entregarles una tupla con las cantidades de filas y
columnas que tendrá la matriz:

\begin{lstlisting}
>>> zeros((3, 2))
array([[ 0.,  0.],
       [ 0.,  0.],
       [ 0.,  0.]])
>>> ones((2, 5))
array([[ 1.,  1.,  1.,  1.,  1.],
       [ 1.,  1.,  1.,  1.,  1.]])
\end{lstlisting}

Lo mismo se cumple para muchas otras funciones que crean arreglos; por
ejemplom la función \lstinline!random!:

\begin{lstlisting}
>>> from numpy.random import random
>>> random((5, 2))
array([[ 0.80177393,  0.46951148],
       [ 0.37728842,  0.72704627],
       [ 0.56237317,  0.3491332 ],
       [ 0.35710483,  0.44033758],
       [ 0.04107107,  0.47408363]])
\end{lstlisting}

\section{Operaciones con arreglos bidimensionales}

Al igual que los arreglos de una dimensión, las operaciones sobre las
matrices se aplican término a término:

\begin{lstlisting}
>>> a = array([[5, 1, 4],
...            [0, 3, 2]])
>>> b = array([[2, 3, -1],
...            [1, 0, 1]])
>>> a + 2
array([[7, 3, 6],
       [2, 5, 4]])
>>> a ** b
array([[25,  1,  0],
      [ 0,  1,  2]])
\end{lstlisting}

Cuando dos matrices aparecen en una operación, ambas deben tener
exactamente la misma forma:

\begin{lstlisting}
>>> a = array([[5, 1, 4],
...            [0, 3, 2]])
>>> b = array([[ 2,  3],
...            [-1,  1],
...            [ 0,  1]])
>>> a + b
Traceback (most recent call last):
  File "<stdin>", line 1, in <module>
ValueError: shape mismatch: objects cannot be
            broadcast to a single shape
\end{lstlisting}

\section{Obtener elementos de un arreglo bidimensional}

Para obtener un elemento de un arreglo, debe indicarse los índices de su
fila \lstinline!i! y su columna \lstinline!j! mediante la sintaxis
\lstinline!a[i, j]!:

\begin{lstlisting}
>>> a = array([[ 3.21,  5.33,  4.67,  6.41],
               [ 9.54,  0.30,  2.14,  6.57],
               [ 5.62,  0.54,  0.71,  2.56],
               [ 8.19,  2.12,  6.28,  8.76],
               [ 8.72,  1.47,  0.77,  8.78]])
>>> a[1, 2]
2.14
>>> a[4, 3]
8.78
>>> a[-1, -1]
8.78
>>> a[0, -1]
6.41
\end{lstlisting}

También se puede obtener secciones rectangulares del arreglo usando el
operador de rebanado con los índices:

\begin{lstlisting}
>>> a[2:3, 1:4]
array([[ 0.54,  0.71,  2.56]])
>>> a[1:4, 0:4]
array([[ 9.54,  0.3 ,  2.14,  6.57],
       [ 5.62,  0.54,  0.71,  2.56],
       [ 8.19,  2.12,  6.28,  8.76]])
>>> a[1:3, 2]
array([ 2.14,  0.71])
>>> a[0:4:2, 3:0:-1]
array([[ 6.41,  4.67,  5.33],
       [ 2.56,  0.71,  0.54]])
>>> a[::4, ::3]
array([[ 3.21,  6.41],
       [ 8.72,  8.78]])
\end{lstlisting}

Para obtener una fila completa, hay que indicar el índice de la fila, y
poner \lstinline!:! en el de las columnas (significa «desde el principio
hasta el final»). Lo mismo para las columnas:

\begin{lstlisting}
>>> a[2, :]
array([ 5.62,  0.54,  0.71,  2.56])
>>> a[:, 3]
array([ 6.41,  6.57,  2.56,  8.76,  8.78])
\end{lstlisting}

Note que el número de dimensiones es igual a la cantidad de rebanados
que hay en los índices:

\begin{lstlisting}
>>> a[2, 3]      # valor escalar (cero dimensiones)
2.56
>>> a[2:3, 3]    # arreglo de una dimension de 1 elemento
array([ 2.56])
>>> a[2:3, 3:4]  # arreglo de dos dimensiones de 1 x 1
array([[ 2.56]])
\end{lstlisting}

\section{Otras operaciones}

La \textbf{trasposicion} consiste en cambiar las filas por las columnas
y viceversa. Para trasponer un arreglo, se usa el método
\lstinline!transpose!:

\begin{lstlisting}
>>> a
array([[ 3.21,  5.33,  4.67,  6.41],
       [ 9.54,  0.3 ,  2.14,  6.57],
       [ 5.62,  0.54,  0.71,  2.56]])
>>> a.transpose()
array([[ 3.21,  9.54,  5.62],
       [ 5.33,  0.3 ,  0.54],
       [ 4.67,  2.14,  0.71],
       [ 6.41,  6.57,  2.56]])
\end{lstlisting}

El método \lstinline!reshape! entrega un arreglo que tiene los mismos
elementos pero otra forma. El parámetro de \lstinline!reshape! es una
tupla indicando la nueva forma del arreglo:

\begin{lstlisting}
>>> a = arange(12)
>>> a
array([ 0, 1, 2, 3, 4, 5, 6, 7, 8, 9, 10, 11])
>>> a.reshape((4, 3))
array([[ 0, 1, 2],
       [ 3, 4, 5],
       [ 6, 7, 8],
       [ 9, 10, 11]])
>>> a.reshape((2, 6))
array([[ 0, 1, 2, 3, 4, 5],
       [ 6, 7, 8, 9, 10, 11]])
\end{lstlisting}

La función \lstinline!diag! aplicada a un arreglo bidimensional entrega
la diagonal principal de la matriz (es decir, todos los elementos de la
forma \lstinline!a[i, i]!):

\begin{lstlisting}
>>> a
array([[ 3.21,  5.33,  4.67,  6.41],
       [ 9.54,  0.3 ,  2.14,  6.57],
       [ 5.62,  0.54,  0.71,  2.56]])
>>> diag(a)
array([ 3.21,  0.3 ,  0.71])
\end{lstlisting}

Además, \lstinline!diag! recibe un segundo parámetro opcional para
indicar otra diagonal que se desee obtener. Las diagonales sobre la
principal son positivas, y las que están bajo son negativas:

\begin{lstlisting}
>>> diag(a, 2)
array([ 4.67,  6.57])
>>> diag(a, -1)
array([ 9.54,  0.54])
\end{lstlisting}

La misma función \lstinline!diag! también cumple el rol inverso: al
recibir un arreglo de una dimensión, retorna un arreglo bidimensional
que tiene los elementos del parámetro en la diagonal:

\begin{lstlisting}
>>> diag(arange(5))
array([[0, 0, 0, 0, 0],
       [0, 1, 0, 0, 0],
       [0, 0, 2, 0, 0],
       [0, 0, 0, 3, 0],
       [0, 0, 0, 0, 4]])
\end{lstlisting}

