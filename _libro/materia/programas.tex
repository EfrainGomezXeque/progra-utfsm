\chapter{Desarrollo de programas}

Un \textbf{programa} es un archivo de texto que contiene código para ser
ejecutado por el computador.

En el caso del lenguaje Python, el programa es ejecutado por un
\textbf{intérprete}. El intérprete es un programa que ejecuta programas.

Los programas escritos en Python deben estar contenidos en un archivo
que tenga la extensión \lstinline!.py!. En Windows, el programa puede
ser ejecutado haciendo doble clic sobre el ícono del archivo.

Para probar cómo hacerlo, descargue el programa
\href{../\_static/programas/cuadratica.py}{cuadratica.py} que sirve para
resolver ecuaciones cuadráticas.

\section{Edición de programas}

Un programa es un
\href{http://es.wikipedia.org/wiki/Archivo\_de\_texto}{archivo de
texto}. Por lo tanto, puede ser creado y editado usando cualquier
\href{http://es.wikipedia.org/wiki/Editor\_de\_texto}{editor de texto},
como el Bloc de Notas.

Lo que no se puede usar es un procesador de texto, como Microsoft Word.

Haga la prueba: abra el programa \lstinline!cuadratica.py! con el Bloc
de Notas (u otro editor) y verá su contenido.

Otros editores de texto (mucho mejores que el Bloc de Notas) que usted
puede instalar son:

\begin{itemize}
\item
  en Windows: \href{http://notepad-plus-plus.org/}{Notepad++},
  \href{http://www.textpad.com/}{Textpad};
\item
  en Mac:
  \href{http://www.barebones.com/products/textwrangler/}{TextWrangler},
  \href{http://macromates.com/}{TextMate};
\item
  en Linux: \href{http://projects.gnome.org/gedit/}{Gedit},
  \href{http://kate-editor.org/}{Kate}.
\end{itemize}

\section{Instalación del intérprete de Python}

Una cosa es editar el programa, y otra es ejecutarlo. Para poder
ejecutar un programa en Python hay que instalar el \textbf{intérprete}.

En la \href{http://www.python.org/download/}{página de descargas de
Python} está la lista de instaladores. Debe descargar el indicado para
su computador y su sistema operativo.

La versión que debe instalar es la \textbf{2.7.1}, no la 3.1.3.

No use los instaladores que dicen \lstinline!x86-64! a no ser que esté
seguro que su computador tiene una arquitectura de 64 bits (lo más
probable es que no sea así).

\section{Ejecución de un programa}

Una vez escrito el programa e instalado el intérprete, es posible
ejecutar los programas. Para hacerlo, haga doble clic en el ícono del
programa.

\section{Uso de la consola}

La ejecución de programas no es la única manera de ejecutar el
intérprete. Si uno ejecuta Python sin pasarle ningún programa, se abre
la \textbf{consola} (o \textbf{intérprete interactivo}).

La consola permite ingresar un programa línea por línea. Además, sirve
para evaluar expresiones y ver su resultado inmediatamente. Esto permite
usarla como si fuera una calculadora.

La consola interactiva siempre muestra el símbolo \lstinline!>>>!, para
indicar que ahí se puede ingresar código. En todos los libros sobre
Python, y a lo largo de este apunte, cada vez que aparezca un ejemplo en
el que aparezca este símbolo, significa que debe ejecutarse en la
consola, y no en un programa. Por ejemplo:

\begin{lstlisting}
>>> a = 5
>>> a > 10
False
>>> a ** 2
25
\end{lstlisting}

En este ejemplo, al ingresar las expresiones \lstinline!a > 10! y
\lstinline!a ** 2!, el intérprete interactivo entrega los resultados
\lstinline!False! y \lstinline!25!.

No hay ningún motivo para tipear el símbolo \lstinline!>>>! ni en un
programa ni en un certamen, pues no es parte de la sintaxis del
lenguaje.

\section{Entornos de desarollo}

En general, usar un simple editor de texto para escribir programas no es
la manera más eficiente de trabajar.

Los \textbf{entornos de desarrollo} (también llamados \emph{IDE}, por
sus siglas en inglés) son aplicaciones que hacen más fácil la tarea de
escribir programas.

Python viene con su propio entorno de desarrollo llamado \textbf{IDLE}.
IDLE viene con una consola y un editor de texto.

Además, hay otros buenos entornos de desarrollo más avanzados para
Python:

\begin{itemize}
\item
  \href{http://code.google.com/p/pyscripter/downloads/list}{PyScripter},
\item
  \href{http://www.wingware.com/downloads/wingide-101/3.2.12-1/binaries}{WingIDE
  101}
\end{itemize}

Usted puede probar éstos y usar el que más le acomode durante el
semestre.

El siguiente video muestra cómo usar IDLE para desarrollar un programa y
para usar la consola interactiva.

Si desea trabajar con PyScripter en vez de IDLE, puede ver
\href{http://www.youtube.com/watch?v=bzF5rDtQLS4}{este otro video} con
una demostración de cómo usarlo.

