\chapter{Procesamiento de texto}

Hasta ahora, hemos visto cómo los tipos de datos básicos (strings,
enteros, reales, booleanos) y las estructuras de datos permiten
representar y manipular información compleja y abstracta en un programa.

Sin embargo, en muchos casos la información no suele estar disponible ya
organizada en estructuras de datos convenientes de usar, sino en
documentos de texto.

Por ejemplo, las páginas webs son archivos de puro texto, que describen
la estructura de un documento en un lenguaje llamado HTML. Usted puede
ver el texto de una página web buscando una instrucción «Ver código
fuente» (o algo parecido) en el navegador. A partir de este texto, el
navegador extrae la información necesaria para reconstruir la página que
finalmente usted ve.

Un texto siempre es un string, que puede ser tan largo y complejo como
se desee. El procesamiento de texto consiste en manipular strings, ya
sea para extraer información del string, para convertir un texto en
otro, o para codificar información en un string.

En Python, el tipo \lstinline!str! provee muchos métodos convenientes
para hacer procesamiento de texto, además de las operaciones más simples
que ya aprendimos (como \lstinline!s + t!, \lstinline!s[i]! y
\lstinline!s in t!).

\section{Saltos de línea}

Un string puede contener caracteres de \textbf{salto de línea}, que
tienen el efecto equivalente al de presionar la tecla Enter. El caracter
de salto de línea se representa con \lstinline!\n!:

\begin{lstlisting}
>>> a = 'piano\nviolin\noboe'
>>> print a
piano
violin
oboe
\end{lstlisting}

Los saltos de línea sólo son visibles al imprimir el string mediante la
sentencia \lstinline!print!. Si uno quiere ver el valor del string en la
consola, el salto de línea aparecerá representado como \lstinline!\n!:

\begin{lstlisting}
>>> a
'piano\nviolin\noboe'
>>> print a
piano
violin
oboe
\end{lstlisting}

Aunque dentro del string se representa como una secuencia de dos
símbolos, el salto de línea es un único caracter:

\begin{lstlisting}
>>> len('a\nb')
3
\end{lstlisting}

En general, hay varios caracteres especiales que se representan
comenzando con una barra invertida (\lstinline!\!) seguida de una letra.
Experimente, y determine qué significan los caracteres especiales
\lstinline!\t! y \lstinline!\b!:

\begin{lstlisting}
print 'abcde\tefg\thi\tjklm'
print 'abcde\befg\bhi\bjklm'
\end{lstlisting}

Para incluir una barra invertida dentro de un string, hay que hacerlo
con \lstinline!\\!:

\begin{lstlisting}
>>> print 'c:\\>'
c:\>
>>> print '\\o/  o\n |  /|\\\n/ \\ / \\'
\o/  o
 |  /|\
/ \ / \
\end{lstlisting}

\section{Reemplazar secciones del string}

El método \lstinline!s.replace(antes, despues)! busca en el string
\lstinline!s! todas las apariciones del texto \lstinline!antes! y las
reemplaza por \lstinline!despues!:

\begin{lstlisting}
>>> 'La mar astaba sarana'.replace('a', 'e')
'Le mer estebe serene'
>>> 'La mar astaba sarana'.replace('a', 'i')
'Li mir istibi sirini'
>>> 'La mar astaba sarana'.replace('a', 'o')
'Lo mor ostobo sorono'
\end{lstlisting}

Hay que tener siempre muy claro que esta operación no modifica el
string, sino que retorna uno nuevo:

\begin{lstlisting}
>>> orden = 'Quiero arroz con pollo'
>>> orden.replace('arroz', 'pure').replace('pollo', 'huevo')
'Quiero pure con huevo'
>>> orden
'Quiero arroz con pollo'
\end{lstlisting}

\section{Separar y juntar strings}

\lstinline!s.split()! separa el strings en varios strings, usando los
espacios en blanco como separador. El valor retornado es una lista de
strings:

\begin{lstlisting}
>>> oracion = 'El veloz murcielago hindu  comia feliz  cardillo y kiwi'
>>> oracion.split()
['El', 'veloz', 'murcielago', 'hindu', 'comia', 'feliz', 'cardillo', 'y', 'kiwi']
\end{lstlisting}

Además, es posible pasar un parámetro al método \lstinline!split! que
indica cuál será el separador a usar (en vez de los espacios en blanco):

\begin{lstlisting}
>>> s = 'Ana lavaba las sabanas'
>>> s.split()
['Ana', 'lavaba', 'las', 'sabanas']
>>> s.split('a')
['An', ' l', 'v', 'b', ' l', 's s', 'b', 'n', 's']
>>> s.split('l')
['Ana ', 'avaba ', 'as sabanas']
>>> s.split('aba')
['Ana lav', ' las s', 'nas']
\end{lstlisting}

Esto es muy útil para pedir al usuario que ingrese datos en un programa
de una manera más conveniente, y no uno por uno. Por ejemplo, antes
hacíamos programas que funcionaban así:

Ahora podemos hacerlos así:

En este caso, el código del programa podría ser:

\begin{lstlisting}
entrada = raw_input('Ingrese lados del triangulo: ')
lados = entrada.split()
a = int(lados[0])
b = int(lados[1])
c = int(lados[2])
print 'El triangulo es', tipo_triangulo(a, b, c)
\end{lstlisting}

O usando la función \lstinline!map!, más simplemente:

\begin{lstlisting}
entrada = raw_input('Ingrese lados del triangulo: ')
a, b, c = map(int, entrada.split())
print 'El triangulo es', tipo_triangulo(a, b, c)
\end{lstlisting}

\lstinline!s.join(lista_de_strings)! une todos los strings de la lista,
usando al string \lstinline!s! como «pegamento»:

\begin{lstlisting}
>>> valores = map(str, range(10))
>>> pegamento = ' '
>>> pegamento.join(valores)
'0 1 2 3 4 5 6 7 8 9'
>>> ''.join(valores)
'0123456789'
>>> ','.join(valores)
'0,1,2,3,4,5,6,7,8,9'
>>> ' --> '.join(valores)
'0 --> 1 --> 2 --> 3 --> 4 --> 5 --> 6 --> 7 --> 8 --> 9'
\end{lstlisting}

\section{Mayúsculas y minúsculas}

\lstinline!s.isupper()! y \lstinline!s.islower()! indican si el string
está, respectivamente, en mayúsculas o minúsculas:

\begin{lstlisting}
>>> s = 'hola'
>>> t = 'Hola'
>>> u = 'HOLA'
>>> s.isupper(), s.islower()
(False, True)
>>> t.isupper(), t.islower()
(False, False)
>>> u.isupper(), u.islower()
(True, False)
\end{lstlisting}

\lstinline!s.upper()! y \lstinline!s.lower()! entregan el string
\lstinline!s! convertido, respectivamente, a mayúsculas y minúsculas:

\begin{lstlisting}
>>> t
'Hola'
>>> t.upper()
'HOLA'
>>> t.lower()
'hola'
\end{lstlisting}

\lstinline!s.swapcase()! cambia las minúsculas a mayúsculas,
respectivamente, a mayúsculas y minúsculas:

\begin{lstlisting}
>>> t.swapcase()
'hOLA'
\end{lstlisting}

Lamentablemente, ninguno de estos métodos funcionan con acentos y eñes:

\begin{lstlisting}
>>> print 'ñandú'.upper()
ñANDú
\end{lstlisting}

\section{Revisar contenidos del string}

\lstinline!s.startswith(t)! y \lstinline!s.endswith(t)! indican si el
string \lstinline!s! comienza y termina, respectivamente, con el string
\lstinline!t!:

\begin{lstlisting}
>>> objeto = 'paraguas'
>>> objeto.startswith('para')
True
>>> objeto.endswith('aguas')
True
>>> objeto.endswith('x')
False
>>> objeto.endswith('guaguas')
False
\end{lstlisting}

Nuestro conocido operador \lstinline!in! indica si un string está
contenido dentro de otro:

\begin{lstlisting}
>>> 'pollo' in 'repollos'
True
>>> 'pollo' in 'gallinero'
False
\end{lstlisting}

\section{Alineación de strings}

Los métodos \lstinline!s.ljust(n)!, \lstinline!s.rjust(n)! y
\lstinline!s.center(n)! rellenan el string con espacios para que su
largo sea igual a \lstinline!n!, de modo que el contenido quede
alineado, respectivamente, a la izquierda, a la derecha y al centro:

\begin{lstlisting}
>>> contenido.ljust(20)
'hola                '
>>> contenido.center(20)
'        hola        '
>>> contenido.rjust(20)
'                hola'
\end{lstlisting}

Estos métodos son útiles para imprimir tablas bien alineadas:

\begin{lstlisting}
datos = [
    ('Pepito', (1991, 12, 5), 'Osorno', '***'),
    ('Yayita', (1990, 1, 31), 'Arica', '*'),
    ('Fulanito', (1992, 10, 29), 'Porvenir', '****'),
]

for n, (a, m, d), c, e in datos:
    print n.ljust(10),
    print str(a).rjust(4), str(m).rjust(2), str(d).rjust(2),
    print c.ljust(10), e.center(5)
\end{lstlisting}

Este programa imprime lo siguiente:

\section{Interpolación de strings}

El método \lstinline!format! permite usar un string como una plantilla
que se puede completar con distintos valores dependiendo de la
situación.

Las posiciones en que se deben rellenar los valores se indican dentro
del string usando un número entre paréntesis de llaves:

\begin{lstlisting}
>>> s = 'Soy {0} y vivo en {1}'
\end{lstlisting}

Estas posiciones se llaman \emph{campos}. En el ejemplo, el string
\lstinline!s! tiene dos campos, numerados del cero al uno.

Para llenar los campos, hay que llamar al método \lstinline!format!
pasándole los valores como parámetros:

\begin{lstlisting}
>>> s.format('Perico', 'Valparaiso')
'Soy Perico y vivo en Valparaiso'
>>> s.format('Erika', 'Berlin')
'Soy Erika y vivo en Berlin'
>>> s.format('Wang Dawei', 'Beijing')
'Soy Wang Dawei y vivo en Beijing'
\end{lstlisting}

El número indica en qué posición va el parámetro que está asociado al
campo:

\begin{lstlisting}
>>> '{1}{0}{2}{0}'.format('a', 'v', 'c')
'vaca'
>>> '{0} y {1}'.format('carne', 'huevos')
'carne y huevos'
>>> '{1} y {0}'.format('carne', 'huevos')
'huevos y carne'
\end{lstlisting}

Otra opción es referirse a los campos con un nombre. En este caso, hay
que llamar al método \lstinline!format! diciendo explícitamente el
nombre del parámetro para asociarlo al valor:

\begin{lstlisting}
>>> s = '{nombre} estudia en la {universidad}'
>>> s.format(nombre='Perico', universidad='UTFSM')
'Perico estudia en la UTFSM'
>>> s.format(nombre='Fulana', universidad='PUCV')
'Fulana estudia en la PUCV'
>>> s.format(universidad='UPLA', nombre='Yayita')
'Yayita estudia en la UPLA'
\end{lstlisting}

