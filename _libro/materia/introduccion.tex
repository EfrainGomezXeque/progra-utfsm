\section{Introducción a la programación}

\begin{quote}
\emph{Se suele decir que una persona no entiende algo de verdad hasta
que puede explicárselo a otro. En realidad, no lo entiende de verdad
hasta que puede explicárselo a un computador.} ---
\href{http://es.wikipedia.org/wiki/Donald_Knuth}{Donald Knuth}.
\end{quote}

Si tuvieramos que resumir el propósito de la programación en una frase,
ésta debería ser:

\begin{quote}
que el computador haga el trabajo por nosotros.
\end{quote}

Los computadores son buenos para hacer tareas rutinarias. Idealmente,
cualquier problema tedioso y repetitivo debería ser resuelto por un
computador, y los seres humanos sólo deberíamos encargarnos de los
problemas realmente interesantes: los que requieren creatividad,
pensamiento crítico y subjetividad.

La \textbf{programación} es el proceso de transformar un método para
resolver problemas en uno que pueda ser entendido por el computador.

\subsection{Algoritmos}

\begin{quote}
\emph{La informática se trata de computadores tanto como la astronomía
se trata de telescopios}. ---
\href{http://es.wikipedia.org/wiki/Edsger_Dijkstra}{Edsger Dijkstra}.
\end{quote}

Al diseñar un programa, el desafío principal es crear y describir un
procedimiento que esté completamente bien definido, que no tenga
ambigüedades, y que efectivamente resuelva el problema.

Así es como la programación no es tanto sobre computadores, sino sobre
resolver problemas de manera estructurada. El objeto de estudio de la
programación no son los programas, sino los algoritmos.

Un \textbf{algoritmo} es un procedimiento bien definido para resolver un
problema.

Todo el mundo conoce y utiliza algoritmos a diario, incluso sin darse
cuenta:

\begin{itemize}
\item
  Una receta de cocina es un algoritmo; si bien podríamos cuestionar que
  algunos pasos son ambiguos (¿cuánto es «una pizca de sal»? ¿qué
  significa «agregar a gusto»?), en general las instrucciones están lo
  suficientemente bien definidas para que uno las pueda seguir sin
  problemas.

  La entrada de una receta son los ingredientes y algunos datos como:
  ¿para cuántas personas se cocinará? El proceso es la serie de pasos
  para manipular los ingredientes. La salida es el plato terminado.

  En principio, si una receta está suficientemente bien explicada,
  podría permitir preparar un plato a alguien que no sepa nada de
  cocina.
\end{itemize}

\begin{itemize}
\item
  El 
  %\href{http://es.wikipedia.org/wiki/Algoritmo_de_multiplicación}
  números a mano que aprendimos en el colegio es un
  algoritmo. Dado cualquier par de números enteros, si seguimos paso a
  paso el procedimiento siempre obtendremos el producto:

  %\includegraphics{../diagramas/multiplicacion.gif}

  La entrada del algoritmo de multiplicación son los dos factores. El
  proceso es la secuencia de pasos en que los dígitos van siendo
  multiplicados las reservas van siendo sumadas, y los productos
  intermedios son finalmente sumados. La salida del algoritmo es el
  producto obtenido.
\end{itemize}

Un algoritmo debe poder ser usado mecánicamente, sin necesidad de usar
inteligencia, intuición ni habilidad.

A lo largo de esta asignatura, haremos un recorrido por los conceptos
elementales de la programación, con énfasis en el aspecto práctico de la
disciplina.

Al final del semestre, usted tendrá la capacidad de identificar
problemas que pueden ser resueltos por el computador, y de diseñar y
escribir programas sencillos. Además, entenderá qué es lo que ocurre
dentro del computador los programas que usted usa.

\begin{quote}
\emph{Los computadores son inútiles: sólo pueden darte respuestas}. ---
\href{http://es.wikipedia.org/wiki/Pablo_Picasso}{Pablo Picasso}.
\end{quote}
