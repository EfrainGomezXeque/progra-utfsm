\chapter{Funciones}

Supongamos que necesitamos escribir un programa que calcule el
%\href{http://es.wikipedia.org/wiki/Número_combinatorio}
\(C(m, n)\), definido como:
\[C(m, n) = \frac{m!}{(m - n)!\,n!},\]
donde \(n!\) (el \href{http://es.wikipedia.org/wiki/Factorial}{factorial}
de \(n\)) es el producto de los números enteros desde 1 hasta \(n\):

\[n! = 1\cdot 2\cdot\cdots\cdot(n - 1)\cdot n = \prod_{i=1}^n i.\]

El código para calcular el factorial de un número entero \(n\) es
sencillo:

\begin{lstlisting}
f = 1
for i in range(1, n + 1):
    f *= i
\end{lstlisting}

Sin embargo, para calcular el número combinatorio, hay que hacer lo
mismo tres veces:

\begin{lstlisting}
comb = 1

# multiplicar por m!
f = 1
for i in range(1, m + 1):
    f = f * i
comb = comb * f

# dividir por (m - n)!
f = 1
for i in range(1, m - n + 1):
    f = f * i
comb = comb / f

# dividir por n!
f = 1
for i in range(1, n + 1):
    f = f * i
comb = comb / f
\end{lstlisting}

La única diferencia entre los tres cálculos de factoriales es el valor
de término de cada ciclo \lstinline!for! (\lstinline!m!,
\lstinline!m - n! y \lstinline!n!, respectivamente).

Escribir el mismo código varias veces es tedioso y propenso a errores.
Además, el código resultante es mucho más dificil de entender, pues no
es evidente a simple vista qué es lo que hace.

Lo ideal sería que existiera una función llamada \lstinline!factorial!
que hiciera el trabajo sucio, y que pudiéramos usar de la siguiente
manera:

\begin{lstlisting}
factorial(m) / (factorial(m - n) * factorial(n))
\end{lstlisting}

Ya vimos anteriormente que Python ofrece «de fábrica» algunas funciones,
como \lstinline!int!, \lstinline!min! y \lstinline!abs!. Ahora veremos
cómo crear nuestras propias funciones.

\section{Funciones}

En programación, una \textbf{función} es una sección de un programa que
calcula un valor de manera independiente al resto del programa.

Una función tiene tres componentes importantes:

\begin{itemize}
\item
  los \textbf{parámetros}, que son los valores que recibe la función
  como entrada;
\item
  el \textbf{código de la función}, que son las operaciones que hace la
  función; y
\item
  el \textbf{resultado} (o \textbf{valor de retorno}), que es el valor
  final que entrega la función.
\end{itemize}

En esencia, una función es un mini programa. Sus tres componentes son
análogos a la entrada, el proceso y la salida de un programa.

En el ejemplo del factorial, el parámetro es el entero al que queremos
calcularle el factorial, el código es el ciclo que hace las
multiplicaciones, y el resultado es el valor calculado.

\section{Definición de funciones}

Las funciones en Python son creadas con la sentencia
\lstinline!def!:

\begin{lstlisting}
def nombre(parametros):
    # codigo de la funcion
\end{lstlisting}

Los parámetros son variables en las que quedan almacenados los valores
de entrada.

La función contiene código igual al de cualquier programa. La diferencia
es que, al terminar, debe entregar su resultado usando la sentencia
\lstinline!return!.

Por ejemplo, la función para calcular el factorial puede ser definida de
la siguiente manera:

\begin{lstlisting}
def factorial(n):
    f = 1
    for i in range(1, n + 1):
        f *= i
    return f
\end{lstlisting}

En este ejemplo, el resultado que entrega una llamada a la función es el
valor que tiene la variable \lstinline!f! al llegar a la última línea de
la función.

Una vez creada, la función puede ser usada como cualquier otra, todas
las veces que sea necesario:

\begin{lstlisting}
>>> factorial(0)
1
>>> factorial(12) + factorial(10)
482630400
>>> factorial(factorial(3))
720
>>> n = 3
>>> factorial(n ** 2)
362880
\end{lstlisting}

Las variables que son creadas dentro de la función (incluyendo los
pará\-me\-tros y el resultado) se llaman \textbf{variables locales}, y sólo
son visibles dentro de la función, no desde el resto del programa.

Por otra parte, las variables creadas fuera de alguna función se llaman
\textbf{variables globales}, y son visibles desde cualquier parte del
programa. Sin embargo, su valor no puede ser modificado, ya que una
asignación crearía una variable local del mismo nombre.

En el ejemplo, las variables locales son \lstinline!n!, \lstinline!f! e
\lstinline!i!. Una vez que la llamada a la función termina, estas
variables dejan de existir:

\begin{lstlisting}
>>> factorial(5)
120
>>> f
Traceback (most recent call last):
  File "<console>", line 1, in <module>
NameError: name 'f' is not defined
\end{lstlisting}

Después de definir la función \lstinline!factorial!, podemos crear otra
función llamada \lstinline!comb! para calcular números combinatorios:

\begin{lstlisting}
def comb(m, n):
    fact_m = factorial(m)
    fact_n = factorial(n)
    fact_m_n = factorial(m - n)
    c = fact_m / (fact_n * fact_m_n)
    return c
\end{lstlisting}

Esta función llama a \lstinline!factorial! tres veces, y luego usa los
resultados para calcular su resultado. La misma función puede ser
escrita también de forma más sucinta:

\begin{lstlisting}
def comb(m, n):
    return factorial(m) / (factorial(n) * factorial(m - n))
\end{lstlisting}

El programa completo es el siguiente:
\begin{lstlisting}
def factorial(n):
    p = 1
    for i in range(1, n + 1):
        p *= i
    return p


def comb(m, n):
    return factorial(m) / (factorial(n) * factorial(m - n))


m = int(raw_input('Ingrese m: '))
n = int(raw_input('Ingrese n: '))
c = comb(m, n)
print '(m n) =', c
\end{lstlisting}



Note que, gracias al uso de las funciones, la parte principal del
programa ahora tiene sólo cuatro líneas, y es mucho más fácil de
entender.

\section{Múltiples valores de retorno}

En Python, una función puede retornar más de un valor.
Por ejemplo, la siguiente función recibe una cantidad de segundos, y
retorna el equivalente en horas, minutos y segundos:

\begin{lstlisting}
def convertir_segundos(segundos):
    horas = segundos / (60 * 60)
    minutos = (segundos / 60) % 60
    segundos = segundos % 60
    return horas, minutos, segundos
\end{lstlisting}

Al llamar la función, se puede asignar un nombre a cada uno de los
valores retornados:

\begin{lstlisting}
>>> h, m, s = convertir_segundos(9814)
>>> h
2
>>> m
43
>>> s
34
\end{lstlisting}

Técnicamente, la función está retornando una \textbf{tupla} de valores,
un tipo de datos que veremos más adelante:

\begin{lstlisting}
>>> convertir_segundos(9814)
(2, 43, 34)
\end{lstlisting}

\section{Funciones que no retornan nada}

Una función puede realizar acciones sin entregar necesariamente un
resultado.
Por ejemplo, si un programa necesita imprimir cierta información muchas
veces, conviene encapsular esta acción en una función que haga los
\lstinline!print!

\begin{lstlisting}
def imprimir_datos(nombre, apellido, rol, dia, mes, anno):
    print 'Nombre completo:', nombre, apellido
    print 'Rol:', rol
    print 'Fecha de nacimiento:', dia, '/', mes, '/', anno

imprimir_datos('Perico', 'Los Palotes', '201101001-1',  3, 1, 1993)
imprimir_datos('Yayita', 'Vinagre',     '201101002-2', 10, 9, 1992)
imprimir_datos('Fulano', 'De Tal',      '201101003-3', 14, 5, 1990)
\end{lstlisting}

En este caso, cada llamada a la función \lstinline!imprimir_datos!
muestra los datos en la pantalla, pero no entrega un resultado. Este
tipo de funciones son conocidas en programación como
\textbf{procedimientos} o \textbf{subrutinas}, pero en Python son
funciones como cualquier otra.

Técnicamente, todas las funciones retornan valores. En el caso de las
funciones que no tienen una sentencia \lstinline!return!, el valor de
retorno siempre es \lstinline!None!. Pero como la llamada a la función
no aparece en una asignación, el valor se pierde, y no tiene ningún
efecto en el programa.
