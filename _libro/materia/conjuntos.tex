\chapter{Conjuntos}

Un \textbf{conjunto} es una colección desordenada de valores no
repetidos.
Los conjuntos de Python son análogos a los conjuntos matemáticos. El
tipo de datos que representa a los conjuntos se llama \lstinline!set!.
El tipo \lstinline!set! es mutable: una vez que se ha creado un
conjunto, puede ser modificado.

\section{Cómo crear conjuntos}

Las dos maneras principales de crear un conjunto son:

\begin{itemize}
\item
  usar un conjunto literal, entre llaves:
\begin{lstlisting}
>>> colores = {'azul', 'rojo', 'blanco', 'blanco'}
>>> colores
{'rojo', 'azul', 'blanco'}
\end{lstlisting}
  Note que el conjunto no incluye elementos repetidos, y que los
  elementos no quedan en el mismo orden en que fueron agregados.

\item
  usar la función \lstinline!set! aplicada sobre un iterable:
\begin{lstlisting}
>>> set('abracadabra')
{'a', 'r', 'b', 'c', 'd'}
>>> set(range(50, 2000, 400))
{1250, 50, 1650, 850, 450}
>>> set([(1, 2, 3), (4, 5), (6, 7, 8, 9)])
{(4, 5), (6, 7, 8, 9), (1, 2, 3)}
\end{lstlisting}

  El conjunto vacío debe ser creado usando \lstinline!set()!, ya que
  \lstinline!{}! representa al diccionario vacío.
\end{itemize}

Los elementos de un conjunto deben ser inmutables. Por ejemplo, no es
posible crear un conjunto de listas, pero sí un conjunto de tuplas:

\begin{lstlisting}
>>> s = {[2, 4], [6, 1]}
Traceback (most recent call last):
  File "<stdin>", line 1, in <module>
TypeError: unhashable type: 'list'
>>> s = {(2, 4), (6, 1)}
>>>
\end{lstlisting}

Como un conjunto no es ordenado, no tiene sentido intentar obtener un
elemento usando un índice:

\begin{lstlisting}
>>> s = {'a', 'b', 'c'}
>>> s[0]
Traceback (most recent call last):
  File "<stdin>", line 1, in <module>
TypeError: 'set' object does not support indexing
\end{lstlisting}

Sin embargo, sí es posible iterar sobre un conjunto usando un ciclo
\lstinline!for!:

\begin{lstlisting}
>>> for i in {'a', 'b', 'c'}:
...     print i
...
a
c
b
\end{lstlisting}

\section{Operaciones sobre conjuntos}

\lstinline!len(s)! entrega el número de elementos del conjunto
\lstinline!s!:

\begin{lstlisting}
>>> len({'azul', 'verde', 'rojo'})
3
>>> len(set('abracadabra'))
5
>>> len(set())
0
\end{lstlisting}

\lstinline!x in s! permite saber si el elemento \lstinline!x! está en el
conjunto \lstinline!s!:

\begin{lstlisting}
>>> 3 in {2, 3, 4}
True
>>> 5 in {2, 3, 4}
False
\end{lstlisting}

\lstinline!x not in s! permite saber si \lstinline!x! no está en
\lstinline!s!:

\begin{lstlisting}
>>> 10 not in {2, 3, 4}
True
\end{lstlisting}

\lstinline!s.add(x)! agrega el elemento \lstinline!x! al conjunto
\lstinline!s!:

\begin{lstlisting}
>>> s = {6, 1, 5, 4, 3}
>>> s.add(-37)
>>> s
{1, 3, 4, 5, 6, -37}
>>> s.add(4)
>>> s
{1, 3, 4, 5, 6, -37}
\end{lstlisting}

\lstinline!s.remove(x)! elimina el elemento \lstinline!x! del conjunto
\lstinline!s!:

\begin{lstlisting}
>>> s = {6, 1, 5, 4, 3}
>>> s.remove(1)
>>> s
{3, 4, 5, 6}
\end{lstlisting}

Si el elemento \lstinline!x! no está en el conjunto, ocurre un
\textbf{error de llave}:

\begin{lstlisting}
>>> s.remove(10)
Traceback (most recent call last):
  File "<stdin>", line 1, in <module>
KeyError: 10
\end{lstlisting}

\lstinline!&! y \lstinline!|! son, respectivamente, los operadores de
intersección y unión:

\begin{lstlisting}
>>> a = {1, 2, 3, 4}
>>> b = {2, 4, 6, 8}
>>> a & b
{2, 4}
>>> a | b
{1, 2, 3, 4, 6, 8}
\end{lstlisting}

\lstinline!s - t! entrega la diferencia entre \lstinline!s! y
\lstinline!t!; es decir, los elementos de \lstinline!s! que no están en
\lstinline!t!:

\begin{lstlisting}
>>> a - b
{1, 3}
\end{lstlisting}

\lstinline!s ^ t! entrega la diferencia simétrica entre \lstinline!s! y
\lstinline!t!; es decir, los elementos que están en \lstinline!s! o en
\lstinline!t!, pero no en ambos:

\begin{lstlisting}
>>> a ^ b
{1, 3, 6, 8}
\end{lstlisting}

El operador \lstinline!<! aplicado sobre conjuntos significa «es
subconjunto de»:

\begin{lstlisting}
>>> {1, 2} < {1, 2, 3}
True
>>> {1, 4} < {1, 2, 3}
False
\end{lstlisting}

\lstinline!s <= t! también indica si \lstinline!s! es subconjunto de
\lstinline!t!. La distinción ocurre cuando los conjuntos son iguales:

\begin{lstlisting}
>>> {1, 2, 3} < {1, 2, 3}
False
>>> {1, 2, 3} <= {1, 2, 3}
True
\end{lstlisting}

