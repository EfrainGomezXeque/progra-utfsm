\chapter{Errores y excepciones}

No siempre los programas que escribiremos están correctos. Existen
muchos tipos de errores que pueden estar presentes en un programa.

No todos los errores pueden ser detectados por el computador. Por
ejemplo, el siguiente programa tiene un error lógico bastante evidente:
\begin{lstlisting}
n = int(raw_input('Escriba un numero: '))
doble = 3 * n
print 'El doble de n es', doble
\end{lstlisting}

El computador no se dará cuenta del error, pues todas las instrucciones
del programa son correctas. El programa simplemente entregará siempre la
respuesta equivocada.

Existen otros errores que sí pueden ser detectados. Cuando un error es
detectado \emph{durante} la ejecución del programa ocurre una
\textbf{excepción}.

El intérprete anuncia una excepción deteniendo el programa y mostrando
un mensaje con la descripción del error. Por ejemplo, podemos crear el
siguiente programa y llamarlo \lstinline!division.py!:

\begin{lstlisting}
n = 8
m = 0
print n / m
print 'Listo'
\end{lstlisting}

Al ejecutarlo, el intérprete lanzará una excepción, pues la división por
cero es una operación inválida:

\begin{lstlisting}
Traceback (most recent call last):
  File "division.py", line 3, in <module>
    print n / m
ZeroDivisionError: division by zero
\end{lstlisting}

La segunda línea del mensaje indica cómo se llama el archivo donde está
el error y en qué línea del archivo está. En este ejemplo, el error esta
en la línea 3 de \lstinline!division.py!. La última línea muestra el
nombre de la excepción (en este caso es \lstinline!ZeroDivisionError!) y
un mensaje explicando cuál es el error.

Los errores y excepciones presentados aquí son los más básicos y
comunes.

\section{Error de sintaxis}

Un \textbf{error de sintaxis} ocurre cuando el programa no cumple las
reglas del lenguaje. Cuando ocurre este error, significa que el programa
está mal escrito. El nombre del error es \lstinline!SyntaxError!.

Los errores de sintaxis siempre ocurren \emph{antes} de que el programa
sea ejecutado. Es decir, un programa mal escrito no logra ejecutar
ninguna instrucción. Por lo mismo, el error de sintaxis no es una
excepción.

A continuación veremos algunos ejemplos de errores de sintaxis

\begin{lstlisting}
>>> 2 * (3 + 4))
  File "<stdin>", line 1
    2 * (3 + 4))
               ^
SyntaxError: invalid syntax
\end{lstlisting}

\begin{lstlisting}
>>> n + 2 = 7
  File "<stdin>", line 1
SyntaxError: can't assign to operator
\end{lstlisting}

\begin{lstlisting}
>>> True = 1000
  File "<stdin>", line 1
SyntaxError: assignment to keyword
\end{lstlisting}

\section{Error de nombre}

Un \textbf{error de nombre} ocurre al usar una variable que no ha sido
creada con anterioridad.
El nombre de la excepción es \lstinline!NameError!:
\begin{lstlisting}
>>> x = 20
>>> 5 * x
100
>>> 5 * y
Traceback (most recent call last):
  File "<stdin>", line 1, in <module>
NameError: name 'y' is not defined
\end{lstlisting}

Para solucionar este error, es necesario asignar un valor a la variable
antes de usarla.

\section{Error de tipo}

En general, todas las operaciones en un programa pueden ser aplicadas
sobre valores de tipos bien específicos. Un \textbf{error de tipo}
ocurre al aplicar una operación sobre operandos de tipo incorrecto.
El nombre de la excepción es \lstinline!TypeError!.

Por ejemplo, no se puede multiplicar dos strings:

\begin{lstlisting}
>>> 'seis' * 'ocho'
Traceback (most recent call last):
  File "<stdin>", line 1, in <module>
TypeError: can't multiply sequence by non-int of type 'str'
\end{lstlisting}

Tampoco se puede obtener el largo de un número:

\begin{lstlisting}
>>> len(68)
Traceback (most recent call last):
  File "<stdin>", line 1, in <module>
TypeError: object of type 'int' has no len()
\end{lstlisting}

Cuando ocurre un error de tipo, generalmente el programa está mal
diseñado. Hay que revisarlo, idealmente hacer un ruteo para entender el
error, y finalmente corregirlo.

\section{Error de valor}

El \textbf{error de valor} ocurre cuando los operandos son del tipo
correcto, pero la operación no tiene sentido para ese valor.
El nombre de la excepción es \lstinline!ValueError!.

Por ejemplo, la función \lstinline!int! puede convertir un string a un
entero, pero el string debe ser la representación de un número entero.
Cualquier otro valor lanza un error de valor:

\begin{lstlisting}
>>> int('41')
41
>>> int('perro')
Traceback (most recent call last):
  File "<stdin>", line 1, in <module>
ValueError: invalid literal for int() with base 10: 'perro'
>>> int('cuarenta y uno')
Traceback (most recent call last):
  File "<stdin>", line 1, in <module>
ValueError: invalid literal for int() with base 10: 'cuarenta y uno'
\end{lstlisting}

Para corregir el error, hay que preocuparse de siempre usar valores
adecuados.

\section{Error de división por cero}

El \textbf{error de division por cero} ocurre al intentar dividir por cero.
El nombre de la excepción es \lstinline!ZeroDivisionError!:

\begin{lstlisting}
>>> 1/0
Traceback (most recent call last):
  File "<stdin>", line 1, in <module>
ZeroDivisionError: division by zero
\end{lstlisting}

\section{Error de desborde}

El \textbf{error de desborde} ocurre cuando el resultado de una
operación es tan grande que el computador no puede representarlo
internamente.
El nombre de la excepción es \lstinline!OverflowError!:

\begin{lstlisting}
>>> 20.0 ** 20.0 ** 20.0
Traceback (most recent call last):
  File "<stdin>", line 1, in <module>
OverflowError: (34, 'Numerical result out of range')
\end{lstlisting}

Los interesados en profundizar más pueden revisar
\href{http://docs.python.org/library/exceptions.html}{la sección sobre
excepciones} en la documentación oficial de Python.
