\section{Listas}

Una \textbf{lista} es una colección ordenada de valores. Una lista puede
contener cualquier cosa.

En Python, el tipo de datos que representa a las listas se llama
\lstinline!list!.

\subsection{Cómo crear listas}

Las dos maneras principales de crear una lista son:

\begin{itemize}
\item
  usar una lista literal, con los valores entre corchetes:

\begin{lstlisting}
>>> primos = [2, 3, 5, 7, 11]
>>> primos
[2, 3, 5, 7, 11]
>>> []
[]
>>> [1.0 + 2.0, 3.0 + 4.0 + 5.0]
[3.0, 12.0]
>>> ['hola ' + 'mundo', 24 * 7, True or False]
['hola mundo', 168, True]
\end{lstlisting}
\end{itemize}

\begin{itemize}
\item
  usar la función \lstinline!list! aplicada sobre un iterable:

\begin{lstlisting}
>>> list('hola')
['h', 'o', 'l', 'a']
>>> list(range(10))
[0, 1, 2, 3, 4, 5, 6, 7, 8, 9]
>>> list()
[]
\end{lstlisting}
\end{itemize}

\subsection{Operaciones sobre listas}

\lstinline!len(l)! entrega el largo de la lista; es decir, cuántos
elementos tiene:

\begin{lstlisting}
>>> colores = ['azul', 'rojo', 'verde', 'amarillo']
>>> len(colores)
4
>>> len([True, True, True])
3
>>> len([])
0
\end{lstlisting}

\lstinline!l[i]! entrega el \lstinline!i!-ésimo valor de la lista. El
valor \lstinline!i! se llama \textbf{índice} del valor. Al igual que
para los strings, los índices parten de cero:

\begin{lstlisting}
>>> colores = ['azul', 'rojo', 'verde', 'amarillo']
>>> colores[0]
'azul'
>>> colores[3]
'amarillo'
\end{lstlisting}

Además, es posible modificar el valor del \lstinline!i!-ésimo elemento:

\begin{lstlisting}
>>> colores[1] = 'negro'
>>> colores
['azul', 'negro', 'verde', 'amarillo']
\end{lstlisting}

Si el índice \lstinline!i! indica un elemento que no está en la lista,
ocurre un \textbf{error de índice}:

\begin{lstlisting}
>>> colores[4]
Traceback (most recent call last):
  File "<stdin>", line 1, in <module>
IndexError: list index out of range
\end{lstlisting}

Si el índice es negativo, los elementos se cuentan desde el final hacia
atrás:

\begin{lstlisting}
>>> colores[-1]
'amarillo'
>>> colores[-4]
'azul'
\end{lstlisting}

\lstinline!l.append(x)! agrega el elemento \lstinline!x! al final de la
lista:

\begin{lstlisting}
>>> primos = [2, 3, 5, 7, 11]
>>> primos.append(13)
>>> primos.append(17)
>>> primos
[2, 3, 5, 7, 11, 13, 17]
\end{lstlisting}

Un comentario al margen: \lstinline!append! es un \textbf{método}. Los
métodos son funciones que están dentro de un objeto. Cada lista tiene su
propia función \lstinline!append!. Es importante tener esta distinción
clara, ya que hay operaciones que están implementadas como funciones y
otras como métodos.

\lstinline!sum(x)! entrega la suma de los valores de la lista:

\begin{lstlisting}
>>> sum([1, 2, 1, -1, -2])
1
>>> sum([])
0
\end{lstlisting}

\lstinline!l1 + l2! concatena las listas \lstinline!l1! y
\lstinline!l2!:

\begin{lstlisting}
>>> list('perro') + [2, 3, 4]
['p', 'e', 'r', 'r', 'o', 2, 3, 4]
\end{lstlisting}

\lstinline!l * n! repite \lstinline!n! veces la lista \lstinline!l!:

\begin{lstlisting}
>>> [3.14, 6.28, 9.42] * 2
[3.14, 6.28, 9.42, 3.14, 6.28, 9.42]
>>> [3.14, 6.28, 9.42] * 0
[]
\end{lstlisting}

Para saber si un elemento \lstinline!x! está en la lista \lstinline!l!,
se usa \lstinline!x in l!. La versión negativa de \lstinline!in! es
\lstinline!not in!:

\begin{lstlisting}
>>> r = range(0, 20, 2)
>>> r
[0, 2, 4, 6, 8, 10, 12, 14, 16, 18]
>>> 12 in r
True
>>> 15 in r
False
>>> 15 not in r
True
\end{lstlisting}

\lstinline!l[i:j]! es el operador de rebanado, que entrega una nueva
lista que tiene desde el \lstinline!i!-ésimo hasta justo antes del
\lstinline!j!-ésimo elemento de la lista \lstinline!l!:

\begin{lstlisting}
>>> x = [1.5, 3.3, 8.4, 3.1, 2.9]
>>> x[2:4]
[8.4, 3.1]
\end{lstlisting}

\lstinline!l.count(x)! cuenta cuántas veces está el elemento
\lstinline!x! en la lista:

\begin{lstlisting}
>>> letras = list('paralelepipedo')
>>> letras.count('p')
3
\end{lstlisting}

\lstinline!l.index(x)! entrega cuál es el índice del valor
\lstinline!x!:

\begin{lstlisting}
>>> colores = ['azul', 'rojo', 'verde', 'amarillo']
>>> colores.index('verde')
2
>>> colores.index('fucsia')
Traceback (most recent call last):
  File "<stdin>", line 1, in <module>
ValueError: 'fucsia' is not in list
\end{lstlisting}

\lstinline!l.remove(x)! elimina el elemento \lstinline!x! de la lista:

\begin{lstlisting}
>>> l = [7, 0, 3, 9, 8, 2, 4]
>>> l.remove(2)
>>> l
[7, 0, 3, 9, 8, 4]
\end{lstlisting}

\lstinline!del l[i]! elimina el \lstinline!i!-ésimo elemento de la
lista:

\begin{lstlisting}
>>> l = [7, 0, 3, 9, 8, 2, 4]
>>> del l[2]
>>> l
[7, 0, 9, 8, 2, 4]
\end{lstlisting}

\lstinline!l.reverse()! invierte la lista:

\begin{lstlisting}
>>> l = [7, 0, 3, 9, 8, 2, 4]
>>> l.reverse()
>>> l
[4, 2, 8, 9, 3, 0, 7]
\end{lstlisting}

\lstinline!l.sort()! ordena la lista:

\begin{lstlisting}
>>> l = [7, 0, 3, 9, 8, 2, 4]
>>> l.sort()
>>> l
[0, 2, 3, 4, 7, 8, 9]
\end{lstlisting}

Para todas estas operaciones, siempre hay que tener muy claro si la
lista es modificada o no. Por ejemplo, el rebanado no modifica la lista,
sino que crea una nueva:

\begin{lstlisting}
>>> ramos = ['Progra', 'Mate', 'Fisica', 'Ed.Fisica']
>>> ramos[:2]
['Progra', 'Mate']
>>> len(ramos)    # la lista sigue teniendo cuatro elementos
4
\end{lstlisting}

\subsection{Iteración sobre una lista}

Una lista es un objeto \textbf{iterable}. Esto significa que sus valores
se pueden recorrer usando un ciclo \lstinline!for!:

\begin{lstlisting}
valores = [6, 1, 7, 8, 9]
for i in valores:
    print i ** 2
\end{lstlisting}

En cada iteración del \lstinline!for!, la variable \lstinline!i! toma
uno de los valores de la lista, por lo que este programa imprime los
siguientes valores:
