\chapter{Resolución de sistemas lineales}

Repasemos el producto matriz-vector:

\includegraphics{../diagramas/dieta-1.png}

Esta operación tiene dos operandos: una matriz y un vector. El resultado
es un vector. A los operandos los denominaremos respectivamente
\lstinline!A! y \lstinline!x!, y al resultado, \lstinline!b!.

Un problema recurrente en Ingeniería consiste en obtener cuál es el
vector \lstinline!x! cuando \lstinline!A! y \lstinline!b! son dados:

\includegraphics{../diagramas/dieta-2.png}

La ecuación matricial \(Ax = b\) es una manera abreviada de expresar un
\href{http://es.wikipedia.org/wiki/Sistema\_de\_ecuaciones\_lineales}{sistema
de ecuaciones lineales}. Por ejemplo, la ecuación del diagrama es
equivalente al siguiente sistema de tres ecuaciones que tiene las tres
incógnitas \(w\), \(y\) y \(z\):

\[\begin{align}
36w + 51y + 13z &= 3 \\
52w + 34y + 74z &= 45 \\
7y + 1.1z &= 33 \\
\end{align}\]

En matemáticas, este sistema se representa matricialmente así:

\[\begin{bmatrix}
36 & 51 & 13 \\
52 & 34 & 74 \\
&  7 & 1.1 \\
\end{bmatrix}
\begin{bmatrix}
w \\ y \\ z \\
\end{bmatrix}
=
\begin{bmatrix}
3 \\ 45 \\ 33 \\
\end{bmatrix}\]

La teoría detrás de la resolución de problemas de este tipo usted la
aprenderá en sus ramos de matemáticas. Sin embargo, como este tipo de
problemas aparece a menudo en la práctica, aprenderemos cómo obtener
rápidamente la solución usando Python.

Dentro de los varios módulos incluídos en NumPy (por ejemplo, ya vimos
\lstinline!numpy.random!), está el módulo \lstinline!numpy.linalg!, que
provee algunas funciones que implementan algoritmos de álgebra lineal,
que es la rama de las matemáticas que estudia los problemas de este
tipo. En este módulo está la función \lstinline!solve!, que entrega la
solución \lstinline!x! de un sistema a partir de la matriz \lstinline!A!
y el vector \lstinline!b!:

\begin{lstlisting}
>>> a = array([[ 36. ,  51. ,  13. ],
...            [ 52. ,  34. ,  74. ],
...            [  0. ,   7. ,   1.1]])
>>> b = array([  3.,  45.,  33.])
>>> x = solve(a, b)
>>> x
array([-7.10829222,  4.13213834,  3.70457422])
\end{lstlisting}

Podemos ver que el vector \lstinline!x! en efecto satisface la ecuación
\lstinline!Ax = b!:

\begin{lstlisting}
>>> dot(a, x)
array([  3.,  45.,  33.])
>>> b
array([  3.,  45.,  33.])
\end{lstlisting}

Sin embargo, es importante tener en cuenta que los valores de tipo real
casi nunca están representados de manera exacta en el computador, y que
el resultado de un algoritmo que involucra muchas operaciones puede
sufrir de algunos errores de redondeo. Por esto mismo, puede ocurrir que
aunque los resultados se vean iguales en la consola, los datos obtenidos
son sólo aproximaciones y no exactamente los mismos valores:

\begin{lstlisting}
>>> (dot(a, x) == b).all()
False
\end{lstlisting}

