\chapter{Expresiones}

Una \textbf{expresión} es una combinación de valores y operaciones que,
al ser evaluados, entregan un valor como resultado.

Algunos elementos que pueden formar parte de una expresión son: valores
\textbf{lite\-rales} (como \lstinline!2!, \lstinline!"hola"! o
\lstinline!5.7!), \textbf{variables}, \textbf{operadores} y
\textbf{llamadas a funciones}.

Por ejemplo, la expresión \lstinline!4 * 3 - 2! entrega el valor 10 al
ser evaluada por el intérprete:

\begin{lstlisting}
>>> 4 * 3 - 2
10
\end{lstlisting}

El valor de la siguiente expresión depende del valor que tiene la
variable \lstinline!n! en el momento de la evaluación:

\begin{lstlisting}
>>> n / 7 + 5
\end{lstlisting}

Una expresión está compuesta de otras expresiones, que son evaluadas
recursivamente hasta llegar a sus componentes más simples, que son los
literales y las variables.

\section{Operadores}

Un \textbf{operador} es un símbolo en una expresión que representa una
operación aplicada a los valores sobre los que actúa.

Los valores sobre los que actúa un operador se llaman
\textbf{operandos}. Un \textbf{operador binario} es el que tiene dos
operandos, mientras que un \textbf{operador unario} es el que tiene sólo
uno.

Por ejemplo, en la expresión \lstinline!2.0 + x! el operador
\lstinline!+! es un operador binario que en este contexto representa la
operación de adición. Sus operandos son \lstinline!2.0! y \lstinline!x!.

Las operaciones más comunes se pueden clasificar en: aritméticas,
relacionales, lógicas y de texto.

\subsection{Operadores aritméticos}

Las \textbf{operaciones aritméticas} son las que operan sobre valores
numéricos y entregan otro valor numérico como resultado. Los valores
numéricos son los que tienen tipo entero, real o complejo.

Las siguientes son algunas operaciones aritméticas básicas, junto con el
operador que las representa en Python:

\begin{itemize}
\item
  la \textbf{suma} \lstinline!+!;
\item
  la \textbf{resta} \lstinline!-!;
\item
  la \textbf{multiplicación} \lstinline!*!;
\item
  la \textbf{división} \lstinline!/!;
\item
  el \textbf{módulo} \lstinline!%! (resto de la división);
\item
  la \textbf{potencia} \lstinline!**! («elevado a»).
\end{itemize}

En general, si los operandos son de tipo entero, el resultado también
será de tipo entero. Pero basta que uno de los operandos sea real para
que el resultado también lo sea:

\begin{lstlisting}
>>> 8 - 5
3
>>> 8 - 5.0
3.0
>>> 8.0 - 5
3.0
>>> 8.0 - 5.0
3.0
\end{lstlisting}

Esta regla suele causar confusión en el caso de la división. Al dividir
números enteros, el resultado siempre es entero, y es igual al resultado
real \textbf{truncado}, es decir, sin su parte decimal:

\begin{lstlisting}
>>> 5 / 2
2
>>> 5 / -2
-3
\end{lstlisting}

Si uno de los operandos es complejo, el resultado también será complejo:

\begin{lstlisting}
>>> 3 + 4
7
>>> 3 + (4+0j)
(7+0j)
\end{lstlisting}

El operador de módulo entrega el resto de la división entre sus
operandos:

\begin{lstlisting}
>>> 7 % 3
1
\end{lstlisting}

Un uso bastante común del operador de módulo es usarlo para determinar
si un número es divisible por otro:

\begin{lstlisting}
>>> 17 % 5   # 17 no es divisible por 5
2
>>> 20 % 5   # 20 si es divisible por 5
0
\end{lstlisting}

Una relación entre \lstinline!/! y \lstinline!%! que siempre se cumple
para los números enteros es:

\begin{lstlisting}
(a / b) * b + (a % b) == a
\end{lstlisting}

Hay dos operadores aritméticos unarios:

\begin{itemize}
\item
  el \textbf{positivo} \lstinline!+!, y
\item
  el \textbf{negativo} \lstinline!-!.
\end{itemize}

El positivo entrega el mismo valor que su operando, y el negativo
también pero con el signo cambiado:

\begin{lstlisting}
>>> n = -4
>>> +n
-4
>>> -n
4
\end{lstlisting}

\subsection{Operaciones relacionales}

Las \textbf{operaciones relacionales} sirven para comparar valores. Sus
operandos son cualquier cosa que pueda ser comparada, y sus resultados
siempre son valores lógicos.

Algunas operaciones relacionales son:

\begin{itemize}
\item
  el \textbf{igual a} \lstinline!==! (no confundir con el \lstinline!=!
  de las asignaciones);
\item
  el \textbf{distinto a} \lstinline"!=";
\item
  el \textbf{mayor que} \lstinline!>!;
\item
  el \textbf{mayor o igual que} \lstinline!>=!;
\item
  el \textbf{menor que} \lstinline!<!;
\item
  el \textbf{menor o igual que} \lstinline!<=!;
\end{itemize}

Algunos ejemplos en la consola:

\begin{lstlisting}
>>> a = 5
>>> b = 9
>>> c = 14
>>> a < b
True
>>> a + b != c
False
>>> 2.0 == 2
True
>>> 'amarillo' < 'negro'
True
\end{lstlisting}

Los operadores relacionales pueden ser encadenados, tal como se acostumbra en
matemáticas, de la siguiente manera:

\begin{lstlisting}
>>> x = 4
>>> 0 < x <= 10
True
>>> 5 <= x <= 20
False
\end{lstlisting}

La expresión \lstinline!0 < x <= 10! es equivalente a
\lstinline!(0 < x) and (x <= 10)!.

\subsection{Operaciones lógicas}

% TODO: encerrar celdas en \lstinline.
% Revisar este tip: http://tex.stackexchange.com/a/59788/341
\begin{table}
  \centering
  \begin{tabular}{*{5}{l}}
    \toprule
    p     & q       & p and q & p or q & not p \\
    \midrule
    True  & True    & True    & True   & False \\
    True  & False   & False   & True   &       \\
    False & True    & False   & True   & True  \\
    False & False   & False   & False  &       \\
    \bottomrule
  \end{tabular}
  \caption{Resultados posibles de los operadores booleanos.}
  \label{tbl:operadores-booleanos}
\end{table}

Los \textbf{operadores lógicos} son los que tienen operandos y resultado
de tipo lógico.

En Python, hay tres operaciones lógicas:

\begin{itemize}
\item
  la conjunción lógica \textbf{and} (en español: «y»),
\item
  la disyunción lógica \textbf{or} (en español: «o»), y
\item
  la negación lógica \textbf{not} (en español: «no»).
\end{itemize}

Los operadores \lstinline!and! y \lstinline!or! son binarios, mientras
que \lstinline!not! es unario:

\begin{lstlisting}
>>> True and False
False
>>> not True
False
\end{lstlisting}

La tabla~\ref{tbl:operadores-booleanos} muestra todos los resultados posibles de las
operaciones lógicas. Las primeras dos columnas representan los valores
de los operandos, y las siguientes tres, los resultados de las
operaciones.

\subsection{Operaciones de texto}

Los operadores \lstinline!+! y \lstinline!*! tienen otras
interpretaciones cuando sus operandos son strings.

\lstinline!+! es el operador de \textbf{concatenación} de strings: pega
dos strings uno después del otro:

\begin{lstlisting}
>>> 'perro' + 'gato'
'perrogato'
\end{lstlisting}

La concatenación no es una suma. Ni siquiera es una operación
conmutativa.

\lstinline!*! es el operador de \textbf{repetición} de strings. Recibe
un operando string y otro entero, y entrega como resultado el string
repetido tantas veces como indica el entero:

\begin{lstlisting}
>>> 'waka' * 2
'wakawaka'
\end{lstlisting}

Más adelante veremos muchas más operaciones para trabajar sobre texto.
Por ahora utilizaremos las más elementales. Otras operaciones que pueden
serle útiles por el momento son:

\begin{itemize}
\item
  obtener el \(i\)-ésimo caracter de un string (partiendo desde cero)
  usando los corchetes:

\begin{lstlisting}
>>> nombre = 'Perico'
>>> nombre[0]
'P'
>>> nombre[1]
'e'
>>> nombre[2]
'r'
\end{lstlisting}
\item
  comprarar strings alfabéticamente con los operadores relacionales
  (lamentablemente no funciona con acentos y eñes):

\begin{lstlisting}
>>> 'a' < 'abad' < 'abeja'
True
>>> 'zapato' <= 'alpargata'
False
\end{lstlisting}
\item
  obtener el largo de un string con la función \lstinline!len!:

\begin{lstlisting}
>>> len('papalelepipedo')
14
>>> len("")
0
\end{lstlisting}
\item
  verificar si un string está dentro de otro con el operador
  \lstinline!in!:

\begin{lstlisting}
>>> 'pollo' in 'repollos'
True
>>> 'pollo' in 'gallinero'
False
\end{lstlisting}
\end{itemize}

\section{Precedencia}

La \textbf{precedencia de operadores} es un conjunto de reglas que
especifica en qué orden deben ser evaluadas las operaciones de una
expresión.

La precedencia está dada por la siguiente lista, en que los operadores
han sido listados en orden de menor a mayor precedencia:

\begin{itemize}
\item
  \lstinline!or!
\item
  \lstinline!and!
\item
  \lstinline!not!
\item
  \lstinline!<!, \lstinline!<=!, \lstinline!>!, \lstinline!>=!,
  \lstinline"!=", \lstinline!==!
\item
  \lstinline!+!, \lstinline!-! (suma y resta)
\item
  \lstinline!*!, \lstinline!/!, \lstinline!%!
\item
  \lstinline!+!, \lstinline!-! (positivo y negativo)
\item
  \lstinline!**!
\end{itemize}

Esto significa, por ejemplo, que las multiplicaciones se evalúan antes
que las sumas, y que las comparaciones se evalúan antes que las
operaciones lógicas:

\begin{lstlisting}
>>> 2 + 3 * 4
14
>>> 1 < 2 and 3 < 4
True
\end{lstlisting}

Operaciones dentro de un mismo nivel son evaluadas en el orden en que
aparecen en la expresión, de izquierda a derecha:

\begin{lstlisting}
>>> 15 * 12 % 7    # es igual a (15 * 12) % 7
5
\end{lstlisting}

La única excepción a la regla anterior son las potencias, que son
evaluadas de derecha a izquierda:

\begin{lstlisting}
>>> 2 ** 3 ** 2    # es igual a 2 ** (3 ** 2)
512
\end{lstlisting}

Para forzar un orden de evaluación distinto a la regla de precedencia,
debe usarse paréntesis:

\begin{lstlisting}
>>> (2 + 3) * 4
20
>>> 15 * (12 % 7)
75
>>> (2 ** 3) ** 2
64
\end{lstlisting}

Otra manera de forzar el orden es ir guardando los resultados
intermedios en variables:

\begin{lstlisting}
>>> n = 12 % 7
>>> 15 * n
75
\end{lstlisting}

Como ejemplo, consideremos la siguiente expresión:

\begin{lstlisting}
15 + 59 * 75 / 9 < 2 ** 3 ** 2 and (15 + 59) * 75 % n == 1
\end{lstlisting}

y supongamos que la variable \lstinline!n! tiene el valor 2. Aquí
podemos ver cómo la expresión es evaluada hasta llegar al resultado
final, que es \lstinline!False!:

\begin{lstlisting}
15 + 59 * 75 / 9 < 2 ** 3 ** 2 and (15 + 59) * 75 % n == 1
15 + 59 * 75 / 9 < 2 **   9    and (15 + 59) * 75 % n == 1
15 + 59 * 75 / 9 < 512         and (15 + 59) * 75 % n == 1
15 +  4425   / 9 < 512         and (15 + 59) * 75 % n == 1
15 +        491  < 512         and (15 + 59) * 75 % n == 1
15 +        491  < 512         and    74     * 75 % n == 1
15 +        491  < 512         and          5550  % n == 1
15 +        491  < 512         and          5550  % 2 == 1
15 +        491  < 512         and                0   == 1
  506            < 512         and                0   == 1
                True           and                0   == 1
                True           and                  False
                              False
\end{lstlisting}

La operación entre paréntesis \lstinline!(15 + 59)! debe ser evaluada
antes de la multiplicación por 75, ya que es necesario conocer su
resultado para poder calcular el producto. El momento preciso en que
ello ocurre no es importante.

Lo mismo ocurre con la evaluación de la variable \lstinline!n!: sólo
importa que sea evaluada antes de ser usada por el operador de módulo.

En el ejemplo, ambos casos fueron evaluados inmediatamente antes de que
su valor sea necesario.

Las reglas completas de precedencia, incluyendo otros operadores que aún
no hemos visto, pueden ser consultados en
\href{http://docs.python.org/reference/expressions.html\#summary}{la
sección sobre expresiones} de la documentación oficial de Python.

\subsection{¿Cómo aprenderse las reglas de precedencia?}

La respuesta es: mejor no aprendérselas. Las reglas de precedencia son
muchas y no siempre son intuitivas,

Un programa queda mucho más fácil de entender si uno explícitamente
indica el orden de evaluación usando paréntesis o guardando en variables
los resultados intermedios del cálculo.

Un buen programador siempre se preocupa de que su código sea fácil de
entender por otras personas, ¡e incluso por él mismo en unas semanas más
adelante!

\section{Llamadas a función}

Los operadores forman un conjunto bastante reducido de operaciones. Más
comúnmente, las operaciones más generales son representadas como
\textbf{funciones}.

Al igual que en matemáticas, las funciones tienen un nombre, y reciben
\textbf{parámetros} (o \textbf{argumentos}) que van entre paréntesis
después del nombre. La operación de usar la función para obtener un
resultado se llama \textbf{llamar la función}.

Ya conocemos la función \lstinline!raw_input!, que entrega como
resultado el texto ingresado por el usuario mediante el teclado.

La función \lstinline!abs! entrega el valor absoluto de su argumento:

\begin{lstlisting}
>>> abs(4 - 5)
1
>>> abs(5 - 4)
1
\end{lstlisting}

La función \lstinline!len! recibe un string y entrega su largo. (más
adelante veremos otros usos de la función \lstinline!len!):

\begin{lstlisting}
>>> len('hola mundo')
10
>>> len('hola' * 10)
40
\end{lstlisting}

Los nombres de los tipos también sirven como funciones, que entregan el
equivalente de su parámetro en el tipo correspondiente:

\begin{lstlisting}
>>> int(3.8)
3
>>> float('1.5')
1.5
>>> str(5 + 6)
'11'
>>> int('5' + '6')
56
\end{lstlisting}

Las funciones \lstinline!min! y \lstinline!max! entregan el mínimo y el
máximo de sus argumentos:

\begin{lstlisting}
>>> min(6, 1, 8)
1
>>> min(6.0, 1.0, 8.0)
1.0
>>> max(6, 1, 4, 8)
8
\end{lstlisting}

La función \lstinline!round! redondea un número real al entero más
cercano:

\begin{lstlisting}
>>> round(4.4)
4.0
>>> round(4.6)
5.0
\end{lstlisting}

Algunas funciones matemáticas como la exponencial, el logaritmo y las
trigonométricas pueden ser usadas, pero deben ser importadas primero
usando la sentencia \lstinline!import!, que veremos en detalle más
adelante:

\begin{lstlisting}
>>> from math import exp
>>> exp(2)
7.3890560989306504
>>> from math import sin, cos
>>> cos(3.14)
-0.9999987317275395
>>> sin(3.14)
0.0015926529164868282
\end{lstlisting}

La lista completa de funciones matemáticas que pueden ser importadas
está en la \href{http://docs.python.org/library/math.html}{descripción
del módulo math} en la documentación de Python.

Más adelante también aprenderemos a crear nuestras propias funciones.
Por ahora, sólo necesitamos saber cómo llamarlas.

Por supuesto, siempre es necesario que los argumentos de una llamada
tengan el tipo apropiado:

\begin{lstlisting}
>>> round('perro')
Traceback (most recent call last):
  File "<console>", line 1, in <module>
TypeError: a float is required
>>> len(8)
Traceback (most recent call last):
  File "<console>", line 1, in <module>
TypeError: object of type 'int' has no len()
\end{lstlisting}

