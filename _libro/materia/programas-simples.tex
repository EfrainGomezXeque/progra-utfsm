\chapter{Programas simples}

Un programa es una secuencia de \textbf{sentencias}. Una sentencia
representa una instrucción bien definida que es ejecutada por el
computador. En Python, cada línea del código representa una sentencia.

Hay que distinguir entre:

\begin{enumerate}
\item
  \textbf{sentencias simples}: son una única instrucción; y
\item
  \textbf{sentencias de control}: contienen varias otras sentencias, que
  a su vez pueden ser simples o de control.
\end{enumerate}

Las sentencias simples son ejecutadas secuencialmente, una después de la
otra.

Todas las sentencias siguen ciertas reglas acerca de cómo deben ser
escritas. Si no son seguidas, el programa está incorrecto y no se
ejecutará. A este conjunto de reglas se le denomina \textbf{sintaxis}.

A continuación veremos algunas sentencias simples, con las que se pueden
escribir algunos programas sencillos. Más adelante introduciremos las
sentencias de control.

Como ejemplo, consideremos el siguiente programa, que pide al usuario
ingresar una temperatura en grados Fahrenheit y entrega como resultado
el equivalente en grados Celsius:

\begin{lstlisting}
f = float(raw_input('Ingrese temperatura en Fahrenheit: '))
c = (f - 32.0) * (5.0 / 9.0)
print 'El equivalente en Celsius es:', c

raw_input()
\end{lstlisting}

Escriba este programa en el computador
y ejecútelo para convencerse de que funciona correctamente.

\section{Expresiones y variables}

Una \textbf{expresión} es una combinación de valores y operaciones que
son evaluados durante la ejecución del algoritmo para obtener un
resultado.

Por ejemplo, \lstinline!2 + 3! es una expresión aritmética que, al ser
evaluada, siempre entrega el valor \lstinline!5! como resultado. En esta
expresión, \lstinline!2! y \lstinline!3! son \textbf{valores literales}
y \lstinline!+! es el operador de adición.

En el programa de conversión de temperaturas aparece la expresión
\lstinline!(f - 32.0) * (5.0 / 9.0)!, cuyo resultado depende de cuál es
el valor de \lstinline!f! al momento de la evaluación. A diferencia de
los valores literales, \lstinline!f! es una \textbf{variable}, que tiene
un valor específico que puede ser distinto cada vez que la expresión es
evaluada.

En esta expresión, \lstinline!*! es el operador de multiplicación y
\lstinline!/! el de división.

Una expresión puede ser usada como una sentencia de un programa por sí
sola, pero la mayoría de las veces esto no tiene ningún efecto. El
programa evaluará la expresión, pero no hará nada con el resultado
obtenido.

\section{Asignaciones}

Una \textbf{asignación} es una sentencia que asocia un nombre al
resultado de una expresión. El nombre asociado al valor se llama
\textbf{variable}.

La sintaxis de una asignación es la siguiente:

\begin{lstlisting}
variable = expresion
\end{lstlisting}

Por ejemplo, el programa de conversión de temperaturas tiene la
siguiente asignación:

\begin{lstlisting}
c = (f - 32.0) * (5.0 / 9.0)
\end{lstlisting}

Cuando aparece una asignación en un programa, es interpretada de la
siguiente manera:

\begin{enumerate}
\item
  primero la expresión a la derecha del signo \lstinline!=! es evaluada,
  utilizando los valores que tienen en ese momento las variables que
  aparecen en ella;
\item
  una vez obtenido el resultado, el valor de la variable a la izquierda
  del signo \lstinline!=! es reemplazado por ese resultado.
\end{enumerate}

Bajo esta interpretación, es perfectamente posible una asignación como
ésta:

\begin{lstlisting}
i = i + 1
\end{lstlisting}

Primero la expresión \lstinline!i + 1! es evaluada, entregando como
resultado el sucesor del valor actual de \lstinline!i!. A continuación,
la variable \lstinline!i! toma el nuevo valor. Por ejemplo, si
\lstinline!i! tiene el valor 15, después de la asignación tendrá el
valor 16.

Esto no significa que \(15 = 16\). Una asignación no es una igualdad
matemática ni una ecuación.

Por ejemplo, las siguientes asignaciones son correctas, suponiendo que
las variables que aparecen en ellas ya fueron asignadas previamente:

\begin{lstlisting}
nombre = 'Perico Los Palotes'
discriminante = b ** 2 - 4 * a * c
pi = 3.14159
r = 5.0
perimetro = 2 * pi * r
sucesor = n + 1
a = a
es_menor = x < 4
x0 = x1 + x2
r = 2 * abs(x - x0)
nombre = raw_input('Ingrese su nombre')
\end{lstlisting}

Las siguientes no son asignaciones válidas, pues no respetan la sintaxis
\lstinline!nombre = expresion! (tarea: identifique los errores):

\begin{lstlisting}
n + 1 = 5
7 = a
2_pi_r = 2 * pi * r
area del circulo = pi * r ** 2
x ** 2 = x * x
\end{lstlisting}

\section{Entrada}

La \textbf{entrada} es la parte del programa en que el usuario ingresa
datos.

La manera más simple de ingresar datos es hacerlo a través del teclado.
La función \lstinline!raw_input(mensaje)! pide al usuario ingresar un
valor, que puede ser asignado a una variable para ser usado por el
programa. El \lstinline!mensaje! es lo que se mostrará al usuario antes
de que él ingrese el valor.

El valor ingresado por el usuario siempre es interpretado como texto,
por lo que es de tipo \lstinline!str!. Si es necesario usarlo como si
fuera de otro tipo, hay que convertirlo explícitamente.

Por ejemplo, en el programa de conversión de temperaturas, la entrada es
realizada por la sentencia:

\begin{lstlisting}
f = float(raw_input('Ingrese temperatura en Fahrenheit: '))
\end{lstlisting}

Cuando el programa llega a esta línea, el mensaje
\lstinline!Ingrese temperatura en Fahrenheit:! es mostrado al usuario,
que entonces debe ingresar un valor, que es convertido a un número real
y asociado al nombre \lstinline!f!.

Desde esa línea en adelante, la variable \lstinline!f! puede ser usada
en el programa para referirse al valor ingresado.

\section{Salida}

La \textbf{salida} es la parte del programa en que los resultados son
entregados al usuario.

La manera más simple de entregar la salida es mostrando texto en la
pantalla. En Python, la salida del programa es realizada por la
sentencia \textbf{print} (\emph{imprimir} en inglés).

Si se desea imprimir un texto tal cual, la sintaxis es la siguente:

\begin{lstlisting}
print valor_a_imprimir
\end{lstlisting}

Si los valores a imprimir son varios, deben ser puestos separados por
comas. Por ejemplo, el programa de conversión de temperaturas tiene la
siguiente sentencia de salida:

\begin{lstlisting}
print 'El equivalente en Celsius es:', c
\end{lstlisting}

En este caso, se está imprimiendo el mensaje
\texttt{El equivalente en Celsius es:} y a continuación, en la misma
línea, el valor de la variable \lstinline!c!.

Las comillas sólo sirven para representar un string en el código, y no
forman parte del string. Al imprimir el string usando \lstinline!print!
las comillas no aparecen:

\begin{lstlisting}
>>> 'Hola'
'Hola'
>>> print 'Hola'
Hola
\end{lstlisting}

\section{Comentarios}

Un \textbf{comentario} es una sección del código que es ignorada por el
intérprete. Un comentario puede ser utilizado por el programador para
dejar un mensaje en el código que puede ser útil para alguien que tenga
que leerlo en el futuro.

En Python, cualquier texto que aparezca a la derecha de un signo
\lstinline!#! es un comentario:

\begin{lstlisting}
>>> 2 + 3  # Esto es una suma
5
>>> # Esto es ignorado
>>>
\end{lstlisting}

La excepción son los signos \lstinline!#! que aparecen en un string:

\begin{lstlisting}
>>> "123 # 456" # 789
'123 # 456'
\end{lstlisting}

\section{Evitar que se cierre el programa}

La ejecución de programas en Windows presenta un inconveniente práctico:
cuando el programa termina, la ventana de ejecución se cierra
inmediatamente, por lo que no es posible alcanzar a leer la salida del
programa.

Por ejemplo, al ejecutar el programa \lstinline!temperatura.py! tal como
está arriba, el usuario verá el mensaje
\lstinline!Ingrese temperatura...! y a continuación ingresará el valor.
Una vez que el programa entrega como resultado el equivalente en grados
Celcius, no quedan más sentencias para ejecutar, por lo que el programa
se cierra.

Existen otras maneras de ejecutar programas con las que este problema no
ocurre.. Por ejemplo, al ejecutar un programa desde una IDE,
generalmente la salida aparece en una ventana que no se cierra.

Una solución para evitar que la ventana se cierre es agregar un
\lstinline!raw_input()! al final del código. De este modo, el programa
quedará esperando que el usuario ingrese cualquier cosa (un enter basta)
antes de cerrarse.

Los programas presentados en este apunte no tendrán el
\lstinline!raw_input()! al final, pero usted puede agregarlo por su
cuenta si así lo desea. En controles y certámenes, no será necesario
hacerlo.
