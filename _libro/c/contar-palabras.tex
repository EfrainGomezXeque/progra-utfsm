\chapter{Contar palabras en un archivo}

El último programa que estudiaremos cuenta cuántas veces aparecen en un
archivo de texto un conjunto de palabras indicadas por el usuario.

Copie y pegue este programa en su editor de texto y compílelo.

El programa \lstinline!contar-palabras! está diseñado para recibir
parámetros por línea de comandos. Al momento de ejecutar el programa,
usted debe indicar inmediatamente después del nombre del programa cuál
es el archivo que quiere leer, y cuáles son las palabras que quiere
contar:

\begin{lstlisting}
$ ./contar-palabras archivo.txt perro gato
\end{lstlisting}

Para probar el programa, descargue
\href{http://www.gutenberg.org/ebooks/2000.txt.utf8}{El Quijote de la
Mancha} en formato de texto plano. El archivo se llama
\lstinline!pg2000.txt!; guárdelo en el mismo directorio donde está el
programa compilado. Contemos cuántas veces aparecen los nombres del
Quijote, de Sancho Panza y de Dulcinea en el libro:

\begin{lstlisting}
$ ./contar-palabras pg2000.txt Sancho Dulcinea Quijote
   950  Sancho
   165  Dulcinea
   894  Quijote
\end{lstlisting}

Contemos también cuántas veces aparecen los artículos del idioma español
en toda la obra:

\begin{lstlisting}
$ ./contar-palabras pg2000.txt el la los las
  7957  el
 10200  la
  4680  los
  3423  las
\end{lstlisting}

Pruebe qué ocurre al ejecutar el programa si:

\begin{itemize}
\item
  no se le entrega ningún parámetro:

\begin{lstlisting}
$ ./contar-palabras
\end{lstlisting}
\item
  se le pasa el archivo pero ninguna palabra:

\begin{lstlisting}
$ ./contar-palabras pg2000.txt
\end{lstlisting}
\item
  se le pasa un archivo que no existe:

\begin{lstlisting}
$ ./contar-palabras no-existe.txt perro gato
\end{lstlisting}
\end{itemize}

\section{Lectura de archivos de texto}

Ya aprendimos a escribir en un archivo de texto, y ahora veremos cómo
leer datos de él.

Primero que todo, hay que abrir el archivo en modo lectura:

\begin{lstlisting}
FILE *f = fopen("archivo.txt", "r");
\end{lstlisting}

Por supuesto, hay que verificar que \lstinline!f! no es \lstinline!NULL!
para asegurarnos que el archivo sí pudo ser abierto.

La manera más sencilla de leer datos desde el archivo es usar la función
\lstinline!fscanf! de la misma manera que usamos \lstinline!scanf! para
leer de la entrada estándar. Como en nuestro programa nos interesa leer
palabra por palabra, usamos el descriptor de formato \lstinline!%s!.

Para comprobar si ya se llegó al final del archivo, y por lo tanto ya no
queda nada más que leer, se usa la función \lstinline!feof!. Una manera
típica de leer todo el archivo es hacerlo como lo hicimos en nuestro
programa: un ciclo \lstinline!while! que va verificando antes de cada
lectura si quedan o no cosas por leer:

\begin{lstlisting}
while (!feof(f)) {
     fscanf(f, "%s", s);

    /* ... */
}
\end{lstlisting}

\section{Arreglos son punteros, punteros son arreglos}

\section{Parámetros del programa por línea de comandos}

Para que nuestro programa reciba parámetros al momento de ejecutarlo,
debemos modificar la declaración de \lstinline!main! para que incluya
dos parámetros:

\begin{lstlisting}
int main(int argc, char **argv) {
}
\end{lstlisting}

La variable \lstinline!argc! tomará como valor la cantidad de argumentos
pasados en la línea de comandos, \emph{incluyendo el nombre del
programa}.

El puntero \lstinline!argv! apunta a un arreglo de \lstinline!argc!
strings, que son precisamente estos parámetros.

(Recordemos que un string es un arreglo de \lstinline!char!, y que un
arreglo es en la práctica un puntero) Por eso \lstinline!argv! es un
puntero a puntero a \lstinline!char!).

Por ejemplo, cuando ejecutamos el programa de la siguiente manera:

\begin{lstlisting}
$ ./contar-palabras abc.txt azul rojo verde "amarillo patito"
\end{lstlisting}

entonces \lstinline!argc! tendrá el valor 6 y los valores del arreglo
\lstinline!argv! serán:

\begin{verbatim}
argv[0]  →  "./contar-palabras"
argv[1]  →  "abc.txt"
argv[2]  →  "azul"
argv[3]  →  "rojo"
argv[4]  →  "verde"
argv[5]  →  "amarillo patito"
\end{verbatim}

\section{Aritmética de punteros}

Un puntero es una dirección de memoria, y una dirección de memoria no es
más que un entero. ¿Estará permitido entonces aplicar operaciones
aritméticas a los punteros para obtener otros punteros?

En C sí es posible hacerlo. Sin embargo, los punteros tienen sus propias
reglas para hacer aritmética.

La única operación permitida es «puntero + entero», y el resultado es un
puntero del mismo tipo,

Para verificarlo con sus propios ojos, puede ejecutar el siguiente
programa:

Un \lstinline!char! ocupa un byte en la memoria. Por lo tanto,
\lstinline!p + 1! apuntará a un byte más que \lstinline!p!.

Un \lstinline!float! ocupa cuatro bytes. Luego, \lstinline!q + 1!
apuntará a cuatro bytes más allá de \lstinline!q!.

La aritmética de punteros es útil cuando hay arreglos involucrados. Si
\lstinline!p! apunta a \lstinline!arreglo[0]!, entonces
\lstinline!p + 1! apunta a \lstinline!arreglo[1]!, independientemente
del tipo del arrego.

En otras palabras, \lstinline!p + 1! siempre apunta a lo que hay en la
memoria inmediatamente después de lo apuntado por \lstinline!p!.

En nuestro contador de palabras, contamos desde el principio con un
arreglo con todos los parámetros del programa. Pero las palabras que
interesan están sólo desde el tercer parámetro en adelante. En vez de
declarar un nuevo arreglo (con el consiguiente uso extra de memoria) y
copiar allí las palabras, simplemente introducimos el puntero
\lstinline!palabras! que apunta al tercer elemento de \lstinline!argv!.
Hacer esto es muy fácil gracias a la aritmética de punteros:

\begin{lstlisting}
palabras = argv + 2;
\end{lstlisting}

Desde esta línea en adelante, \lstinline!palabras! y \lstinline!argv! se
ven como dos arreglos que comparten su memoria. \lstinline!palabras[0]!
es lo mismo que \lstinline!argv[2]!:

\begin{lstlisting}
+----------+      +----------+
|          |<-----+----*     | argv
+----------+      +----------+
|          |
+----------+      +----------+
|          |<-----+----*     | palabras
+----------+      +----------+
|          |
+----------+
|          |
+----------+
|          |
+----------+
|          |
+----------+
\end{lstlisting}

En C siempre se cumple que \lstinline!a[i]! es lo mismo que
\lstinline!*(a + i)!. ¿Puede darse cuenta de por qué? Esta relación
debería resultarle natural después de estudiar arreglos, punteros y su
aritmética.

\section{Reserva de memoria dinámica}

\section{Ejercicios}

El programa es incapaz de distinguir cuando una palabra que está siendo
buscada aparece en mayúsculas, o tiene pegada a ella un signo de
puntuación.

Por ejemplo, para este archivo \lstinline!test.txt!:

\begin{lstlisting}
Da da da? Da da da da (da da) da. Da Da!
\end{lstlisting}

vamos a obtener esta salida:

\begin{lstlisting}
$ ./contar-palabras test.txt da
     4 da
\end{lstlisting}

\begin{itemize}
\item
  Modifique el programa para que cuente cada palabra independiente de si
  aparece con mayúsculas o minúsculas en el archivo.
\item
  Modifique el programa para que cuente cada palabra incluso si aparece
  precedida o sucedida de un signo de puntuación.
\end{itemize}
