\chapter{Calculadora simple}

El siguiente programa es una calculadora simple:

Al ejecutar el programa, primero uno ingresa la operación que será
aplicada, que puede ser:

\ctable[pos = H, center, botcap]{ll}
{% notes
}
{% rows
\FL
Signo & Operación
\\\noalign{\medskip}
------- & ---------------
\\\noalign{\medskip}
\lstinline!+! & Suma
\\\noalign{\medskip}
\lstinline!-! & Resta
\\\noalign{\medskip}
\lstinline!*! & Multiplicación
\\\noalign{\medskip}
\lstinline!/! & División
\\\noalign{\medskip}
\lstinline!^! & Potencia
\LL
}

La multiplicación también puede ser indicada con una \lstinline!x!
minúscula.

A continuación, se debe ingresar los dos operandos. Finalmente, el
programa muestra el resultado de la operación.

Escriba, compile y ejecute este programa.

En este programa puede ver que es posible asignar un valor inicial a una
variable al momento de declararla:

\begin{lstlisting}
float resultado = 1.0;
int valido = 1;
\end{lstlisting}

También note que tanto en el \lstinline!if! como en el \lstinline!else!
del final se ha omitido los paréntesis de llave (\lstinline!{}!) ya que
en ambos casos hay incluída solamente una única sentencia.

\section{Definición de funciones}

Al principio del programa, se ha definido una función llamada
\lstinline!potencia!. Ella recibe como parámetros la base (un número
real) y el exponente (un entero), y retorna el resultado de elevar la
base al exponente.

En C no existe un operador «elevado a» (como el \lstinline!**! de
Python), por lo que sí es útil definir una función como ésta.

Es necesario especificar explícitamente cuál será el tipo del valor
retornado (en este caso \lstinline!float!) y los tipos de cada uno de
los parámetros (en el ejemplo, \lstinline!float! e \lstinline!int!).

Las variables declaradas dentro de la función se llaman
\textbf{variables locales}. Estas variables comienzan a existir al
momento de llamar a la función, y desaparecen cuando la función termina.
Son invisibles desde fuera de la función.

En nuestro programa, las dos funciones \lstinline!main! y
\lstinline!potencia! tienen una variable local llamada
\lstinline!resultado!. Ambas variables son distintas, y sus valores
respectivos están almacenados en regiones diferentes de la memoria.

\section{Tipo char}

El tipo \lstinline!char! se usa para representar caracteres (símbolos)
solitarios. La variable \lstinline!op! que almacena la operación es de
este tipo.

Un valor de tipo \lstinline!char! se representa en un programa entre
comillas simples. Por ejemplo, el signo más está representado como
\lstinline!'+'!.

Técnicamente, los valores de tipo \lstinline!char! son números enteros
que están comprendidos entre −128 y 127. Cada número está asociado a un
caracter a través de la
\href{http://es.wikipedia.org/wiki/C\%C3\%B3digo\_ASCII\#Caracteres\_imprimibles\_ASCII}{tabla
ASCII}. Los enteros y los caracteres asociados son intercambiables; por
ejemplo, la expresión \lstinline!'m' == 109! es evaluada como verdadera.

No hay que confundir un caracter con un string de largo uno:
\lstinline!'a'! y \lstinline!"a"! son dos cosas distintas.

\section{Sentencia switch}

El \textbf{switch} es una sentencia de control condicional que permite
indicar qué hacer si el resultado de una expresión es igual a alguno de
ciertos valores constantes indicados

Un ejemplo de uso de \lstinline!switch! es el siguiente:

\begin{lstlisting}
switch (expresion) {
    case 1:
        /* que hacer cuando expresion == 1 */

    case 2:
        /* que hacer cuando expresion == 2 */

    default:
        /* que hacer cuando la expresion no es igual
         * a ninguno de los casos anteriores */
}
\end{lstlisting}

Cuando el resultado de la expresión es igual a alguno de los valores
indicados, la ejecución del programa salta al \lstinline!case! con ese
valor. Si el valor con el resultado no existe, salta a
\lstinline!default!.

Hay que tener cuidado con una característica extraña del
\lstinline!switch!: cuando se cumple un caso, los casos que vienen a
continuación también se ejecutan. En este ejemplo:

\begin{itemize}
\item
  si \lstinline!expresion == 1!, el programa saltará a
  \lstinline!case 1!, y luego continuará con \lstinline!case 2! y
  \lstinline!default!;
\item
  si \lstinline!expresion == 2!, el programa saltará a
  \lstinline!case 2!, y luego continuará con \lstinline!default!;
\item
  si \lstinline!expresion! no es ni 1 ni 2, el programa saltará a
  \lstinline!default!.
\end{itemize}

Para evitar que los casos siguientes sean ejecutados, debe ponerse un
\lstinline!break! al final de cada caso. Esto es lo que se hizo en el
programa de la calculadora.

\section{Conversión de tipos}

El segundo parámetro de la función \lstinline!potencia! es entero, pero
los operandos ingresados por el usuario son almacenados como números
reales.

Para convertir el exponente de real a entero, basta con anteponer al
valor el tipo entre paréntesis.

En este caso particular, la conversión se hace truncando los decimales
del número real. Así, si \lstinline!y! vale \lstinline!5.9!, entonces
\lstinline!(int) y! vale \lstinline!5!. Para conversiones entre otros
tipos, se siguen otras reglas diferentes.

En inglés, el nombre de esta operación es \emph{cast}. Posiblemente
usted escuche más de una vez a alguien refiriéndose a esta operación
como «castear».

\section{Ejercicios}

Modifique el programa de modo que soporte una nueva operación: obtener
el \href{http://es.wikipedia.org/wiki/Coeficiente\_binomial}{coeficiente
binomial} entre \lstinline!x! e \lstinline!y!. Esta operación debe ser
indicada con el símbolo \lstinline!b!:

El coeficiente binomial es una operación entre números enteros. Tenga
cuidado y use conversiones apropiadas.
