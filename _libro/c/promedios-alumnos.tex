\chapter{Promedios de alumnos}

El siguiente programa pide al usuario ingresar las notas de uno o más
alumnos, y va mostrando los promedios de cada uno de ellos:

El especificador de formato \lstinline!%.1f! sirve para mostrar un
número \lstinline!float! con una cifra decimal.

Por ejemplo, una ejecución del programa podría verse así:

Escriba, compile y ejecute este programa.

\section{Arreglos}

Un \textbf{arreglo} es una región continua en la memoria del computador
en la que se almacenan varios valores del mismo tipo. En C se usa los
arreglos como colecciones de valores, tal como se hacía con las listas
de Python.

Un arreglo llamado \lstinline!notas! de tipo \lstinline!int! y tamaño
\lstinline!10! se declara de la siguiente manera:

\begin{lstlisting}
int notas[10];
\end{lstlisting}

Los arreglos son mucho más limitados que las listas de Python. Todos los
elementos de un arreglo deben ser del mismo tipo. El tamaño de un
arreglo está fijo, y debe estar especificado al momento de compilar el
programa. Por ejemplo, es ilegal hacer lo siguiente:

\begin{lstlisting}
int n;
scanf("%d", &n);

float arreglo[n];   /* ilegal */
\end{lstlisting}

Por lo tanto, lo que suele hacerse es declarar arreglos suficientemente
grandes, y llevar la cuenta de cuántos elementos han sido asignados. Hay
que tener en cuenta que, al igual que todas las variables, cada elemento
del arreglo siempre tiene un valor, aunque no haya sido asignado
explícitamente:

\begin{lstlisting}
int a[5];
a[0] = 1000;
a[1] = 700;
printf("%d\n", a[2]); /* Esto algo va a imprimir,
                         pero no sabemos que. */
\end{lstlisting}

Cada elemento está identificado a través de su índice, que es su
posición dentro del arreglo. Los índices parten de cero: si el arreglo
tiene diez elementos, entonces los índices van de cero a nueve. Cada
elemento del arreglo puede ser considerado por sí solo como una
variable, a la que se accede usando el índice entre corchetes:

\begin{lstlisting}
int a[10];
/* Todas las instrucciones a continuacion
   son validas. */
a[0] = 5;
a[1] = a[0] + 3;
a[0]++;
a[2] = (a[0] + a[1]) / 2.0;
\end{lstlisting}

Es ilegal tratar de acceder a un elemento del arreglo cuyo índice está
fuera de los límites definidos por su tamaño. Lamentablemente, nunca se
verifica que los índices utilizados sean válidos, ni al momento de
compilar ni durante la ejecución del programa. Esto es una fuente de
errores difíciles de detectar. Por ejemplo, al ejecutar este código el
programa podría seguir funcionando, o también podría caerse
estrepitosamente:

\begin{lstlisting}
int a[10];
a[20] = 5;  /* ilegal */
\end{lstlisting}

\section{Funciones que reciben arreglos como parámetros}

La función \lstinline!promedio! recibe como primer parámetro el arreglo
con los valores que se van a promediar. Lo ideal es que la función sirva
para arreglos de cualquier tamaño, no sólo para los de tamaño 10 como el
del ejemplo.

En la declaración del parámetro, hay que especificar que se trata de un
arreglo, pero no su tamaño. Para esto, hay que poner los corchetes sin
el tamaño:

\begin{lstlisting}
int valores[]
\end{lstlisting}

Sin embargo, cada vez que se llame a la función sí es importante conocer
el tamaño del arreglo. De otro modo, sería imposible saber hasta qué
valor promediar. Por lo tanto, es imprescindible pasar el tamaño del
arreglo como parámetro adicional, que en esta función hemos bautizado
como \lstinline!cantidad!.

Note que aunque siempre estamos pasando el mismo arreglo
\lstinline!notas! a la función, \lstinline!cantidad! no necesariamente
tiene el mismo valor cada vez. Esto no es importante para la función,
que operará sólo con la cantidad de valores que se le indica. Eso sí, la
cantidad debe ser siempre menor o igual que el tamaño verdadero del
arreglo (en este caso, 10).

\section{Strings}

En C no existe un tipo de datos para representar los strings, como el
tipo \lstinline!str! de Python. En C, \textbf{un string es simplemente
un arreglo de caracteres}.

Ya vimos que los arreglos deben tener un tamaño fijo. Sin embargo, en
general uno no conoce de antemano el largo de los textos que serán
almacenado. Esto en teoría representa un problema: ¿cómo sabe el
programa cuáles de los caracteres del arreglo son parte del texto, y
cuáles son simplemente caracteres que están allí sólo porque el arreglo
es más largo de lo que corresponde?

La manera con la que C resuelve este problema es marcando el final del
texto con un caracter especial representado como \lstinline!'\0'!.

Por ejemplo, después de ingresar el nombre \lstinline!Perico!, el
contenido del arreglo \lstinline!nombre! podría ser el siguiente:

\begin{verbatim}
                0   1   2   3   4   5   6   7   8  ...  19
              +---+---+---+---+---+---+---+---+---+---+---+
      nombre: | P | e | r | i | c | o | \0| x | m |...| q |
              +---+---+---+---+---+---+---+---+---+---+---+
\end{verbatim}

Lo que hay a continuación del caracter \lstinline!'\0'! es irrelevante.
Todas las operaciones de strings saben que el texto llega solamente
hasta ahí.

Como el texto \lstinline!"Perico"! tiene seis caracteres, se utilizará
siete casillas del arreglo para almacenarlo. En general, siempre debe
declararse un arreglo de caracteres cuyo tamaño sea uno más que el más
largo de los textos que se podría almacenar.

Para leer un string como entrada usando la función \lstinline!scanf!, se
debe usar el descriptor de formato \lstinline!%s!. Una diferencia
importante con la lectura de otros tipos de variables es que, al leer
strings, el segundo parámetro del \lstinline!scanf! no debe ir con el
operador \lstinline!&!, sino que debe ser la variable desnuda:

\begin{lstlisting}
scanf("%s", nombre);
\end{lstlisting}

Hay una razón técnica muy precisa para esto que será más sencilla de
comprender una vez que sepamos más sobre la organización de la memoria,
pero por ahora aceptémoslo como un dogma: los strings se leen sin
\lstinline!&!, valores de otros tipos con \lstinline!&!.

Todas las operaciones de strings están implementadas como funciones
cuyas declaraciones están en la cabecera \lstinline!string.h!:

\begin{itemize}
\item
  \lstinline!strlen(s)! retorna el largo del string \lstinline!s!, sin
  incluir el \lstinline!'\0'! del final.
\item
  \lstinline!strcpy(s, t)! copia el contenido del string \lstinline!t!
  en el string \lstinline!s!; es necesario que el tamaño del arreglo
  \lstinline!s! sea al menor tan grande como \lstinline!t!.
\item
  \lstinline!strcat(s, t)! concatena el string \lstinline!t! al string
  \lstinline!s!; por ejemplo, al ejecutar el siguiente código, el string
  \lstinline!s! queda con el contenido \lstinline!"Hola mundo"!:

\begin{lstlisting}
char s[30], t[30];
strcpy(s, "Hola ");
strcpy(t, "mundo");
strcat(s, t);
printf("%s\n", s);         /* imprime Hola mundo */
printf("%d\n", strlen(s)); /* imprime 10 */
\end{lstlisting}
\end{itemize}

\section{Ciclo do-while}

El \textbf{do while} es un ciclo similar al \lstinline!while!. El código
es ejecutado mientras la condición es verdadera. La única diferencia es
que la condición del \lstinline!do while! es evaluada al final de cada
iteración, mientras que la del \lstinline!while! es evaluada al
principio.

En otras palabras, esto significa que el \lstinline!do while! hace algo
una o más veces, mientras que el \lstinline!while! lo hace cero o más
veces.

En nuestro programa de ejemplo es apropiado usar \lstinline!do while!,
ya que no tiene sentido ejecutar el programa para no calcular ningún
promedio. Por lo tanto, calculamos uno y al final decidimos si queremos
continuar.

La sintaxis del ciclo \lstinline!do while! es:

\begin{lstlisting}
do {
    /* ... */
}
while (condicion);
\end{lstlisting}

El punto y coma al final es obligatorio.

\section{Ejercicios}

¿Qué ocurre con el programa si intenta ingresar más de una palabra al
ingresar el nombre de un alumno (por ejemplo el nombre completo:
\lstinline!Perico Los Palotes!)? Haga la prueba. Investigue cómo hacer
para que el programa sea capaz de leer un nombre con espacios.

¿Qué ocurre si intenta ingresar un nombre que tenga más de 20
caracteres, como por ejemplo \lstinline!Periiiiiiiiiiiiiiiiico!)? Haga
la prueba.
