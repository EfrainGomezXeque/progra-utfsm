\section{Cómo funciona la memoria}

Todas las variables de un programa en C tienen asociado un lugar en la
memoria del computador.

La memoria puede ser vista como un gran arreglo de \textbf{bits}. Un
\href{http://en.wikipedia.org/wiki/Bit}{bit} es la unidad básica de
información que se puede representar en un computador, y puede tener los
valores 0 o 1:

\begin{verbatim}
Memoria
+----------------------------------------------+
|...0001001111011000001010110111110001010011...|
+----------------------------------------------+
\end{verbatim}

Los bits están agrupados en \textbf{bytes}. En cualquier computador
moderno, un byte está formado por ocho bits:

\begin{verbatim}
+---+--------+--------+--------+--------+--------+---+
|...|00010011|11011000|00101011|01111100|01010011|...|
+---+--------+--------+--------+--------+--------+---+
\end{verbatim}

Para que no pase vergüenzas: byte se pronuncia «bait».

Cualquier valor que aparezca en un programa debe ser representado como
uno o varios bytes. Tanto el tamaño como la manera de interpretar un
valor están determinados por su \textbf{tipo de datos}.

\subsection{Tipos de datos}

El tamaño de un valor de tipo \lstinline!char! es de 1 byte, y el
significado de los bits está determinado usando la codificación
\href{http://en.wikipedia.org/wiki/ASCII}{ASCII} (pronúnciese «asqui»).

Una variable de tipo \lstinline!char! estará almacenada, por lo tanto,
en uno de los bytes de la memoria. Casi nunca nos interesará cuáles son
exactamente los bits del valor, por lo que podemos considerar que lo que
hay en ese byte es un caracter:

\begin{verbatim}
char a;

         a
+---+--------+--------+--------+--------+--------+---+
|...|   'u'  |11011000|00101011|01111100|01010011|...|
+---+--------+--------+--------+--------+--------+---+
\end{verbatim}

El tamaño de un \lstinline!int! no es igual en todas las plataformas. Si
su procesador tiene una arquitectura de 32 bits, lo más probable es que
los enteros sean de 4 bytes. Si es de 64 bits, probablemente son de 8
bytes.

Los enteros son almacenados en su
\href{http://en.wikipedia.org/wiki/Binary\_numeral\_system}{representación
binaria}. La manera más común de representar los enteros negativos es el
\href{http://en.wikipedia.org/wiki/Two's\_complement}{complemento a
dos}.

El operador \lstinline!sizeof! entrega el tamaño en bytes de una
variable o de un tipo. Para conocer los tamaños que tienen varios tipos
de datos en la plataforma que está usando, ejecute el siguiente
programa:

\subsection{Direcciones de memoria}

Todos los bytes en la memoria tienen una \textbf{dirección}, que no es
más que un índice correlativo.

Por conveniencia, las direcciones de memoria suelen escribirse en
\href{http://en.wikipedia.org/wiki/Hexadecimal}{notación hexadecimal},
pero no hay que espantarse: se trata simplemente de un número entero.

\begin{verbatim}
Dirección   Valor
          +--------+
          |  ...   |
0xf1e568  |00010011|
0xf1e569  |11011000|
0xf1e56a  |00101011|
0xf1e56b  |01111100|
0xf1e56c  |01010011|
          |  ...   |
          +--------+
\end{verbatim}

En C, el operador unario \lstinline!&! permite obtener la dirección de
la memoria en que está almacenada una variable, o más precisamente, la
dirección de su primer byte.

Las variables locales de una función están almacenadas consecutivamente
en una región de la memoria llamada
\href{http://es.wikipedia.org/wiki/Pila\_de\_llamadas}{pila de
llamadas}.

Copie y ejecute este programa, e interprete cómo están distribuidas las
variables en la pila:

Por ejemplo, yo ejecuté el programa una vez en el computador y obtuve
esta salida:

\begin{verbatim}
var     address   sizeof
n:      0xe49808  4
c:      0xe4980f  1
x:      0xe49804  4
h:      0xe497f0  12
d:      0xe4980e  1

h.anno: 0xe497f0  4
h.mes : 0xe497f4  4
h.dia : 0xe497f8  4
\end{verbatim}

Esto significa que la pila está organizada más o menos así:

\begin{verbatim}
           +--------+
0xe497f0   |   2012 |  h.anno
0xe497f1   |        |
0xe497f2   |        |
0xe497f3   |        |
           +--------+
0xe497f4   |      2 |  h.mes
0xe497f5   |        |
0xe497f6   |        |
0xe497f7   |        |
           +--------+
0xe497f8   |     29 |  h.dia
0xe497f9   |        |
0xe497fa   |        |
0xe497fb   |        |
           +--------+
0xe497fc   | ?????? |
0xe497fd   | ?????? |
0xe497fe   | ?????? |
0xe497ff   | ?????? |
0xe49800   | ?????? |
0xe49801   | ?????? |
0xe49802   | ?????? |
0xe49803   | ?????? |
           +--------+
0xe49804   | 3.1415 |  x
0xe49805   |        |
0xe49806   |        |
0xe49807   |        |
           +--------+
0xe49808   |     10 |  n
0xe49809   |        |
0xe4980a   |        |
0xe4980b   |        |
           +--------+
0xe4980c   | ?????? |
0xe4980d   | ?????? |
           +--------+
0xe4980e   |    '!' |  d
           +--------+
0xe4980f   |   '\n' |  c
           +--------+
\end{verbatim}
