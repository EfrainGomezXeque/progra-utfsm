\chapter{Punteros}

Un \textbf{puntero} es un tipo de datos cuyo valor es una dirección de
memoria.

Nunca olvide esta sencilla definición. Los punteros son un concepto que
suele causar mucha confusión a quienes están aprendiendo C. Sin embargo,
no se trata de un concepto difícil si uno comprende cómo están
representadas las variables en la memoria.

Por otra parte, el uso incorrecto de punteros es una fuente muy común de
errores críticos, y que no siempre son fáciles de depurar. Por esto es
importante siempre entender muy bien lo que se está haciendo cuando hay
punteros involucrados.

Cuando una variable de tipo puntero tiene almacenada una dirección de
memoria, se dice que «apunta» al valor que está en esa dirección.

\begin{verbatim}
           +----------+
0x3ad900   |      398 |  int n
           +----------+
0x3ad904   | ???????? |
           +----------+
0x3ad908   | 0x3ad900 |  int *p
           +----------+
0x3ad90c   | 0x3ad904 |  float *q
           +----------+
0x3ad910   |    2.717 |  float x
           +----------+
\end{verbatim}

En general no importa cuál es valor exacto de un puntero, sino que basta
con comprender qué es lo que hay «al otro lado». Por esto, en los
diagramas de la memoria se suele preferir usar flechas en lugar de las
direcciones explícitas:

\begin{verbatim}
          +----------+
   int n  |      398 |<----+
          +----------+     |
          | ???????? |     |
          +----------+     |
  int *p  |     *----+-----+
          +----------+
float *q  |     *----+----+
          +----------+    |
 float x  |    2.717 |<---+
          +----------+
\end{verbatim}

Un valor especial llamado \lstinline!NULL! puede ser asignado a
cualquier puntero para indicar aún no está apuntando a ninguna parte de
la memoria.

\section{Declaración de punteros}

La siguiente es la manera de declarar un puntero que apunte a un entero:

\begin{lstlisting}
int *x;
\end{lstlisting}

Esto se puede leer «lo apuntado por \lstinline!x! es un entero». En este
caso, \lstinline!*! no es una multiplicación, sino una
derreferenciación, como veremos más abajo.

Una vez declarada \lstinline!x! de la manera ya mostrada, los únicos
valores válidos que se puede asignar a \lstinline!x! son
\lstinline!NULL! o una dirección de memoria donde haya un entero:

\begin{lstlisting}
int a, b, c;
float z;
int *p;
int *q;

p = NULL;   /* valido */
p = &a;     /* valido */
p = &b;     /* valido */
p = &z;     /* invalido (z no es un entero) */
p = 142857; /* invalido (142857 no es una direccion de memoria) */

q = &b;     /* valido */
q = p;      /* valido */
q = NULL;   /* valido */
q = &p;     /* invalido (p no es un entero) */
\end{lstlisting}

Por supuesto, es posible cambiar \lstinline!int! por cualquier otro tipo
para declarar punteros a datos de otra naturaleza.

Ojo con la siguiente sutileza al declarar varios punteros de una vez:

\begin{lstlisting}
int *x, *y;    /* x e y son punteros */
int *x, y;     /* x es puntero, y es entero */
\end{lstlisting}

\section{Derreferenciación de punteros}

El operador unario \lstinline!*! de los punteros es el operador de
\textbf{derreferenciación}. Lo que hace es entregar el valor que está en
la dirección de memoria.

En otras palabras, \lstinline!*! significa «lo apuntado por».

Al derreferenciar un puntero a entero, se obtiene un entero. El puntero
derreferenciado puede ser usado en cualquier contexto en que un entero
sea válido:

\begin{lstlisting}
int x, y;
int *p;

x = 5;
p = &x;

printf("%d\n", *p);      /* imprime 5 */
y = *p * 10 - 7;         /* y toma el valor 43 */
*p = 9;                  /* x toma el valor 9 */
\end{lstlisting}

En la última sentencia, se está asignando el valor \lstinline!9! en la
memoria que está reservada para la variable \lstinline!x!, por lo que la
asignacion cambia en efecto el valor de la variable \lstinline!x! de
manera indirecta.

Derreferenciar el puntero \lstinline!NULL! no está permitido. Al
hacerlo, lo más probable es que el programa se termine abruptamente y
sin razón aparente. Errores de este tipo son muy fastidiosos, pues son
difíciles de detectar, e incluso pueden ocurrir en un programa que ha
estado funcionando correctamente durante mucho tiempo.

Si existe alguna remota posibilidad de que un puntero pueda tener el
valor \lstinline!NULL!, lo sensato es revisar su valor antes de
derreferenciarlo:

\begin{lstlisting}
if (p != NULL)
    hacer_algo(*p);
\end{lstlisting}

\section{Ejercicio}

¿Qué imprime el siguiente programa?
