\section{Escribir tabla trigonométrica}

El siguiente programa crea un archivo llamado \lstinline!trig.txt! que
contiene una tabla con los senos y los cosenos de números reales entre 0
y π, de 0.1 en 0.1.

Si intenta compilar este programa tal como viene haciéndolo hasta ahora,
es probable que el compilador arroje un error parecido a «referencia
indefinida a sin». Para evitar este error, debe agregar la opción
\lstinline!-lm! al momento de compilar ---ya explicaremos por qué---:

\begin{lstlisting}
$ gcc trig.c -lm -o trig
\end{lstlisting}

Transcriba, compile y ejecute el programa. Verá que un archivo llamado
\lstinline!trig.txt! fue creado. Vea el contenido de ese archivo. Si
está trabajando en la consola, puede usar la instrucción
\lstinline!cat!:

\begin{lstlisting}
$ cat trig.txt
\end{lstlisting}

\subsection{Escritura de archivos de texto}

La escritura de datos en un archivo no puede hacerse directamente en él,
sino que debe hacerse a través de operaciones sobre un \textbf{flujo},
que es una abstracción que permite al programa interactuar con
dispositivos físicos.

El tipo de datos para representar flujos en C se llama \lstinline!FILE!,
que es una estructura definida en \lstinline!stdio.h!. Para crear un
flujo de datos, primero hay que declarar un puntero a \lstinline!FILE!
(como la variable \lstinline!a_trig! de nuestro programa) y luego
inicializarlo con el valor que retorna la función \lstinline!fopen!, que
es la que abre el flujo para poder interactuar con él.

El primer parámetro de \lstinline!fopen! es el nombre del archivo, y el
segundo es un string que describe el modo en que éste será abierto. Si
el modo contiene una \lstinline!w!, significa que el archivo se abrirá
para escribir datos en él. Si tiene una \lstinline!r!, es que se leerán
datos de él. Además, hay
\href{http://c.conclase.net/librerias/?ansifun=fopen}{varios otros}
modos posibles.

En nuestro ejemplo estamos creando un archivo de texto. Para escribir el
texto, usamos la función \lstinline!fprintf!, pasando como primer
parámetro el flujo con el que abrimos el archivo.

A esta altura ya no debería sorprender que si hubiéramos abierto un
archivo de texto para lectura, entonces leeríamos los datos usando una
función llamada \lstinline!fscanf!.

Por supuesto, siempre que se abre un archivo hay que cerrarlo después de
terminar de usarlo. Para eso, usamos la función \lstinline!fclose!.

La apertura de un archivo puede fallar por varios motivos. Por ejemplo,
el disco duro podría estar lleno o uno podría no tener permisos para
escribir en ese directorio. Cuando la apertura falla, entonces
\lstinline!fopen! retorna \lstinline!NULL!. Es importante verificar que
la apertura fue exitosa antes de hacer cualquier lectura o escritura con
el flujo.

\subsection{Tipos de datos de coma flotante}

En nuestro programa decidimos declarar la variable \lstinline!theta!
como \lstinline!double!, que es otro tipo de datos asociado a los
números reales.

Los números reales no pueden ser representados exactamente en un
computador, sino que deben ser aproximados de alguna manera que sea lo
suficientemente general como para abarcar a la vez valores muy grandes y
muy pequeños, como los que suelen aparecer en ciencia e ingeniería.

La manera estándar que utilizan los computadores para esto es la
\href{http://es.wikipedia.org/wiki/Coma\_flotante}{representación de
coma flotante}, que es una especie de notación científica en base 2.

El tipo \lstinline!float! que habíamos usado hasta ahora utiliza 32 bits
para representar números reales. Lo podemos verificar al imprimir el
valor de \lstinline!sizeof(float)!. De estos 32 bits, 1 es para
almacenar el signo del número, 8 son para almacenar el exponente de la
base, y el resto son para el factor que la acompaña (llamado
\emph{mantisa}). Esto permite al tipo \lstinline!float! representar
números reales con aproximadamente 7 cifras decimales de precisión.

A veces esta precisión no es suficiente, y para ello existe el tipo
\lstinline!double!, que dedica 64 bits para representar el número. Esto
permite alcanzar una precisión de aproximadamente 15 cifras decimales.

A pesar de lo que parecen indicar sus nombres, ambos tipos son
representaciones de coma flotante. A los valores \lstinline!float! se le
llama «de precisión simple» y a los \lstinline!double!, bueno, «de
precisión doble».

\subsection{Biblioteca matemática}

La cabecera \lstinline!math.h! provee declaraciones de funciones y
constantes matemáticas de la biblioteca estándar de C.

Las constantes \emph{e} y π están declaradas, respectivamente, como
\lstinline!M_E! y \lstinline!M_PI!. Además, también están disponibles
otras constantes precalculadas como π/2 (\lstinline!M_PI_2!) y la raíz
de 2 (\lstinline!M_SQRT2!).

La mayoría de las funciones matemáticas viene en dos versiones: una para
\lstinline!float! y una para \lstinline!double!. Las primeras llevan una
letra \lstinline!f! al final de su nombre.

Las funciones \lstinline!sin! y \lstinline!cos! que usamos en nuestro
programa están declaradas en \lstinline!math.h!. Si \lstinline!theta!
hubiera sido declarada como \lstinline!float! en vez de
\lstinline!double!, habríamos tenido que usar las funciones
\lstinline!sinf! y \lstinline!cosf!.

Para explorar todas las funciones que están disponibles, consulte el
manual de \lstinline!math.h!.

\subsection{Enlazado de bibliotecas}

¿Por qué hubo que agregar el \lstinline!-lm! al compilar?

En general, al usar bibliotecas, no basta con sólo incluir al archivo de
cabecera, sino que además es necesario indicar al compilador que debe
enlazar nuestro programa compilado con la biblioteca.

La razón es que la cabecera contiene sólo las declaraciones, pero no las
implementaciones de las funciones, que están en algún archivo ya
compilado que se encuentra instalado en alguna parte de nuestro sistema.

Las funciones de la mayoría de las cabeceras estándares están
implementadas en una biblioteca llamada \lstinline!libc!. que es
enlazada automáticamente al compilar cualquier programa.

Por
\href{http://stackoverflow.com/questions/1033898/why-do-you-have-to-link-the-math-library-in-c}{razones
históricas} que podemos obviar, las funciones matemáticas no están
implementadas en \lstinline!libc! sino en otra biblioteca llamada
\lstinline!libm!, que no es enlazada automáticamente.

Para enlazar \lstinline!libm! al momento de compilar es que se agrega la
opción \lstinline!-lm!. Si quisiera enlazar explícitamente con
\lstinline!libc!, podría agregar \lstinline!-lc!, pero esto no tendría
ningún efecto.

Cuando usted, más adelante, sea ya una experta programadora en C y
comience a usar bibliotecas escritas por otros desarrolladores (o
incluso por usted misma), deberá siempre tener el cuidado de enlazarlas
correctamente al momento de compilar usando la opción
\lstinline!-lnombre_de_la_biblioteca!.

\subsection{Ejercicios}

Interprete qué es lo que ocurre al usar los siguientes descriptores de
formato para imprimir los valores en el archivo:

\begin{itemize}
\item
  \lstinline!%.2lf!
\item
  \lstinline!% .2lf!
\item
  \lstinline!%+.2lf!
\item
  \lstinline!%10.2lf!
\item
  \lstinline!%lf!
\item
  \lstinline!%lg!
\end{itemize}

Modifique el programa para que escriba una columna adicional con el
logaritmo natural de cada número.
