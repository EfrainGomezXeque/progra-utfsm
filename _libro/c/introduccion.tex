\section{El lenguaje C}

Aprender a programar no es lo mismo que aprender un lenguaje de
programación. Los conceptos importantes de la programación que aparecen
en un lenguaje generalmente son traspasables a otro.

En términos de paradigmas de programación, C y Python pueden ser
clasificados como lenguajes procedurales, y como tales comparten muchos
de sus componentes fundamentales: expresiones, variables, sentencias,
condicionales, ciclos, funciones, etcétera.

Sintácticamente, ambos lenguajes se ven diferentes a simple vista, pero
veremos que muchas de las diferencias son sólo cosméticas:

\begin{lstlisting}
es_primo = True
for d in range(2, n):
    if n % d == 0:
        es_primo = False
        break
\end{lstlisting}

\begin{lstlisting}
es_primo = 1;
for (d = 2; d < n; d++) {
    if (n % d == 0) {
        es_primo = 0;
        break;
    }
}
\end{lstlisting}

No todas las diferencias sintácticas son tan sutiles, y haremos énfasis
en las que son más importantes.

Más allá de las diferencias visibles en el código, ambos lenguajes son
fundamentalmente diferentes en la manera que usan los recursos del
computador. Estas diferencias no son apreciables con sólo mirar el
código, sino que deben ser comprendidas desde el principio. La imagen
mental que uno se forma sobre el programa que está escribiendo es mucho
más importante al programar en C que en Python.

Este apunte está diseñado para que usted pueda familiarizarse
rápidamente con el lenguaje C después de haber aprendido Python. No nos
detendremos mucho tiempo en aspectos que son similares a Python, sino
que nos enfocaremos en las diferencias.

\subsection{Compilación versus interpretación}

Cuando programamos en Python, en cierto modo estamos haciendo trampa. El
código Python no es ejecutado físicamente por el computador, sino por un
\textbf{intérprete}, que es el programa que ejecuta los programas. El
lenguaje C permite hacer «menos trampa», ya que sí es un medio para dar
instrucciones al procesador.

El \textbf{procesador} es el componente del computador que ejecuta las
instrucciones de un programa.

Las instrucciones que el procesador recibe no están en un lenguaje de
programación como Python o C, sino en un
\href{http://en.wikipedia.org/wiki/Machine\_code}{lenguaje de máquina},
que es mucho más básico. Cada procesador viene diseñado «de fábrica»
para entender su propio lenguaje de máquina, que se compone de
\href{http://en.wikipedia.org/wiki/Instruction\_set}{instrucciones} muy
básicas, como leer un dato, aplicar una operación sobre un par de datos
o saltar a otra parte de la memoria para leer una nueva instrucción.

Si bien es posible programar directamente en el lenguaje de la máquina,
esto es extremadamente engorroso, y lo más probable es que usted nunca
tenga la necesidad de hacerlo. Decimos que el lenguaje de máquina es
\href{http://en.wikipedia.org/wiki/Low-level\_programming\_language}{de
bajo nivel}, que en computación no es un término peyorativo, sino que
significa que está tan ligado al hardware que no es lo suficientemente
expresivo para describir algoritmos de manera abstracta.

Es más razonable programar en un lenguaje de programación de alto nivel,
que nos ofrezca abstracciones como: variables, condicionales, ciclos,
funciones, tipos de datos, etc., que permiten describir algoritmos en
términos más humanos y menos «ferreteros».

C y Python son lenguajes tales, pero difieren en la forma en que son
ejecutados. Python es un lenguaje pensado para ser interpretado,
mientras que C debe ser compilado.

Un programa llamado \textbf{compilador} recibe como entrada el código C
y genera como salida código \textbf{binario} que el procesador es capaz
de entender. El binario puede ser un programa ejecutable, o una
biblioteca con funciones que pueden ser llamadas desde un programa.

\begin{quote}
  \texttt{.c}
  \(\to\)
  Compilador
  \(\to\)
  \texttt{BIN}
  \(\to\)
  Procesador
\end{quote}

A pesar de que el compilador actúa de intermediario entre nuestro código
y el procesador, el lenguaje C sigue siendo de más bajo nivel que
Python. El programador tiene la libertad (y la responsabilidad) de
lidiar con aspectos de la ejecución que no son accesibles desde Python.
Principalmente, la administración de la memoria que usa el programa.

\subsection{Lecturas adicionales}

Aquí termina el «bla bla» de este apunte. De aquí en adelante, usted
estará probando y analizando línea por línea programas enteros escritos
en C.

Para profundizar acerca de la relevancia del lenguaje C y las razones
para estudiarlo, le sugerimos leer los siguientes enlaces.

\begin{itemize}
\item
  \href{http://c.learncodethehardway.org/book/learn-c-the-hard-wayli3.html}{The
  Cartesian Dream of C}.
\item
  \href{http://en.wikibooks.org/wiki/C\_Programming/Why\_learn\_C\%3F}{Why
  learn C} en el wikibook de C.
\item
  \href{http://c2.com/cgi/wiki?CeeLanguage}{C Language} en WikiWikiWeb.
\end{itemize}
