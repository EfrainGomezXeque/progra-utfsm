\chapter{Referencias adicionales}

Este tutorial pretende ser sólo una introducción, y no es de ninguna
manera una referencia completa acerca del lenguaje C. Para complementar
su aprendizaje y para usar como referencia futura, le recomendamos los
siguientes libros.

\section{Brian Kernighan, Dennis Ritchie. \emph{The C programming
language}}

Conocido como K\&R (por las iniciales de sus autores), \emph{The C
programming language} es la referencia clásica sobre C. Dennis Ritchie,
uno de sus autores, fue el creador del lenguaje.

K\&R es considerado uno de los textos técnicos mejor escritos del siglo
XX, y su estilo ha marcado la pauta de cómo redactar un libro sobre
programación. Si bien está un poco viejito, y en algunos casos no
refleja las prácticas modernas de desarrollo, seguirá siendo un clásico
por mucho tiempo más.

Léalo para entender barbaridades como ésta:

\begin{lstlisting}
while (*p++ = *q++)
    ;
\end{lstlisting}

\section{Zed Shaw. \emph{Learn C the hard way}}

Un libro que aún está siendo escrito, pero cuyo borrador está disponible
para ser leído en la web. Presenta un enfoque muy práctico y moderno,
orientado a las buenas prácticas y al uso de las herramientas adecuadas,
siempre desde la perspectiva muy personal del autor.

A pesar de estar inconcluso, es ya un excelente tutorial para aprender
C,

\section{Mike Banahan et al. \emph{The C book}}

Un buen libro que sus autores han decidido publicar libremente en la
web.
