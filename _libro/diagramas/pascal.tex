\documentclass[dvipsnames]{minimal}
\usepackage[pdftex,active,tightpage]{preview}
\usepackage{mathpazo}
\usepackage{tikz}
\usetikzlibrary{positioning,arrows}
\PreviewEnvironment{tikzpicture}

\begin{document}
\begin{tikzpicture}[xscale=1.2, yscale=0.8]

  % Thanks to Paul Gaborit for this code.
  % http://www.texample.net/tikz/examples/pascals-triangle-and-sierpinski-triangle/

  % Pascal's triangle
  % row #0 => value is 1
  \node (p-0-0) at (0,0) {1};
  \foreach \row in {1,...,6} {
     % col #0 => value is 1
    \node (p-\row-0) at (-\row/2,-\row) {1};
    \pgfmathsetmacro{\value}{1};
    \foreach \col in {1,...,\row} {
      % iterative formula : val = precval * (row-col+1)/col
      % (+ 0.5 to bypass rounding errors)
      \pgfmathtruncatemacro{\value}{\value*((\row-\col+1)/\col)+0.5};
      \global\let\value=\value
      % position of each value
      \coordinate (pos) at (-\row/2+\col,-\row);
      \node (p-\row-\col) at (pos) {\value};
      % for arrows and plus sign
      \ifnum \col<\row
        \node[Red, above=0mm of p-\row-\col]{+};
        \pgfmathtruncatemacro{\prow}{\row-1}
        \pgfmathtruncatemacro{\pcol}{\col-1}
        \draw[Red, -latex'] (p-\prow-\pcol) -- (p-\row-\col);
        \draw[Red, -latex'] (p-\prow-\col) -- (p-\row-\col);
      \fi
    }
  }

\end{tikzpicture}
\end{document}

