\newcommand\polytwo  [1]{-- ++(#1, 0) -- cycle}
\newcommand\polyfour [3]{-- ++(#1, 0) -- ++(0, #2) \polytwo{#3}}
\newcommand\polysix  [5]{-- ++(#1, 0) -- ++(0, #2) \polyfour{#3}{#4}{#5}}
\newcommand\polyeight[7]{-- ++(#1, 0) -- ++(0, #2) \polysix{#3}{#4}{#5}{#6}{#7}}
\newcommand\polyten  [9]{-- ++(#1, 0) -- ++(0, #2) \polyeight{#3}{#4}{#5}{#6}{#7}{#8}{#9}}

\tikzstyle{piece} = [draw=black]
\tikzstyle{row} = [pattern color=gray, pattern=north west lines, thick]
\tikzstyle{fcall} = [font=\footnotesize,yshift=-.5cm]
\tikzstyle{darrow} = [very thick, -latex']

\def\colJ{blue!80!black}
\def\colZ{red!40!blue!80!black}
\def\colT{orange!90!black}
\def\colO{red!80!black}
\def\colS{yellow!90!black}
\def\colL{green!80!black}

\begin{tikzpicture}[yscale=-1, scale=.6]

  \def\juego{
    \draw[gray] (0, 0) grid (7, 10);

    \fill[\colJ, piece] (0, 10) \polysix{1}{-2}{1}{-1}{-2};
    \fill[\colT, piece] (3, 10) \polyeight{1}{-1}{1}{-1}{-3}{1}{1};
    \fill[\colS, piece] (6, 10) \polyeight{1}{-2}{-1}{-1}{-1}{2}{1};
    \fill[\colS, piece] (1,  7) \polyeight{1}{-2}{-1}{-1}{-1}{2}{1};
    \fill[\colO, piece] (2,  7) \polyfour{2}{-2}{-2};
    \fill[\colL, piece] (2,  8) \polysix{3}{-2}{-1}{1}{-2};
    \fill[\colT, piece] (4,  6) \polyeight{1}{-3}{-1}{1}{-1}{1}{1};
    \fill[\colZ, piece] (1,  5) \polyeight{2}{-1}{-1}{-1}{-2}{1}{1};
  }

  \def\pieza{
    \fill[\colL, piece] (6, 8) \polysix{1}{-3}{-2}{1}{1};
  }

  \begin{scope}
    \juego
    \draw[thick] (0, 0) rectangle (7, 10);
    \node[anchor=south] at (3.5, 0) {\texttt{juego}};
    \node (c) at (5.5, 10) {};
    \node[anchor=north] (l) at (5.5, 11) {\texttt{columna}};
    \draw[-latex', thick] (l) -- (c);
    \node (from1) at (7, 5) {};
  \end{scope}

  \begin{scope}[yshift=-10cm, xshift=-2.5cm]
    \draw[gray] (5, 5) grid ++(2, 3);
    \pieza
    \draw[thick] (5, 5) rectangle ++(2, 3);
    \node[anchor=south] at (6, 5) {\texttt{pieza}};
  \end{scope}

  \begin{scope}[xshift=12cm]
    \juego
    \fill[\colL, piece] (6, 8) \polysix{1}{-3}{-2}{1}{1};
    \draw[thick] (0, 0) rectangle (7, 10);
    \fill[row] (0, 6) rectangle ++(7, -1);
    \fill[row] (0, 8) rectangle ++(7, -1);
    \node (to1) at (0, 5) {};
    \node (from2) at (7, 5) {};
  \end{scope}

  \begin{scope}[xshift=24cm]
    \draw[gray] (0, 0) grid (7, 10);

    \fill[\colJ, piece] (0, 10) \polyfour{1}{-2}{-1};
    \fill[\colT, piece] (3, 10) \polyeight{1}{-1}{1}{-1}{-3}{1}{1};
    \fill[\colS, piece] (6, 10) \polysix{1}{-2}{-2}{1}{1};
    \fill[\colS, piece] (1,  8) \polyfour{1}{-1}{-1};
    \fill[\colS, piece] (0,  7) \polyfour{1}{-1}{-1};
    \fill[\colO, piece] (2,  8) \polyfour{2}{-1}{-2};
    \fill[\colL, piece] (4,  8) \polyfour{1}{-1}{-1};
    \fill[\colT, piece] (3,  7) \polysix{2}{-2}{-1}{1}{-1};
    \fill[\colZ, piece] (1,  7) \polyeight{2}{-1}{-1}{-1}{-2}{1}{1};
    \fill[\colL, piece] (6,  8) \polyfour{1}{-1}{-1};
    \node (to2) at (0, 5) {};
    \draw[thick] (0, 0) rectangle (7, 10);
  \end{scope}

  \draw[darrow] (from1) -- (to1);
  \node[fcall] at ($ (from1)!.5!(to1) $) {\verb+poner_pieza+};

  \draw[darrow] (from2) -- (to2);
  \node[fcall] at ($ (from2)!.5!(to2) $) {\verb+liberar_lineas+};

\end{tikzpicture}
