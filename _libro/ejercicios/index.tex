\section{Ejercicios}

\subsection{Acerca de los enunciados}

Muchos de los enunciados de los ejercicios vienen acompañados de
\textbf{casos de prueba}, que muestran una sesión de ejemplo del
programa.

En estos casos de prueba, el texto que aparece en negrita indica qué es
lo que el usuario ingresa. Lo que aparece en texto normal es lo que el
programa muestra.

Por ejemplo, el siguiente es un caso de prueba de un programa:

En este ejemplo, el valor \lstinline!2! en la primera línea fue
ingresado por el usuario. El resto del texto fue impreso por el
programa, ya sea usando la sentencia \lstinline!print! o la función
\lstinline!raw_input!.

Para probar su programa, usted puede ejecutarlo e ingresar los datos tal
como se muestra en el caso de prueba, y debe obtener la misma salida.

Si su programa pasa todos los casos de prueba exitosamente, no significa
que esté correcto, pero por lo menos funciona para algunas entradas.
