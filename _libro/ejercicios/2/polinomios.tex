\section{Polinomios}

Un \emph{polinomio} de grado \(n\)
es una función matemática de la forma:
\[
  p(x) = a_0 + a_1 x + a_2 x^2 + a_3 x^3 + \cdots + a_n x^n,
\]
donde \(x\) es el parámetro y \(a_0, a_1, \dots, a_n\) son números reales
dados.

Algunos ejemplos de polinomios son:
\begin{itemize}
 \item \(p(x) = 1 + 2x + x^2\),
 \item \(q(x) = 4 - 17x\),
 \item \(r(x) = -1 - 5x^3 + 3x^5\),
 \item \(s(x) = 5x^{40} + 2x^{80}\).
\end{itemize}
Los grados de estos polinomios son, respectivamente, 2, 1, 5 y 80.

Evaluar un polinomio significa reemplazar \(x\) por un valor y obtener el
resultado. Por ejemplo, si evaluamos el polinomio \(p\) en el valor
\(x = 3\), obtenemos el resultado:
\[
  p(3) = 1 + 2\cdot 3 + 3^2 = 16.
\]

Un polinomio puede ser representado como una lista con los valores
\(a_0, a_1, \dots, a_n\). Por ejemplo, los polinomios anteriores pueden
ser representados así en un programa:
\begin{lstlisting}
>>> p = [1, 2, 1]
>>> q = [4, -17]
>>> r = [-1, 0, 0, -5, 0, 3]
>>> s = [0] * 40 + [5] + [0] * 39 + [2]
\end{lstlisting}

\begin{enumerate}

  \item
    Escriba la función \lstinline!grado(p)! que entregue el grado de un
    polinomio:
\begin{lstlisting}
>>> grado(r)
5
>>> grado(s)
80
\end{lstlisting}

  \item
    Escriba la función \lstinline!evaluar(p, x)! que evalúe el polinomio
    \lstinline!p! (representado como una lista) en el valor \lstinline!x!:
\begin{lstlisting}
>>> evaluar(p, 3)
16
>>> evaluar(q, 0.0)
4.0
>>> evaluar(r, 1.1)
-2.82347
>>> evaluar([4, 3, 1], 3.14)
23.2796
\end{lstlisting}

  \item
    Escriba la función \lstinline!sumar_polinomios(p1, p2)! que entregue
    la suma de dos polinomios:
\begin{lstlisting}
>>> sumar_polinomios(p, r)
[0, 2, 1, -5, 0, 3]
\end{lstlisting}

  \item
    Escriba la función \lstinline!derivar_polinomio(p)! que entregue la
    derivada del polinomio \lstinline!p!:
\begin{lstlisting}
>>> derivar_polinomio(r)
[0, 0, -15, 0, 15]
\end{lstlisting}

  \item
    Escriba la función \lstinline!multiplicar_polinomios(p1, p2)! que
    entregue el producto de dos polinomios:
\begin{lstlisting}
>>> multiplicar_polinomios(p, q)
[4, -9, -30, -17]
\end{lstlisting}

\end{enumerate}
