\section{Palabras especiales}

\begin{enumerate}
\item
  Dos palabras son \emph{anagramas} si tienen las mismas letras pero
  en otro orden. Por ejemplo, «torpes» y «postre» son anagramas,
  mientras que «aparta» y «raptar» no lo son, ya que «raptar» tiene una
  \emph{r} de más y una \emph{a} de menos.

  Escriba la función \lstinline!son_anagramas(p1, p2)! que indique si
  las palabras \lstinline!p1! y \lstinline!p2! son anagramas:

\begin{lstlisting}
>>> son_anagramas('torpes', 'postre')
True
>>> son_anagramas('aparta', 'raptar')
False
\end{lstlisting}
\item
  Las palabras \emph{panvocálicas} son las que tienen las cinco
  vocales. Por ejemplo: centrifugado, bisabuelo, hipotenusa.

  Escriba la función \lstinline!es_panvocalica(palabra)! que indique si
  una palabra es panvocálica o no:

\begin{lstlisting}
>>> es_panvocalica('educativo')
True
>>> es_panvocalica('pedagogico')
False
\end{lstlisting}
\item
  Escriba la función \lstinline!cuenta_panvocalicas(oracion)! que cuente
  cuántas palabras panvocálicas tiene una oración:

\begin{lstlisting}
>>> cuenta_panvocalicas('la contertulia estudiosa va a casa')
2
>>> cuenta_panvocalicas('los hipopotamos bailan al amanecer')
0
oracion = 'el abuelito mordisquea el aceituno con contundencia'
>>> cuenta_panvocalicas(oracion)
4
\end{lstlisting}
\item
  Escriba la función \lstinline!tiene_letras_en_orden(palabra)! que
  indique si las letras de la palabra están en orden alfabético:

\begin{lstlisting}
>>> tiene_letras_en_orden('himnos')
True
>>> tiene_letras_en_orden('abenuz')
True
>>> tiene_letras_en_orden('zapato')
False
\end{lstlisting}
\item
  Escriba la función \lstinline!tiene_letras_dos_veces(palabra)! que
  indique si cada letra de la palabra aparece exactamente dos veces:

\begin{lstlisting}
>>> tiene_letras_dos_veces('aristocraticos')
True
>>> tiene_letras_dos_veces('quisquilloso')
True
>>> tiene_letras_dos_veces('aristocracia')
False
\end{lstlisting}
\item
  Escriba la función \lstinline!palabras_repetidas(oracion)! que
  entregue una lista de las palabras que están repetidas en la oración:

\begin{lstlisting}
>>> palabras_repetidas('El partido termino cero a cero')
['cero']
>>> palabras_repetidas('El sobre esta sobre el mueble')
['el', 'sobre']
>>> palabras_repetidas('Ay, ahi no hay pan')
[]
\end{lstlisting}
\item
  Un \emph{pangrama} es un texto que tiene todas las letras del
  alfabeto, de la \emph{a} a la \emph{z}
  (por las limitaciones de Python 2.7 excluiremos la \emph{ñ}).
  Escriba la función \lstinline!es_pangrama(texto)!
  que indique si el texto es o no un pangrama:

\begin{lstlisting}
>>> a = 'Sylvia wagt quick den Jux bei Pforzheim.'
>>> b = 'Cada vez que trabajo, Felix me paga un whisky.'
>>> c = 'Cada vez que trabajo, Luis me invita a una cerveza.'
>>> es_pangrama(a)
True
>>> es_pangrama(b)
True
>>> es_pangrama(c)
False
\end{lstlisting}
\end{enumerate}
