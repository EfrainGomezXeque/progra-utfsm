\section{Expresiones con diccionarios}

Considere las siguientes asignaciones:
\begin{lstlisting}
>>> a = {'a': 14, 'b': 23, 'c': 88}
>>> b = {12: True, 55: False, -2: False}
>>> c = dict()
>>> d = {1: [2, 3, 4], 5: [6, 7, 8, 9], 10: [11]}
>>> e = {2 + 3: 4, 5: 6 + 7, 8: 9, 10: 11 + 12}
\end{lstlisting}

Sin usar el computador, indique cuál es el resultado y el tipo de las
si\-guien\-tes expresiones. A continuación, verifique sus respuestas en el
computador.

\begin{itemize}
  \item \lstinline!a['c']!
  \item \lstinline!a[23]!
  \item \lstinline!b[-2] or b[55]!
  \item \lstinline!23 in a!
  \item \lstinline!'a' in a!
  \item \lstinline!5 in d[5]!
  \item \lstinline!sum(b)!
  \item \lstinline!len(c)!
  \item \lstinline!len(d)!
  \item \lstinline!len(d[1])!
  \item \lstinline!len(b.values())!
  \item \lstinline!len(e)!
  \item \lstinline!sum(a.values())!
  \item \lstinline!max(list(e))!
  \item \lstinline!d[1] + d[5] + d[10]!
  \item \lstinline!max(map(len, d.values()))!
\end{itemize}
