\section{Compatibilidad entre personas}

Para este problema, consideraremos las siguientes características de una
persona:

\begin{itemize}
\item
  nombre,
\item
  género (masculino o femenino),
\item
  edad,
\item
  música favorita, y
\item
  signo zodiacal.
\end{itemize}

En el programa a realizar, una persona será representada como una tupla:

\begin{lstlisting}
persona_1 = ('Pepito', 'M', 27, 'rock', 'leo')
persona_2 = ('Yayita', 'F', 23, 'cumbia', 'virgo')
\end{lstlisting}

Dos personas son compatibles si:

\begin{itemize}
\item
  son de géneros opuestos (un hombre y una mujer),
\item
  tienen menos de diez años de diferencia,
\item
  les gusta la misma música, y
\item
  sus signos zodiacales son compatibles.
\end{itemize}

Para saber los signos compatibles, existe un conjunto
\lstinline!signos_compatibles! que tiene tuplas
\lstinline!(signo_mujer, signo_hombre)!, que
\href{../../\_static/signos.py}{usted puede descargar aquí}. Si una
tupla está en el conjunto, significa que los signos son compatibles:

\begin{lstlisting}
>>> ('aries', 'tauro') in signos_compatibles
True

# Significa que mujer aries
# es compatible con hombre tauro.

>>> ('capricornio', 'libra') in signos_compatibles
False

# Significa que mujer capricornio
# no es compatible con hombre libra.
\end{lstlisting}

Escriba la función \lstinline!compatibles(p1, p2)! que indique si dos
personas son compatibles o no.
