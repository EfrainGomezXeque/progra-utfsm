\section{Expresiones con listas}

Considere las siguientes listas:
\begin{lstlisting}
>>> a = [5, 1, 4, 9, 0]
>>> b = range(3, 10) + range(20, 23)
>>> c = [[1, 2], [3, 4, 5], [6, 7]]
>>> d = ['perro', 'gato', 'jirafa', 'elefante']
>>> e = ['a', a, 2 * a]
\end{lstlisting}

Sin usar el computador, indique cuál es el resultado y el tipo de las
si\-guien\-tes expresiones. A continuación, verifique sus respuestas en el
computador.
\begin{itemize}
  \item \lstinline!a[2]!
  \item \lstinline!b[9]!
  \item \lstinline!c[1][2]!
  \item \lstinline!e[0] == e[1]!
  \item \lstinline!len(c)!
  \item \lstinline!len(c[0])!
  \item \lstinline!len(e)!
  \item \lstinline!c[-1]!
  \item \lstinline!c[-1][+1]!
  \item \lstinline!c[2:] + d[2:]!
  \item \lstinline!a[3:10]!
  \item \lstinline!a[3:10:2]!
  \item \lstinline!d.index('jirafa')!
  \item \lstinline!e[c[0][1]].count(5)!
  \item \lstinline!sorted(a)[2]!
  \item \lstinline!complex(b[0], b[1])!
\end{itemize}
