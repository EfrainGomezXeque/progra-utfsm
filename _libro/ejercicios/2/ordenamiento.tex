\section{Ordenamiento}

El método \lstinline!sort! de las listas ordena sus elementos de menor a
mayor:
\begin{lstlisting}
>>> a.sort()
>>> a
[0, 1, 4, 6, 9]
\end{lstlisting}

A veces necesitamos ordenar los elementos de acuerdo a otro criterio.
Para esto, el método \lstinline!sort! acepta un parámetro con nombre
llamado \lstinline!key!, que debe ser una función que asocia a cada
elemento el valor que será usado para ordenar.
Por ejemplo, para ordenar la lista de mayor a menor uno puede usar una
función que cambie el signo de cada número:
\begin{lstlisting}
>>> def negativo(x):
...     return -x
...
>>> a = [6, 1, 4, 0, 9]
>>> a.sort(key=negativo)
>>> a
[9, 6, 4, 1, 0]
\end{lstlisting}

Esto significa que la lista es ordenada comparando los negativos de sus
elementos, aunque son los elementos originales los que aparecen en el
resultado.

Como segundo ejemplo, veamos cómo ordenar una lista de números por su
último dígito, de menor a mayor:

\begin{lstlisting}
>>> def ultimo_digito(n):
...     return n % 10
...
>>> a = [65, 71, 39, 30, 26]
>>> a.sort(key=ultimo_digito)
>>> a
[30, 71, 65, 26, 39]
\end{lstlisting}

Resuelva los siguientes problemas de ordenamiento, escribiendo la
función criterio para cada caso, e indicando qué es lo que debe ir en la
línea marcada con «\lstinline!???????!».

\begin{itemize}
\item
  Ordenar una lista de strings de la más corta a la más larga:

\begin{lstlisting}
>>> animales
['conejo', 'ornitorrinco', 'pez', 'hipopotamo', 'tigre']
>>> # ???????
>>> animales
['pez', 'tigre', 'conejo', 'hipopotamo', 'ornitorrinco']
\end{lstlisting}
\item
  Ordenar una lista de strings de la más larga a la más corta:

\begin{lstlisting}
>>> animales
['conejo', 'ornitorrinco', 'pez', 'hipopotamo', 'tigre']
>>> # ???????
>>> animales
['ornitorrinco', 'hipopotamo', 'conejo', 'tigre', 'pez']
\end{lstlisting}
\item
  Ordenar una lista de listas según la suma de sus elementos, de menor a
  mayor:

\begin{lstlisting}
>>> a = [
...   [6, 1, 5, 9],
...   [0, 0, 4, 0, 1],
...   [3, 2, 12, 1],
...   [1000],
...   [7, 6, 1, 0],
... ]
>>> # ??????
>>>
>>> a
[[0, 0, 4, 0, 1], [7, 6, 1, 0], [3, 2, 12, 1],
[6, 1, 5, 9], [1000]]
\end{lstlisting}

  (Las sumas en la lista ordenada son, respectivamente, 5, 14, 18, 21 y
  1000).
\item
  Ordenar una lista de tuplas
  \lstinline!(nombre, apellido, (anno, mes, dia))! por orden alfabético
  de apellidos:

\begin{lstlisting}
>>> personas = [
...     ('John',   'Doe',         (1992, 12, 28)),
...     ('Perico', 'Los Palotes', (1992, 10, 8)),
...     ('Yayita', 'Vinagre',     (1991,  4, 17)),
...     ('Fulano', 'De Tal',      (1992, 10, 4)),
... ]
>>> # ???????
>>> from pprint import pprint
>>> pprint(personas)
[('Fulano', 'De Tal', (1992, 10, 4)),
 ('John', 'Doe', (1992, 12, 28)),
 ('Perico', 'Los Palotes', (1992, 10, 8)),
 ('Yayita', 'Vinagre', (1991, 4, 17))]
\end{lstlisting}

  (La función \lstinline!pprint! sirve para imprimir estructuras de
  datos hacia abajo en vez de hacia el lado).
\item
  Ordenar una lista de tuplas
  \lstinline!(nombre, apellido, (anno, mes, dia))! por fecha de
  nacimiento, de la más antigua a la más reciente:

\begin{lstlisting}
>>> # ???????
>>> pprint(personas)
[('Yayita', 'Vinagre', (1991, 4, 17)),
 ('Fulano', 'De Tal', (1992, 10, 4)),
 ('Perico', 'Los Palotes', (1992, 10, 8)),
 ('John', 'Doe', (1992, 12, 28))]
\end{lstlisting}
\item
  Ordenar una lista de tuplas
  \lstinline!(nombre, apellido, (anno, mes, dia))! por fecha de
  nacimiento, pero ahora de la más reciente a la más antigua:

\begin{lstlisting}
>>> # ???????
>>> pprint(personas)
[('John', 'Doe', (1992, 12, 28)),
 ('Perico', 'Los Palotes', (1992, 10, 8)),
 ('Fulano', 'De Tal', (1992, 10, 4)),
 ('Yayita', 'Vinagre', (1991, 4, 17))]
\end{lstlisting}
\item
  Ordenar una lista de meses según la cantidad de días, de más a menos:

\begin{lstlisting}
>>> meses = ['agosto', 'noviembre', 'abril', 'febrero']
>>> # ???????
>>> meses
['febrero', 'noviembre', 'abril', 'agosto']
\end{lstlisting}
\item
  Hacer que queden los números impares a la izquierda y los pares a la
  derecha:

\begin{lstlisting}
>>> from random import randrange
>>> valores = []
>>> for i in range(12):
...     valores.append(randrange(256))
...
>>> valores
[55, 222, 47, 81, 82, 44, 218, 82, 20, 96, 82, 251]
>>> # ???????
>>> valores
[55, 47, 81, 251, 222, 82, 44, 218, 82, 20, 96, 82]
\end{lstlisting}
\item
  Hacer que queden los palíndromos a la derecha y los no palíndromos a
  la izquierda:

\begin{lstlisting}
>>> a = [12321, 584, 713317, 8990, 44444, 28902]
>>> # ????????
>>> a
[584, 8990, 28902, 12321, 713317, 44444]
\end{lstlisting}
\end{itemize}
