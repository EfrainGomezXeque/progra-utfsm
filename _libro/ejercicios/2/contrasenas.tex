\section{Contraseñas}

Actualmente un método muy utilizado para escoger una contraseña es
cambiar ciertas letras de una determinada palabra por números, por
ejemplo:

\begin{lstlisting}
me gusta el futbol.
\end{lstlisting}

\begin{lstlisting}
m3 gust4 3l futb0l!
\end{lstlisting}

Por lo tanto, para poder hacer más fácil esta tarea, realice una función
que utilizando:

\begin{itemize}
\item
  una frase.
\item
  un diccionario con los caracteres a reemplazar.
\end{itemize}

pueda entregar la contraseña con los caracteres reemplazados.

Recuerde la función \emph{replace()}.

\begin{lstlisting}
frase = "quiero mi password segura."
diccionario = {'a':4,'e':3,'i':1,'o':0,'.':'?'}
cambia(frase,diccionaio)
"qu13r0 m1 p4ssw0rd s3gur4?"
\end{lstlisting}

\begin{lstlisting}
frase = "gatito bonito"
diccionario = {'t':'n','o':''}
cambia(frase,diccionario)
"ganin bonin"
\end{lstlisting}

Además, como siempre se desea una contraseña mucho más segura modifique
la función anterior, para poder cambiar los caracteres sólo una cantidad
\emph{n} de veces.

Por ejemplo:

\begin{lstlisting}
frase = "pablito clavo un clavito"
diccionario = {'a':4,'o':0,'i':1}
cambia(frase,diccionario,1)
"p4bl1t0 clavo un clavito"
cambia(frase,diccionario,3)
"p4bl1t0 cl4v0 un cl4v1t0"
\end{lstlisting}

