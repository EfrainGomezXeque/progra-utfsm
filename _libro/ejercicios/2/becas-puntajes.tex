\section{Becas a mejores puntajes}

El Instituto Tecnológico de Putre, por intermedio de su Departamento de
Informática, desea llevar un control de todos los puntajes de ingresos
obtenidos por los postulantes a dicha universidad.

Fueron mil alumnos los que postularon, y sus puntajes puede encontrarlos
\href{../../\_static/puntajes.txt}{aqui}

La cantidad máxima de vacantes es de 850 estudiantes, por lo que deberá
seleccionar los mejores puntajes.

También desea premiar a todos los alumnos que ingresen con un puntaje
superior a 764 puntos, con una beca mensual de \$60.000.

Con esta información, se pide que desarrolle los siguientes ejercicios.

\begin{enumerate}
\item
  Escriba la función para guardar el contenido del archivo en una lista
  llamada \lstinline!obtener_puntajes(archivo)!.
\item
  Escriba la función para obtener dos listas, una con los alumnos
  aceptados y otra con los alumnos rechazados, llamada
  \lstinline!obtener_listas(alumnos)!
\item
  Escriba la función llamada \lstinline!calcular_becas(alumnos)! para
  poder determinar la cantidad mensual que deberá desembolsar el
  Instituto debido a las becas que otorgará.
\item
  Escriba la función llamada \lstinline!puntaje_promedio(alumnos)! para
  saber el promedio de los puntajes de las personas aceptadas en la
  universidad.
\item
  Utilizando la lista de las personas rechazadas, realice una función
  que permita obtener la
  \href{http://es.wikipedia.org/wiki/Desviaci\%C3\%B3n\_est\%C3\%A1ndar}{desviación
  estándar} de sus puntajes.
\end{enumerate}
