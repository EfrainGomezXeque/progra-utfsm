\section{Desviación estándar}

Desarrolle una función llamada \lstinline!desviacion_estandar(valores)!
cuyo parámetro \lstinline!valores! sea una lista de números reales,
que retorne la \emph{desviación estándar} de éstos,
definida como:
\[\sigma = \sqrt{\sum_{i} \frac{(x_i - \bar{x})^2}{n - 1}}\]
donde \(n\) es la cantidad de valores, \(\bar{x}\) es el promedio de los
valores, y los \(x_i\) son cada uno de los valores.

Esto significa que hay que hacerlo siguiendo estos pasos:

\begin{itemize}
\item
  calcular el promedio de los valores;
\item
  a cada valor hay que restarle el promedio, y el resultado elevarlo al
  cuadrado;
\item
  sumar todos los valores obtenidos;
\item
  dividir la suma por la cantidad de valores; y
\item
  sacar la raíz cuadrada del resultado.
\end{itemize}

\begin{lstlisting}
>>> desviacion_estandar([1.3, 1.3, 1.3])
0.0
>>> desviacion_estandar([4.0, 1.0, 11.0, 13.0, 2.0, 7.0])
4.88535225615
>>> desviacion_estandar([1.5, 9.5])
5.65685424949
\end{lstlisting}

