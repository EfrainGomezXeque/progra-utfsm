\section{Campeonato de fútbol}

Los resultados de un campeonato de fútbol están almacenados en un
diccionario. Las llaves son los partidos y los valores son los
resultados. Cada partido es representado como una tupla con los dos
equipos que jugaron, y el resultado es otra tupla con los goles que hizo
cada equipo:
\begin{lstlisting}
resultados = {
   ('Honduras', 'Chile'):    (0, 1),
   ('Espana',   'Suiza'):    (0, 1),
   ('Suiza',    'Chile'):    (0, 1),
   ('Espana',   'Honduras'): (3, 0),
   ('Suiza',    'Honduras'): (0, 0),
   ('Espana',   'Chile'):    (2, 1),
}
\end{lstlisting}

\begin{enumerate}
\item
  Escriba la función \lstinline!obtener_lista_equipos(resultados)! que
  reciba como parámetro el diccionario de resultados y retorne una lista
  con todos los equipos que participaron del campeonato:

\begin{lstlisting}
>>> obtener_lista_equipos(resultados)
['Honduras', 'Suiza', 'Espana', 'Chile']
\end{lstlisting}
\item
  El equipo que gana un partido recibe tres puntos y el que pierde,
  cero. En caso de empate, ambos equipos reciben un punto.

  Escriba la función \lstinline!calcular_puntos(equipo, resultados)! que
  entregue la cantidad de puntos obtenidos por un equipo:

\begin{lstlisting}
>>> calcular_puntos('Chile', resultados)
6
>>> calcular_puntos('Suiza', resultados)
4
\end{lstlisting}
\item
  La \emph{diferencia de goles} de un equipo es el total de goles que
  anotó un equipo menos el total de goles que recibió.

  Escriba la función
  \lstinline!calcular_diferencia_de_goles(equipo, resultados)! que
  entregue la diferencia de goles de un equipo:

\begin{lstlisting}
>>> calcular_diferencia_de_goles('Chile', resultados)
1
>>> calcular_diferencia_de_goles('Honduras', resultados)
-4
\end{lstlisting}
\item
  Escriba la función \lstinline!posiciones(resultados)! que reciba como
  parámetro el diccionario de resultados, y retorne una lista con los
  equipos ordenados por puntaje de mayor a menor. Los equipos que tienen
  el mismo puntaje deben ser ordenados por diferencia de goles de mayor
  a menor. Si tienen los mismos puntos y la misma diferencia de goles,
  deben ser ordenados por los goles anotados:

\begin{lstlisting}
>>> posiciones(resultados)
['Espana', 'Chile', 'Suiza', 'Honduras']
\end{lstlisting}

  En este ejemplo, España queda clasificado en primer lugar porque tiene
  6 puntos y diferencia de goles de \(+3\), mientras que Chile también tiene
  6 puntos, pero diferencia de goles de \(+1\).
\end{enumerate}
