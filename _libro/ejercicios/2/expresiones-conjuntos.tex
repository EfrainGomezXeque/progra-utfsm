\section{Expresiones con conjuntos}

Considere las siguientes asignaciones:
\begin{lstlisting}
>>> a = {5, 2, 3, 9, 4}
>>> b = {3, 1}
>>> c = {7, 5, 5, 1, 8, 6}
>>> d = [6, 2, 4, 5, 5, 3, 1, 3, 7, 8]
>>> e = {(2, 3), (3, 4), (4, 5)}
>>> f = [{2, 3}, {3, 4}, {4, 5}]
\end{lstlisting}

Sin usar el computador, indique cuál es el resultado y el tipo de las
si\-guien\-tes expresiones. A continuación, verifique sus respuestas en el
computador.
\begin{itemize}
  \item \lstinline!len(c)!
  \item \lstinline!len(set(d))!
  \item \lstinline!a & (b | c)!
  \item \lstinline!(a & b) | c!
  \item \lstinline!c - a!
  \item \lstinline!max(e)!
  \item \lstinline!f[0] < a!
  \item \lstinline!set(range(4)) & a!
  \item \lstinline!(set(range(4)) & a) in f!
  \item \lstinline!len(set('perro'))!
  \item \lstinline!len({'perro'})!
\end{itemize}
