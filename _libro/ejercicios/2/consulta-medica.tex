\section{Consulta médica}

Una consulta médica tiene un archivo \lstinline!pacientes.txt! con los
datos personales de sus pacientes. Cada línea del archivo tiene el rut,
el nombre y la edad de un paciente, separados por un símbolo
«\lstinline!:!». Así se ve el archivo:

\begin{lstlisting}[language=file]
12067539-7:Anastasia Lopez:32
15007265-4:Andres Morales:26
8509454-8:Pablo Munoz:45
7752666-8:Ignacio Navarro:49
8015253-1:Alejandro Pacheco:51
9217890-0:Patricio Pimienta:39
9487280-4:Ignacio Rosas:42
12393241-2:Ignacio Rubio:33
11426761-9:Romina Perez:35
15690109-1:Francisco Ruiz:26
6092377-9:Alfonso San Martin:65
9023365-3:Manuel Toledo:38
10985778-5:Jesus Valdes:38
13314970-8:Abel Vazquez:30
7295601-k:Edison Munoz:60
5106360-0:Andrea Vega:71
8654231-5:Andres Zambrano:55
10105321-0:Antonio Almarza:31
13087677-3:Jorge Alvarez:28
9184011-1:Laura Andrade:47
12028339-1:Jorge Argandona:29
10523653-0:Camila Avaria:40
12187197-1:Felipe Avila:36
5935556-2:Aquiles Barriga:80
14350739-4:Eduardo Bello:29
6951420-0:Cora Benitez:68
11370775-5:Hugo Berger:31
11111756-k:Cristobal Borquez:34
\end{lstlisting}

Además, cada vez que alguien se atiende en la consulta, la visita es
registrada en el archivo \lstinline!atenciones.txt!, agregando una línea
que tiene el rut del paciente, la fecha de la visita (en formato
\texttt{dia-mes-año}) y el precio de la atención, también separados
por «\lstinline!:!». El archivo se ve así:

\begin{lstlisting}[language=file]
8015253-1:4-5-2010:69580
12393241-2:6-5-2010:57274
10985778-5:8-5-2010:73206
8015253-1:10-5-2010:30796
8015253-1:12-5-2010:47048
12028339-1:12-5-2010:47927
11426761-9:13-5-2010:39117
10985778-5:15-5-2010:86209
7752666-8:18-5-2010:41916
8015253-1:18-5-2010:74101
12187197-1:20-5-2010:38909
8654231-5:20-5-2010:75018
8654231-5:22-5-2010:64944
5106360-0:24-5-2010:53341
8015253-1:27-5-2010:76047
9217890-0:30-5-2010:57726
7752666-8:1-6-2010:54987
8509454-8:2-6-2010:76483
6092377-9:2-6-2010:62106
11370775-5:3-6-2010:67035
11370775-5:7-6-2010:47299
8509454-8:7-6-2010:73254
8509454-8:10-6-2010:82955
11111756-k:10-6-2010:56520
7752666-8:10-6-2010:40820
12028339-1:12-6-2010:79237
11111756-k:13-6-2010:69094
5935556-2:14-6-2010:73174
11111756-k:21-6-2010:70417
11426761-9:22-6-2010:80217
12067539-7:25-6-2010:31555
11370775-5:26-6-2010:75796
10523653-0:26-6-2010:34585
6951420-0:28-6-2010:45433
5106360-0:1-7-2010:48445
8654231-5:4-7-2010:76458
\end{lstlisting}

Note que las fechas están ordenadas de menos a más reciente, ya que las
nuevas líneas siempre se van agregando al final.

\begin{enumerate}
\item
  Escriba la función \lstinline!costo_total_paciente(rut)! que entregue
  el costo total de las atenciones del paciente con el rut dado:

\begin{lstlisting}
>>> costo_total_paciente('8015253-1')
297572
>>> costo_total_paciente('14350739-4')
0
\end{lstlisting}
\item
  Escriba la función \lstinline!pacientes_dia(dia, mes, ano)! que
  entregue una lista con los nombres de los pacientes que se atendieron
  el día señalado:

\begin{lstlisting}
>>> pacientes_dia(2, 6, 2010)
['Pablo Munoz', 'Alfonso San Martin']
>>> pacientes_dia(23, 6, 2010)
[]
\end{lstlisting}
\item
  Escriba la función \lstinline!separar_pacientes()! que construya dos
  nuevos archivos:

  \begin{itemize}
  \item
    \lstinline!jovenes.txt!, con los datos de los pacientes menores de
    30 años;
  \item
    \lstinline!mayores.txt!, con los datos de todos los pacientes
    mayores de 60 años.
  \end{itemize}

  Por ejemplo, el archivo \lstinline!jovenes.txt! debe verse así:

\begin{lstlisting}[language=file,inputencoding=utf8/latin1]
15007265-4:Andres Morales:26
15690109-1:Francisco Ruiz:26
13087677-3:Jorge Alvarez:28
12028339-1:Jorge Argandona:29
14350739-4:Eduardo Bello:29
\end{lstlisting}
\item
  Escribir una función \lstinline!ganancias_por_mes()! que construya un
  nuevo archivo llamado \lstinline!ganancias.txt! que tenga el total de
  ganancias por cada mes en el siguiente formato:

\begin{lstlisting}[language=file]
5-2010:933159
6-2010:1120967
7-2010:124903
\end{lstlisting}
\end{enumerate}
