\section{Vocales y consonantes}

Escriba una función que determine si una letra es vocal o consonante.
Decida usted qué es lo que retornará la función. Por ejemplo, podría ser
así:
%
\begin{lstlisting}
>>> es_vocal('a')
True
>>> es_vocal('b')
False
\end{lstlisting}
%
O así:
%
\begin{lstlisting}
>>> es_consonante('a')
False
>>> es_consonante('b')
True
\end{lstlisting}
%
O incluso así:
%
\begin{lstlisting}
>>> tipo_de_letra('a')
'vocal'
>>> tipo_de_letra('b')
'consonante'
\end{lstlisting}

A continuación, escriba una función
%\lstinline!contar_vocales_y_consonantes(palabra)!
que retorne las cantidades de vocales y consonantes de la palabra.
Esta función debe llamar a la función que usted escribió antes.

\begin{lstlisting}
>>> v, c = contar_vocales_y_consonantes('edificio')
>>> v
5
>>> c
3
\end{lstlisting}

Finalmente, escriba un programa que pida al usuario ingresar una palabra
y le muestre cuántas vocales y consonantes tiene:
%
\begin{lstlisting}[language=testcase]
Ingrese palabra: `edificio`
Tiene 5 vocales y 3 consonantes
\end{lstlisting}
