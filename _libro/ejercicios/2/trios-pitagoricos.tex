\section{Trios pitagóricos}

Un trío pitagórico se define como un conjunto de tres números, \emph{a},
\emph{b} y \emph{c} que cumplen con la relación.

\[a^{2} + b^{2} = c^{2}\]

Desarrolle un programa que contenga la función
\lstinline!son_pitagoricos(a, b, c)! que retorne \lstinline!True! si
\lstinline!a!, \lstinline!b! y \lstinline!c! son un trío pitagórico, y
\lstinline!False! si no lo son:

\begin{lstlisting}
>>> son_pitagoricos(3, 4, 5)
True
>>> son_pitagoricos(4, 6, 9)
False
>>> son_pitagoricos(5, 12, 13)
True
\end{lstlisting}

A continuación, en el mismo programa escriba la función
\lstinline!pitagoricos(n)! que retorne la lista de todos los tríos
pitagóricos (como tuplas) todos los tríos pitagóricos cuyos valores son
menores que \lstinline!n!:

\begin{lstlisting}
>>> pitagoricos(18)
[(3, 4, 5), (4, 3, 5), (5, 12, 13), (6, 8, 10), (8, 6, 10), (8, 15, 17), (9, 12, 15), (12, 5, 13), (12, 9, 15), (15, 8, 17)]
\end{lstlisting}

