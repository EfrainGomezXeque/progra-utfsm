\section{Recorrido de diccionarios}

\begin{enumerate}


\item
Escriba la función
\lstinline!hay_llaves_pares(d)! que indique si el diccionario
\lstinline!d! tiene alguna llave que sea un número par.

A continuación, escriba una función \lstinline!hay_valores_pares(d)! que
indique si el diccionario \lstinline!d! tiene algún valor que sea un
número par.

Para probar las funciones, ocupe diccionarios cuyas llaves y valores
sean sólo números enteros:

\begin{lstlisting}
>>> d1 = {1: 2, 3: 5}
>>> d2 = {2: 1, 6: 7}
>>> hay_valores_pares(d1)
True
>>> hay_valores_pares(d2)
False
>>> hay_llaves_pares(d1)
False
>>> hay_llaves_pares(d2)
True
\end{lstlisting}

\item
Escriba la función \lstinline!maximo_par(d)! que
entregue el valor máximo de la suma de una llave y un valor del
diccionario \lstinline!d!:

\begin{lstlisting}
>>> d = {5: 1, 4: 7, 9: 0, 2: 2}
>>> maximo_par(d)
11
\end{lstlisting}

\item
Escriba la función \lstinline!invertir(d)! que
entregue un diccionario cuyas llaves sean los valores de \lstinline!d! y
cuyos valores sean las llaves respectivas:

\begin{lstlisting}
>>> invertir({1: 2, 3: 4, 5: 6})
{2: 1, 4: 3, 6: 5}
>>> apodos = {
...   'Suazo': 'Chupete',
...   'Sanchez': 'Maravilla',
...   'Medel': 'Pitbull',
...   'Valdivia': 'Mago',
... }
>>> invertir(apodos)
{'Maravilla': 'Sanchez', 'Mago': 'Valdivia',
'Chupete': 'Suazo', 'Pitbull': 'Medel'}
\end{lstlisting}

\end{enumerate}
