\section{Cartones de Loto}

Para los siguientes ejercicios, descarge el archivo
\href{../../\_static/juegos.txt}{juegosttxt}. Este archivo tiene una
lista de todos los cartones jugados para un sorteo de Loto. Cada línea
del archivo tiene la lista de números jugados en un cartón.

Este archivo se puede abrir con cualquier editor de texto para ver su
contenido, pero para resolver los problemas, hay que escribir funciones
que analicen los datos.

Todas las funciones deben hacer lo siguiente:

\begin{itemize}
\item
  abrir el archivo con \lstinline!archivo = open('juegos.txt')!;
\item
  leer los datos y analizarlos,
\item
  cerrar el archivo con \lstinline!archivo.close()!.
\end{itemize}

Como cada línea del archivo es un string, hay que convertirlo a un
conjunto de números para poder analizarlos, de la siguiente manera:

\begin{lstlisting}
numeros_carton = set()
for n in linea.split():
    numeros_carton.add(int(n))
\end{lstlisting}

También se puede hacer así:

\begin{lstlisting}
numeros_carton = set(map(int, linea.split()))
\end{lstlisting}

\begin{enumerate}
\item
  ¿Cuántos cartones de Loto fueron jugados?

  (Para responder la pregunta, escriba una función
  \lstinline!contar_cartones! que cuente los cartones del archivo).
\item
  De todos los cartones jugados, ¿cuántos escogieron el número 7?

  Para responder la pregunta, escriba una función
  \lstinline!contar_numero_en_cartones(n)! que cuente cuántos cartones
  tienen el número \lstinline!n!.
\item
  Escriba la función \lstinline!hay_ganadores(numeros)!, cuyo parámetro
  \lstinline!numeros! es el conjunto de los seis números de un sorteo,
  que indique si alguien se ganó el Loto:

\begin{lstlisting}
>>> hay_ganadores({13, 33, 5, 38, 1, 19})
True
>>> hay_ganadores({14, 21, 1, 36, 9, 17})
False
\end{lstlisting}
\item
  Escriba la función \lstinline!n_aciertos(numeros, n)!, que indique
  cuántas personas tuvieron \lstinline!n! aciertos, donde
  \lstinline!numeros! es el conjunto de los seis números de un sorteo:

\begin{lstlisting}
>>> n_aciertos({13, 33, 5, 38, 1, 19}, 4)
17
>>> n_aciertos({20, 39, 6, 27, 12, 4}, 3)
229
>>> n_aciertos({1, 2, 3, 4, 5, 6}, 5)
2
\end{lstlisting}
\end{enumerate}
