\section{Intersección de circunferencias}

En el plano, una circunferencia está determinada por su centro
(\emph{x}, \emph{y}) y por su radio \emph{r}.

En un programa, podemos representarla como una tupla
\lstinline!(centro, radio)!, donde a su vez \lstinline!centro! es una
tupla \lstinline!(x, y)!:

\begin{lstlisting}
c = ((4.0, 5.1), 2.8)
\end{lstlisting}

\begin{enumerate}[1.]
\item
  Escriba la función \lstinline!distancia(p1, p2)! que entregue la
  distancia entre los puntos \lstinline!p1! y \lstinline!p2!:

\begin{lstlisting}
>>> distancia((2, 2), (7, 14))
13.0
>>> distancia((2, 5), (1, 9))
4.1231056256176606
\end{lstlisting}
\item
  Escriba la función \lstinline!se_intersectan(c1, c2)! que indique si
  las circunferencias \lstinline!c1! y \lstinline!c2! se intersectan:

  \includegraphics{../../diagramas/circunferencias.png}

\begin{lstlisting}
>>> A = (( 5.0,  4.0), 3.0)
>>> B = (( 8.0,  6.0), 2.0)
>>> C = (( 8.4, 12.7), 3.0)
>>> D = (( 8.0, 12.0), 2.0)
>>> E = ((16.0,  7.8), 2.7)
>>> F = ((15.5,  2.7), 2.1)
>>> se_intersectan(A, B)
True
>>> se_intersectan(C, D)
False
>>> se_intersectan(E, F)
False
\end{lstlisting}
\end{enumerate}
