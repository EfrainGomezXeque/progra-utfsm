\section{Intersección de circunferencias}

En el plano, una circunferencia está determinada por su centro
\((x, y)\) y por su radio \(r\).
En un programa, podemos representarla como una tupla
\lstinline!(centro, radio)!, donde a su vez \lstinline!centro! es una
tupla \lstinline!(x, y)!:
\begin{lstlisting}
c = ((4.0, 5.1), 2.8)
\end{lstlisting}

\begin{enumerate}

  \item
    Escriba la función \lstinline!distancia(p1, p2)! que entregue la
    distancia entre los puntos \lstinline!p1! y \lstinline!p2!:
\begin{lstlisting}
>>> distancia((2, 2), (7, 14))
13.0
>>> distancia((2, 5), (1, 9))
4.1231056256176606
\end{lstlisting}

  \item

    \begin{figure}
      \centering
      \documentclass{minimal}
\usepackage[pdftex,active,tightpage]{preview}
\usepackage[utf8]{inputenc}
\usepackage{mathpazo}
\usepackage{tikz}
\usetikzlibrary{calc,arrows,decorations}

\begin{document}
\begin{preview}
\begin{tikzpicture}[scale=.5]

  \newcommand\circulo[5]{
    \node[fill=#4, inner sep=1pt] (p1) at (#1, #2) {};
    \draw[very thick, #4] (p1) circle (#3 cm);
    \node[#4] at ([shift=(130:{.5 + {#3}})] p1) {#5};
  }

  \draw[gray] (0, 0) grid (20, 16);
  \foreach\x in {0,...,20}
    \node[gray, anchor=north] at (\x, -.1) {\x};
  \foreach\y in {0,...,16}
    \node[gray, anchor=east] at (-.1, \y) {\y};

  \circulo{5}{4}{3}{blue!30!black}{A}
  \circulo{8}{6}{2}{blue!50!black}{B}

  \circulo{8.4}{12.7}{3}{red!30!black}{C}
  \circulo{8}{12}{2}{red!50!black}{D}

  \circulo{16}{7.8}{2.7}{green!30!black}{E}
  \circulo{15.5}{2.7}{2.1}{green!50!black}{F}

\end{tikzpicture}
\end{preview}

\end{document}


      \caption{Algunos casos de intersecciones
        entre pares de circunferencias.}
      \label{fig:circunferencias}
    \end{figure}

    Escriba la función \lstinline!se_intersectan(c1, c2)! que indique si
    las circunferencias \lstinline!c1! y \lstinline!c2! se intersectan.
    Por ejemplo,
    con las circunferencias de la figura~\ref{fig:circunferencias}:
\begin{lstlisting}
>>> A = (( 5.0,  4.0), 3.0)
>>> B = (( 8.0,  6.0), 2.0)
>>> C = (( 8.4, 12.7), 3.0)
>>> D = (( 8.0, 12.0), 2.0)
>>> E = ((16.0,  7.8), 2.7)
>>> F = ((15.5,  2.7), 2.1)
>>> se_intersectan(A, B)
True
>>> se_intersectan(C, D)
False
>>> se_intersectan(E, F)
False
\end{lstlisting}
\end{enumerate}
