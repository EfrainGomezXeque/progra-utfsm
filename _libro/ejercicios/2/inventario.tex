\section{Inventario}

Una tienda tiene la información de sus productos en un archivo llamado
\lstinline!productos.txt!. Cada línea del archivo tiene tres datos:

\begin{itemize}
\item
  el código del producto (un número entero),
\item
  el nombre del producto, y
\item
  la cantidad de unidades del producto que quedan en bodega.
\end{itemize}

Los datos están separados por un símbolo \lstinline!/!. Por ejemplo, el
siguiente puede ser el contenido del archivo:

\begin{lstlisting}
1265/Reloj/26
613/Cuaderno/87
9801/Vuvuzela/3
321/Lápiz/12
5413/Tomate/5
\end{lstlisting}

\begin{enumerate}[1.]
\item
  Escriba la función \lstinline!existe_producto(codigo)! que indique si
  existe el producto con el código dado:

\begin{lstlisting}
>>> existe_producto(1784)
False
>>> existe_producto(321)
True
>>> existe_producto(613)
True
>>> existe_producto(0)
False
\end{lstlisting}
\item
  Escriba la función \lstinline!por_reabastecer()! que cree un nuevo
  archivo llamado \lstinline!por_reabastecer.txt! que contenga los datos
  de todos los productos de los que queden menos de 10 unidades.

  En este caso, el archivo \lstinline!por_reabastecer.txt! debe quedar
  así:
\end{enumerate}

\begin{lstlisting}
9801/Vuvuzela/3
5413/Tomate/5
\end{lstlisting}

