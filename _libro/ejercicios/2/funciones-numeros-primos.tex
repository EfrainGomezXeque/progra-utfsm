\section{Funciones de números primos}

Para el ejercicio~\ref{sec:primos} usted debió
desarrollar programas sobre números primos.
Muchos de estos programas sólo tenían pequeñas diferencias entre ellos,
por lo que había que repetir mucho código al escribirlos. En este
ejercicio, usted deberá implementar algunos de esos programas como
funciones, reusando componentes para evitar escribir código
repetido.

\begin{enumerate}
\item
  Escriba la función \lstinline!es_divisible(n, d)! que indique si
  \lstinline!n! es divisible por \lstinline!d!:

\begin{lstlisting}
>>> es_divisible(15, 5)
True
>>> es_divisible(15, 6)
False
\end{lstlisting}
\item
  Usando la función \lstinline!es_divisible!, escriba una función
  \lstinline!es_primo(n)! que determine si el número \lstinline!n!
  es primo o no:

\begin{lstlisting}
>>> es_primo(17)
True
>>> es_primo(221)
False
\end{lstlisting}
\item
  Usando la función \lstinline!es_primo!, escriba la función
  \lstinline!i_esimo_primo(i)! que entregue el \lstinline!i!-ésimo número primo:

\begin{lstlisting}
>>> i_esimo_primo(1)
2
>>> i_esimo_primo(20)
71
\end{lstlisting}
\item
  Usando las funciones anteriores, escriba la función
  \lstinline!primeros_primos(m)! que entregue una lista de los primeros
  \lstinline!m! números primos:

\begin{lstlisting}
>>> primeros_primos(10)
[2, 3, 5, 7, 11, 13, 17, 19, 23, 29]
\end{lstlisting}
\item
  Usando las funciones anteriores, escriba la función
  \lstinline!primos_hasta(m)! que entregue una lista de los primos
  menores o iguales que \lstinline!m!:

\begin{lstlisting}
>>> primos_hasta(19)
[2, 3, 5, 7, 11, 13, 17, 19]
\end{lstlisting}
\item
  Cree un módulo llamado \lstinline!primos.py! que contenga todas las
  funciones ante\-rio\-res.
  Al ejecutar \lstinline!primos.py! como un programa por sí solo, debe
  mostrar, a modo de prueba, los veinte primeros números primos. Al
  importarlo como un módulo, esto no debe ocurrir.
\item
  Un \emph{primo de Mersenne} es un número primo de la forma \(2^p - 1\).
  Una propiedad conocida de los primos de Mersenne es que \(p\) debe ser
  también un número primo.

  Escriba un programa llamado \lstinline!mersenne.py! que pregunte al
  usuario un entero \(n\), y muestre como salida los primeros \(n\) primos
  de Mersenne:
\begin{lstlisting}[language=testcase]
Cuantos primos de Mersenne: `5`
3
7
31
127
8191
\end{lstlisting}

  Su programa debe importar el módulo \lstinline!primos! y usar las
  funciones que éste contiene.
\end{enumerate}
