\section{Iguales o distintos}

Escriba la función \lstinline!todos_iguales(lista)! que indique si todos
los elementos de una lista son iguales:

\begin{lstlisting}
>>> todos_iguales([6, 6, 6])
True
>>> todos_iguales([6, 6, 1])
False
>>> todos_iguales([0, 90, 1])
False
\end{lstlisting}

A continuación, escriba una función \lstinline!todos_distintos(lista)!
que indique si todos los elementos de una lista son distintos:

\begin{lstlisting}
>>> todos_distintos([6, 6, 6])
False
>>> todos_distintos([6, 6, 1])
False
>>> todos_distintos([0, 90, 1])
True
\end{lstlisting}

Sus funciones deben ser capaces de aceptar listas de cualquier tamaño y
con cualquier tipo de datos:

\begin{lstlisting}
>>> todos_iguales([7, 7, 7, 7, 7, 7, 7, 7, 7])
True
>>> todos_distintos(list(range(1000)))
True
>>> todos_iguales([12])
True
>>> todos_distintos(list('hiperblanduzcos'))
True
\end{lstlisting}

