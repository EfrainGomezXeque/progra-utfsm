\section{Calcular la raíz cuadrada}

Desarrolle una función que permita calcular aproximadamente la raíz
cuadrada de un número de acuerdo al siguiente procedimiento:

\begin{itemize}
\item
  Se toma el número inicial y se le resta el primer número impar (el
  uno), a este resultado se le resta el siguiente número impar y así
  sucesivamente hasta que el resultado de la resta sea menor o igual a
  cero.
\item
  Si el resultado final es igual a cero se trata de un número con raíz
  entera y estará dada por la cantidad de veces que se hizo la resta,
  incluyendo el cero.
\item
  Si el resultado es menor que cero, el número no tiene raíz perfecta y
  el resultado aproximado (truncado) estará dada por la cantidad de
  veces que se hizo la resta menos uno.
\end{itemize}

Por ejemplo:

\begin{lstlisting}
36

36 - 1  = 35
35 - 3  = 32
32 - 5  = 27
27 - 7  = 20
20 - 9  = 11
11 - 11 = 0

6 veces
raíz aproximada: 6
\end{lstlisting}

\begin{lstlisting}
8

8 - 1 = 7
7 - 3 = 4
4 - 5 = -1

3 veces
raíz aproximada: 3
\end{lstlisting}

La dinámica de la función deberá ser la siguiente:

\begin{lstlisting}
>>> raiz_aproximada(25)
5
>>> raiz_aproximada(6)
2
>>> raiz_aproximada(1)
1
\end{lstlisting}

