\section{Manos de póker}

En los juegos de naipes, una carta tiene dos atributos: un valor (2, 3,
4, 5, 6, 7, 8, 9, 10, J, Q, K, A) y un palo (♥, ♦, ♣, ♠).

En un programa, el valor puede ser representado como un número del 1 al
13, y el palo como un string: ♥ → \lstinline!'C'!, ♦ → \lstinline!'D'!,
♣ → \lstinline!'T'! y ♠ → \lstinline!'P'!.

Una carta puede ser representada como una tupla de dos elementos: el
valor y el palo:

\begin{lstlisting}
carta1 = (5, 'T')
carta2 = (10, 'D')
\end{lstlisting}

Para simplificar, se puede representar el as como un 1, y los «monos» J,
Q y K como 11, 12 y 13:

\begin{lstlisting}
# as de picas y reina de corazones
carta3 = (1, 'P')
carta4 = (12, 'C')
\end{lstlisting}

En el juego de póker, una mano tiene cinco cartas, lo que en un programa
vendría a ser un conjunto de cinco tuplas:

\begin{lstlisting}
mano = {(1, 'P'), (1, 'C'), (1, 'T'), (13, 'D'), (12, 'P')}
\end{lstlisting}

\begin{enumerate}
\item
  Un \emph{full} es una mano en que tres cartas tienen el mismo valor, y
  las otras dos tienen otro valor común. Por ejemplo, A♠ A♥ 6♣ A♦ 6♦ es
  un full (tres ases y dos seis), pero 2♣ A♥ Q♥ A♦ 6♦ no.

  Escriba la función que indique si la mano es un full:

\begin{lstlisting}
>>> mano_1 = {(1, 'P'), (1, 'C'), (6, 'T'), (1, 'D'), (6, 'D')}
>>> mano_2 = {(2, 'T'), (1, 'C'), (12, 'C'), (1, 'D'), (6, 'D')}
>>> es_full(mano_1)
True
>>> es_full(mano_2)
False
\end{lstlisting}
\item
  Un \emph{color} es una mano en que todas las cartas tienen el mismo
  palo. Por ejemplo, 8♠ K♠ 4♠ 9♠ 2♠ es un color (todas las cartas son
  picas), pero Q♣ A♥ 5♥ 2♥ 2♦ no lo es.

  Escriba la función que indique si la mano es un color:

\begin{lstlisting}
>>> mano_1 = {(8, 'P'), (13, 'P'), (4, 'P'), (9, 'P'), (2, 'P')}
>>> mano_2 = {(12, 'T'), (1, 'C'), (5, 'C'), (2, 'C'), (2, 'D')}
>>> es_color(mano_1)
True
>>> es_color(mano_2)
False
\end{lstlisting}
\item
  Una \emph{escalera} es una mano en que las cartas tienen valores
  consecutivos. Por ejemplo, 4♠ 7♥ 3♥ 6♣ 5♣ es una escalera (tiene los
  valores 3, 4, 5, 6 y 7), pero Q♣ 7♥ 3♥ Q♥ 5♣ no lo es.

  Escriba la función que indique si la mano es una escalera:

\begin{lstlisting}
>>> mano_1 = {(4, 'P'), (7, 'C'), (3, 'C'), (6, 'T'), (5, 'T')}
>>> mano_2 = {(12, 'T'), (7, 'C'), (3, 'C'), (12, 'C'), (5, 'T')}
>>> es_escalera(mano_1)
True
>>> es_escalera(mano_2)
False
\end{lstlisting}
\item
  Una \emph{escalera de color} es una escalera en la que todas las
  cartas tienen el mismo palo. Por ejemplo, 4♦ 7♦ 3♦ 6♦ 5♦ es una
  escalera de color (son sólo diamantes, y los valores 3, 4, 5, 6 y 7
  son consecutivos).

  Escriba la función que indique si la mano es una escalera de color:

\begin{lstlisting}
>>> mano_1 = {(4, 'P'), (7, 'C'), (3, 'C'), (6, 'T'), (5, 'T')}
>>> mano_2 = {(8, 'P'), (13, 'P'), (4, 'P'), (9, 'P'), (2, 'P')}
>>> mano_3 = {(4, 'D'), (7, 'D'), (3, 'D'), (6, 'D'), (5, 'D')}
>>> es_escalera_de_color(mano_1)
False
>>> es_escalera_de_color(mano_2)
False
>>> es_escalera_de_color(mano_3)
True
\end{lstlisting}
\item
  Escriba las funciones para identificar las demás `manos del póker`\_.
\item
  Escriba un programa que pida al usuario ingresar cinco cartas, y le
  indique qué tipo de mano es:
\end{enumerate}
