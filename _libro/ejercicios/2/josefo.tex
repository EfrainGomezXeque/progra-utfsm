\section{Problema de Josefo}

El
\href{http://es.wikipedia.org/wiki/Problema\_de\_Flavio\_Josefo}{problema
de Josefo} es el siguiente: `m` personas están en un círculo, y son
ejecutadas en orden contando cada `n` personas; el que queda solo al
final es el sobreviviente.

Por ejemplo, con `m = 12` y `n = 3`, el sobreviviente es la persona 10:

\includegraphics{http://img.thedailywtf.com/images/200907/Josephus.gif}

Escriba la función que reciba los parámetros \lstinline!m! y
\lstinline!n!, y entregue como resultado quién es el sobreviviente:

\begin{lstlisting}
>>> sobreviviente(12, 3)
10
\end{lstlisting}

