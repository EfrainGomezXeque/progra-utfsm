\section{Traslape de rectángulos}

Un rectángulo que está en el plano \emph{xy} y cuyos lados son paralelos a
los ejes cartesianos puede ser representado mediante cuatro datos:
las coordenadas \(x\) e \(y\) de su vértice inferior izquierdo,
su ancho \(w\) y su altura \(h\).

\begin{figure}
  \centering
  \documentclass{minimal}
\usepackage[pdftex,active,tightpage]{preview}
\usepackage[utf8]{inputenc}
\usepackage{mathpazo}
\usepackage{tikz}
\usetikzlibrary{calc,arrows,decorations}

\begin{document}
\begin{preview}
\begin{tikzpicture}[scale=.5]

  \def\gridsize{10}
  \def\Rw{5}
  \def\Rh{6}
  \def\Rx{3}
  \def\Ry{2}
  \draw[gray] (0, 0) grid (\gridsize, \gridsize);
  \draw[thick,->] (-1, 0) -- (\gridsize, 0);
  \draw[thick,->] (0, -1) -- (0, \gridsize);
  \draw[blue, thick, fill=blue!30, opacity=.8] (\Rx, \Ry) rectangle ++(\Rw, \Rh);
  \fill[blue] (\Rx, \Ry) circle (.2cm);
  \foreach\i in {0,...,{\gridsize}} {
      \node[gray] at (\i, -.5) {\i};
      \node[gray] at (-.5, \i) {\i};
  }
  \node[anchor=east] at (\Rx, \Ry) {\((x, y)\)};
  \draw[|<->|, thick] (\Rx + \Rw + 0.5, \Ry) -- node[fill=white] {\(h\)} ++(0, \Rh);
  \draw[|<->|, thick] (\Rx, \Ry + \Rh + 0.5) -- node[fill=white] {\(w\)} ++(\Rw, 0);


\end{tikzpicture}
\end{preview}

\end{document}


  \caption{El rectángulo \lstinline!(3, 2, 5, 6)!.}
  \label{fig:rect1}
\end{figure}

En un programa en Python, esto se traduce a una tupla
\lstinline!(x, y, w, h)! de cuatro elementos.
Por ejemplo, vea el rectángulo de la figura~\ref{fig:rect1}.

\begin{enumerate}
\item
  Escriba la función \lstinline!ingresar_rectangulo()! que pida al
  usuario ingresar los datos de un rectángulo, y retorne la tupla con
  los datos ingresados. La función no tiene parámetros. Al ejecutar la
  función, la sesión debe verse así:

\begin{lstlisting}[language=testcase]
Ingrese x: `3`
Ingrese y: `2`
Ingrese ancho: `5`
Ingrese alto: `6`
\end{lstlisting}

  Con esta entrada, la función retornaría la tupla
  \lstinline!(3, 2, 5, 6)!.
\item

  \begin{figure}
    \centering
    \documentclass{minimal}
\usepackage[pdftex,active,tightpage]{preview}
\usepackage[utf8]{inputenc}
\usepackage{mathpazo}
\usepackage{tikz}
\usetikzlibrary{calc,arrows,decorations}

\newcommand\drawrectangle[6]{
  \draw[#5, thick, fill=#5!30, opacity=.8] (#1, #2) rectangle ++(#3, #4);%
  \fill[#5] (#1, #2) circle (.2cm);
  \node[#5] at ({#1 + #3/2}, {#2 + #4/2}) {#6};
}

\begin{document}
\begin{preview}
\begin{tikzpicture}[scale=.4]

  \def\gridsize{15}
  \draw[gray] (0, 0) grid (\gridsize, \gridsize);
  \draw[thick,->] (-1, 0) -- (\gridsize, 0);
  \draw[thick,->] (0, -1) -- (0, \gridsize);

  \drawrectangle{1}{8}{8}{5}{blue}{A}
  \drawrectangle{7}{6}{3}{6}{green!60!black}{B}
  \drawrectangle{4}{2}{9}{3}{red!60!black}{C}

  \foreach\i in {0,...,\gridsize} {
      \node[gray, anchor=north] at (\i, -.1) {\i};
      \node[gray, anchor=east] at (-.1, \i) {\i};
  }

\end{tikzpicture}
\end{preview}

\end{document}


    \caption{Los rectángulos A y B se traslapan.
      Los rectángulos A y C no se traslapan.}
    \label{fig:rect2}
  \end{figure}

  Escriba la función \lstinline!se_traslapan(r1, r2)! que reciba como
  parámetros dos rectángulos \lstinline!r1! y \lstinline!r2!, y entregue
  como resultado si los rectángulos se traslapan o no.

  Por ejemplo, con los rectángulos de la figura~\ref{fig:rect2}:
\begin{lstlisting}
>>> a = (1, 8, 8, 5)
>>> b = (7, 6, 3, 6)
>>> c = (4, 2, 9, 3)
>>> se_traslapan(a, b)
True
>>> se_traslapan(b, c)
False
>>> se_traslapan(a, c)
False
\end{lstlisting}
\item
  Escriba un programa que pida al usuario ingresar varios rectángulos, y
  termine cuando se ingrese uno que se traslape con alguno de los
  ingresados anteriormente. La salida debe indicar cuáles son los
  rectángulos que se traslapan.

\begin{lstlisting}[language=testcase]
Rectangulo 0
Ingrese x: `4`
Ingrese y: `2`
Ingrese ancho: `9`
Ingrese alto: `3`

Rectangulo 1
Ingrese x: `1`
Ingrese y: `8`
Ingrese ancho: `8`
Ingrese alto: `5`

Rectangulo 2
Ingrese x: `11`
Ingrese y: `7`
Ingrese ancho: `1`
Ingrese alto: `9`

Rectangulo 3
Ingrese x: `2`
Ingrese y: `6`
Ingrese ancho: `7`
Ingrese alto: `1`

Rectangulo 4
Ingrese x: `7`
Ingrese y: `6`
Ingrese ancho: `3`
Ingrese alto: `6`
El rectangulo 4 se traslapa con el rectangulo 1
El rectangulo 4 se traslapa con el rectangulo 3
\end{lstlisting}

%\item
%  (¡Difícil!). Escriba la función
%  \lstinline!contar_regiones_continuas(rectangulos)! que reciba como
%  parámetro una lista de rectángulos, y retorne la cantidad de regiones
%  continuas formadas por rectángulos traslapados.
%
%  Por ejemplo, en el siguiente diagrama hay 15 rectángulos que forman 6
%  regiones continuas de rectángulos traslapados:
%
%%  \documentclass{minimal}
\usepackage[pdftex,active,tightpage]{preview}
\usepackage[utf8]{inputenc}
\usepackage{mathpazo}
\usepackage{tikz}
\usetikzlibrary{calc,arrows,decorations}
\PreviewEnvironment{tikzpicture}

\newcommand\drawrectangle[6]{
  \draw[#5, thick, fill=#5!30, opacity=.8] (#1, #2) rectangle ++(#3, #4);%
  \fill[#5] (#1, #2) circle (.2cm);
  \node[#5] at ({#1 + #3/2}, {#2 + #4/2}) {#6};
}

\begin{document}
\begin{tikzpicture}[scale=.2]

  \def\gridsize{20}
  \draw[gray] (0, 0) grid (\gridsize, \gridsize);
  \draw[thick,->] (-1, 0) -- (\gridsize, 0);
  \draw[thick,->] (0, -1) -- (0, \gridsize);

  \drawrectangle{ 1}{ 8}{8}{5}{green!60!black}{}
  \drawrectangle{ 7}{ 6}{3}{6}{green!60!black}{}
  \drawrectangle{ 1}{ 1}{6}{4}{red!60!black}{}
  \drawrectangle{ 4}{ 2}{9}{3}{red!60!black}{}
  \drawrectangle{15}{ 3}{5}{4}{yellow!80!red!80!black}{}
  \drawrectangle{12}{ 6}{4}{6}{yellow!80!red!80!black}{}
  \drawrectangle{14}{10}{5}{1}{yellow!80!red!80!black}{}
  \drawrectangle{13}{ 7}{2}{2}{yellow!80!red!80!black}{}
  \drawrectangle{ 2}{16}{3}{3}{red!60!green!80!black}{}
  \drawrectangle{ 5}{15}{4}{3}{blue!60!orange}{}
  \drawrectangle{ 8}{16}{4}{3}{blue!60!orange}{}
  \drawrectangle{13}{14}{2}{4}{gray}{}
  \drawrectangle{14}{13}{3}{2}{gray}{}
  \drawrectangle{16}{14}{3}{4}{gray}{}
  \drawrectangle{14}{17}{3}{2}{gray}{}

  %\foreach\i in {0,...,\gridsize} {
  %    \node[gray] at (\i, -.5) {\i};
  %    \node[gray] at (-.5, \i) {\i};
  %}

\end{tikzpicture}

\end{document}


%
%  Los rectángulos de la figura son los siguientes:
%
%\begin{lstlisting}
%rs = [
%    ( 4,  2, 9, 3), (14, 10, 5, 1), (14, 17, 3, 2),
%    (13,  7, 2, 2), ( 8, 16, 4, 3), (13, 14, 2, 4),
%    ( 1,  8, 8, 5), ( 1,  1, 6, 4), (16, 14, 3, 4),
%    (12,  6, 4, 6), ( 7,  6, 3, 6), ( 5, 15, 4, 3),
%    (14, 13, 3, 2), (15,  3, 5, 4), ( 2, 16, 3, 3),
%]
%\end{lstlisting}
%
%  Puede usar esta lista para probar su función:
%\begin{lstlisting}
%>>> contar_regiones_continuas(rs)
%6
%\end{lstlisting}
\end{enumerate}
