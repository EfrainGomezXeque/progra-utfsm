\section{Cumpleaños}

Las fechas pueden ser representadas como tuplas
\lstinline!(año, mes, dia)!.

Para asociar a cada persona su fecha de nacimiento, se puede usar un
diccionario:

\begin{lstlisting}
>>> n = {
...     'Pepito': (1990, 10, 20),
...     'Yayita': (1992, 3, 3),
...     'Panchito': (1989, 10, 20),
...     'Perica': (1989, 12, 8),
...     'Fulanita': (1991, 2, 14),
... }
\end{lstlisting}

\textbf{Ejercicio 1:} escriba una función
\lstinline!mismo_dia(fecha1, fecha2)! que indique si las dos fechas
ocurren el mismo día del año (aunque sea en años diferentes):

\begin{lstlisting}
>>> mismo_dia((2010, 6, 11), (1990, 6, 11))
True
>>> mismo_dia((1981, 8, 12), (1981, 5, 12))
False
\end{lstlisting}

\textbf{Ejercicio 2:} escriba una función \lstinline!mas_viejo(n)! que
indique quién es la persona más vieja según las fechas de nacimiento del
diccionario \lstinline!n!:

\begin{lstlisting}
>>> mas_viejo(n)
'Panchito'
\end{lstlisting}

\textbf{Ejercicio 3:} escriba una función \lstinline!primer_cumple(n)!
que indique quién es la persona que tiene el primer cumpleaños del año:

\begin{lstlisting}
>>> primer_cumple(n)
'Fulanita'
\end{lstlisting}

