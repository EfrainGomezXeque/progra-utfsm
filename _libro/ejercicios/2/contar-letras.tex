\section{Contar letras y palabras}

\begin{enumerate}
\item
  Escriba la función \lstinline!contar_letras(oracion)! que retorne un
  diccionario asociando a cada letra la cantidad de veces que aparece en
  la oracion:

\begin{lstlisting}
>>> contar_letras('El elefante avanza hacia Asia')
{'a': 8, 'c': 1, 'e': 4, 'f': 1, 'h': 1, 'i': 2, 'l': 2,
'n': 2, 's': 1, 't': 1, 'v': 1, 'z': 1}
\end{lstlisting}

  Cada valor del diccionario debe considerar tanto las apariciones en
  mayúscula como en minúscula de la letra correspondiente. Los espacios
  deben ser ignorados.
\item
  Escriba la función \lstinline!contar_vocales(oracion)! que retorne un
  diccionario asociando a cada vocal la cantidad de veces que aparece en
  la oracion. Si una vocal no aparece en la oración, de todos modos debe
  estar en el diccionario asociada al valor 0:

\begin{lstlisting}
>>> contar_vocales('El elefante avanza hacia Asia')
{'a': 8, 'e': 4, 'i': 2, 'o': 0, 'u': 0}
\end{lstlisting}
\item
  Escriba la función \lstinline!contar_iniciales(oracion)! que retorne
  un diccionario asociando a cada letra la cantidad de veces que aparece
  al principio de una palabra:

\begin{lstlisting}
>>> contar_iniciales('El elefante avanza hacia Asia')
{'e': 2, 'h': 1, 'a': 2}
>>> contar_iniciales('Varias vacas vuelan sobre Venezuela')
{'s': 1', 'v': 4}
\end{lstlisting}
\item
  Escriba la función \lstinline!obtener_largo_palabras(oracion)! que
  retorne un diccionario asociando a cada palabra su cantidad de letras:

\begin{lstlisting}
>>> obtener_largo_palabras('el gato y el pato son amigos')
{'el': 2, 'son': 3, 'gato': 4, 'y': 1, 'amigos': 6, 'pato': 4}
\end{lstlisting}
\item
  Escriba la función \lstinline!contar_palabras(oracion)! que retorne un
  diccionario asociando a cada palabra la cantidad de veces que aparece
  en la oración:

\begin{lstlisting}
>>> contar_palabras('El sobre esta sobre el pupitre')
{'sobre': 2, 'pupitre': 1, 'el': 2, 'esta': 1}
\end{lstlisting}
\item
  Escriba la función \lstinline!palabras_repetidas(oracion)! que retorne
  una lista con las palabras que están repetidas:

\begin{lstlisting}
>>> palabras_repetidas('El partido termino cero a cero')
['cero']
>>> palabras_repetidas('El sobre esta sobre el mueble')
['el', 'sobre']
>>> palabras_repetidas('Ay, ahi no hay pan')
[]
\end{lstlisting}
\end{enumerate}
Para obtener la lista de palabras de la oración, puede usar el método
\lstinline!split! de los strings:
\begin{lstlisting}
>>> s = 'el gato y el pato'
>>> s.split()
['el', 'gato', 'y', 'el', 'pato']
\end{lstlisting}
Para obtener un string en minúsculas, puede usar el método
\lstinline!lower!:
\begin{lstlisting}
>>> s = 'Venezuela'
>>> s.lower()
'venezuela'
\end{lstlisting}

