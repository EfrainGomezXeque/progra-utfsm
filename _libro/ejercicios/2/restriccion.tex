\section{Restricción vehicular}

\emph{Este problema apareció en el certamen 2 del segundo semestre de
2011 en el campus Vitacura.}

La ciudad de Pitonia tiene una alta congestión de vehículos circulando
por sus calles. Las autoridades han decidido aplicar restricción
vehicular para descongestionar la ciudad en base a las patentes de los
vehículos.

La patente de un vehículo es un string de cuatro letras y dos dígitos, y
la restricción depende sólo del \emph{penúltimo} dígito. Por ejemplo,
para la patente \lstinline!GEEA78!, el dígito \lstinline!7! es el
utilizado para evaluar la restricción.

La restricción vehícular de los días hábiles de la semana se encuentra
ingresada en el diccionario \lstinline!digitos!, cuyas llaves son los
días de la semanas, y cuyos valores son tuplas de cuatro enteros que
representan los dígitos con restricción de ese día.

\begin{itemize}
\item
  Implemente la función
  \lstinline!esta_con_restriccion(digitos, dia, patente)!, que indique
  si el vehículo está o no con restricción.
\item
  Implemente la función
  \lstinline!dias_con_restriccion(digitos, patente)!, que retorne la
  lista de los días en que el vehículo no puede circular.
\item
  Implemente la función
  \lstinline!dias_sin_restriccion(digitos, patente)!, que retorne el
  conjunto de los días en que el vehículo sí puede circular.
\end{itemize}

\begin{lstlisting}
>>> digitos = {'lunes':     (3, 4, 5, 6), 'martes': (7, 8, 9, 0),
...            'miercoles': (1, 2, 3, 4), 'jueves': (5, 6, 7, 8),
...            'viernes':   (9, 0, 1, 2)}
>>> esta_con_restriccion(digitos, 'lunes', 'BBDT35')
True
>>> dias_con_restriccion(digitos, 'BBDT35')
['lunes', 'miercoles']
>>> dias_sin_restriccion(digitos, 'BBDT35')
set(['jueves', 'martes', 'viernes'])
\end{lstlisting}

