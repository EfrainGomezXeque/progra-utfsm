\section{Asistencia}

La asistencia de los alumnos a clases puede ser llevada en una tabla
como la siguiente:

\begin{quote}
\ctable[pos = H, center, botcap]{llllllll}
{% notes
}
{% rows
\FL
\parbox[b]{0.15\columnwidth}{\raggedright
Clase
} & \parbox[b]{0.06\columnwidth}{\raggedright
1
} & \parbox[b]{0.06\columnwidth}{\raggedright
2
} & \parbox[b]{0.06\columnwidth}{\raggedright
3
} & \parbox[b]{0.06\columnwidth}{\raggedright
4
} & \parbox[b]{0.06\columnwidth}{\raggedright
5
} & \parbox[b]{0.06\columnwidth}{\raggedright
6
} & \parbox[b]{0.06\columnwidth}{\raggedright
7
}
\ML
\parbox[t]{0.15\columnwidth}{\raggedright
Pepito
} & \parbox[t]{0.06\columnwidth}{\raggedright
✓
} & \parbox[t]{0.06\columnwidth}{\raggedright
✓
} & \parbox[t]{0.06\columnwidth}{\raggedright
✓
} & \parbox[t]{0.06\columnwidth}{\raggedright
} & \parbox[t]{0.06\columnwidth}{\raggedright
} & \parbox[t]{0.06\columnwidth}{\raggedright
} & \parbox[t]{0.06\columnwidth}{\raggedright
}
\\\noalign{\medskip}
\parbox[t]{0.15\columnwidth}{\raggedright
Yayita
} & \parbox[t]{0.06\columnwidth}{\raggedright
✓
} & \parbox[t]{0.06\columnwidth}{\raggedright
✓
} & \parbox[t]{0.06\columnwidth}{\raggedright
✓
} & \parbox[t]{0.06\columnwidth}{\raggedright
} & \parbox[t]{0.06\columnwidth}{\raggedright
✓
} & \parbox[t]{0.06\columnwidth}{\raggedright
} & \parbox[t]{0.06\columnwidth}{\raggedright
✓
}
\\\noalign{\medskip}
\parbox[t]{0.15\columnwidth}{\raggedright
Fulanita
} & \parbox[t]{0.06\columnwidth}{\raggedright
✓
} & \parbox[t]{0.06\columnwidth}{\raggedright
✓
} & \parbox[t]{0.06\columnwidth}{\raggedright
✓
} & \parbox[t]{0.06\columnwidth}{\raggedright
✓
} & \parbox[t]{0.06\columnwidth}{\raggedright
✓
} & \parbox[t]{0.06\columnwidth}{\raggedright
✓
} & \parbox[t]{0.06\columnwidth}{\raggedright
✓
}
\\\noalign{\medskip}
\parbox[t]{0.15\columnwidth}{\raggedright
Panchito
} & \parbox[t]{0.06\columnwidth}{\raggedright
✓
} & \parbox[t]{0.06\columnwidth}{\raggedright
✓
} & \parbox[t]{0.06\columnwidth}{\raggedright
✓
} & \parbox[t]{0.06\columnwidth}{\raggedright
} & \parbox[t]{0.06\columnwidth}{\raggedright
✓
} & \parbox[t]{0.06\columnwidth}{\raggedright
✓
} & \parbox[t]{0.06\columnwidth}{\raggedright
✓
}
\LL
}
\end{quote}

En un programa, esta informacion puede ser representada usando listas:

\begin{lstlisting}
>>> alumnos = ['Pepito', 'Yayita', 'Fulanita', 'Panchito']
>>> asistencia = [
...  [True, True, True, False, False, False, False],
...  [True, True, True, False, True,  False, True ],
...  [True, True, True, True,  True,  True,  True ],
...  [True, True, True, False, True,  True,  True ]]
>>>
\end{lstlisting}

\begin{enumerate}
\item
  Escriba la función \lstinline!total_por_alumno(tabla)! que reciba como
  parámetro la tabla de asistencia y retorne una lista con el número de
  clases a las que asistió cada alumno:

\begin{lstlisting}
>>> total_por_alumno(asistencia)
[3, 5, 7, 6]
\end{lstlisting}
\item
  Escriba la función \lstinline!total_por_clase(tabla)! que reciba como
  parámetro la tabla de asistencia y retorne una lista con el número de
  alumnos que asistió a cada clase:

\begin{lstlisting}
>>> total_por_clase(asistencia)
[4, 4, 4, 1, 3, 2, 3]
\end{lstlisting}
\item
  Escriba la función \lstinline!alumno_estrella(asistencia)! que indique
  qué alumno asistió más a clases:

\begin{lstlisting}
>>> alumno_estrella(asistencia)
'Fulanita'
\end{lstlisting}
\end{enumerate}
