\section{Estadísticos de localización}

\begin{enumerate}

  \item
    La \emph{media aritmética} (o promedio) de un conjunto de datos es la
    suma de los valores dividida por la cantidad de datos.

    Escriba la función \lstinline!media_aritmetica(datos)!, donde
    \lstinline!datos! es una lista de números, que entregue la media
    aritmética de los datos:

\begin{lstlisting}
>>> media_aritmetica([6, 1, 4, 8])
4.75
\end{lstlisting}

  \item

    La \emph{media armónica} de un conjunto de datos es el recíproco de la
    suma de los recíprocos de los datos, multiplicada por la cantidad de
    datos:
    \[H = \frac{n}{
    \frac{1}{x_1} +
    \frac{1}{x_2} +
    \cdots +
    \frac{1}{x_n} +
    }\]

    Escriba la función \lstinline!media_armonica(datos)!, que entregue la
    media armónica de los datos:
\begin{lstlisting}
>>> media_armonica([6, 1, 4, 8])
2.5945945945945943
\end{lstlisting}

  \item
    La \emph{mediana} de un conjunto de datos reales es el valor para el
    que el conjunto tiene tantos datos mayores como menores a él.

    Más rigurosamente, la mediana es definida de la siguiente manera:

    \begin{itemize}
    \item
      si la cantidad de datos es impar, la mediana es el valor que queda en
      la mitad al ordenar los datos de menor a mayor;
    \item
      si la cantidad de datos es par, la mediana es el promedio de los dos
      valores que quedan al centro al ordenar los datos de menor a mayor.
    \end{itemize}

    Escriba la función \lstinline!mediana(datos)!, que entregue la mediana
    de los datos:
\begin{lstlisting}
>>> mediana([5.0, 1.4, 3.2])
3.2
>>> mediana([5.0, 1.4, 3.2, 0.1])
2.3
\end{lstlisting}

    La función no debe modificar la lista que recibe como argumento:
\begin{lstlisting}
>>> x = [5.0, 1.4, 3.2]
>>> mediana(x)
3.2
>>> x
[5.0, 1.4, 3.2]
\end{lstlisting}

  \item
    La \emph{moda} de un conjunto de datos es el valor que más se repite.

    Escriba la función \lstinline!modas(datos)!, donde \lstinline!datos! es
    una lista, que entregue una lista con las modas de los datos:
\begin{lstlisting}
>>> modas([5, 4, 1, 4, 3, 3, 4, 5, 0])
[4]
>>> modas([5, 4, 1, 4, 3, 3, 4, 5, 3])
[3, 4]
>>> modas([5, 4, 5, 4, 3, 3, 4, 5, 3])
[3, 4, 5]
\end{lstlisting}

  \item
    Usando las funciones definidas anteriormente, escriba un programa que:
    \begin{itemize}
    \item
      pregunte al usuario cuántos datos ingresará,
    \item
      le pida que ingrese los datos uno por uno, y
    \item
      muestre un reporte con las medias aritmética y armónica, la mediana y
      las modas de los datos ingresados.
    \end{itemize}

    Si alguno de los datos es cero, el reporte no debe mostrar la media
    armónica.

\end{enumerate}
