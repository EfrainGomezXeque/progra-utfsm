\section{Expresiones con estructuras de datos anidadas}

Considere el siguiente trozo de programa:

\begin{lstlisting}
d = {
  (1, 2): [{1, 2}, {3}, {1, 3}],
  (2, 1): [{3}, {1, 2}, {1, 2, 3}],
  (2, 2): [{}, {2, 3}, {1, 3}],
}
\end{lstlisting}

Indique el valor y el tipo de las siguientes expresiones:

\begin{itemize}
\item
  \lstinline!len(d)! (respuesta: el valor es \lstinline!3!, el tipo es
  \lstinline!int!)
\item
  \lstinline!d[(1, 2)][2]! (respuesta: el valor es \lstinline!{1, 3}!,
  el tipo es \lstinline!set!)
\item
  \lstinline!d[(2, 2)][0]!
\item
  \lstinline!(1, 2)!
\item
  \lstinline!(1, 2)[1]!
\item
  \lstinline!d[(1, 2)][1]!
\item
  \lstinline!d[(1, 2)]!
\item
  \lstinline!d[1, 2]!
\item
  \lstinline!len(d[2, 1])!
\item
  \lstinline!len(d[2, 1][1])!
\item
  \lstinline!d[2, 2][1] & d[1, 2][2]!
\item
  \lstinline!(d[2, 2] + d[2, 1])[4]!
\item
  \lstinline!max(map(len, d.values()))!
\item
  \lstinline!min(map(len, d[2, 1]))!
\item
  \lstinline!d[1, 2][-3] & d[2, 1][-2] & d[2, 2][-1]!
\item
  \lstinline!d[len(d[2, 1][1]), len(d[1, 2][-1])][1]!
\end{itemize}

Puede verificar sus respuestas usando la consola interactiva. Para
obtener el tipo, use la función \lstinline!type!:

\begin{lstlisting}
>>> v = d[(1, 2)][2]
>>> v
{1, 3}
>>> type(v)
<class 'set'>
\end{lstlisting}

