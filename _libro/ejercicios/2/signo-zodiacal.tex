\section{Signo zodiacal}

El signo zodiacal de una persona está determinado por su día de
nacimiento.

El diccionario \lstinline!signos! asocia a cada signo el período del año
que le corresponde. Cada período es una tupla con la fecha de inicio y
la fecha de término, y cada fecha es una tupla \lstinline!(mes, dia)!:

\begin{lstlisting}
signos = {
   'aries':       (( 3, 21), ( 4, 20)),
   'tauro':       (( 4, 21), ( 5, 21)),
   'geminis':     (( 5, 22), ( 6, 21)),
   'cancer':      (( 6, 22), ( 7, 23)),
   'leo':         (( 7, 24), ( 8, 23)),
   'virgo':       (( 8, 24), ( 9, 23)),
   'libra':       (( 9, 24), (10, 23)),
   'escorpio':    ((10, 24), (11, 22)),
   'sagitario':   ((11, 23), (12, 21)),
   'capricornio': ((12, 22), ( 1, 20)),
   'acuario':     (( 1, 21), ( 2, 19)),
   'piscis':      (( 2, 20), ( 3, 20)),
}
\end{lstlisting}

Por ejemplo, para que una persona sea de signo libra debe haber nacido
entre el 24 de septiembre y el 23 de octubre.

Escriba la función \lstinline!determinar_signo(fecha_de_nacimiento)! que
reciba como parámetro la fecha de nacimiento de una persona,
representada como una tupla \lstinline!(anno, mes, dia)!, y que retorne
el signo zodiacal de la persona:

\begin{lstlisting}
>>> determinar_signo((1990, 5, 7))
'tauro'
>>> determinar_signo((1904, 11, 24))
'sagitario'
>>> determinar_signo((1998, 12, 28))
'capricornio'
>>> determinar_signo((1999, 1, 11))
'capricornio'
\end{lstlisting}

