\section{Módulo de listas}

Desarrolle un módulo llamado \lstinline!listas.py! que contenga las
siguientes funciones.

\begin{itemize}
\item
  Una función \lstinline!promedio(l)!, cuyo parámetro \lstinline!l! sea
  una lista de números reales, y que entregue el promedio de los
  números:

\begin{lstlisting}
>>> promedio([7.0, 3.1, 1.7])
3.933333333333333
>>> promedio([1, 4, 9, 16])
7.5
\end{lstlisting}
\item
  Una función \lstinline!cuadrados(l)!, que entregue una lista con los
  cuadrados de los valores de \lstinline!l!:

\begin{lstlisting}
>>> cuadrados([1, 2, 3, 4, 5])
[1, 4, 9, 16, 25]
>>> cuadrados([3.4, 1.2])
[11.559999999999999, 1.44]
\end{lstlisting}
\item
  Una función \lstinline!mas_largo(palabras)!, cuyo parámetro
  \lstinline!palabras! es una lista de strings, que entregue cuál es el
  string más largo:

\begin{lstlisting}
>>> mas_largo(['raton', 'hipopotamo', 'buey', 'jirafa'])
'hipopotamo'
>>> mas_largo(['****', '**', '********', '**'])
'********'
\end{lstlisting}

  Si las palabras más largas son varias, basta que entregue una de
  ellas.
\end{itemize}
