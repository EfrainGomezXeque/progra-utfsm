\section{Reporte de notas}

Las notas de un ramo están almacenadas en un archivo llamado
\lstinline!notas.txt!, que contiene lo siguiente:
\begin{lstlisting}[language=file]
Pepito:5.3:3.7:6.7:6.7:7.1:5.5
Yayita:5.5:5.2:2.0:5.6:6.0:2.0
Fulanita:7.1:6.6:6.4:5.1:5.8:6.3
Moya:5.2:4.7:1.8:3.5:2.7:4.5
\end{lstlisting}

Cada línea tiene el nombre del alumno y sus seis notas, separadas por
dos puntos («\lstinline!:!»).

Escriba un programa que cree un nuevo archivo llamado
\lstinline!reporte.txt!, en que cada línea indique si el alumno está
aprobado (promedio \(\ge 4.0\)) o reprobado (promedio \(< 4.0\)):
\begin{lstlisting}
Pepito aprobado
Yayita aprobado
Fulanita aprobado
Moya reprobado
\end{lstlisting}

