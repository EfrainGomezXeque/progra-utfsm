\section{Binarios}

El \href{http://es.wikipedia.org/wiki/Sistema\_binario}{sistema binario}
es un sistema de numeración en los cuales los números se representan
sólo utilizando 0 y 1.

Realice una función llamada \lstinline!binario(n)!, la cual entrada el
número binario correspondiente para un número \lstinline!n! entero.

Para transformar un número decimal a binario se debe dividir el número
por 2 y almacenar el resto, proceso que se repite hasta que el resultado
de dicha división sea 0. Finalmente, el número deseado es el recorrido
en orden inverso de los restos almacenados.

Por ejemplo, al transformar el número decimal 3 al binario, se obtiene:

\begin{lstlisting}
3 : 2 = 1 con resto = 1
1 : 2 = 0 con resto = 1

Por lo tanto, 3 en binario es 11
\end{lstlisting}

Para el número 7, por ejemplo:

\begin{lstlisting}
7 : 2 = 3 con resto = 1
3 : 2 = 1 con resto = 1
1 : 2 = 0 con resto = 1

Por lo tanto, 7 en binario es 111
\end{lstlisting}

De la misma forma, para poder pasar un número binario a entero decimal,
se logra entender el procedimiento.

Primero se invierte el número y se comienza multiplicar cada elemento
por las potencia de 2:

\[Nro original: 1101
Nro invertido: 1011
Mult: 1 \times 2^{0} + 0 \times 2^{1} + 1 \times 2^{3} + 1 \times 2^{4} : 25\]

Desarrolle una función \lstinline!decimal(n)! la cual haga el trabajo de
transformar el binario a una representación de entero decimal:

\begin{lstlisting}
>>> binario(10)
1010
>>> binario(123)
1111011
>>> decimal(10111)
23
>>> decimal(1101111)
111
\end{lstlisting}

