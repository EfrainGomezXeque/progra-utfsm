\section{Láminas}

El módulo \lstinline!random! provee funciones para entregar valores al
azar. En computación, los números creados al azar se llaman
\textbf{números aleatorios}.

La función \lstinline!randrange(n, m)! entrega un número aleatorio en el
rango entre \lstinline!n! y \lstinline!m!:

\begin{lstlisting}
>>> from random import randrange
>>> randrange(5, 15)
14
>>> randrange(5, 15)
6
>>> randrange(5, 15)
14
>>> randrange(5, 15)
5
>>> randrange(5, 15)
12
>>> randrange(5, 15)
9
\end{lstlisting}

(Recuerde que los rangos incluyen el primer valor pero no el último, así
que la función retorna números entre 5 y 14).

\begin{enumerate}
\item
  Suponga que Pepito colecciona un álbum de láminas. Las láminas están
  numeradas desde 1 hasta 640, y se compran en sobres de cinco láminas
  al azar.

  Escriba la función \lstinline!nuevo_sobre()! que entregue una lista
  con las láminas que vienen en un sobre recién comprado:

\begin{lstlisting}
>>> nuevo_sobre()
[27, 31, 207, 455, 529]
>>> nuevo_sobre()
[66, 577, 481, 171, 602]
>>> nuevo_sobre()
[275, 493, 167, 25, 592]
>>> nuevo_sobre()
[113, 35, 592, 560, 244]
\end{lstlisting}
\item
  Pepito lleva un registro de sus láminas en una lista llamada
  \lstinline!laminas_pepito!. Cada ciertos días, Pepito va al quiosco y
  compra algunos sobres, y los agrega a su lista.

  Escriba la función \lstinline!agregar_laminas(lista_laminas, m)!, que
  agregue las láminas de \lstinline!m! nuevos sobres a la
  \lstinline!lista_laminas!:

\begin{lstlisting}
>>> laminas_pepito = []
>>> agregar_laminas(laminas_pepito, 1)
>>> laminas_pepito
[190, 130, 53, 537, 167]
>>> agregar_laminas(laminas_pepito, 2)
>>> laminas_pepito
[190, 130, 53, 537, 167, 572, 537, 375, 496, 469, 249, 545, 95, 279, 131]
>>>
>>> agregar_laminas(laminas_pepito, 14)
>>> len(laminas_pepito)
85
\end{lstlisting}

  Note que la función no retorna nada. Sólo modifica la lista que recibe
  como parámetro.
\item
  Escriba la función \lstinline!faltantes(lista_laminas)! que entregue
  el conjunto de las láminas que faltan para completar el álbum:

\begin{lstlisting}
>>> laminas_pepito = []
>>> agregar_laminas(laminas_pepito, 128)
>>> faltantes(laminas_pepito)
{514, 3, 5, 7, 10, 523, 12, 525, 14, 16, 529, ...}
\end{lstlisting}

  Note que Pepito compró 128 sobres, que en total tienen el mismo número
  de láminas del álbum, pero como hay muchas láminas repetidas y otras
  que no salen, no es suficiente para completar el álbum.
\item
  Escriba la función \lstinline!cuenta(lista_laminas)! que entregue un
  diccionario que asocie a cada lámina el número de veces que está en la
  lista de láminas:

\begin{lstlisting}
>>> laminas_pepito = [4, 6, 9, 12, 9, 9, 6, 12, 2]
>>> cuenta(laminas_pepito)
{9: 3, 2: 1, 4: 1, 6: 2, 12: 2}
\end{lstlisting}
\item
  Pepito intercambia láminas con Yayita, que también colecciona el
  álbum. A Pepito le interesa obtener las láminas que Yayita tiene
  repetidas y que a él le faltan, y viceversa.

  Escriba la función
  \lstinline!cuales_me_sirven(lista_quiere, lista_tiene)! que entregue
  el conjunto de las láminas que le faltan a \lstinline!lista_quiere! y
  que \lstinline!lista_tiene! tiene repetidas:

\begin{lstlisting}
>>> laminas_pepito = [4, 6, 9, 12, 9, 9, 6, 12, 2]
>>> laminas_yayita = [4, 9, 7, 7, 4, 4, 8]
>>> cuales_me_sirven(laminas_pepito, laminas_yayita)
{7}
>>> cuales_me_sirven(laminas_yayita, laminas_pepito)
{12, 6}
\end{lstlisting}

  A Pepito le falta la lámina 7, que Yayita tiene repetida. También le
  falta la 8, pero ella no la tiene repetida, así que no le sirve.
  Yayita tiene repetida la 4, pero Pepito ya la tiene, así que tampoco
  le sirve.
\item
  El sobre de láminas vale \$250. Pepito quiere saber cuánto va a gastar
  en láminas para completar el álbum.

  Escriba la función \lstinline!costo_laminas()! que vaya comprando
  sobres hasta completar las 640 láminas distintas, y que retorne cuál
  fue el gasto total:

\begin{lstlisting}
# Si no sale ninguna repetida, el resultado será:
>>> costo_laminas()
32000

# Si salen algunas repetidas:
>>> costo_laminas()
54250

# Muy mala suerte:
>>> costo_laminas()
241750
\end{lstlisting}
\item
  Vladimiro es un fanfarrón: él desea sacar pica a Yayita por las
  láminas que él tiene y que ella no.

  Escriba la función \lstinline!tengo_y_tu_no(mis_laminas, tus_laminas)!
  que entregue el conjunto de láminas que están en
  \lstinline!mis_laminas! y no en \lstinline!tus_laminas!:

\begin{lstlisting}
>>> laminas_vladimiro = [6, 1, 3, 3, 4, 7]
>>> laminas_yayita = [8, 4, 9, 12, 2, 11, 4, 6, 13, 14]
>> tengo_y_tu_no(laminas_vladimiro, laminas_yayita)
{1, 3, 7}
\end{lstlisting}
\end{enumerate}
