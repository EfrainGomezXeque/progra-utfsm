\section{Dígito verificador}

Desarrolle un programa que calcule el dígito verificador de un rol
UTFSM.

Para calcular el dígito verificador, se deben realizar los siguiente
pasos:

\begin{enumerate}
\item
  Obtener el rol sin guión ni dígito verificador.
\item
  Invertir el número. (e.g: de 201012341 a 143210102).
\item
  Multiplicar los dígitos por la secuencia 2, 3, 4, 5, 6, 7, si es que
  se acaban los números, se debe comenzar denuevo, por ejemplo, con
  143210102:
\end{enumerate}

\[1\times2+ 4\times3+ 3\times4+ 2\times5+ 1\times6+ 0\times7+ 1\times2+ 0\times3+ 2\times4 = 52\]

\begin{enumerate}
\item
  Al resultado obtenido, es decir, 52, debemos sacarle el módulo 11, es
  decir:

  \begin{quote}
  52 \% 11 = 8
  \end{quote}
\item
  Con el resultado obtenido en el paso anterior, debemos restarlo de 11:

  \begin{quote}
  11 − 8 = 3
  \end{quote}
\item
  Finalmente, el dígito verificador será el obtenido en la resta:
  201012341-3.
\end{enumerate}
