\section{Números primos}

Un \href{http://es.wikipedia.org/wiki/N\%C3\%BAmero\_primo}{número
primo} es un número natural que sólo es divisible por 1 y por sí mismo.

Los números que tienen más de un divisor se llaman números compuestos.
El número 1 no es ni primo ni compuesto.

\begin{enumerate}
\item
  Escriba un programa que reciba como entrada un número natural, e
  indique si es primo o compuesto:
\item
  Escriba un programa que muestre los `n` primeros números primos, donde
  `n` es ingresado por el usuario:
\item
  Escriba un programa que muestre los números primos menores que `m`,
  donde `m` es ingresado por el usuario:
\item
  Escriba un programa que cuente cuántos son los números primos menores
  que `m`, donde `m` es ingresado por el usuario:

  En matemáticas, a este valor se le llama `función π`\_.
\item
  Todos los números naturales mayores que 1 pueden ser factorizados de
  una única manera como un `producto de divisores primos`\_.

  Escriba un programa que muestre los factores primos de un número
  entero ingresado por el usuario:
\item
  La `conjetura de Goldbach`\_ sugiere que todo número par mayor que dos
  puede ser escrito como la suma de dos números primos. Hasta ahora no
  se conoce ningún número para el que esto no se cumpla.

  Escriba un programa que reciba un número par como entrada y muestre
  todas las maneras en que puede ser escrito como una suma de dos
  primos:

  Muestre sólo una de las maneras de escribir cada suma (por ejemplo, si
  muestra 61 + 271, no muestre 271 + 61).
\item
  Escriba programas que respondan las siguientes preguntas:

  \begin{itemize}
  \item
    ¿Cuántos primos menores que diez mil terminan en 7?
  \item
    ¿Cuál es la suma de los cuadrados de los números primos entre 1 y
    1000? (Respuesta: 49.345.379).
  \item
    ¿Cuál es el producto de todos los números primos menores que 100 que
    tienen algún dígito 7? (Respuesta: 7 × 17 × 37 × 47 × 67 × 71 × 73 ×
    79 × 97 = 550.682.633.299.463).
  \end{itemize}
\end{enumerate}
