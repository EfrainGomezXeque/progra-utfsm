\section{e}

El número de Euler, \emph{e} ≈ 2,71828, puede ser representado como la
siguiente suma infinita:

\[e = \frac{1}{0!} +  \frac{1}{1!} +  \frac{1}{2!} +  \frac{1}{3!} +  \frac{1}{4!} + \ldots\]

Desarrolle un programa que entregue un valor aproximado de \emph{e},
calculando esta suma hasta que la diferencia entre dos sumandos
consecutivos sea menor que 0,0001.

Recuerde que el factorial \emph{n}! es el producto de los números de 1 a
\emph{n}.
