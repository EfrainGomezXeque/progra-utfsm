\section{Votaciones de la CONFECH}

\emph{Este problema apareció en el certamen 1 del segundo semestre de
2011 en el campus Vitacura.}

La CONFECH, en su afán de agilizar el proceso de recuento de las
votaciones, le ha encargado el desarrollo de un programa de registro de
votación por universidades.

Primero, el programa debe solicitar al usuario ingresar la cantidad de
universidades que participan en el proceso.

Luego, para cada una de las universidades, el usuario debe ingresar el
nombre de la universidad y los votos de sus alumnos, que pueden ser:
\emph{aceptar} (\lstinline!A!), \emph{rechazar} (\lstinline!R!),
\emph{nulo} (\lstinline!N!) o \emph{blanco} (\lstinline!B!). El término
de la votación se indica ingresando una \lstinline!X!, tras lo cual se
debe mostrar los totales de votos de la universidad, con el formato que
se muestra en el ejemplo.

Finalmente, el programa debe mostrar el resultado de la votación,
indicando la cantidad de universidades que aceptan, que rechazan y en
las que hubo empate entre estas dos opciones.
