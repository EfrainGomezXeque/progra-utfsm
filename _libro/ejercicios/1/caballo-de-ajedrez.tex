\section{Caballo de ajedrez}

Un tablero de ajedrez es una grilla de \(8\times 8\) casillas. Cada celda puede
ser representada mediante las coordenadas de su fila y su columna,
numeradas desde 1 hasta 8.

\begin{figure}
  \centering
  \documentclass{minimal}
\usepackage[pdftex,active,tightpage]{preview}
\usepackage[utf8]{inputenc}
\usepackage{mathpazo}
\usepackage{tikz}
\usetikzlibrary{calc,arrows,decorations}
\PreviewEnvironment{tikzpicture}

\begin{document}
\tikzstyle{knight}=[draw, shape=circle, fill=white]
\tikzstyle{jump}=[->, thick]
\tikzstyle{odd}=[black!40!red]
\tikzstyle{even}=[black!60]
\begin{tikzpicture}[yscale=-1, scale=.9]
  \foreach\n in {1,...,8} {
    \node at (-.25, \n - .5) {\n};
    \node at (\n - .5, -.25) {\n};
  }
  \foreach\row in {0,2,4,6} {
    \foreach\col in {1,3,5,7} {
      \fill[odd]  (\row, \col)     rectangle ++(1, 1);
      \fill[even] (\row, \col - 1) rectangle ++(1, 1);
    }
  }
  \foreach\row in {1,3,5,7} {
    \foreach\col in {0,2,4,6} {
      \fill[odd]  (\row, \col) rectangle ++(1, 1);
      \fill[even] (\row, \col + 1) rectangle ++(1, 1);
    }
  }
  \draw[white] (0, 0) grid      ++(8, 8);
  \draw[thick] (0, 0) rectangle ++(8, 8);

  \def\row{2}
  \def\col{8}
  \node[knight] (c1) at (\col - .5, \row - .5) {C};
  \draw [jump, out=180, in= 90] (c1) to (\col - 2 - .5, \row - 1 - .5);
  \draw [jump, out=180, in=-90] (c1) to (\col - 2 - .5, \row + 1 - .5);
  \draw [jump, out= 90, in=  0] (c1) to (\col - 1 - .5, \row + 2 - .5);

  \def\row{3}
  \def\col{4}
  \node[knight] (c1) at (\col - .5, \row - .5) {C};
  \draw [jump, out=180, in= 90] (c1) to (\col - 2 - .5, \row - 1 - .5);
  \draw [jump, out=180, in=-90] (c1) to (\col - 2 - .5, \row + 1 - .5);
  \draw [jump, out=-90, in=  0] (c1) to (\col - 1 - .5, \row - 2 - .5);
  \draw [jump, out= 90, in=  0] (c1) to (\col - 1 - .5, \row + 2 - .5);
  \draw [jump, out=  0, in= 90] (c1) to (\col + 2 - .5, \row - 1 - .5);
  \draw [jump, out=  0, in=-90] (c1) to (\col + 2 - .5, \row + 1 - .5);
  \draw [jump, out=-90, in=180] (c1) to (\col + 1 - .5, \row - 2 - .5);
  \draw [jump, out= 90, in=180] (c1) to (\col + 1 - .5, \row + 2 - .5);

\end{tikzpicture}

\end{document}


  \caption{Ejemplos de los movimientos posibles del caballo de ajedrez.}
  \label{fig:caballo-ajedrez}
\end{figure}

El \href{http://es.wikipedia.org/wiki/Caballo\_(ajedrez)}{caballo} es
una pieza que se desplaza en forma de L: su movimiento consiste en
avanzar dos casillas en una dirección y luego una casilla en una
dirección perpendicular a la primera,
como se muestra en la figura~\ref{fig:caballo-ajedrez}.

Escriba un programa que reciba como entrada las coordenadas en que se
encuentra un caballo, y entregue como salida todas las casillas hacia
las cuales el caballo puede desplazarse.

Todas las coordenadas mostradas deben estar dentro del tablero.

Si la coordenada ingresada por el usuario es inválida, el programa debe
indicarlo.

\begin{lstlisting}[language=testcase]
Ingrese coordenadas del caballo.
Fila: `2`
Columna: `8`

El caballo puede saltar de 2 8 a:
1 6
3 6
4 7
\end{lstlisting}

\begin{lstlisting}[language=testcase]
Ingrese coordenadas del caballo.
Fila: `3`
Columna: `4`

El caballo puede saltar de 3 4 a:
1 3
1 5
2 2
2 6
4 2
4 6
5 3
5 5
\end{lstlisting}

\begin{lstlisting}[language=testcase]
Ingrese coordenadas del caballo.
Fila: `1`
Columna: `9`

Posicion invalida.
\end{lstlisting}

