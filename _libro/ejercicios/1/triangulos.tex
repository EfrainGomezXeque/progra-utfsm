\section{Triángulos}

Los tres lados \(a\), \(b\) y \(c\) de un triángulo deben
satisfacer la
\href{http://es.wikipedia.org/wiki/Desigualdad\_triangular}{desigualdad
triangular}: cada uno de los lados no puede ser más largo que la suma de
los otros dos.

Escriba un programa que reciba como entrada los tres lados de un
triángulo, e indique:

\begin{itemize}
\item
  si acaso el triángulo es inválido; y
\item
  si no lo es, qué tipo de triángulo es.
\end{itemize}

\begin{lstlisting}[language=testcase]
Ingrese a: `3.9`
Ingrese b: `6.0`
Ingrese c: `1.2`
No es un triangulo valido.
\end{lstlisting}

\begin{lstlisting}[language=testcase]
Ingrese a: `1.9`
Ingrese b: `2`
Ingrese c: `2`
El triangulo es isoceles.
\end{lstlisting}

\begin{lstlisting}[language=testcase]
Ingrese a: `3.0`
Ingrese b: `5.0`
Ingrese c: `4.0`
El triangulo es escaleno.
\end{lstlisting}

