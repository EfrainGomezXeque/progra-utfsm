\section{Contar combinaciones de dados}

Un jugador debe lanzar dos dados numerados de 1 a 6, y su puntaje es la
suma de los valores obtenidos.

Un puntaje dado puede ser obtenido con varias combinaciones posibles.
Por ejemplo, el puntaje 4 se logra con las siguientes tres
combinaciones: \(1+3\), \(2+2\) y \(3+1\).

Escriba un programa que pregunte al usuario un puntaje, y muestre como
resultado la cantidad de combinaciones de dados con las que se puede
obtener ese puntaje.

Escriba un programa que pregunte al usuario un puntaje,
y muestre como resultado la cantidad de combinaciones de dados
con las que se puede obtener ese puntaje.

\begin{lstlisting}[language=testcase]
Ingrese el puntaje: `4`
Hay 3 combinaciones para obtener 4
\end{lstlisting}

\begin{lstlisting}[language=testcase]
Ingrese el puntaje: `11`
Hay 2 combinaciones para obtener 11
\end{lstlisting}

\begin{lstlisting}[language=testcase]
Ingrese el puntaje: `17`
Hay 0 combinaciones para obtener 17
\end{lstlisting}

