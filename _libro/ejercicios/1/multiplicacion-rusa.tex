\section{Multiplicación rusa}

El método de
\href{http://mathworld.wolfram.com/RussianMultiplication.html}{multiplicación
rusa} consiste en multiplicar sucesivamente por 2 el multiplicando y
dividir por 2 el multiplicador hasta que el multiplicador tome el valor
1. Luego, se suman todos los multiplicandos correspondientes a los
multiplicadores impares.

Dicha suma es el producto de los dos números. La siguiente tabla muestra
el cálculo realizado para multiplicar 37 por 12, cuyo resultado final es
12 + 48 + 384 = 444.

\ctable[pos = H, center, botcap]{llll}
{% notes
}
{% rows
\FL
Multiplicador & Multiplicando & Multiplicador impar & Suma
\ML
37 & 12 & sí & 12
\\\noalign{\medskip}
18 & 24 & no & 
\\\noalign{\medskip}
9 & 48 & sí & 60
\\\noalign{\medskip}
4 & 96 & no & 
\\\noalign{\medskip}
2 & 192 & no & 
\\\noalign{\medskip}
1 & 384 & sí & 444
\LL
}

Desarrolle un programa que reciba como entrada el multiplicador y el
multiplicando, y entrege como resultado el producto de ambos, calculado
mediante el método de multiplicación rusa.
