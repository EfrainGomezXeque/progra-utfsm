\section{Rango}

\emph{Este problema apareció en el certamen 1 del primer semestre de
2011.}

En estadística descriptiva, se define el \emph{rango} de un conjunto de
datos reales como la diferencia entre el mayor y el menor de los datos.

Por ejemplo, si los datos son:

\begin{quote}
{[}5.96, 6.74, 7.43, 4.99, 7.20, 0.56, 2.80{]},
\end{quote}

entonces el rango es 7.43 − 0.56 = 6.87.

Escriba un programa que:

\begin{itemize}
\item
  pregunte al usuario cuántos datos serán ingresados,
\item
  pida al usuario ingresar los datos uno por uno, y
\item
  entregue como resultado el rango de los datos.
\end{itemize}

Suponga que todos los datos ingresados son válidos.
