\section{Años bisiestos}

Cuando la Tierra completa una órbita alrededor del Sol, no han
transcurrido exactamente 365 rotaciones sobre sí misma, sino un poco
más. Más precisamente, la diferencia es de más o menos un cuarto de día.

Para evitar que las estaciones se desfasen con el calendario, el
calendario juliano introdujo la regla de introducir un día adicional en
los años divisibles por 4 (llamados
\href{http://es.wikipedia.org/wiki/A\%C3\%B1o\_bisiesto}{bisiestos}),
para tomar en consideración los cuatro cuartos de día acumulados.

Sin embargo, bajo esta regla sigue habiendo un desfase, que es de
aproximadamente 3/400 de día.

Para corregir este desfase, en el año 1582 el papa Gregorio XIII
introdujo un nuevo calendario, en el que el último año de cada siglo
dejaba de ser bisiesto, a no ser que fuera divisible por 400.

Escriba un programa que indique si un año es bisiesto o no, teniendo en
cuenta cuál era el calendario vigente en ese año:
