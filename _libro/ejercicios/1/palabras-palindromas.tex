\section{Palabras palíndromas}

Así como hay \href{es-numero-palindromo.html}{números palíndromos},
también hay palabras palíndromas, que son las que no cambian al invertir
el orden de sus letras.
Por ejemplo, «reconocer», «Neuquén» y «acurruca» son palíndromos.

\begin{enumerate}

  \item
    Escriba un programa que reciba como entrada una palabra e indique si
    es palíndromo o no. Para simplificar, suponga que la palabra no tiene
    acentos y todas sus letras son minúsculas:

\begin{lstlisting}[language=testcase]
Ingrese palabra: `sometemos`
Es palindromo
\end{lstlisting}

\begin{lstlisting}[language=testcase]
Ingrese palabra: `rascar`
No es palindromo
\end{lstlisting}

  \item
    Modifique su programa para que reconozca oraciones palíndromas. La
    dificultad radica en que hay que ignorar los espacios:

\begin{lstlisting}[language=testcase]
Ingrese oracion: `dabale arroz a la zorra el abad`
Es palindromo
\end{lstlisting}

\begin{lstlisting}[language=testcase]
Ingrese oracion: `eva usaba rimel y le miraba suave`
Es palindromo
\end{lstlisting}

\begin{lstlisting}[language=testcase]
Ingrese oracion: `puro chile es tu cielo azulado`
No es palindromo
\end{lstlisting}

\end{enumerate}
