\section{Espiral}

\emph{Ejercicio sacado de Project Euler\footnotemark.}
\footnotetext{
  Publicado en \url{http://projecteuler.net/problem=28}
  bajo licencia Creative Commons BY-NC-SA 2.0.
}

La siguiente espiral de \(5 × 5\) se forma comenzando por el número 1, y
moviéndose a la derecha en el sentido de las agujas del reloj:

\begin{tabular}{*{5}{r}}
  \textbf{21} &         22  &         23  &         24  & \textbf{25} \\
          20  & \textbf{ 7} &          8  & \textbf{ 9} &         10  \\
          19  &          6  & \textbf{ 1} &          2  &         11  \\
          18  & \textbf{ 5} &          4  & \textbf{ 3} &         12  \\
  \textbf{17} &         16  &         15  &         14  & \textbf{13} \\
\end{tabular}

La suma de las diagonales de esta espiral es 101.

Escriba un programa que entregue la suma de las diagonales de una
espiral de \(1001 × 1001\) creada de la misma manera.
