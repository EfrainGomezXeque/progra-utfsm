\section{Promoción con descuento}

\emph{Este problema apareció en el certamen 1 del segundo semestre de
2011 en el campus Vitacura.}

El supermercado Pitón Market ha lanzado una promoción para todos sus
clientes que posean la tarjeta Raw Input. La promoción consiste en
aplicar un descuento por cada \emph{n} productos que pasan por caja.

El primer descuento es de 20\%, y se aplica sobre los primeros \emph{n}
productos ingresados. Luego, cada descuento es la mitad del anterior, y
es aplicado sobre los siguientes \emph{n} productos.

Por ejemplo, si \emph{n} = 3 y la compra es de 11 productos, entonces
los tres primeros tienen 20\% de descuento, los tres siguientes 10\%,
los tres siguientes 5\%, y los dos últimos no tienen descuento.

Escriba un programa que pida al usuario ingresar \emph{n} y la cantidad
de productos, y luego los precios de cada producto. Al final, el
programa debe entregar el precio total, el descuento total y el precio
final después de aplicar el descuento.

Si al aplicar el descuento el precio queda con decimales, redondee el
valor hacia abajo.
