\section{Media armónica}

La media armónica de una secuencia de \(n\) números reales
\(x_1, x_2, \ldots, x_n\) se define como:

\[H = \frac{n}{\frac{1}{x_1} + \frac{1}{x_2} + \frac{1}{x_3} + \cdots + \frac{1}{x_n}}.\]

Desarrolle un programa que calcule la media armónica de una secuencia de
números.

El programa primero debe preguntar al usuario cuántos números ingresará.
A continuación, pedirá al usuario que ingrese cada uno de los \(n\)
números reales. Finalmente, el programa mostrará el resultado.

\begin{lstlisting}[language=testcase]
Cuantos numeros: `3`
n1 = `1`
n2 = `3`
n3 = `2`
H = 1.63636363636363636363
\end{lstlisting}
