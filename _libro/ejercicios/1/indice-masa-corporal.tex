\section{Índice de masa corporal}

\emph{Ejercicio sacado de} {[}Camp09{]}\_.

El riesgo de que una persona sufra enfermedades coronarias depende de su
edad y su índice de masa corporal:

\begin{quote}
\ctable[pos = H, center, botcap]{lll}
{% notes
}
{% rows
\FL
\parbox[b]{0.24\columnwidth}{\raggedright
} & \parbox[b]{0.22\columnwidth}{\raggedright
edad \textless{} 45
} & \parbox[b]{0.22\columnwidth}{\raggedright
edad ≥ 45
}
\ML
\parbox[t]{0.24\columnwidth}{\raggedright
\textbf{IMC \textless{} 22.0}
} & \parbox[t]{0.22\columnwidth}{\raggedright
bajo
} & \parbox[t]{0.22\columnwidth}{\raggedright
medio
}
\\\noalign{\medskip}
\parbox[t]{0.24\columnwidth}{\raggedright
\textbf{IMC ≥ 22.0}
} & \parbox[t]{0.22\columnwidth}{\raggedright
medio
} & \parbox[t]{0.22\columnwidth}{\raggedright
alto
}
\LL
}
\end{quote}

El índice de masa corporal es el cuociente entre el peso del individuo
en kilos y el cuadrado de su estatura en metros.

Escriba un programa que reciba como entrada la estatura, el peso y la
edad de una persona, y le entregue su condición de riesgo.
