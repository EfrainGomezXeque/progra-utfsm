\section{Huevos a la copa}

\emph{Ejercicio sacado de} {[}Lang09{]}\_.

Cuando un huevo es hervido en agua, las proteínas comienzan a coagularse
cuando la temperatura sobrepasa un punto crítico. A medida que la
temperatura aumenta, las reacciones se aceleran.

En la clara, las proteínas comienzan a coagularse para temperaturas
sobre 63°C, mientras que en la yema lo hacen para temperaturas sobre
70°C. Para hacer un huevo a la copa, la clara debe haber sido calentada
lo suficiente para coagularse a más de 63°C, pero la yema no debe
sobrepasar los 70°C para evitar obtener un huevo duro.

El tiempo en segundos que toma al centro de la yema alcanzar `T\_y` °C
está dado por la fórmula:

\[t = \frac{M^{2/3} c \rho^{1/3}}
{K\pi^2(4\pi/3)^{2/3}}
\ln\left[
0.76\frac{T_o - T_w}
{T_y - T_w}
\right],\]

donde `M` es la masa del huevo, `rho` su densidad, `c` su capacidad
calorífica específica y `K` su conductividad térmica. Algunos valores
típicos son:

\begin{itemize}
\item
  `M = 47,{[}text\{g\}{]}` para un huevo pequeño y `M =
  67,{[}text\{g\}{]}` para uno grande,
\item
  `rho = 1.038,{[}text\{g\},text\{cm\}\^{}\{-3\}{]}`,
\item
  `c = 3.7,{[}text\{J\},text\{g\}\^{}\{-1\} text\{K\}\^{}\{-1\}{]}`, y
\item
  `K = 5.4cdot 10\^{}\{-3\},{[}text\{W\},text\{cm\}\^{}\{-1\}
  text\{K\}\^{}\{-1\}`{]}.
\end{itemize}

`T\_w` es la temperatura de ebullición del agua y `T\_o` la temperatura
original del huevo antes de meterlo al agua, ambos en grados Celsius.

Escriba un programa que reciba como entrada la temperatura original del
huevo y muestre como salida el tiempo en segundos que le toma alcanzar
la temperatura máxima para prepararlo a la copa.
