\section{Números de Fibonacci}

Los
\href{http://es.wikipedia.org/wiki/N\%C3\%BAmeros\_de\_Fibonacci}{números
de Fibonacci} \(F_k\) son una sucesión de números naturales definidos de
la siguiente manera:
\begin{align*}
  F_0 &= 0, \\
  F_1 &= 1, \\
  F_k &= F_{k - 1} + F_{k - 2}, \qquad\text{cuando } k\ge 2.
\end{align*}

En palabras simples, la sucesión de Fibonacci comienza con 0 y 1, y los
siguientes términos siempre son la suma de los dos anteriores.

\begin{table}
  \begin{tabular}{l*{14}{r}}
    \toprule
    $n$   & 0 & 1 & 2 & 3 & 4 & 5 & 6 &  7 &  8 &  9 & 10 & 11 &  12 & \ldots{} \\
    \midrule
    $F_n$ & 0 & 1 & 1 & 2 & 3 & 5 & 8 & 13 & 21 & 34 & 55 & 89 & 144 & \ldots{} \\
    \bottomrule
  \end{tabular}
  \caption{Los primeros términos \(F_n\) de la sucesión de Fibonacci.}
  \label{tbl:fibonacci}
\end{table}

En la tabla~\ref{tbl:fibonacci} podemos ver los números de Fibonacci desde el
0-ésimo hasta el duodécimo.

\begin{enumerate}

  \item
    Escriba un programa que reciba como entrada un número entero \(n\),
    y entregue como salida el \(n\)-ésimo número de Fibonacci:

\begin{lstlisting}[language=testcase]
Ingrese n: `11`
F11 = 89
\end{lstlisting}

  \item
    Escriba un programa que reciba como entrada un número entero e indique
    si es o no un número de Fibonacci:

\begin{lstlisting}[language=testcase]
Ingrese un numero: `34`
34 es numero de Fibonacci
\end{lstlisting}

\begin{lstlisting}[language=testcase]
Ingrese un numero: `78`
78 no es numero de Fibonacci
\end{lstlisting}

  \item
    Escriba un programa que muestre los \(m\) primeros números de
    Fibonacci, donde \(m\) es un número ingresado por el usuario:

\begin{lstlisting}[language=testcase]
Ingrese m: `7`
Los 7 primeros numeros de Fibonacci son:
0 1 1 2 3 5 8
\end{lstlisting}

\end{enumerate}
