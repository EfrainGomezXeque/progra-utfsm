\section{Método de Newton}

\emph{Ejercicio sacado de} {[}Abel96{]}\_
(\href{http://mitpress.mit.edu/sicp/full-text/book/book-Z-H-4.html\#\%\_toc\_\%\_sec\_1.1.7}{fuente}).

El método computacional más común para calcular raíces cuadradas (y
otras funciones también) es el
\href{http://es.wikipedia.org/wiki/M\%C3\%A9todo\_de\_Newton}{método de
Newton} de aproximaciones sucesivas. Cada vez que tenemos una estimación
`y` del valor de la raíz cuadrada de un número `x`, podemos hacer una
pequeña manipulación para obtener una mejor aproximación (una más
cercana a la verdadera raíz cuadrada) promediando `y` con `x/y`.

Por ejemplo, calculemos la raíz cuadrada de 2 usando la aproximación
inicial `sqrt\{2\}approx 1`:

\ctable[pos = H, center, botcap]{lll}
{% notes
}
{% rows
\FL
Estimación `y` & Cuociente `x/y` & Promedio
\ML
`1` & `2/1 = 2` & `(2 + 1 )/2 = 1.5`
\\\noalign{\medskip}
`1.5` & `2/1.5 = 1.3333` & `(1.3333 + 1.5 )/2 = 1.4167`
\\\noalign{\medskip}
`1.4167` & `2/1.4167 = 1.4118` & `(1.4118 + 1.4167)/2 = 1.4142`
\\\noalign{\medskip}
`1.4142` & \ldots{} & \ldots{}
\LL
}

Al continuar este proceso, obtenemos cada vez mejores estimaciones de la
raíz cuadrada.

El algoritmo debe detenerse cuando la estimación es «suficientemente
buena», que debe ser un criterio bien definido.

\begin{enumerate}
\item
  Escriba un programa que reciba como entrada un número real `x` y
  calcule su raíz cuadrada usando el método de Newton. El algoritmo debe
  detenerse cuando el cuadrado de la raíz cuadrada estimada difiera de
  `x` en menos de 0,0001.

  (Este criterio de detención no es muy bueno).
\item
  Escriba un programa que reciba como entrada el número real `x` y un
  número entero indicando con cuántas cifras decimales de precisión se
  desea obtener su raíz cuadrada.

  El método de Newton debe detenerse cuando las cifras de precisión
  deseadas no cambien de una iteración a la siguiente.

  Por ejemplo, para calcular `sqrt\{2\}` con dos cifras de precisión,
  las estimaciones sucesivas son aproximadamente 1; 1,5; 1,416667 y
  1,414216. El algoritmo debe detenerse en la cuarta iteración, pues en
  ella las dos primeras cifras decimales no cambiaron con respecto a la
  iteración anterior:

  (La cuarta aproximación es bastante cercana a la verdadera raíz
  1.4142135623730951).
\end{enumerate}
