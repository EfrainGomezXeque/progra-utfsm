\section{Adivinar el número}

Escriba un programa que «piense» un número entre 0 y 100, y entregue
pistas al usuario para que lo adivine.

El programa puede obtener un número al azar entre 0 y 100 de la
siguiente manera (¡haga la prueba!):

\begin{lstlisting}
>>> from random import randrange
>>> n = randrange(101)
>>> print n
72
\end{lstlisting}

El usuario debe ingresar su intento, y el programa debe decir si el
número pensado es mayor, menor, o el correcto:

Una vez que complete ese ejercicio, es hora de invertir los roles: ahora
usted pensará un número y el computador lo adivinará.

Escriba un programa que intente adivinar el número pensado por el
usuario. Cada vez que el computador haga un intento, el usuario debe
ingresar \lstinline!<!, \lstinline!>! o \lstinline!=!, dependiendo si el
intento es menor, mayor o correcto.

La estrategia que debe seguir el programa es recordar siempre cuáles son
el menor y el mayor valor posibles, y siempre probar con el valor que
está en la mitad. Por ejemplo, si usted piensa el número 82, y no hace
trampa al jugar, la ejecución del programa se verá así:
