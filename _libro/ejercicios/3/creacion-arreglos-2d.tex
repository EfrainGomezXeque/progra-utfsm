\section{Creación de arreglos bidimensionales}

La función \lstinline!arange! retorna un arreglo con números en el rango
indicado:

\begin{lstlisting}
>>> from numpy import arange
>>> a = arange(12)
>>> a
array([ 0,  1,  2,  3,  4,  5,  6,  7,  8,  9, 10, 11])
\end{lstlisting}

A partir del arreglo \lstinline!a! definido arriba, indique cómo obtener
los siguientes arreglos de la manera más simple que pueda:

\begin{lstlisting}
>>> # ???
array([[ 0,  1,  2,  3],
       [ 4,  5,  6,  7],
       [ 8,  9, 10, 11]])
>>> # ???
array([[  0,   1,   4,   9],
       [ 16,  25,  36,  49],
       [ 64,  81, 100, 121]])
>>> # ???
array([[ 0,  4,  8],
       [ 1,  5,  9],
       [ 2,  6, 10],
       [ 3,  7, 11]])
>>> # ???
array([[ 0,  1,  2],
       [ 4,  5,  6],
       [ 8,  9, 10]])
>>> # ???
array([[ 11.5,  10.5,   9.5],
       [  8.5,   7.5,   6.5],
       [  5.5,   4.5,   3.5],
       [  2.5,   1.5,   0.5]])
>>> # ???
array([[100, 201, 302, 403],
       [104, 205, 306, 407],
       [108, 209, 310, 411]])
>>> # ???
array([[100, 101, 102, 103],
       [204, 205, 206, 207],
       [308, 309, 310, 311]])
\end{lstlisting}

