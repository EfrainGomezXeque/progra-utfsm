\section{Producción de autos}

Una fábrica de autos produce tres modelos: sedán, camioneta y económico.
Cada auto necesita para su producción materiales, personal, impuestos y
transporte. Los costos en unidades por cada concepto son los siguientes:

\ctable[pos = H, center, botcap]{llll}
{% notes
}
{% rows
\FL
(Costos) & Sedán & Camioneta & Económico
\ML
Material & 7 & 8 & 5
\\\noalign{\medskip}
Personal & 10 & 9 & 7
\\\noalign{\medskip}
Impuestos & 5 & 3 & 2
\\\noalign{\medskip}
Transporte & 2 & 3 & 1
\LL
}

Semanalmente, se producen 60 sedanes, 40 camionetas y 90 económicos.

Los costos de una unidad de material, personal, impuestos y transporte
son respectivamente 5, 15, 7 y 2.

Escriba un programa que muestre:

\begin{itemize}
\item
  las unidades semanales necesarias de material, personal, impuestos y
  transporte,
\item
  el costo total de un auto de cada modelo,
\item
  el costo total de la producción semanal.
\end{itemize}
