\section{Rotar matrices}

\begin{enumerate}
\item
  Escriba la función \lstinline!rotar90(a)! que retorne el arreglo
  \lstinline!a! rotado 90 grados en el sentido contrario a las agujas
  del reloj:

\begin{lstlisting}
>>> a = arange(12).reshape((3, 4))
>>> a
array([[ 0,  1,  2,  3],
       [ 4,  5,  6,  7],
       [ 8,  9, 10, 11]])
>>> rotar90(a)
array([[ 3,  7, 11],
       [ 2,  6, 10],
       [ 1,  5,  9],
       [ 0,  4,  8]])
\end{lstlisting}

  Hay dos maneras de hacerlo: la larga (usando ciclos anidados) y la
  corta (usando operaciones de arreglos). Trate de hacerlo de las dos
  maneras.
\item
  Escriba las funciones \lstinline!rotar180(a)! y
  \lstinline!rotar270(a)!:

\begin{lstlisting}
>>> rotar180(a)
array([[11, 10,  9,  8],
       [ 7,  6,  5,  4],
       [ 3,  2,  1,  0]])
>>> rotar270(a)
array([[ 8,  4,  0],
       [ 9,  5,  1],
       [10,  6,  2],
       [11,  7,  3]])
\end{lstlisting}

  Hay tres maneras de hacerlo: la larga (usando ciclos anidados), la
  corta (usando operaciones de arreglos) y la astuta. Trate de hacerlo
  de las tres maneras.
\item
  Escriba el módulo \lstinline!rotar.py! que contenga estas tres
  funciones. %Le será útil más adelante:

\begin{lstlisting}
>>> from rotar import rotar90
>>> a = array([[6, 3, 8],
...            [9, 2, 0]])
>>> rotar90(a)
array([[8, 0],
       [3, 2],
       [6, 9]])
\end{lstlisting}
\end{enumerate}
