\section{Matrices especiales}

\begin{enumerate}
\item
  Una matriz \lstinline!a! es \emph{simétrica} si para todo par de
  índices \lstinline!i! y \lstinline!j! se cumple que
  \lstinline!a[i, j] == a[j, i]!.

  Escriba la función \lstinline!es_simetrica(a)! que indique si la
  matriz \lstinline!a! es simétrica o no.
  Cree algunas matrices simétricas y otras que no lo sean para probar su
  función.

\item
  Una matriz \lstinline!a! es \emph{antisimétrica} si para todo par de
  índices \lstinline!i! y \lstinline!j! se cumple que
  \lstinline!a[i, j] == -a[j, i]! (note el signo menos).

  Escriba la función \lstinline!es_antisimetrica(a)! que indique si la
  matriz \lstinline!a! es antisimétrica o no.
  Cree algunas matrices antisimétricas y otras que no lo sean para
  probar su función.

\item
  Una matriz \lstinline!a! es \emph{diagonal} si todos sus elementos
  que no están en la diagonal principal tienen el valor cero. Por
  ejemplo, la siguiente matriz es diagonal:

  \[\begin{bmatrix}
  9 & 0 & 0 & 0 \\
  0 & 2 & 0 & 0 \\
  0 & 0 & 0 & 0 \\
  0 & 0 & 0 & -1 \\
  \end{bmatrix}\]

  Escriba la función \lstinline!es_diagonal(a)! que indique si la matriz
  \lstinline!a! es diagonal o no.

\item
  Una matriz \lstinline!a! es \emph{triangular superior} si todos sus
  elementos que están bajo la diagonal principal tienen el valor cero.
  Por ejemplo, la siguiente matriz es triangular superior:

  \[\begin{bmatrix}
  9 & 1 & 0 & 4 \\
  0 & 2 & 8 & -3 \\
  0 & 0 & 0 & 7 \\
  0 & 0 & 0 & -1 \\
  \end{bmatrix}\]

  Escriba la función \lstinline!es_triangular_superior(a)! que indique
  si la matriz \lstinline!a! es trangular superior o no.

\item
  No es dificil deducir qué es lo que es una matriz \emph{triangular
  inferior}. Escriba la función \lstinline!es_triangular_inferior(a)!.
  Para ahorrarse trabajo, llame a \lstinline!es_triangular_superior!
  desde dentro de la función.

\item
  Una matriz es \emph{idempotente} si el resultado del producto
  matricial consigo misma es la misma matriz. Por ejemplo:

  \[\begin{bmatrix}
  2 & -2 & -4 \\
  -1 &  3 &  4 \\
  1 & -2 & -3 \\
  \end{bmatrix}
  \begin{bmatrix}
  2 & -2 & -4 \\
  -1 &  3 &  4 \\
  1 & -2 & -3 \\
  \end{bmatrix}
  =
  \begin{bmatrix}
  2 & -2 & -4 \\
  -1 &  3 &  4 \\
  1 & -2 & -3 \\
  \end{bmatrix}\]

  Escriba la función \lstinline!es_idempotente(a)! que indique si la
  matriz \lstinline!a! es idempotente o no.

\item
  Se dice que dos matrices \(A\) y \(B\) \emph{conmutan} si los
  productos matriciales entre \(A\) y \(B\) y entre \(B\) y
  \(A\) son iguales.

  Por ejemplo, estas dos matrices sí conmutan:

  \[\begin{bmatrix}
  1 & 3 \\ 3 & 2 \\
  \end{bmatrix}
  \begin{bmatrix}
  -1 & 3 \\ 3 & 0 \\
  \end{bmatrix} =
  \begin{bmatrix}
  -1 & 3 \\ 3 & 0 \\
  \end{bmatrix}
  \begin{bmatrix}
  1 & 3 \\ 3 & 2 \\
  \end{bmatrix} =
  \begin{bmatrix}
  8 & 3 \\ 3 & 9 \\
  \end{bmatrix}\]

  Escriba la función \lstinline!conmutan! que indique si dos matrices
  conmutan o no. Pruebe su función con estos ejemplos:
\begin{lstlisting}
>>> a = array([[ 1, 3], [3, 2]])
>>> b = array([[-1, 3], [3, 0]])
>>> conmutan(a, b)
True
>>> a = array([[3, 1, 2], [9, 2, 4]])
>>> b = array([[1, 7], [2, 9]])
>>> conmutan(a, b)
False
\end{lstlisting}
\end{enumerate}
