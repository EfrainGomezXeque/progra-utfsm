Un polígono puede ser representado
como una lista de vértices.
Cada vértice es un punto \verb!(x, y)!
en el plano.

\begin{minipage}[T]{0.7\textwidth}
  Por ejemplo,
  el polígono de la figura de la derecha
  quedaría representado así:
  \begin{lstlisting}
p = [(40, 40), (50, 0), (20, 10),
     (10, -10), (-10, 0), (0, 30)]
  \end{lstlisting}

  Escriba un módulo llamado \texttt{poligonos}
  que contenga las funciones
  mostradas a continuación.
\end{minipage}
\hfill
\begin{minipage}[T]{0.25\textwidth}
  \begin{tikzpicture}[scale=0.4]
    \draw[dotted] (-2, -2) grid (6, 5); 
    \draw[-latex'] (-2, 0) -- (6, 0); 
    \draw[-latex'] (0, -2) -- (0, 5); 
    \draw[very thick, Blue]
          ( 4, 4) --
          ( 5, 0) --
          ( 2, 1) --
          ( 1, -1) --
          (-1, 0) --
          ( 0, 3) --
          cycle;
  \end{tikzpicture}
\end{minipage}

\lstinputlisting{poligonos/ejemplos.txt}
