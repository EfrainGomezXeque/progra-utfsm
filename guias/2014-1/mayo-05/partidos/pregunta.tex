Los resultados de un campeonato de fútbol
se pueden almacenar en un diccionario
en que las llaves son los partidos
(tuplas de dos equipos)
y cuyos valores son los resultados
(tuplas con los goles anotados por los equipos):

\lstinputlisting[linerange=PARTIDOS-FIN\ PARTIDOS]{partidos/campeonato.py}

\begin{enumerate}[leftmargin=0pt,label=\emph{\alph*})]
  \item
    Escriba la función \texttt{puntos(equipo, partidos)}
    que entregue la cantidad de puntos obtenida por el equipo
    en el campeonato:
    \lstinputlisting[linerange=PUNTOS-FIN\ PUNTOS]{partidos/ejemplo.txt}

  \newpage
  \item
    Escriba la función \verb!obtener_equipos(campeonato)!
    que retorne la lista de los equipos que participan
    en el campeonato:

    \lstinputlisting[linerange=OBTENER\ EQUIPOS-FIN\ OBTENER\ EQUIPOS]{partidos/ejemplo.txt}

  \item
    Escriba la función \verb!calcular_tabla(campeonato)!
    que retorne un diccionario con los puntos
    que obtuvo cada equipo:

    \lstinputlisting[linerange=CALCULAR\ TABLA-FIN\ CALCULAR\ TABLA]{partidos/ejemplo.txt}

  \item
    Escriba la función \verb!determinar_campeon(campeonato)!
    que retorne el nombre del equipo que obtuvo más puntos.
    Si hay varios equipos empatados en el primer lugar,
    debe retornar cualquiera de ellos.

  \item
    El módulo \verb!chile2001! contiene un diccionario
    con todos los partidos del campeonato chileno del año 2001.
    Escriba un programa que muestre el nombre del equipo
    que fue el campeón en ese año.

\end{enumerate}
