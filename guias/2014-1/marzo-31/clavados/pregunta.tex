En las competencias de clavados
(como las de los Juegos Olímpicos),
cada salto es evaluado
por un panel de siete jueces.

Cada juez entrega una puntuación
en una escala de 1 a 10,
con incrementos de~\sfrac{1}{2}.
La puntuación más alta y la más baja son eliminadas.
La suma de los cinco puntajes restantes
es multiplicada por~\sfrac{3}{5},
y el resultado es multiplicado
por el grado de dificultad del salto.
El valor obtenido es el puntaje total del salto.

El ganador de la competencia
es el clavadista que obtiene
el puntaje total más alto.

\begin{minipage}[t]{.52\textwidth}
  \begin{enumerate}[leftmargin=0pt,label=\emph{\alph*})]
    \item
      Escriba un programa que pida al usuario
      ingresar el nombre de un clavadista,
      el grado de dificultad de su salto
      y los puntajes entregados por los jueces.
      Como salida,
      el programa debe imprimir
      el puntaje total obtenido,
      como se muestra en el ejemplo de la derecha.
    \item
      Escriba un programa que pregunte
      los datos de los saltos de varios clavadistas,
      y al final imprima el nombre del ganador.
      El programa debe terminar
      cuando el nombre ingresado sea \texttt{FIN}.
  \end{enumerate}
\end{minipage}
\hfill
\begin{minipage}[t]{.43\textwidth}
  \lstinputlisting[language=testcase,frame=single]{clavados/caso.txt}
\end{minipage}


