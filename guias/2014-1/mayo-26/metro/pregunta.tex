El módulo \texttt{metro.py} contiene dos objetos:
\begin{itemize}[leftmargin=0pt]
  \item el diccionario \verb!lineas!,
    que asocia a cada línea del metro
    (\verb!'1'!, \verb!'2!', \verb!'4'!, \verb!'4A'!, \verb!'5'!)
    la lista de sus estaciones
    en el orden de su recorrido;
  \item el diccionario \verb!estaciones!,
    que asocia a cada estación de metro
    una tupla con sus coordenadas geográficas
    (latitud y longitud).
\end{itemize}

\begin{enumerate}[leftmargin=0pt,label=\emph{\alph*})]
  \item
    Escriba un programa llamado \texttt{combinacion.py}
    que imprima \textbf{todas} las estaciones de combinación
    siguiendo el siguiente formato:
    \lstinputlisting[
      linerange=SALIDA-FIN\ SALIDA,
      frame=single
    ]{metro/combinacion.py}

  \item
    Escriba un programa llamado \texttt{extremas.py}
    que muestre cuáles son las estaciones que están
    más al norte, al sur, al este y al oeste
    de toda la red:
    \lstinputlisting[
      linerange=SALIDA-FIN\ SALIDA,
      frame=single
    ]{metro/extremas.py}

  \newpage
  \item
    Escriba las siguientes funciones:
    \lstinputlisting{metro/casos.txt}
\end{enumerate}


