Las cartas de una baraja tienen
un palo (♡, ♢, ♣, ♠)
y un valor (A, 2, 3, 4, 5, 6, 7, 8, 9, 10, J, Q, K).

En un programa en Python
podemos representar cada carta
como una tupla \texttt{(valor, palo)},
donde el palo es un string
(\lstinline!'C'!, \lstinline!'D'!, \lstinline!'T'! o \lstinline!'P'!)
y el valor es un número entre 1 y 13.
Por ejemplo,
la carta ``5♡'' sería \lstinline!(5, 'C')!.

Una mano de póker es un conjunto de cinco cartas:
\begin{lstlisting}
mano = {(2, 'T'), (1, 'C'), (12, 'C'), (1, 'D'), (6, 'D')}
\end{lstlisting}

Según las cartas que tenga,
una mano se puede clasificar como:
pareja, doble pareja, trío, escalera,
color, full, póker, escalera de color o escalera real.
Las reglas para hacer esta clasificación
las puede ver en esta página:

\url{https://es.wikipedia.org/wiki/Manos_de_p%C3%B3quer#Manos}

Escriba las funciones
\lstinline!es_pareja!,
\lstinline!es_doble_pareja!,
\lstinline!es_trio!,
\dots,
que reciban como parámetro una mano
y que retornen \verb!True! o \verb!False!
depen\-diendo de si la mano cumple o no la clasificación.
\begin{lstlisting}
>>> m = {(3, 'T'), (8, 'C'), (3, 'C'), (8, 'P'), (3, 'D')}
>>> es_full(m)
True
>>> es_escalera(m)
False
\end{lstlisting}
