\documentclass[12pt,spanish]{article}
\usepackage[utf8]{inputenc}
\usepackage{babel}
\usepackage{mathpazo}
\usepackage{enumitem}
\usepackage{parskip}
\usepackage{fullpage}

\begin{document}
  \thispagestyle{empty}
  \pagestyle{empty}
  \section*{Miércoles 17 de octubre---Instrucciones de la actividad}

  \begin{enumerate}[leftmargin=0pt]

    \item
      Se debe trabajar en equipos de 2 o 3 personas,
      ni más ni menos.

    \item
      Cada equipo puede usar sólo un computador.

    \item
      No puede incluir en su equipo
      a personas que no asistieron a la clase.

    \item
      Un único integrante del equipo debe subir a Moodle
      \textbf{al final de la clase del miércoles}
      el archivo con todos los problemas que se hayan alcanzado a resolver.

    \item
      El archivo debe incluir un comentario al principio
      con el rol y el nombre de todos los integrantes del equipo.

    \item
      Opcionalmente,
      el equipo puede mejorar su código o resolver más problemas
      una vez terminada la clase.
      Hay plazo hasta el domingo en la noche
      para subir el archivo mejorado a Moodle.
      Este archivo será usado en la segunda parte de la actividad.

    \item
      Durante la clase del lunes
      el profesor evaluará oralmente a un integrante del equipo.
      La nota obtenida será asignada a todo el equipo.
      Se descontará nota si algún integrante
      falta a la clase sin justificación.

    \item
      No se aceptará entregas fuera de plazo o por correo electrónico.

    \item
      Está completamente prohibido compartir código con otros equipos.

  \end{enumerate}

\end{document}

