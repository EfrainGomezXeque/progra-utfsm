\documentclass[12pt,spanish]{article}
\usepackage[utf8]{inputenc}
\usepackage{babel}
\usepackage{fullpage}
\usepackage[sc]{mathpazo}
\usepackage{courier}
\usepackage{enumitem}
\usepackage{parskip}
\usepackage{microtype}

\begin{document}
  \section*{22 de octubre}

  Tras la clase del miércoles pasado,
  el profesor de programación comentó a su amigo,
  el periodista deportivo Sapytho Tuplestone,
  que sus estudiantes habían desarrollado un módulo
  para calcular estadísticas de torneos de fútbol.

  Sapytho se entusiasmó con la noticia,
  e inmediatamente escribió dos programas:

  \begin{enumerate}[leftmargin=0pt]
    \item \verb!tabla.py!,
      que muestra las estadísticas de los nueve equipos de las eliminatorias;
    \item \verb!grupos_mundial.py!,
      que muestra las estadisticas de los ocho grupos de cuatro equipos
      del mundial de Sudáfrica 2010.
  \end{enumerate}

  Los datos que ocupan ambos programas
  se encuentran en los módulos \verb!eliminatorias.py! y \verb!mundial.py!, respectivamente.
  Usted debe descargar los cuatro archivos.

  Para la clase de hoy,
  su tarea es asegurarse que los programas escritos por Sapytho funcionen correctamente.
  Para esto, su equipo debe hacer todas las reparaciones necesarias
  al módulo \verb!torneos.py! desarrollado en la clase anterior.

  Primero que todo,
  su equipo debe probar que ambos programas se ejecuten sin errores.
  ¡Esto es lo más importante!
  Si el programa arroja errores, significa que ni siquiera lo ha probado.

  Una vez logrado esto,
  debe verificar que los resultados que muestran los programas son correctos
  (se muestran todos los equipos, se muestran los puntos correctos,
  los equipos están correctamente ordenados).

  Durante la clase,
  el profesor interrogará a uno de los integrantes del equipo.

  Al final de la clase,
  un integrante del equipo debe subir a Moodle el módulo terminado.
  Al principio del archivo deben estar comentados los nombres de los integrantes.

\end{document}

