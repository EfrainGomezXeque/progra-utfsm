En el básquetbol existen tres diferentes tipos de anotaciones:
\begin{itemize}
  \item el tiro libre (\li!L!), que vale un punto,
  \item el doble (\li!D!), que vale dos puntos, y
  \item el triple (\li!T!), que vale tres puntos.
\end{itemize}

\begin{minipage}[t]{0.52\textwidth}
  \parskip=1ex
  Las anotaciones hechas por todos los equipos
  durante los partidos de un día
  están registradas en un archivo
  llamado \texttt{partidos-\textit{año}-\textit{mes}-\textit{día}.txt}.

  En cada línea del archivo aparece el nombre de un equipo
  seguido de dos puntos y un espacio (\verb+: +),
  y a continuación las anotaciones del equipo durante el partido.

  Los equipos que jugaron un partido entre ellos
  aparecen en líneas consecutivas.
  Los partidos están separados entre ellos
  por una línea en blanco.

\end{minipage}
\hfill
\begin{minipage}[t]{0.43\textwidth}
  \centering
  Archivo \verb+partidos-2012-07-02.txt+
  \small
  \lstinputlisting[language=file,frame=single]{reporte-basquet/partidos-2012-07-02.txt}
\end{minipage}

Por ejemplo,
del archivo mostrado arriba a la derecha
podemos inferir que el 2 de julio de 2012
Mentecatos derrotó a Tarugos por un marcador de 39--27.

\begin{enumerate}[leftmargin=0pt,label=\emph{\alph*})]

  \item
    Escriba la función \li!calcular_puntos(anotaciones)!
    que reciba como parámetro un string de anotaciones
    y retorne el total de puntos correspondiente.
    \lstinputlisting[linerange=CASO\ 1-FIN\ CASO\ 1]{reporte-basquet/programa.py}

  \item
    Escriba la función \li!leer_partidos(nombre_archivo)!
    que lea los datos de los partidos del archivo recibido como parámetro,
    y retorne un diccionario
    que asocie a cada partido (tupla de equipos)
    su resultado (tupla de puntajes).
    \lstinputlisting[linerange=CASO\ 2-FIN\ CASO\ 2]{reporte-basquet/programa.py}

\end{enumerate}
