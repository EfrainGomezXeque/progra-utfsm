En un campo de batalla
hay objetivos que se desea destruir.
Todos los objetivos están representados
como tuplas \li!(x, y)!
indicando su posición en el campo.

Durante la próxima misión
se lanzará varias bombas sobre el campo de batalla.
Cada bomba está representada
como una tupla \li!(x, y, r)!,
donde \li!x! e \li!y! indican el punto de impacto
y \li!r! es el radio de acción de la bomba.
Al caer en el punto de impacto,
una bomba destruye todo lo que está
a una distancia menor o igual que \li!r!.

\begin{minipage}{.70\textwidth}
  En el mapa de la derecha están visualizados
  los objetivos y la bomba siguientes:
  \lstinputlisting[linerange=MAPA-FIN\ MAPA]{bombas/programa.py}
  En este ejemplo,
  la bomba destruye tres objetivos.
\end{minipage}
\hfill
\begin{minipage}{.25\textwidth}
  \hfill
  \begin{tikzpicture}[scale=.4, auto]
    \tikzstyle{target}=[inner sep=2pt, circle, fill=gray!40]
    \tikzstyle{friend}=[inner sep=2pt, circle, fill=gray!90]
    \draw[dotted] (0, 0) grid (10, 10);
    \draw (0, 0) rectangle (10, 10);
    \node[target] (f0) at (3.9, 2.1) {0};
    \node[target] (f1) at (8.7, 0.3) {1};
    \node[target] (f2) at (9.1, 8.2) {2};
    \node[target] (f3) at (0.9, 9.5) {3};
    \node[target] (f4) at (5.5, 5.5) {4};
    \node[target] (f5) at (4.5, 4.5) {5};
    \node[target] (f6) at (1.4, 6.2) {6};
    \node[target] (f7) at (7.7, 8.4) {7};
    \node[target] (f8) at (9.1, 3.1) {8};
    \draw (5.1, 3.7) circle (3cm);
    \draw[-latex'] (5.1, 3.7) --
        node [midway, below] {\(r\)}
        +(30:3cm);
    \node[inner sep=1pt, fill, circle] at (5.1, 3.7) {};
  \end{tikzpicture}
\end{minipage}

\begin{enumerate}[leftmargin=0pt,label=\emph{\alph*})]

  \item
    Escriba la función \li!contar_impactos(bomba, objetivos, amigos)!.
    La función debe retornar dos valores:
    la cantidad de objetivos destruídos por la \li!bomba!
    y la cantidad de posiciones amigas destruídas por la bomba.
    Los parámetros \li!objetivos! y \li!amigos! son, respectivamente,
    una lista de objetivos y una lista de posiciones amigas.

  \item
    Una bomba es útil si destruye por lo menos un objetivo
    y no destruye ninguna posición amiga.

    Escriba la función \li!bombas_utiles(bombas, objetivos, amigos)!,
    cuyo primer parámetro es una lista de bombas,
    que retorne una nueva lista que contenga sólo las bombas que son útiles.

  \item
    El módulo \li!random! provee la función \li!random!
    que retorna un número al azar entre 0 y 1:
\begin{lstlisting}
>>> from random import random
>>> random()
0.23982804170484318
>>> random()
0.7492150316162312
>>> random()
0.7981306319510799
\end{lstlisting}
    Escriba la función \li!bomba_aleatoria()!
    que retorne una bomba con radio de acción 0.5,
    y cuyo punto de impacto sea un lugar al azar
    dentro del cuadrado definido por los puntos~\((0, 0)\) y~\((10, 10)\).

\end{enumerate}

