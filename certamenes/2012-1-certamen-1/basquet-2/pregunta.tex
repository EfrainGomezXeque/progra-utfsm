En el básquetbol existen tres diferentes tipos de anotaciones:
\begin{itemize}
  \item el tiro libre (\li!L!), que vale un punto,
  \item el doble (\li!D!), que vale dos puntos, y
  \item el triple (\li!T!), que vale tres puntos.
\end{itemize}
Un partido de básquetbol está dividido
en varios períodos.
%en cuatro períodos de diez minutos cada uno.
%Si un partido termina empatado, se juegan tantos períodos adicionales
%como sean necesarios para desempatar el partido.
%Por lo tanto, es imposible saber de antemano
%cuántos períodos se jugaran en un partido.

Usted debe escribir un programa que reciba como entrada una única línea,
que contenga todas las anotaciones realizadas
por un equipo de básquetbol durante un partido.
Las anotaciones de períodos distintos deben ir separadas por un espacio.
Como salida,
debe mostrar la cantidad de puntos obtenidos en cada período y los puntos totales,
siguiendo el formato del ejemplo.

\begin{minipage}[t]{24em}
  \lstinputlisting[language=testcase,frame=single]{basquet-2/casos.txt}
\end{minipage}

