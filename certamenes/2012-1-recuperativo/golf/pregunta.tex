En un torneo de golf,
cada jugador debe intentar meter la bola
en cada uno de los 18 hoyos del campo
usando la menor cantidad de golpes posible.

Cada uno de los hoyos tiene asignado un \emph{par},
que es la cantidad de golpes
en que se espera que un jugador consiga el hoyo.
El par de un hoyo suele ser 3, 4 o 5,
pero podría ser cualquier valor.

Al final del día,
cada jugador debe entregar una tarjeta
indicando cuántos golpes hizo en cada hoyo.
Esta tarjeta antes se hacía en papel,
pero hoy en día los jugadores simplemente envían por correo electrónico
un archivo de texto de 18 líneas,
cada una de las cuales tiene un número entero.
El nombre del archivo debe ser
\texttt{golpes-\textit{nombre}-\textit{apellido}.txt},
reemplazando con el nombre y el apellido del jugador en minúsculas.

\begin{minipage}[t]{.65\textwidth}
  La organización del torneo usa dos archivos adicionales
  para calcular el marcador del juego.
  El archivo \texttt{pares.txt} también tiene 18 líneas,
  cada una de las cuales tiene el par del hoyo correspondiente.
  El archivo \texttt{jugadores.txt} tiene los nombres de los jugadores
  que participan del torneo.
\end{minipage}
\hfill
\begin{minipage}[t]{.30\textwidth}
  Archivo \texttt{jugadores.txt}:
  \lstinputlisting[language=file,frame=single]{golf/jugadores.txt}
\end{minipage}

Para calcular el puntaje final de un jugador,
se suman las diferencias entre el par de cada hoyo
y el número de golpes que necesitó para completarlo.
Un puntaje negativo es mejor que uno positivo,
porque significa que necesitó menos golpes en total.

\begin{enumerate}[leftmargin=0pt,label=\emph{\alph*})]

  \item
    Escriba la función \li!leer_valores(nombre_archivo)!,
    que recibe como parámetro el nombre de un archivo
    que tiene 18 enteros (uno por línea),
    y retorne un arreglo con esos valores.
    \lstinputlisting[linerange=CASO\ 1-FIN\ CASO\ 1]{golf/programa.py}

    (De este ejemplo podemos deducir
    que el puntaje final de Arnoldo Palmeras fue \(+2\)).

  \item
    Escriba la función \li!obtener_marcador()!,
    que retorne un diccionario
    que asocie a cada jugador su puntaje final.
    \lstinputlisting[linerange=CASO\ 2-FIN\ CASO\ 2]{golf/programa.py}

  \item
    Escriba la función \li!obtener_ganador()!,
    que retorne el nombre del ganador del torneo.
    \lstinputlisting[linerange=CASO\ 3-FIN\ CASO\ 3]{golf/programa.py}

    Si hay varios jugadores empatados en el primer lugar,
    elija a cualquiera (¡pero sólo a uno!) de ellos.

\end{enumerate}

