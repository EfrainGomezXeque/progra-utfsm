La empresa Recuperativix necesita desarrollar un programa
para manipular listas de clientes.
Cada lista de clientes siempre contiene tuplas
con el nombre, el apellido, la fecha de nacimiento y el email de cada cliente.
La fecha de nacimiento es una tupla con el año, el mes y el día, en ese orden.
\lstinputlisting[linerange=CLIENTES-FIN\ CLIENTES]{clientes/programa.py}

La parte de la dirección de email
que va después del último punto se llama TLD,
y está asociada al país de origen del cliente.
Por ejemplo,
el TLD del email de Luisa González es \verb+cl+,
e indica que ella es chilena.

El programa que se está desarrollando
tiene dos variables globales:
el diccionario \li!paises!,
que asocia a cada TLD el país correspondiente,
y la lista \li!meses!,
que tiene los nombres de los doce meses del año.
\lstinputlisting[linerange=PAISES-FIN\ PAISES]{clientes/programa.py}
\lstinputlisting[linerange=MESES-FIN\ MESES]{clientes/programa.py}

\begin{enumerate}[leftmargin=0pt,label=\emph{\alph*})]

  \item
    Escriba la función \li!contar_paises(clientes)!
    que retorne un diccionario que asocie a cada país
    la cantidad de clientes que provienen de él.
    \lstinputlisting[linerange=CASO\ 1-FIN\ CASO\ 1]{clientes/programa.py}

  \item
    Escriba la función \li!persona_mas_joven(clientes)!
    que retorne el nombre completo
    del cliente más joven de la lista.
    \lstinputlisting[linerange=CASO\ 2-FIN\ CASO\ 2]{clientes/programa.py}

  \item
    Al comienzo de cada mes,
    Recuperativix desea enviar un email
    felicitando a todas las personas que tienen cumpleaños durante el mes.

    Escriba la función \li!obtener_mails_cumpleaneros(clientes, mes)!
    que retorne el conjunto de las direcciones de email
    de los \li!clientes! que tienen cumpleaños en el \li!mes! indicado.
    \lstinputlisting[linerange=CASO\ 3-FIN\ CASO\ 3]{clientes/programa.py}

    Note que el valor del parámetro debe ser el nombre del mes,
    no su número.

\end{enumerate}

