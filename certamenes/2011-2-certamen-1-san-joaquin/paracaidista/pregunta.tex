\def\rPlaza{47}
\def\rExtVerde{35}
\def\rIntVerde{20}
\def\rPileta{7}
\tikzstyle{cemento}=[fill=black!10]
\tikzstyle{pileta}=[fill=black!90]
\tikzstyle{verde}=[fill=Green]
\tikzstyle{gente}=[pattern=north west lines, pattern color=Brown]
Para celebrar el aniversario de la fundación de Kreisdorf,
la alcaldesa ha organizado un espectáculo aéreo,
que finalizará con el aterrizaje de un paracaidista
en la plaza del pueblo.

\begin{minipage}[T]{.6\textwidth}
  La plaza es circular, y su radio es de \rPlaza{} metros.
  En su centro hay una pileta
  (\tikz\filldraw[pileta] (0, 0) circle (.6ex) ;)
  de \rPileta{} metros de radio.
  Además, hay cuatro áreas verdes
  (\tikz\filldraw[verde]
        ( 67 :       1.0ex) arc
        ( 67 : 112 : 1.0ex) --
        (      112 : 2.1ex) arc
        (112 :  67 : 2.1ex) -- cycle ;),
  en las posiciones indicadas en la figura;
  los radios interno y externo de las áreas verdes
  son, respectivamente, \rIntVerde{} y \rExtVerde{} metros.
  El resto de la superficie es cemento.
  \\[.6ex]
  El público que irá a presenciar el espectáculo
  se ubicará en el sector de la plaza
  indicado en la figura
  (\tikz\filldraw[gente]
        (0, 0) -- (0:3ex) arc (0:30:3ex) -- cycle;).
  \\[.6ex]
  El paracaidista ha encargado el desarrollo de un programa
  que le permita planificar su trayectoria
  para evitar caer en un lugar peligroso.
  El punto de aterrizaje
  está definido por dos valores:
  su distancia al centro de la plaza
  y el ángulo con respecto a las coordenadas indicadas en la figura.

\end{minipage}
\hfill
\begin{minipage}[T]{.3\textwidth}
  \begin{tikzpicture}[scale=.046]
    % plaza
    \filldraw[cemento] (0, 0) circle (\rPlaza);

    % lineas y ángulos
    \tikzstyle{lineas}=[dashed, gray!50, thick]
    \foreach\angulo in {0,45,...,315} {
      \draw[lineas] (0, 0) -- (\angulo : \rPlaza);
      \node[font=\tiny] at (\angulo: 52) {\angulo\(^{\circ}\)};
    }
    \draw[lineas] (0, 0) circle (\rExtVerde);
    \draw[lineas] (0, 0) circle (\rIntVerde);

    % gente
    \fill[gente] (0, 0) -- (45 : \rPlaza) arc (45 : 90 : \rPlaza) -- cycle;

    % pileta
    \filldraw[pileta] (0, 0) circle (\rPileta);

    % areas verdes
    \foreach\i in {0,2,4,6} {
      \filldraw[verde]
        ({\i * 45} :                   \rIntVerde) arc
        ({\i * 45} : {(\i + 1) * 45} : \rIntVerde) --
        (            {(\i + 1) * 45} : \rExtVerde) arc
        ({(\i + 1) * 45} : {\i * 45} : \rExtVerde) -- cycle;
    }

    % distancias
    \foreach\dist in {\rPlaza,\rExtVerde,\rIntVerde, \rPileta} {
      \edef\ldist{\dist + 3.5}
      \node[font=\tiny] at (247: \ldist) {\dist};
    }

  \end{tikzpicture}
\end{minipage}

Escriba un programa que,
a partir de la distancia y el ángulo de aterrizaje,
le indique al paracaidista donde caerá.
El programa debe imprimir
\li!PILETA!, \li!AREA VERDE!, \li!PUBLICO!, \li!CEMENTO! o \li!FUERA DE LA PLAZA!.
Si el paracaidista aterriza justo en la frontera entre dos regiones,
elija cualquiera de ellas.

\begin{minipage}[t]{.26\textwidth}
  \lstinputlisting[language=testcase,frame=single,linerange=CASO\ 1-FIN\ CASO\ 1]{paracaidista/casos-paracaidista.txt}
\end{minipage}
\hfill
\begin{minipage}[t]{.26\textwidth}
  \lstinputlisting[language=testcase,frame=single,linerange=CASO\ 2-FIN\ CASO\ 2]{paracaidista/casos-paracaidista.txt}
\end{minipage}
\hfill
\begin{minipage}[t]{.26\textwidth}
  \lstinputlisting[language=testcase,frame=single,linerange=CASO\ 3-FIN\ CASO\ 3]{paracaidista/casos-paracaidista.txt}
\end{minipage}

