Una \emph{serie de tiempo}
es una secuencia de valores numéricos
obtenidos al medir algún fenómeno cada cierto tiempo.
Algunos ejemplos de series de tiempo son:
el precio del dólar en cada segundo,
el nivel medio mensual de concentración de CO\(_2\) en el aire y
las temperaturas máximas anuales de una ciudad.
En un programa, los valores de una serie de tiempo se pueden guardar en un arreglo.

\begin{enumerate}
  \item Las \emph{medias móviles con retardo \(p\)} de una serie de tiempo
    son la secuencia de todos los promedios de \(p\)~valores consecutivos de la serie.

    %Por ejemplo,
    %si los valores de la serie son \(\{5, 2, 2, 8, -4, -1, 2\}\),
    %entonces las medias móviles con retardo 3 son
    %\(\frac{5 + 2 + 2}{3}\),
    %\(\frac{2 + 2 + 8}{3}\),
    %\(\frac{2 + 8 - 4}{3}\),
    %\(\frac{8 - 4 - 1}{3}\) y
    %\(\frac{-4 -1 + 2}{3}\).

    Escriba la función \li!medias_moviles(serie, p)!
    que retorne el arreglo de las medias móviles con retardo~\(p\) de la serie:
    \begin{lstlisting}
>>> s = array([5, 2, 2, 8, -4, -1, 2])
>>> medias_moviles(s, 3)
array([ 3,  4,  2,  1, -1])
    \end{lstlisting}

    En este ejemplo,
    las medias móviles son:
    \(\frac{5 + 2 + 2}{3}\),
    \(\frac{2 + 2 + 8}{3}\),
    \(\frac{2 + 8 - 4}{3}\),
    etc.

  \item Las \emph{diferencias finitas} de una serie de tiempo
    son la secuencia de todas las diferencias entre un valor y el anterior.

    %Por ejemplo,
    %si los valores de la serie son \(\{5, 2, 2, 8, -4, -1, 2\}\),
    %entonces las diferencias finitas son:
    %\((2 - 5)\),
    %\((2 - 2)\),
    %\((8 - 2)\),
    %\((-4 - 8)\),
    %\((-1 + 4)\) y
    %\((2 + 1)\).

    Escriba la función \li!diferencias_finitas(serie)!
    que retorne el arreglo de las diferencias finitas de la serie:
    \begin{lstlisting}
>>> s = array([5, 2, 2, 8, -4, -1, 2])
>>> diferencias_finitas(s)
array([ -3,   0,   6, -12,   3,   3])
    \end{lstlisting}

\end{enumerate}

