La empresa de telecomunicaciones Python Está Aquí
desea implementar un programa
para monitorear y controlar la asignación de antenas
para los teléfonos móviles de sus clientes
en una zona específica de la ciudad.
Para ello se considera la siguiente representación:
\begin{itemize}
  \item la posición de cada antena es una tupla \li!(a, (x, y))!,
    donde \li!a! es el nombre de la antena
    y \li!(x, y)! es su ubicación en la zona, dada en kilómetros;
  \item la posición de casa cliente es una tupla \li!(c, (x, y))!,
    donde \li!c! es un identificador del cliente
    y \li!(x, y)! su ubicación en la ciudad, dada en kilómetros.
\end{itemize}
El máximo radio de cobertura de una antena es de 3 kilómetros.

Implemente la función \li!mejor_antena(antenas, clientes, c)!,
cuyos parámetros son las listas de antenas y clientes,
junto con el identificador \li!c!
de un cliente en particular.
La función debe retornar el nombre de la antena
que entrega la mejor cobertura (la más cercana) al cliente.
Si existe más de una, elija cualquiera.
Si no hay antenas dentro del radio de cobertura,
debe retornar \li!None!.

\lstinputlisting[linerange=CASO-FIN\ CASO]{antenas/antenas.py}

