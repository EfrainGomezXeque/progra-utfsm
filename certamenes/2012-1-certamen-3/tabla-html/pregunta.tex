Una \emph{página web}
es simplemente un archivo de texto
usando un lenguaje llamado HTML.
Este lenguaje consiste en algunas etiquetas
que van intercaladas en el texto
para describir las distintas partes de un documento

\begin{minipage}[t]{0.7\textwidth}
  \parskip=2ex
  Las \emph{tablas} son uno de los componentes
  que pueden ser parte de una página web.
  Por ejemplo,
  la siguiente es una tabla de tres filas,
  cada una de las cuales tiene tres celdas:

  \hfil
  \begin{tabular}{|l|r|r|}  \hline
    Manzana &  7 & \$615 \\ \hline
    Uva     & 10 & \$399 \\ \hline
    Kiwi    &  2 & \$471 \\ \hline
  \end{tabular}
  \hfill

  \newcommand\lh{\li[language=html]}
  En un archivo HTML,
  el inicio de una tabla se marca con la etiqueta \lh!<table>!,
  y el final con \lh!</table>!.
  Cada fila comienza con la etiqueta \lh!<tr>!
  y termina con \lh!</tr>!.
  El contenido de cada celda va encerrado entre las etiquetas
  \lh!<td>! y \lh!</td>!.
  Por ejemplo,
  nuestra tabla de frutas se escribiría en HTML
  como se muestra en el ejemplo de la derecha.
  Si uno abre este archivo en un navegador web,
  la tabla aparecería tal como se ve más arriba.

  Evidentemente,
  escribir una tabla directamente en HTML resulta muy tedioso.
  Una estrategia más sencilla es crear la tabla en una planilla de cálculo
  (como MS Excel)
  y guardarla en formato CSV (valores separados por comas).
  Al hacer esto,
  la tabla queda guardada en un archivo como el mostrado a la derecha.

  Aunque no tengan la extensión \verb!.txt!,
  los archivos HTML y CSV sí son de texto plano,
  y por lo tanto es sencillo y conveniente escribir programas
  que lean y escriban en estos formatos.
\end{minipage}
\hfill
\begin{minipage}[t]{0.25\textwidth}
  Archivo \verb+frutas.html+:
  \lstinputlisting[language=html,frame=single]{tabla-html/frutas-ejemplo.html}

  Archivo \verb+frutas.csv+:
  \lstinputlisting[frame=single]{tabla-html/frutas.csv}
\end{minipage}

\begin{enumerate}[leftmargin=0pt,label=\emph{\alph*})]

  \item
    Escriba la función \li!convertir_csv_a_html(tabla)!
    que convierta un archivo CSV a otro HTML\@.
    El parámetro es el nombre del archivo sin la extensión.
    Por ejemplo,
    la siguiente llamada crea el archivo \verb!frutas.html!
    a partir de los datos del archivo \verb!frutas.csv!:
    \lstinputlisting[linerange=CASO\ 1-FIN\ CASO\ 1]{tabla-html/programa.py}

    La función no debe retornar nada; sólo debe crear el archivo.

  \item
    Escriba la función \li!obtener_tamano(archivo_html)!
    que retorne una tupla \li!(m, n)!
    con el número de filas y el número de columnas
    que tiene una tabla HTML.
    \lstinputlisting[linerange=CASO\ 2-FIN\ CASO\ 2]{tabla-html/programa.py}

    \newcommand\lh{\li[language=html]}

    Suponga que, al igual que en el archivo de ejemplo,
    las etiquetas \lh!<table>!, \lh!</table>!, \lh!<tr>! y \lh!</tr>!
    siempre ocupan una línea completa,
    y que las etiquetas \lh!<td>! y \lh!</td>!
    están en la misma línea del valor al que rodean.
    Ademas,
    suponga que todas las filas tienen
    la misma cantidad de celdas,

  \item
    Escriba la función \li!obtener_precio(fruta)! que
    retorne el precio de la fruta pasada como parámetro:
    \lstinputlisting[linerange=CASO\ 3-FIN\ CASO\ 3]{tabla-html/programa.py}

    La función debe buscar los datos en el archivo \verb!frutas.csv!.
    Si la fruta no está en el archivo,
    la función debe retornar \li!None!.

\end{enumerate}
