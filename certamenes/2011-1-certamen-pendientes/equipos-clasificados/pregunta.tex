En un campeonato de fútbol de eliminación directa, los equipos se enfrentan de a pares.
Para cada uno de los enfrentamientos se juegan dos partidos:
uno \emph{de ida} y uno \emph{de vuelta}.
El partido de ida se juega en el estadio del primer equipo,
y el partido de vuelta en el del segundo.
En cada partido, el equipo que juega en su propio estadio se llama \emph{local},
y el otro es la \emph{visita}.

Los resultados de los partidos están registrados en el archivo de texto \texttt{ronda1.txt},
en el que cada línea tiene cuatro datos separados por el símbolo ``\verb+:+'':
los dos equipos, el resultado de la ida y el resultado de la vuelta.

Al final de los dos partidos,
se suman los goles anotados por ambos equipos.
El equipo que anotó más goles en total clasifica para la ronda siguiente,
y el otro queda eliminado.

Si ambos equipos anotaron la misma cantidad total de goles,
el equipo que clasifica es el que anotó más goles jugando de visita.
Si los equipos empatan tanto en goles totales como en goles de visita,
entonces clasifica el que jugó de visita el partido de vuelta.

Los equipos que clasifican a la ronda siguiente se enfrentan nuevamente de a pares.
Los nuevos enfrentamientos se forman tomando los equipos clasificados de dos en dos,
en el mismo orden en que aparecen en el archivo.

Escriba un programa que,
a partir de los resultados registrados en el archivo \texttt{ronda1.txt},
determine cuáles equipos clasificaron,
y genere un nuevo archivo llamado \texttt{ronda2.txt}
con los partidos que se jugarán en la ronda siguiente.
Siga el formato del ejemplo.

El archivo \texttt{ronda1.txt} tiene una cantidad par de líneas,
pero que usted no conoce de antemano.

\begin{minipage}[t]{.28\textwidth}
  Archivo \texttt{ronda1.txt}:
  \lstinputlisting[frame=single]{equipos-clasificados/ronda1.txt}
\end{minipage}
\hspace{3em}
\begin{minipage}[t]{.28\textwidth}
  Archivo \texttt{ronda2.txt}:
  \lstinputlisting[frame=single]{equipos-clasificados/ronda2.txt}
\end{minipage}

