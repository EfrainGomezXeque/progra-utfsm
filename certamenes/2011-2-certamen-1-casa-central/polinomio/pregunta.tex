Un \emph{polinomio} de grado \(n\)
es una función matemática que tiene la forma:
\[
  p(x) =
  a_0     +
  a_1 x   +
  a_2 x^2 +
  a_3 x^3 +
  \cdots +
  a_n x^n.
\]
Los valores \(a_0, \ldots, a_n\)
son los \emph{coeficientes} del polinomio,
y \(x\) es la \emph{variable independiente}.

\begin{minipage}[t]{.63\textwidth}
  Desarrolle un programa
  que evalúe un polinomio.
  \\[1ex]
  Primero,
  el usuario debe ingresar \(x\).
  A continuación,
  debe ingresar los coeficientes en orden.
  Para indicar que todos los coeficientes han sido ingresados,
  se debe escribir el texto \li!FIN!.
  Finalmente,
  el programa debe mostrar
  el valor de \(p(x)\).
  \\[1ex]
  El ejemplo de la derecha
  muestra cómo evaluar el polinomio
  \(p(x) = -7 - 3x^2 + 2x^3\) en \(x = 2.1\).
\end{minipage}
\hfill
\begin{minipage}[t]{.26\textwidth}
  \lstinputlisting[language=testcase,frame=single]{polinomio/caso-polinomio.txt}
\end{minipage}

