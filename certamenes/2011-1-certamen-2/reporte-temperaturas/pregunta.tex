Las temperaturas mínimas y máximas
de algunas ciudades de la región
están guardadas en un diccionario
cuyas llaves son las ciudades
y cuyos valores son tuplas \texttt{(minima, maxima)}.

Se desea generar un archivo
cuyo contenido sea un reporte como el del ejemplo de más abajo.
Los nombres de las ciudades en las que hubo más de 25 grados
deben aparecer en mayúsculas.
El nombre del archivo debe incluir la fecha.
El orden en que aparecen las ciudades
dentro del archivo no importa.

Escriba la función \li!crear_reporte(fecha, temperaturas)!,
cuyos parámetros son
la fecha (una tupla \texttt{(año, mes, día)})
y el diccionario de temperaturas,
y que genere el archivo de texto
con el formato del ejemplo.

La función \li!crear_reporte! no debe retornar nada.
Recuerde que \li!s.upper()!
entrega el string \li!s! en mayúsculas.

\begin{minipage}[T]{.45\textwidth}
  \lstinputlisting[linerange=CASO-]{reporte-temperaturas/pauta2.py}
\end{minipage}
\hfill
\begin{minipage}[T]{.41\textwidth}
  Archivo \texttt{reporte-2011-5-14.txt}:
  \lstinputlisting[language=file]{reporte-temperaturas/reporte-2011-5-14.txt}
\end{minipage}

