(Este ejercicio es una continuación del anterior).

Para guardar la información sobre cuáles de sus usuarios son amigos entre sí,
Fookbace utiliza el conjunto \li!amistades!,
que contiene tuplas con los códigos de dos usuarios.
Si la tupla \li!(a, b)! está dentro del conjunto,
significa que los usuarios con códigos \li!a! y \li!b! son amigos.
En todas las tuplas se cumple que \(\text{\li!a!} < \text{\li!b!}\).
\lstinputlisting[linerange=AMIGOS-FIN\ AMIGOS]{fookbace/pauta3-4.py}
\begin{enumerate}
  \item Escriba la función \li!obtener_amigos(u)!,
    que retorne el conjunto de los códigos
    de los amigos de \li!u!.
  \item Escriba la función \li!recomendar_amigos(u)!,
    que retorne el conjunto de los códigos
    de los usuarios que cumplen todas estas condiciones a la vez:
    \begin{itemize}
      \item son amigos de un amigo de \li!u!,
      \item no son amigos de \li!u!,
      \item viven en la misma ciudad que \li!u!, y
      \item tienen menos de diez años de diferencia con \li!u!.
    \end{itemize}
\end{enumerate}
En ambas funciones,
el parámetro \li!u! es el código de un usuario,
y el valor de retorno es un conjunto de códigos de usuarios.
Recuerde que \li!c.add(x)! agrega el valor \li!x! al conjunto \li!c!.

