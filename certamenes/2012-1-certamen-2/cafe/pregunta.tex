Despues de llegar a su destino,
Anacleta decidio abrir un café.
Como ella anda en bicicleta
no le gusta comprar comida de sobra,
así que requiere que usted implemente un sistema de inventario
que le permita calcular exactamente
cuánta comida debe comprar para cada pedido.

Este sistema usa dos diccionarios: \li!inventario! y \li!recetas!.
Las llaves del diccionario \li!inventario!
son los ingredientes de las recetas de Anacleta,
y sus valores son las cantidades que actualmente tiene en stock en el café:
\lstinputlisting[linerange=INVENTARIO-FIN\ INVENTARIO]{cafe/solucion.py}

Las llaves del diccionario \li!recetas!
son los nombres de los platos que vende Anacleta,
y los valores son listas de tuplas \li!(cantidad, ingrediente)!
que requiere cada plato:
\lstinputlisting[linerange=RECETAS-FIN\ RECETAS]{cafe/solucion.py}

\begin{enumerate}[leftmargin=0pt,label=\emph{\alph*})]

  \item
    Escriba la funcion \li!ingredientes(num_porciones, plato)!,
    que retorne un diccionario que indique
    cuánto se necesita de cada ingrediente
    para cocinar \li!num_porciones! del \li!plato!.
    \lstinputlisting[linerange=CASO\ 1-FIN\ CASO\ 1]{cafe/solucion.py}

  \item
    Escriba la funcion \li!numero_de_porciones(plato, inv)!,
    que retorne el número máximo de porciones del plato
    que Anacleta puede cocinar usando el inventario \li!inv!.
    \lstinputlisting[linerange=CASO\ 2-FIN\ CASO\ 2]{cafe/solucion.py}

  \item
    Escriba la funcion \li!cocinar_juntos(pedido, inv)!,
    que retorne \li!True! si el pedido se puede cocinar
    con el inventario en \li!inv!,
    o \li!False! si no hay suficientes ingredientes.
    Un pedido es una lista de tuplas \li!(numero_de_porciones, plato)!.
    \lstinputlisting[linerange=CASO\ 3-FIN\ CASO\ 3]{cafe/solucion.py}

\end{enumerate}

Note que \li!dict(inventario)! crea
una copia del inventario a la función,
así que si modifica el diccionario \li!inv! dentro de la función
el diccionario \li!inventario! no se vera afectado.

Conteste al reverso de esta misma hoja.

