La Universidad Tropical Filomena Santa Marta
ha instaurado un nuevo reglamento de evaluaciones.
Todas las asignaturas deben tener tres certámenes y un examen.
Las notas van entre 0 y 10, con un decimal.

Después de los tres certámenes,
los alumnos con promedio menor que 3 reprueban y
los con promedio mayor o igual que 7 aprueban.
El resto va al examen,
en el que deben sacarse por lo menos un 5 para aprobar.

Además,
para reducir el trabajo de los profesores,
se decidió que los alumnos que se sacan menos de un 2
en los dos primeros certámenes
están automáticamente reprobados.
A su vez,
los que obtienen más de un 9
en los dos primeros certámenes
están automáticamente aprobados.
En ambos casos,
no deben rendir el tercer certamen.

Escriba un programa que pregunte a un estudiante
las notas de las evaluaciones que rindió,
y le diga si está aprobado o reprobado.

\begin{minipage}[t]{.2\textwidth}
  \lstinputlisting[language=testcase,frame=single,linerange=1-4]{evaluaciones-utfsm/caso-utfsm.txt}
\end{minipage}
\hfil
\begin{minipage}[t]{.2\textwidth}
  \lstinputlisting[language=testcase,frame=single,linerange=5-8]{evaluaciones-utfsm/caso-utfsm.txt}
\end{minipage}
\hfil
\begin{minipage}[t]{.2\textwidth}
  \lstinputlisting[language=testcase,frame=single,linerange=10-14]{evaluaciones-utfsm/caso-utfsm.txt}
\end{minipage}
%\hfil
%\begin{minipage}[t]{.2\textwidth}
%  \lstinputlisting[language=testcase,frame=single,linerange=16-20]{evaluaciones-utfsm/caso-utfsm.txt}
%\end{minipage}
%\hfil
%\begin{minipage}[t]{.2\textwidth}
%  \lstinputlisting[language=testcase,frame=single,linerange=22-25]{evaluaciones-utfsm/caso-utfsm.txt}
%\end{minipage}
\hfil
\begin{minipage}[t]{.2\textwidth}
  \lstinputlisting[language=testcase,frame=single,linerange=27-29]{evaluaciones-utfsm/caso-utfsm.txt}
\end{minipage}

