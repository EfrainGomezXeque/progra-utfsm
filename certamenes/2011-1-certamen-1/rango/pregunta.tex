En estadística descriptiva,
se define el \emph{rango} de un conjunto de datos reales
como la diferencia entre el mayor y el menor de los datos.

Por ejemplo, si los datos son:
\(
  \begin{bmatrix}
  5.96 & 6.74 & 7.43 & 4.99 & 7.20 & 0.56 & 2.80 
  \end{bmatrix}
\)
entonces el rango es \(7.43 - 0.56 = 6.87\).

\begin{minipage}[t]{.43\textwidth}
  Escriba un programa que:
  \begin{itemize}
    \item pregunte al usuario cuántos datos serán ingresados,
    \item pida al usuario ingresar los datos uno por uno, y
    \item entregue como resultado el rango de los datos.
  \end{itemize}
  Suponga que todos los datos ingresados son válidos.
\end{minipage}
\hfill
\begin{minipage}[t]{.45\textwidth}
  \lstinputlisting[language=testcase,frame=single]{rango/caso3.txt}
\end{minipage}

