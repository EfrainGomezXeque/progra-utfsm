Los tres lados \(a\), \(b\) y \(c\) de un triángulo
deben satisfacer la \emph{desigualdad triangular}:
cada uno de los lados no puede ser más
largo que la suma de los otros dos.

Escriba un programa que reciba como entrada
los tres lados de un triángulo, e indique:
\begin{itemize}
  \item si acaso el triángulo es inválido
    (no satisface la desigualdad triangular);
  \item si no lo es, qué tipo de triángulo es
    (escaleno, isóceles o equilátero).
\end{itemize}

\begin{minipage}[t]{.40\textwidth}
  \lstinputlisting[language=testcase,frame=single]{triangulos/caso2a.txt}
\end{minipage}
\hfil
\begin{minipage}[t]{.40\textwidth}
  \lstinputlisting[language=testcase,frame=single]{triangulos/caso2b.txt}
\end{minipage}
\\
\begin{minipage}[t]{.40\textwidth}
  \lstinputlisting[language=testcase,frame=single]{triangulos/caso2c.txt}
\end{minipage}
\hfil
\begin{minipage}[t]{.40\textwidth}
  \lstinputlisting[language=testcase,frame=single]{triangulos/caso2d.txt}
\end{minipage}

