La función seno puede ser representada
como la siguiente suma infinita:
\[
  \text{sen}(x) =
    \frac{x^{ 1}}{{ 1}!} -
    \frac{x^{ 3}}{{ 3}!} +
    \frac{x^{ 5}}{{ 5}!} -
    \frac{x^{ 7}}{{ 7}!} +
    \frac{x^{ 9}}{{ 9}!} -
    \frac{x^{11}}{{11}!} +
    \frac{x^{13}}{{13}!} -
    \cdots,
\]
donde \(n! = 1\cdot 2\cdot\,\cdots\,\cdot n\)
es el factorial de \(n\).

Los términos de la suma son cada vez más pequeños,
por lo que tomando algunos de los primeros términos
es posible obtener una buena aproximación de la función.

Escriba un programa que pregunte al usuario
un valor de \(x\) y una precisión \(p\),
y entregue como salida el valor aproximado de \(\text{sen}(x)\),
obtenido al sumar términos de la suma infinita
hasta que la diferencia entre dos sumandos consecutivos
sea menor o igual que \(p\).

\begin{minipage}[t]{.40\textwidth}
  \lstinputlisting[language=testcase,frame=single]{seno/caso3.txt}
\end{minipage}

