Las \emph{ecuaciones de Lotka-Volterra}
describen cómo cambian las poblaciones de dos especies en un ecosistema,
en el que una de ellas es depredadora de la otra.

En un día cualquiera,
\(y\) representa el número de depredadores (por ejemplo, lobos),
\(x\) el número de presas (por ejemplo, conejos),
y el cambio de estas cantidades de un día para el otro
están dados por las ecuaciones:
\begin{align*}
  \Delta x &\approx \phantom{-}x(\alpha - \beta  y) \\
  \Delta y &\approx           -y(\gamma - \delta x)
\end{align*}
donde \(\alpha\), \(\beta\), \(\gamma\) y \(\delta\)
son parámetros que dependen de las características del ecosistema.

Para que el modelo funcione,
\(x\) e \(y\) deben ser tratados como números reales,
no como enteros.

Suponga que en un ecosistema inicialmente hay
\(x = 300\) conejos e \(y = 400\) lobos,
y que la dinámica del sistema es descrita por los valores
\(\alpha = 3\cdot 10^{-2}\),
\(\beta  = 4\cdot 10^{-5}\),
\(\gamma = 5\cdot 10^{-2}\) y
\(\delta = 6\cdot 10^{-6}\).

Escriba un programa que calcule y muestre
en cuántos días se extinguirán los conejos
(es decir, \(y < 1\)).

La salida debe verse así:

\begin{minipage}[t]{.60\textwidth}
  \lstinputlisting[language=testcase,frame=single]{lotka-volterra/caso4.txt}
\end{minipage}

