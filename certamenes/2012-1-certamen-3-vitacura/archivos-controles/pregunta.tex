%Por supuesto,
%el profesor Sansanales no llenará a mano la matriz de controles.
%Durante el semestre,
%él usa el Bloc de Notas para ir llenando un archivo
El Ministerio de Educación exige todos los semestres a los profesores
entregar un reporte con las notas de sus alumnos.

\begin{minipage}[t]{.58\textwidth}
  El reporte debe ser un archivo de texto plano
  con una estructura como la mostrada
  en el ejemplo de la derecha.
  El nombre del archivo debe tener el año y el número de semestre
  (que puede ser \li+1+ o \li+2+).
  Por supuesto,
  el profesor Sansanales quiere evitar el trabajo
  de escribir el reporte a mano.
\end{minipage}
\hfill
\begin{minipage}[t]{.38\textwidth}
  \centering
  \verb+notas-2012-1.txt+
  \lstinputlisting[language=file,frame=single]{archivos-controles/notas-2012-1.txt}
\end{minipage}

\begin{enumerate}[leftmargin=0pt,label=\emph{\alph*})]

%  \item
%    Escriba la función \li!cargar_notas(nombre_archivo, nr_controles, nr_alumnos)!
%    que lea las notas de un archivo 
%    y retorne un arreglo como el descrito en la pregunta anterior:
%    \lstinputlisting[linerange=CASO\ 1-FIN\ CASO\ 1]{archivos-controles/programa.py}

  \item
    Escriba la función \li!crear_reporte(notas, anno, semestre)!,
    que reciba como parámetros el arreglo de notas
    (como el usado en la pregunta anterior),
    el año y el número de semestre,
    y que escriba el archivo con el reporte de notas.
    \lstinputlisting[linerange=CASO\ 1-FIN\ CASO\ 1]{archivos-controles/programa.py}

    La función no debe retornar nada; sólo debe crear el archivo.
    Recuerde que los nombres de los alumnos
    están en la lista global \li!alumnos!.

  \item
    Escriba la función \li!promedio_semestral(nombre_archivo)!,
    que reciba como parámetro el nombre de un archivo
    de reporte como el del ejemplo,
    y retorne el promedio de todas las notas que están en él.
    \lstinputlisting[linerange=CASO\ 2-FIN\ CASO\ 2]{archivos-controles/programa.py}

    El archivo podría tener cualquier cantidad de alumnos
    y cualquier cantidad de controles por alumno.
    El promedio debe estar redondeado
    al entero más cercano (recuerde la función \li!round(x)!).

  \item
    El profesor Sansanales ha hecho clases
    desde el primer semestre de 1995 hasta ahora,
    y ha conservado todos los reportes generados desde entonces.
    Para evaluar su desempeño a lo largo del tiempo,
    él quiere observar cómo ha variado el promedio de sus cursos
    en todos estos años.

    Escriba la función \li!resumen_promedios()! que,
    a partir de los reportes de notas existentes,
    retorne la lista con todos los promedios semestrales
    desde que ha hecho clases.

\end{enumerate}

