Una circunferencia en el plano está definida por tres valores:
las coordenadas \((x, y)\) de su centro, y su radio \(r\).

Escriba un programa que determine si dos circunferencias
se intersectan o no.

\def\grilla{%
  \draw[gray] (0, 0) grid (14, 10);
  \draw[-latex'] (0, 0) to (15, 0);
  \draw[-latex'] (0, 0) to (0, 11);
}
\newcommand\circulo[3]{%
  \node[inner sep=1pt, fill, circle] (p1) at (#1, #2) {};
  \draw[very thick] (p1) circle (#3 cm);
}

\begin{tikzpicture}[scale=.3]
  \grilla
  \circulo{ 5}{6}{3.5}
  \circulo{10}{5}{3}
\end{tikzpicture}
\hfill
\begin{tikzpicture}[scale=.3]
  \grilla
  \circulo{3.5}{5}{2}
  \circulo{10}{4}{3}
\end{tikzpicture}
\hfill
\begin{tikzpicture}[scale=.3]
  \grilla
  \circulo{5}{4.5}{3}
  \circulo{6}{5}{4.5}
\end{tikzpicture}

\begin{minipage}[t]{.28\textwidth}
  \lstinputlisting[language=testcase,frame=single,linerange=CASO\ 1-FIN\ CASO\ 1]{interseccion-circunferencias/caso-circ.txt}
\end{minipage}
\hfill
\begin{minipage}[t]{.28\textwidth}
  \lstinputlisting[language=testcase,frame=single,linerange=CASO\ 2-FIN\ CASO\ 2]{interseccion-circunferencias/caso-circ.txt}
\end{minipage}
\hfill
\begin{minipage}[t]{.28\textwidth}
  \lstinputlisting[language=testcase,frame=single,linerange=CASO\ 3-FIN\ CASO\ 3]{interseccion-circunferencias/caso-circ.txt}
\end{minipage}

